% Copyright 2011 The LaTeX Project
\documentclass{ltnews}
\PassOptionsToPackage{colorlinks}{hyperref}

\usepackage{metalogo,ragged2e}

\AtBeginDocument{
  \renewcommand{\LaTeXNews}{\LaTeX3~News}
  \RaggedRight
}

\usepackage{hologo}

\publicationmonth{January}
\publicationyear{2011}
\publicationissue{5}

\begin{document}
\maketitle


\section{Happy new year}

Seasons greetings for 2011!
As the previous news issue was released late, this season's issue will be shorter than usual.

\section{The LPPL is now OSI-approved}

We are happy to report that earlier this year the \LaTeX\ Project Public License (LPPL) has been approved by the OSI as an open source licence.\footnote{\url{http://www.opensource.org/licenses/lppl}} Frank Mittelbach will be publishing further details on this news in a retrospective on the LPPL in an upcoming TUGboat article.

\section{Reflections on 2010}

We are pleased to see the continued development and discussion in the \TeX\ world.
The \LaTeX\ ecosystem continues to see new developments and a selection of notable news from the second half of last year include:
\begin{itemize}
\item[June] The TUG~2010 conference was held very successfully in San
Francisco; videos, slides, and papers from \LaTeX3 Project members are available from our website.\footnote{\url{http://www.latex-project.org/papers/}}
\item[Aug.]
The \TeX\ Stack Exchange\footnote{\url{http://tex.stackexchange.com}} question\,\&\,answer website was created and has since grown quickly. At time of writing, some 2800 people have asked 2600 questions with 5600 answers total, and 2200 users are currently visiting daily.
\item[Sept.] \TeX\ Live 2010 was released: each year the shipping date is earlier; the production process is becoming more streamlined and we congratulate all involved for their hard work. One of the most notable new components of \TeX\ Live 2010 includes the `restricted shell escape' feature to allow, among other things, automatic EPS figure conversion for pdf\LaTeX\ documents.
\item[Oct.] TLContrib\footnote{\url{http://tlcontrib.metatex.org/}} was opened by Taco Hoekwater as a way to update a \TeX~Live installation with material that is not distributable through \verb|tlmgr| itself. Such material includes executables (e.g., new versions of Lua\TeX), non-free code, or test versions of packages.
\item[Nov.] Philipp Lehman released the first stable version of \textsf{biblatex}. One of the most ambitious \LaTeX\ packages in recent memory, \textsf{biblatex} is a highly flexible package for managing citation cross-referencing and bibliography typesetting. In `beta' status for some years now, reaching this point is a great milestone.
\item[Dec.] Lua\TeX\ 0.65. We are happy to see Lua\TeX\ development steadily continuing. \LaTeX\ users may use Lua\TeX\ with the \verb|lualatex| program. Like \verb|xelatex|, this allows \LaTeX\ documents to use multilingual OpenType fonts and Unicode text input.
\end{itemize}

\section{Current progress}

The \textsf{expl3} programming modules continue to see revision and expansion; we have added a Lua\TeX\ module, but \textsf{expl3} continues to support all three of pdf\LaTeX, \XeLaTeX, and Lua\LaTeX\ equally.

The \textsf{l3fp} module for performing floating-point arithmetic has been extended and improved. Floating point maths is important for some of the calculations required for complex box typesetting performed in the new `coffins' code.
The \textsf{l3coffin} module has been added based on the original \textsf{xcoffins} package introduced at TUG~2010 as reported in the last news issue; this code is now available from CTAN for testing and feedback.

We have consolidated the \textsf{l3int} and \textsf{l3intexpr} modules (which were separate for historical purposes); all integer/count-related functions are now contained within the `\textsf{int}' code and have prefix \verb|\int_|. Backwards compatibility is provided for, but eventually we will drop support for the older \verb|\intexpr_| function names.

\section{Plans for 2011}

In the following year, we plan to use the current \LaTeX3 infrastructure to continue work in building high-level code for designing \LaTeX\ documents using the \textsf{xtemplate} package. Our first priority is to look at section headings and document divisions, as we see this area as one of the most difficult, design-wise, of the areas to address. From there we will broaden our scope to more document elements.

We will also do some low-level work on the `galley', which is the code that \LaTeX3 uses to build material for constructing pages, and we will continue to extend \textsf{expl3} into a more complete system from which we can, one day, create a pure \LaTeX3 format.

\end{document}



