%% OPmac -- Olsak's PlainTeX macros
%%%%%%%%%%%%%%%%%%%%%%%%%%%%%%%%%%%
% User documetation
% This file is encoded in UTF-8

\chyph  % use (pdf)csplain
\input opmac
\input pdfuni
\hyperlinks{\Blue}{\Green}

\typosize[11/13]

\catcode`<=13
\def<#1>{\hbox{$\langle$\it#1\/$\rangle$}}

\activettchar"

\insertoutline{OBSAH} \outlines{0} 

\tit OPmac -- rozšiřující makra plain\TeX{}u
%%%%%%%%%%%%%%%%%%%%%%%%%%%%%%%%%%%%%%%%%%%%

\centerline{\it Petr Olšák, 2012 -- 2019}

\bigskip
\centerline{\url{http://petr.olsak.net/opmac.html}}

\notoc\nonum \sec Obsah
\maketoc

\nonum \sec Úvod
%%%%%%%%%
OPmac je balík jednoduchých doplňujících maker k plain\TeX{}u umožňující
uživatelům základní \LaTeX{}ovou funkcionalitu: změny velikosti písma,
automatickou tvorbu obsahu a rejstříku, práci s {\tt bib} databázemi,
referencemi, možnost proložení referencí hyperlinkovými odkazy atd.

Hlavní zásady balíku OPmac jsou:

\begitems
* V jednoduchosti je síla.
* Makra nejsou univerzální, ale jsou čitelná a srozumitelná.
* Uživatel si makra může snadno předefinovat k obrazu svému. 
\enditems

Každé makro je napsáno s cílem co největší srozumitelnosti pro lidi, kteří
to budou chtít číst a měnit. Troufám si říci, že balík nabízí čtenáři
inspiraci, jak se programují \TeX{}ová makra. Z~kódu maker je cítit jistá
elegance. Technická část dokumentace k OPmac by tedy mohla sloužit jako učebnice
programování \TeX{}ových maker.
To je zásadní rozdíl od koncepce \LaTeX{}u. Když se člověk podívá
do \LaTeX{}ového souboru "latex.ltx", vystřeví se na něj množství zavináčů a
makra, ze kterých je často cítit topornost a mnohdy nepochopení vnitřní
koncepce \TeX{}u. Skoro nikdo se v~tom nevyzná. Soubor "latex.ltx" obsahuje
8000 řádků a schopnosti \LaTeX{}u jsou navíc ukryty v~desítkách různých
dalších makro souborech, zatímco v~OPmac vidíte vše pohromadě a názorně. 
Navíc v některých věcech OPmac výrazně překračuje možnosti \LaTeX{}u: 
generování rejstříků bez externího programu, přímé čtení "*.bib" souborů
bez bib\TeX{}u, listingy externích souborů.

Balík OPmac nabízí podobně jako \LaTeX{} autorům textů {\it rozhraní}, tj.
smluvenou sadu značek na vymezení struktury dokumentu. Je jiná, než v
\LaTeX{}u, možná nabídne napsat zdrojový text článku poněkud přehledněji a
oku více lahodícím způsobem. Balík OPmac ovšem neřeší typografický vzhled
dokumentu. Bez doplňujících maker vyleze jednoduchý střízlivý dokument.
Předpokládá se, že autor dodatečných plain\TeX{}ových maker ušije vzhled
dokumentu na míru konkrétnímu požadavku.

Příklad začátku dokumentu:

\begtt
\input opmac        % zavedení makra OPmac
\chyph              % zapnutí češtiny 
\fontfam [LM fonts] % použití Latin Modern fontů
\typosize[12/14]    % nastavení základní velikosti sazby
\endtt

Makro OPmac spolupracuje s běžnými plain\TeX{}ovými formáty: s Knuthovým
klasickým \hbox{plain}\TeX{}em nebo s \csplain{}em nebo s plain\TeX{}em
doplněným makry z~"etex.src" (ten je v~běžných \TeX{}ových distribucích 
základním formátem pdf\TeX{}u, Xe\TeX{}u a Lua\TeX{}u). 

Uživatelům OPmac nabízím konzultace po emailu a uvítám hlášení o chybách.
Výsledky konzultací typicky zveřejňuji na
\url{http://petr.olsak.net/opmac-tricks.html}, kde najdete desítky
nejrůznějších řešení dílčích problémů.


\sec Výběr fontové rodiny
%%%%%%%%%%%%%%%%%%%%%%%%%

OPmac implicitně rodinu fontů nenastavuje, tj. fonty jsou připraveny stejné
jako v plainTeXu (CM fonty) nebo v \CS{}plainu (CS fonty). Je ale možné 
použít tzv. \uv{fontové soubory} pro zavedení jiné fontové rodiny, tj.
typicky čtyř základních variant "\rm", "\bf", "\it" a "\bi". Tyto soubory
interně používají primitivní příkaz "\font" pro zavedení jednotlivých fontů.

Nemusíte si pamatovat název souboru, který je potřeba pro zavedení fontové
rodiny použít. Stačí napsat "\fontfam[<NázevRodiny>]" a požadovaný
soubor se načte. V argumentu <NázevRodiny> nezáleží na mezerách a velkých
písmenech, takže "\fontfam[Times Roman]" je totéž jako
"\fontfam[TimesRoman]" i jako "\fontfam[timesroman]". Také je připraveno
několik běžných aliasů, takže třeba "\fontfam[times]" rovněž zavede rodinu
Times~Roman.

Použijete-li "\fontfam[?]", zobrazí se na terminálu a v log souboru přehled
všech fontových rodin dostupných pomocí fontových souborů. Část seznamu vypadá
takto:

\begtt
[LM Fonts]  {\caps \sans \ttset ...} {\rm \bf \it \bi } +AMS (8z 8t U)
[TG Heros]  {\caps \cond } {\rm \bf \it \bi } +TX (8z 8t)
\endtt
%
Nejprve je uveden <NázevRodiny>, pak následuje seznam modifikátorů
základních selektorů následovaný základními selektory pro výběr variant v
dané rodině, pak za znakem plus je uvedena implicitní matematická sada fontů
užitá k vybrané rodině a konečně v kulaté závorce je seznam kódování, které
rodina fontů podporuje. Více informací o makru "\fontfam" lze najít v
souboru "fontfam.tex".

Fonty přepínáme pomocí základních selektorů ("\rm", "\bf", "\it", "\bi").
Bezprostředně před základním selektorem mohou předcházet {\em modifikátory}
které dále pozměňují variantu písma. Například "\caps\it" zapne malé kapitálky
v kurzívě nebo "\cond\caps\bf" zapne kondenzovanou modifikaci s malými
kapitálkami tučné varianty.
Modifikátory následované příkazem "\fam" (např. "\caps\fam") modifikují
všechny později použité základní selektory (uvnitř stejné skupiny).
Modifikátory následované sekvencí "\one" (např. "\caps\one") modifikují jen
aktuálně užívanou variantu písma. Modifikátory pracují typicky nezávisle na
sobě a každá rodina fontů může disponovat jinou sadou modifikátorů. Více se
o nich píše v~souboru "cs-heros.tex" nebo v článku "kpfonts-plain.pdf".

Použijete-li "\fontfam[Catalog]", vytiskne se katalog dostupných fontových
rodin.


\sec Velikosti fontů a řádkování
%%%%%%%%%%%%%%%%%%%%%%%%%%%%%%%%
Všechna makra popsaná v této i předchozí sekci nastavují změny ve fontech a dalších
parametrech jen lokálně, takže jsou-li ve skupině, 
za ní se nastavení vrací k původním hodnotám.

Makro "\typosize[<velikost fontu>/<řádkování>]" nastaví velikost textových i
matematických fontů a řádkování. Je-li některý z parametrů prázdný, makro
nastaví jen údaje plynoucí z neprázdného parametru. 
Parametry neobsahují jednotku, jednotka pt se doplní v~makru. Příklady

\begtt
\typosize[10/12]    % to je implicitní nastavení
\typosize[11/12.5]  % font velikosti 11pt, řádkování 12.5pt
\typosize[8/]       % font velikosti 8pt, řádkování nezměněno.
\endtt

Začátek dokumentu tedy může vypadat takto:

\begtt
\input opmac \typosize[11.5/13]  % sazba v písmu 11.5pt s řádkováním 13pt
\endtt

Makro "\typoscale[<faktor-font>/<faktor-řádkování>]" zvětší nebo zmenší velikost
textových i matematických fontů resp. řádkování <faktor>krát aktuální 
velikost fontů resp. řádkování. 
Faktor je celé číslo, přitom 1000 znamená
faktor jedna ku jedné (jako za slovem "scaled" v~příkazu "\font"). Je-li parametr
prázdný, je to stejné, jako by byl roven 1000.

\begtt
\typoscale[800/800]    % fonty i řádkování se zmenší na 80 %
\typoscale[\magstep2/] % \magstep2 je 1440, tj. fonty se zvětší 1,44krát
\endtt

Někdy je žádoucí (např. při přechodu na poznámky pod čarou) zmenšit font vzhledem
ke stále stejné velikosti písma. Stačí psát
"\typobase\typoscale[<font>/<řádkování>]". Pak se provede zvětšení/zmenšení
vzhledem k {\em základnímu písmu}, což je písmo nastavené po prvním použití 
"\typosize" nebo "\typoscale".

Pokud zavedete font příkazem "\font\prepinac=<metrika> <nic-nebo-at-nebo-scaled>",
pak "\prepinac" přepíná do pevně stanoveného fontu, který není ovlivněn
makry na nastavování velikosti. Ale co není, může být. Stačí font
registrovat pomocí "\regfont\prepinac" a nyní i "\prepinac" přepíná do fontu
podle velikosti nastavené pomocí "\typosize" nebo "\typoscale". Příklad:
\par\nobreak

\begtt
\font\zapfchan=pzcmi8z  \regfont\zapfchan
\typosize[20/] Taky \zapfchan přepne do Zapf-Chancery ve velikosti 20pt.
\endtt

OPmac registruje pět fontových přepínačů
ukrytých v makrech "\rm", "\it", "\bf", "\bi", "\tt". Takže tato makra
implicitně nastavují font do stanovené velikosti.

Na místo "\typosize" a "\typoscale" je možno použít 
makra na změnu velikosti jen aktuálního fontu
"\thefontsize[<velikost-fontu>]" a "\thefontscale[<faktor>]".
Tato makra nemění matematické fonty ani řádkování. 

Všechna zde uvedená makra na změnu velikosti fontů jsou vybavena
inteligencí: hledají metriku, která má svou designovanou velikost nejblíže
požadované velikosti. Takže při požadavku na velikost 13pt se použije metrika
"csr12 at13pt", zatímco při velikosti 7.5pt se použije metrika 
"csr8 at7.5pt". Data pro tuto inteligenci jsou přečtena ze souboru
"ams-math.tex", kde je najdete u příkazů "\regtfm".

Poslední poznámka k fontům se týká makra "\em", které 
přepíná variantu písma specifickým způsobem. Je to kontextové makro, které pracuje v
závislosti na aktuálně zvoleném fontu. Implicitně přepíná na "\it". Pokud
ale je aktuálním fontem "\it", přepne na "\rm". Je-li aktuálním fontem "\bf"
přepne na "\bi" a obráceně. Makro navíc správně doplní italické korekce ke
slovu před jeho použitím a za jeho použitím. Takže se o italické korekce
není nutno starat. Příklad:

\begtt
To je {\em zdůrazněný} text.     % jako: To je {\it zdůrazněný\/} text.
\it To je {\em zdůrazněný} text. % jako: To je\/ {\rm zdůrazněný} text.
\bf To je {\em zdůrazněný} text. % jako: To je {\bi zdůrazněný\/} text.
\bi To je {\em zdůrazněný} text. % jako: To je\/ {\bf zdůrazněný} text.
\endtt




\label[cleneni]
\sec Členění dokumentu
%%%%%%%%%%%%%%%%%%%%%%
Dokument se může skládat z kapitol, kapitola ze sekcí a sekce z podsekcí.
Titul dokumentu vyznačte pomocí "\tit <titul><konec-řádku>", kapitolu
zahajte "\chap <titul><konec-řádku>" a podobně novou sekci zahajte
"\sec <titul><konec-řádku>" a podsekci 
"\secc <titul><konec-řádku>".
Takže třeba:

\begtt
\chap Brouci

\sec  Chrousti

\secc O nesmrtelnosti chroustů

Bla bla bla bla ...
Bla bla bla a ještě bla.
\endtt

Kapitoly se automaticky číslují jedním číslem, sekce dvěma čísly (číslo
kapitoly.sekce) a podsekce třemi čísly. Pokud dokument neobsahuje kapitoly,
číslo kapitoly chybí, tj. sekce má jedno číslo a podsekce dvě.

Implicitní vzhled nadpisů kapitol, sekcí a podsekcí je definován v
makrech "\printchap", "\printsec" a "\printsecc". Můžete se na obsah těchto
maker podívat do technické dokumentace nebo do "opmac.tex". Můžete se těmi
makry inspirovat a třeba je předefinovat podle vlastního typografického návrhu. 

První odstavec za titulem kapitoly, sekce a podsekce není odsazen. Pokud jej
chcete mít odsazen jako ostatní odstavce, napište "\let\firstnoindent=\relax".

Jestliže je název kapitoly, sekce nebo podsekce příliš dlouhý, rozlomí se do
řádků. V takovém případě je někdy lepší rozdělit název do řádků manuálně. 
K tomu slouží makro "\nl", které odřádkuje v místě použití (newline).
Toto makro se navíc v obsahu chová jako mezera.

Kapitola, sekce, podsekce se nečísluje, předchází-li "\nonum". Kapitola,
sekce, podsekce se neobjeví v obsahu, předcházlí-li "\notoc".

\label[cislovani]
\sec Další číslované objekty a odkazy na ně
%%%%%%%%%%%%%%%%%%%%%%%%%%%%%%%%%%%%%%%%%%%

Kromě kapitol, sekcí a podsekcí se automaticky číslují ještě rovnice a
popisky pod obrázky a pod tabulkami. 

Pokud je na konci display módu uvedeno
"\eqmark", tato rovnice bude číslovaná. Formát číslování je implicitně jediné
číslo uzavřené v kulaté závorce resetované při každém zahájení nové sekce.
Příklad: "$$ a^2 + b^2 = c^2 \eqmark $$" vytiskne
$$ 
  a^2 + b^2 = c^2 \eqmark 
$$

Je-li potřeba očíslovat jedním číslem více rovnic sestavených pomocí
"\eqalignno", pak použijte "\eqmark" v posledním sloupci, například takto:

\begtt
$$
  \eqalignno{a^2+b^2 &= c^2 \cr 
                   c &= \sqrt{a^2+b^2} & \eqmark \cr}
$$
\endtt
Ukázka dává tento výsledek:

\label[vypocet]
$$
  \eqalignno{a^2+b^2 &= c^2 \cr
                   c &= \sqrt{a^2+b^2} & \eqmark \cr}
$$

Dalšími číslovanými objekty jsou popisky.
Popisek pod obrázky je potřeba uvést slovem "\caption/f" a popisek pod
nebo nad tabulkami slovem "\caption/t". Pak následuje text popisku ukončený
prázdným řádkem. Příklad:

\begtt
\hfil\table{rl}{věk   & hodnota \crl\noalign{\smallskip}
                0--1  & neměřitelná \cr 
                1--6  & projevující se \cr
                6--12 & výrazná \cr
               12--20 & extrémní \cr
               20--60 & mírnější \cr
               60--$\infty$ & umírněná} % vytvoření tabulky
\par\nobreak\medskip
\caption/t Závislost závislosti na počítačích na věku.
\endtt
%
Tato ukázka vytvoří:

\label[zkouším]
\medskip
\hfil\table{rl}{věk   & hodnota \crl\noalign{\smallskip}
           0--1  & neměřitelná \cr 
           1--6  & projevující se \cr
           6--12  & výrazná \cr
           12--20 & extrémní \cr
           20--60 & mírnější \cr
           60--$\infty$ & umírněná}
\par\nobreak\medskip
\caption/t Závislost závislosti na počítačích na věku.
\medskip

Vidíme, že makro "\caption/t" doplnilo slovo \uv{Tabulka} následované číslem.
Toto číslo přebírá číslo sekce a za tečku doplňuje ještě číslo
tabulky. Podobně se chová "\caption/f", jen místo slova \uv{Tabulka} se v textu
zjeví slovo \uv{Obrázek}. Obrázky a tabulky jsou číslovány nezávisle.
Popisek je centrován. Je-li popisek delší na více řádcích, je centrován
poslední řádek.

Způsob číslování lze změnit jinou definicí makra "\thednum" (pro
rovnici), "\thetnum" (pro tabulky) a "\thefnum" (pro obrázky). Makro OPmac
je definuje implicitně takto:

\begtt
\def\thednum{(\the\dnum)} 
\def\thetnum{\thesecnum.\the\tnum}
\def\thefnum{\thesecnum.\the\fnum}
\endtt

Makro OPmac vloží slovo \uv{Tabulka} v závislosti na nastaveném jazyce
příkazem "\chyph", "\shyph", "\ehyph". Při "\shyph" dostaneme \uv{Tabuľka} a
při "\ehyph" máme \uv{Table}. Podobně se chovají slova 
\uv{Obrázek/Obrázok/Figure} a \uv{Kapitola/Kapitola/Chapter}. 
Jiná automaticky generovaná slova OPmac nepoužívá.

Předefinovat tato slova lze pomocí "\sdef", jak ukazuje následující příklad,
který zamění celá slova za zkratky.

\begtt
\sdef{mt:t:cs}{Tab.}  \sdef{mt:t:sk}{Tab.}  \sdef{mt:t:en}{Tab.}
\sdef{mt:f:cs}{Obr.}  \sdef{mt:f:sk}{Obr.}  \sdef{mt:f:en}{Fig.}
\endtt

\LaTeX{}oví uživatelé jsou zvyklí, že jim tabulky a obrázky plavou v
dokumentu, přičemž inklinují k horní části stránky. To se při použití OPmac
implicitně neděje, ale je možno plavání zařídit
pomocí plain\TeX{}ového makra "\topinsert" resp. "\midinsert". Například:

\begtt
\topinsert
   \hfil\table{rl}{...} % vytvoření tabulky
   \medskip
   \caption/t Závislost závislosti na počítačích na věku.
\endinsert
\endtt

Na automaticky číslované objekty je nutno se občas v textu odkazovat.
Protože dopředu nevíme, pod jakým číslem se rovnice, sekce, tabulka atd.
vytiskne, je potřeba použít interní lejblíky k označení odkazovaných
objektů. K tomu slouží makro "\label[<lejblík>]", které musí předcházet
makru, jež generuje číslo. Není nutné, aby "\label" předcházel těsně danému
makru. Tedy například:

\begtt
\label[chroust] \sec O nesmrtelnosti chroustů

\label[zavislaci]
\hfil\table{rl}{...} % vytvoření tabulky
\caption/t Závislost závislosti na počítačích na věku.

\label[pythagoras]
$$ a^2 + b^2 = c^2 \eqmark $$

Nyní můžeme hovořit o~sekci~\ref[chroust] na straně~\pgref[chroust] nebo 
taky o~rovnici~\ref[pythagoras] na straně~\pgref[pythagoras]. Dále bude 
potřeba upozornit na tabulku~\ref[zavislaci] na straně~\pgref[zavislaci], 
která shrnuje jistý druh závislosti.
\endtt

Text z ukázky vytvoří zhruba toto: \uv{Nyní můžeme hovořit o sekci 2.1 
na straně 13 nebo taky o rovnici (1) na straně 15. Dále bude 
potřeba upozornit na tabulku 5.3.1 na straně 42, 
která shrnuje jistý druh závislosti.} 

Jestliže se v textu vyskytují dopředné reference (tj. odkazujeme na objekt,
který ještě není vytištěn) nebo text odkazuje na stránky ("\pgref"), je nutné
\TeX{}ovat dokument aspoň dvakrát. 

Pomocí "\label[<lejblík>]\wlabel{<text>}" se dá vytvořit kdekoli 
obecný cíl <text>, na který je možné odkazovat makry 
"\ref[<lejblík>]" nebo "\pgref[<lejblík>]".

\sec Odrážky
%%%%%%%%%%%%

Jednotlivé myšlenky je občas potřeba vypíchnout odrážkami. Prostředí s
odrážkami se vymezuje sekvencemi "\begitems" a "\enditems". Uvnitř tohoto
prostředí je hvězdička aktivním znakem, který zahajuje odrážky. Prostředí s
odrážkami je možné vnořit do sebe. Pomocí "\style <znak>" hned za slovem
"\begitems" je možné vymezit některé z předdefinovaných vzhledů odrážek:

\begtt
\style o % malý puntík
\style O % velký puntík $\bullet$ (implicitní volba)
\style - % spojovník
\style n % odrážky číslované 1., 2., 3., ...
\style N % odrážky číslované 1), 2), 3), ...
\style i % odrážky číslované (i), (ii), (iii), (iv), ...
\style I % odrážky číslované I, II, III, IV, ...
\style a % odrážky s písmeny a), b), c), ...
\style A % odrážky s písmeny A), B), C), ...
\style x % malý čtvereček
\style X % velký čtvereček
\endtt

Příklad:

\begtt
\begitems \style n
* Tady je první myšlenka.
* A tady druhá, která je rozdělena na
  \begitems \style a
  * podmyšlenku
  * a hned následuje další podmyšlenka,
  * poslední podmyšlenka.
  \enditems
* Tady je třetí myšlenka.
\enditems
\endtt
%
vytvoří následující výstup:

\begitems \style n
* Tady je první myšlenka.
* A tady druhá, která je rozdělena na
  \begitems \style a
  * podmyšlenku
  * a hned následuje další podmyšlenka,
  * poslední podmyšlenka.
  \enditems
* Tady je třetí myšlenka.
\enditems

Chcete-li uvnitř prostředí s odrážkami vytisknout hvězdičku, pište~"\char`\*".

Pomocí "\sdef{item:<písmeno>}{<text>}" si můžete dodefinovat vzhled odrážek
podle svých představ. Implicitní odrážku můžete předefinovat pomocí
"\def\normalitem{<text>}".

Jednotlivá prostředí s odrážkami se odsazují podle
velikosti registru "\iindent", který je nastaven na hodnotu "\parindent" v
době čtení souboru "opmac.tex". Pokud později změníte "\parindent",
doporučuji na stejnou hodnotu nastavit "\iindent".
Vertikální mezera nad a pod prostředím s odrážkami je řízena makrem
"\iiskip".

\sec Tvorba automaticky generovaného obsahu
%%%%%%%%%%%%%%%%%%%%%%%%%%%%%%%%%%%%%%%%%%%

Makro "\maketoc" vytiskne v místě svého použití obsah dokumentu bez nadpisu,
jen jednotlivé řádky obsahu. Odsazení jednotlivých řádků je nastaveno na
násobky registru "\iindent". Často je potřeba dokument \TeX{}ovat vícekrát,
než se obsah zjeví a než se čísla stran srovnají správně, protože po
prvním zjevení obsahu se mohou stránky posunout jinam.

Titulek k obsahu by neměl být číslovaný a neměl by se zjevit v obsahu, takže
jej zapíšeme třeba pomocí

\begtt
\nonum\notoc\sec Obsah
\endtt

Titulky kapitol, sekcí a podsekcí zapisuje OPmac pro účely sestavení obsahu do
externího souboru "*.ref". Může se stát, že uživatel v těchto textech
použije nějaké komplikované makro, které se pak v souboru \uv{rozsype} do
takového stavu, že nejde vzápětí přečíst. V takovém případě je potřeba
makro zabezpečit proti expanzi při zápisu do souboru pomocí
deklarace "\addprotect\makro". Takto deklarované makro je pak zabezpečené
proti expanzi do "*.ref" souboru. Například OPmac deklaruje:

\begtt
\addprotect~ \addprotect\TeX \addprotect\thefontsize \addprotect\em
\endtt 
%
a mnoho dalších. Není možné ale předvídat všechno, co může uživatel
nacpat do titulku sekce nebo kapitoly. Pokud se tedy \uv{rozsype}
REF soubor, je potřeba si tímto způsobem zabezpečit používané makro.

Poznámka: Při přechodu na novější verzi OPmac se může stát, že REF soubor
vygenerovaný starou verzí způsobí chyby při dalším zpracování. Pak je
potřebné nejprve REF soubor smazat.


\sec Sestavení rejstříku
%%%%%%%%%%%%%%%%%%%%%%%%

Makro pro zanášení slov do rejstříku je navrženo s ohledem na optimalizaci
počtu úhozů na klávesnici. Autor už napsal své dílo, má daný termín
odevzdání a nyní ho čeká úmorná práce vyhledávání slov v textu, která by
měla přijít do rejstříku, a jejich vyznačování. Je třeba mu tuto práci co
nejvíce usnadnit.

Pro zanesení slova do rejstříku slouží makro "\ii". Je to zkratka za
\uv{insert to index}.
Jeho parametr je <slovo> bez mezery ukončené mezerou (obecnější tvar parametru
uvedeme později). Toto slovo se přepíše do rejstříku, ve kterém jsou všechna
takto deklarovaná slova seřazena podle abecedy a jsou k nim připojena čísla
stránek, na kterých bylo použito odpovídající makro "\ii <slovo>". 
Příklad:

\begtt
Tady mluvím o jistém
\ii kroutilík
kroutilíku, který provokoval moji zvědavost.
\endtt

Makro "\ii" viditelně neudělá v sazbě nic. Přilepí se na následující slovo (v
našem příkladě slovo \uv{kroutilíku}) jako skrytá značka. Číslo strany,
kde se ta značka objeví, bude v rejstříku vedle slova \uv{kroutilík}.

Je-li "\ii" zapsáno ve vertikálním módu, zahájí se v daném místě odstavec,
aby se mohla neviditelná značka z "\ii" nalepit na následující slovo. Pokud
si to z nějakých důvodů nepřejete, použijte interní variantu makra
"\iindex{<slovo>}", která nezahajuje odstavec.

Pokud se v rejstříku má objevit stejné slovo jako v textu, není nutno je psát
dvakrát. Stačí použít makro "\iid" (zkratka za "\ii" double):

\begtt
Hlavní zásady jsou \iid nestrannost , \iid pravdomluvnost a \iid odvaha .
\endtt
%
To povede ke stejnému výsledku jako

\begtt
Hlavní zásady jsou \ii nestrannost nestrannost,
\ii pravdomluvnost pravdomluvnost a \ii odvaha odvaha.
\endtt
%
Povšimněte si, že čárky a tečky jsou odstrčeny od dublovaného slova, protože
mezera je ukončovací znak parametru "\iid". Do textu se mezera vrátí právě
tehdy, když nenásleduje tečka nebo čárka. V našem příkladě před spojkou \uv{a}
mezera ve výsledku je, ale před tečkou nebo čárkou mezera není.

Vlastnosti makra "\iid" jsou tímto popsány zcela. Vraťme se k makru "\ii",
které poskytuje další možnosti.

Parametr "\ii" je vždy ukončen mezerou. Může obsahovat čárky (bez mezer),
které naznačují, že se do rejstříku dává více slov:

\begtt
{\bf Definice.}
\ii lineární~prostor,vektorový~prostor
{\em Lineárním prostorem} (nebo též vektorovým prostorem) rozumíme ...
\endtt
%
Dostaneme totéž jako při "\ii lineární~prostor \ii vektorový~prostor "
a tato ukázka demonstruje ještě jednu věc: je-li potřeba do parametru "\ii"
dostat mezeru, pište vlnku nebo napište heslo uzavřené ve svorkách.

Pokud se v rejstříku objeví hesla skládající se z více slov, obvykle chceme,
aby u~hesla, které opakuje první slovo, se toto slovo 
v rejstříku nevypisovalo opakovaně, ale
aby bylo nahrazeno pomlčkou. Například:
\par\nobreak
\medskip
\hbox{\indent lineární podprostor \ 12, 16, 18, 29}\nobreak
\hbox{\indent \space --- prostor \ 12, 16--32, 51}\nobreak
\hbox{\indent \space --- závislost \ 18--20, 34}
\medskip

\noindent Při takovém požadavku pište místo vlnky mezi slovy lomítko. Příklad:

\begtt
\ii lineární/prostor,vektorový/prostor
\endtt

Někdy je vhodné kromě hesla lineární/prostor zařadit i heslo
prostor/lineární. Aby se to nemuselo psát dvakrát, je k dispozici zkratka
"@" napsaná za čárku na konci parametru:

\begtt
\ii lineární/prostor,vektorový/prostor,@
% je totéž jako \ii lineární/prostor,vektorový/prostor
%               \ii prostor/lineární,prostor/vektorový
\endtt

Počet lomítek v hesle pro rejstřík není omezen. Můžete tedy vytvořit
víceúrovňový rejstřík. Nicméně je třeba vědět, že zkratka "@" nevytváří
všechny permutace, ale jen přesune první údaj před lomítkem za všechny
ostatní. Takže "\ii a/b/c,@" je totéž jako "\ii a/b/c" "\ii b/c/a".

Samotný rejstřík vznikne v místě příkazu "\makeindex". Rejstřík obsahuje
data z předchozího zpracování dokumentu \TeX{}em, takže je potřeba
\TeX{}ovat aspoň dvakrát. Makro "\makeindex" abecedně seřadí data v rejstříku
podle českých a slovenských pravidel řazení a upraví odkazy na stránky
(aby se stránky neopakovaly a inklinovaly k zápisu ve tvaru 26--28). Makro
"\makeindex" se nestará o prostředí, do kterého sazbu vyvrhne, ani o nadpis.
To musíme udělat sami. OPmac nabízí pro sazbu do více sloupců makra
"\begmulti <počet-sloupců> ... \endmulti". Příklad:

\begtt
\sec Rejstřík
\begmulti 3 \makeindex \endmulti
\endtt

Do rejstříku musejí být zařazena jen \uv{čistá} slova, která neobsahují makra
expandující na primitivní příkazy \TeX{}u. Pokud chcete vytisknout v rejstříku
něco komplikovanějšího, můžete sestavit slovník výjimek pomocí 
maker "\iis <heslo><mezera>{<tisk>}" 
(název makra můžeme číst jako "\ii" speciální). 
Funkci si vysvětlíme na příkladu:

\begtt
\iis chikvadrat {$\chi$-kvadrát}
\iis relax {{\tt \char`\\relax}}
\iis Goedelova/věta/o~neúplnosti {G\"odelova/věta/o~neúplnosti}
\iis věta/o~neúplnosti/Goedelova {věta/o~neúplnosti/G\"odelova}
\endtt
%
Lze pak psát "\ii relax", "\ii chikvadrat" nebo 
"\ii Goedelova/věta/o~neúplnosti,@". 
OPmac abecedně řadí podle těchto hesel, ale když dojde
na potřebu heslo vytisknout do rejstříku, vytiskne místo těchto hesel
materiál, 
který je uveden na pravé straně slovníku. 
Příklad ukazuje, že tím lze
řešit nejen tisk hesel, která je potřeba ošetřit
speciálními makry (v příkladu slovo relax), ale také
výjimky abecedního řazení.
%
Slovník výjimek je možný zapsat kamkoli před "\makeindex", 
typicky se píše na začátek dokumentu. 

Výjimku z řazení dvojhlásky ch (například ve slově mochnátý, tj. mnohonohý)
je možné zařídit pomocí tečky, která má stejně jako ostatní interpunční
znaky, nulovou řadicí platnost (OPmac hesla řadí, jakoby tam interpunkce
nebyla). Takže třeba takto:
\begtt
... \ii moc.hnátý ...
\iis moc.hnátý {mochnátý} 
\endtt
%
Je-li při zpracování "\makeindex" zapnutý anglický jazyk (implicitní
nastavení nebo po přepínači
"\ehyph"), pak se ch neinterpretuje jako dvojhláska. Ostatní pravidla řazení
zůstávají nezměněna.

Pro různé speciální znaky můžete využít znak "@", který se řadí před celou
abecedou. Speciální znak pak nahradíte až ve slovníku výjimek.
Takže třeba "\ii Ernst~@~Young" pro řazení a
"\iis Ernst~@~Young {Ernst \& Young}" pro tisk.


\sec Barvy, vodoznaky
%%%%%%%%%%%%%%%%%%%%%

Makra uvedená v této sekci nastavují barvy jen při přímém
vytváření do PDF. Takže při výstupu do DVI tato makra neudělají nic.
%
Barvu textu můžete nastavit pomocí přepínačů 
{\localcolor\Blue "\Blue"}, 
{\localcolor\Red "\Red"}, 
{\localcolor\Brown "\Brown"},
{\localcolor\Green "\Green"}, 
{\localcolor\Yellow "\Yellow"}, 
{\localcolor\Cyan "\Cyan"}, 
{\localcolor\Magenta "\Magenta"}, 
{"\White"}, 
{\localcolor\Grey "\Grey"}, 
{\localcolor\LightGrey "\LightGrey"} a 
{"\Black"}. 

Implicitně tyto přepínače pracují globálně nezávisle na \TeX{}ové skupině.
Barvu jinou než černou je pak potřeba ukončit explicitním přepínačem
"\Black". Toto chování je možné změnit uvedením příznaku "\localcolor".
Tento příznak je možné nastavit globálně (například na začátku dokumentu)
nebo lokálně uvnitř skupiny. Při globálním nastavení se sazba vrací 
k~původní barvě za všemi \TeX{}ovými skupinami 
a při lokálním nastavení se k původní barvě vrací sazba
za skupinou začínající příznakem "\localcolor" (a za všemi vnořenými skupinami).
Příklad:

\def\podbarvi#1#2#3{\setbox0=\hbox{#3}\leavevmode
  {\localcolor\rlap{#1\strut\vrule width\wd0}#2\box0}}

\begtt
Černý {\localcolor \Blue modrý {\Green zelený \Red červený} modrý} černý. 
\endtt
%
Další příklad vytvoří \podbarvi\Yellow\Brown{podbarvený text:}

\begtt
\def\podbarvi#1#2#3{\setbox0=\hbox{#3}\leavevmode
   {\localcolor\rlap{#1\strut\vrule width\wd0}#2\box0}}
   \podbarvi\Yellow\Brown{Tady je hnědý text na žlutém pozadí.}
\endtt

V původní verzi OPmac bylo možné rozlišovat barvy tenkých linek a písma.
Tuto vlastnost jsem opustil od verze Dec.~2014. Zdůvodnění najdete v
technické dokumentaci.

Kromě uvedených barevných přepínačů si můžete \uv{namíchat} v režimu CMYK 
i barvy vlastní. Stačí se inspirovat, jak jsou uvedené přepínače 
definovány:\par\nobreak

\begtt
\def\Red{\setcmykcolor{0 1 1 0}}
\def\Brown{\setcmykcolor{0 0.67 0.67 0.5}} ...
\endtt

Aktuální barvu ve formě čtyř čísel CMYK je možné přečíst z makra
"\currentcolor" například pomocí
"\let\savedcolor=\currentcolor" a později je možné se k této barvě
vrátit pomocí "\setcmykcolor\savedcolor".


\medskip
{\bf Vodoznakem} je míněn šedý text opakující se na každé stránce,
který je vytištěn pod obvyklým textem. Například OPmac nabízí makro
"\draft", které způsobí, že každá stránka obsahuje šikmo napsaný veliký šedý
nápis DRAFT. Můžete se inspirovat v technické dokumentaci, jak je to
uděláno.


\label[klikani]

\sec Klikací odkazy
%%%%%%%%%%%%%%%%%%%

Pokud napíšete na začátek dokumentu "\hyperlinks{<color-in>}{<color-out>}", pak
se v dokumentu při výstupu do PDF stanou klikacími:

\begitems
* čísla generovaná pomocí "\ref" a "\pgref",
* čísla kapitol, sekcí, podsekcí a stránek v obsahu,
* čísla nebo značky generované pomocí "\cite" (odkazy na literaturu),
* texty tištěné pomocí makra "\url" nebo "\ulink".
\enditems

Poslední z uvedených odkazů je externí a bude mít barvu <color-out>, zatímco
ostatní čísla jsou interními odkazy a budou mít barvu <color-in>. Příklad:

\begtt
\hyperlinks{\Blue}{\Green}   % vnitřní odkazy modré, URL zelené
\endtt

Je možné zobrazit rámečky ohraničující aktivní plochu pro klikání. Tyto
rámečky jsou viditelné jen v PDF prohlížeči, při tisku na tiskárně se
nezobrazují. Stačí těmto rámečkům \uv{namíchat} barvu (tentokrát RGB) a
definovat některé ze sekvencí "\pgborder", "\tocborder", "\citeborder",
"\refborder" a "\urlborder". První část jména kontrolní sekvence 
určuje, jakých odkazů se to týká. Příklad:

\begtt
\def\tocborder{1 0 0}  % odkazy v obsahu vlevo budou mít červený rámeček
\def\pgborder{0 1 0}   % odkazy na stránky budou mít zelený rámeček
\def\citeborder{0 0 1} % odkazy na publikace budou mít modrý rámeček
\endtt
%
Implicitně tato makra nejsou definována, což znamená, že se rámečky netvoří.

Manuálně je možno vytvořit cíl odkazu makrem
"\dest[<typ>:<lejblík>]"
a klikací text mak\-rem "\link[<typ>:<lejblík>]{<color>}{<text>}".
Parametr <typ> je typ odkazu (toc, pg, cite, ref nebo další). 

Makro "\url" vytiskne odkaz do internetu. Příklad: "\url{http://petr.olsak.net}"
vytvoří \url{http://petr.olsak.net}. Text je psán strojopisem a může se
lámat do řádků za lomítky. Je-li nastaveno "\hyperlinks", stává se tento
text aktivním vnějším odkazem. 
Vyskytují-li se v~argumentu "\url" znaky
"%", "\", "#", "$", "{" a "}", je třeba použít 
"\%", "\\", "\#", "\$", "\{" a "\}".
Ostatní speciální znaky "~", "_", "^", "&" lze napsat do parametru "\url"
přímo. 
Dále je možno do parametru "\url" napsat "\|" k označení místa, kde
je dovoleno zlomit řádek.
%
Libovolný text odkazovaný na webovou stránku lze vložit pomocí
"\ulink[<URL>]{<text>}". Například odkaz na
\ulink[http://petr.olsak.net/opmac.html]{stránky OPmac} jsou vytvořeny
pomocí "\ulink[http://petr.olsak.net/opmac.html]{stránky OPmac}".

Obsah dokumentu se dá přesunout do levé záložky PDF prohlížeče tak, že klikáním na
něj se přechází v~dokumentu na požadované místo. Ve specifikaci PDF se tomu
říká \uv{outlines}. Makro, které uvedenou věc zařídí, se jmenuje
"\outlines{<úroveň>}". Záložky budou implicitně rozevřeny do <úrovně>
včetně. Takže třeba při <úroveň>=0 jsou
vidět jen úrovně kapitol. Bohužel písmo v~záložkách typicky
nezvládá správně česká a slovenská písmena. 
%(kódování PDFDocEncoding neobsahuje české znaky). 
Proto OPmac konvertuje texty do
záložek tak, že tam jsou bez hacku a carek.
%\fnote
%{Chcete-li háčky a čárky zachovat, použijte doplňkové makro {\tt pdfuni.tex}.} 
Chcete-li vypnout tuto konverzi, napište "\def\toasciidata{}".
Chcete-li akcenty zachovat, použijte doplňkové makro "pdfuni.tex".

Samotný řádek do záložek vložíte makrem
"\insertoutline{<text>}". Text v~tomto případě nepodléhá konverzi.
V sazbě se neobjeví nic, jen se stane cílem, když uživatel na záložku
s~<textem> klikne. Obsah se do záložek vloží celý během činnosti makra
"\outlines", takže další řádky vložené pomocí "\insertoutline" tomuto
obsahu předcházejí nebo následují podle toho, zda předcházejí nebo následují
místu, kde je použito "\outlines".


\sec Verbatim texty
%%%%%%%%%%%%%%%%%%%

Vytisknout část textu verbatim \uv{tak jak je} bez interpretace 
speciálních znaků lze
v prostředí vymezeném makry "\begtt" a "\endtt". Příklad:

{\def\tthook{\adef/{\string\endtt}}
\begtt
\begtt
Tady je vše
   napsáno bez   interpretace speciálních znaků, jakými
 jsou mezera, %, $, \, ~, ^, _, {, }, #, &.
/
\endtt
}
Ve výstupu se objeví:

\begtt
Tady je vše
   napsáno bez   interpretace speciálních znaků, jakými
 jsou mezera, %, $, \, ~, ^, _, {, }, #, &.
\endtt
%
Není-li za "\endtt" prázdný řádek, nemá následující odstavec výchozí
odsazení. 

Je-li před zahájením "\begtt" nastaven registr "\ttline" na nezápornou
hodnotu, bude makro číslovat řádky. První řádek má číslo "\ttline+1"
a po práci makra se registr "\ttline" posune na číslo posledního
vytištěného řádku. Takže v~dalším prostředí "\begtt" ... "\endtt" 
číslování pokračuje tam, kde přestalo. Implicitně je "\ttline=-1",
takže číslování neprobíhá.

Levé odsazení každého řádku v "\begtt...\endtt" je nastaveno na "\ttindent".
Tento registr má výchozí hodnotu rovnu "\parindent" (v době čtení souboru
"opmac.tex"). Vertikální mezera nad a pod verbatim výpisem je vložena
makrem "\ttskip". 

Makro "\begtt" zahájí skupinu a v ní nastaví všem speciálním znakům plain\TeX{}u 
kategorii 12. Pak spustí makro "\tthook", které
je implicitně prázdné. V něm je možno nastavit další kategorie znaků podle
potřeby. Definici aktivních znaků je potřeba udělat
pomocí "\adef<znak>{<text>}". Normální "\def" nefunguje, důvod je
vysvětlen v~TBN, str.~26. Příklad:

{\def\tthook{\adef/{\string\endtt}}
\begtt
\def\tthook{\adef!{?}}
\begtt
Nyní se každý vykřičník promění v otazník. Že nevěříte? Vyzkoušejte!
/
\endtt
}
Definovaný "\tthook" funguje ve všech verbatim výpisech, dokud jej
nepředefinujete jinak. Tipy:
\par\nobreak
\begtt
\def\tthook{\typosize[9/11]}    % jiná velikost verbatim výpisů
\def\tthook{\ttline=0}          % všechny výpisy číslovány od jedničky
\def\tthook{\adef{ }{\char`\ }} % místo mezer budou vaničky
\endtt

Verbatim lze tisknout i v řádku uvnitř odstavce. Pomocí "\activettchar<znak>" si
uživatel zvolí znak, který bude aktivní a bude zahajovat i končit verbatim
výpisy uvnitř odstavce. Verbatim výpis se v odstavci nikdy nerozlomí (je v
boxu). Autor makra OPmac obvykle nastavuje 
"\activettchar"{\tt\char`\"}, takže pak může psát třeba toto:

\begtt
Je-li před zahájením "\begtt" nastaven registr "\ttline" na nezápornou...
\endtt
%
Znak nastavený pomocí "\activettchar" má lokální platnost a ruší se také
pozdějším nastavením "\activettchar" na jinou hodnotu. Při zahájení každého
řádkového verbatim výpisu se spustí makro "\intthook", které je implicitně
prázdné. {\bf Upozornění:} deklaraci "\activettchar<znak>" proveďte až po přečtení všech
makrosouborů. Důvod: "\activettchar" nastavuje <znak> jako aktivní, což může při
čtení souborů maker vadit. Není taky rozumné aktivovat znak pro derivaci nebo jiný
v~textu používaný znak.

Verbatim výpisy je možné tisknout z externího souboru. Například

\begtt
\verbinput (12-42) program.c
\endtt
%
vytiskne ve stejné úpravě, jako při použití "\begtt", ... "\endtt", 
řádky 12 až 42 ze souboru "program.c". 
Parametry v kulaté závorce mohou vypadat také takto:

\begtt
\verbinput (-60) program.c   % výpis od začátku souboru do řádku 60
\verbinput (61-) program.c   % výpis od řádku 61 do konce souboru
\verbinput (-) program.c     % výpis celého souboru
\verbinput (70+10) program.c % výpis od řádku 70, tiskne 10 řádků
\endtt

V dalších ukázkách OPmac čte od řádku, který následuje za naposledy
přečteným řádkem souboru z předchozího volání "\verbinput". Je-li soubor
čten poprvé, začíná číst prvním řádkem. Tento prvně čtený 
řádek je označen v komentářích jako "n".

\begtt
\verbinput (+10) program.c  % výpis deseti řádků od řádku n
\verbinput (+) program.c    % výpis od řádku n do konce souboru
\vebrinput (-5+7) program.c % vynechá 5 řádků, od n+5 tiskne dalších 7
\verbinput (-3+) program.c  % vynechá 3 řádky, tiskne do konce souboru
\endtt
%
Narazí-li čtení na konec souboru dřív, než je vytištěno vše, co si žádá
uživatel, přepis souboru je ukončen a žádná chyba se nezjeví. 

Výpisy provedené makrem "\verbinput" jsou ovlivněny registrem
"\ttindent" a makrem "\tthook" stejně, jako prostředí "\begtt"\dots"\endtt".
Při "\ttline"{\tt\char`\<-1} se netisknou čísla řádků. Je-li "\ttline=-1", čísluje se podle
řádků souboru. Je-li "\ttline" nezáporné, čísluje se od "\ttline+1".


\sec Jednoduché tabulky
%%%%%%%%%%%%%%%%%%%%%%%

\LaTeX{}oví uživatelé jsou zvyklí při vymezení pravidel zarovnávání v
tabulce používat deklarace typu "{cclr}". Každé písmeno vymezí jeden sloupec
v tabulce, přitom písmeno "c" znamená centrovaný sloupec, "l" je sloupec
zarovnaný doleva a "r" sloupec zarovnaný doprava.
Podobnou možnost deklarace jednoduchých tabulek nabízí
OPmac v makru "\table{<deklarace>}{<data>}". Příklad:

\begtt
\table{||lc|r||}{\crl 
   měsíc     & zboží     & cena\hfil  \crli \tskip.2em
   leden     & noťas     & 14 kKč     \cr
   únor      & skejt     &  2 kKč     \cr
   červenec  & jachtička & 3,4 MKč    \crl}
\endtt

Uvedený příklad povede na následující výsledek:

\medskip
\hfil\table{||lc|r||}{\crl
    měsíc     & zboží  & cena\hfil  \crli \tskip.2em
    leden     & noťas  & 14 kKč     \cr
    únor      & skejt  &  2 kKč     \cr
   červenec  & jachtička & 3,4 MKč  \crl}
\medskip

Ve skutečnosti výsledek nebude uprostřed řádku, ale tam, kam "\table"
napíšete. 

Kromě písmen "c", "l", "r" se v <deklaraci> mohou objevit znaky
\uv{svislítko}, které vymezují svislou čáru mezi sloupci. 
Dále kromě deklarátorů jednotlivých sloupců "c", "l", "r" je v OPmac od verze Jun
2017 připraven deklarátor "p{<rozměr>}", který vymezí sloupec
tabulky s pevnou šířkou sloupce <rozměr>. Delší text se formátuje jako odstavec
dané šířky bez odstavcové zarážky, kratší jednořádkový text je vlevo. Například
"p{42mm}"
vytvoří sloupec pro text široký 42\,mm. Při velmi úzkých sloupcích nastávají
obvykle potíže s formátováním odstavce do bloku. V takovém případě můžete
hned za <rozměr> přidat třeba "\raggedright" a odstavec bude formátován 
s~nezarovnaným pravým okrajem, například "p{42mm\raggedright}".

Pokud v <deklaraci> použijete "(<text>)", je tento text aplikován v každém
řádku tabulky v~místě odpovídajícím umístění v <deklaraci>. Například "r(\kern10pt)l"
vloží mezi sloupce "r" a "l" dodatečnou desetibodovou mezeru.

V <deklaraci> je možné místo opakování stejného deklarátoru použít číslo
následované deklarátorem, tedy třeba "4c" je totéž jako "cccc". Opakovat se
mohou celé úseky <deklarace>, pokud za číslem pokračuje úsek ve svorkách.
Například "c 3{|c}" je totéž jako "c|c|c|c".  Mezery v~<deklaraci> se ignorují a
je možné je použít pro zvýšení přehlednosti.

V datové části tabulky musí být přesně tolik sloupců, kolik jich bylo deklarováno.
Jsou odděleny znakem "&" nebo symbolem pro konec řádku "\cr". Z toho vyplývá,
že na každém řádku musí být v datové části o jeden znak "&" méně, než je počet sloupců.
Nedodržíte-li toto pravidlo, \TeX{} se pomstí chybovou hláškou
\begtt
! Extra alignment tab has been changed to \cr
\endtt  
%
nebo vytvoří nedomrlou tabulku. 

Místo symbolu pro konec řádku "\cr" je možno
použít "\crl" (přidá jednoduchou vodorovnou čáru) nebo "\crll" (přidá dvojitou
čáru), "\crli" (přidá čáru přerušenou svislými dvojitými linkami, 
tj.~{\bf i}nterrupted) a 
"\crlli" (přidá dvojitou čáru přerušenou svislými dvojitými linkami).
Konečně lze použít "\crlp{<seznam>}", což vloží přerušovanou čáru jako
"\crli", ale jen ve sloupcích, jejichž pořadová čísla oddělená čárkami 
jsou vyjmenována v <seznamu>. Takže třeba užití "\crlp{1,2}" v ukázce tabulky výše
(místo "\crli") vytvoří čáru jen pod slovy \uv{měsíc} a \uv{zboží}. V
<seznamu> lze používat zkratky typu "<od>-<do>", napříkad "1-3,5" je totéž jako
"1,2,3,5".

Těsně za "\cr", "\crl" atd. může následovat "\tskip<dimen>",
což vytvoří vertikální mezeru velikosti <dimen>, přitom se nepřeruší svislé
čáry v tabulce.

Za povšimnutí stojí, že v ukázce u slova \uv{cena} je připojeno
"\hfil", což vloží pružnou mezeru vpravo od položky. Protože sloupec "r"
obsahuje implicitní stejnou pružnou mezeru vlevo, je slovo \uv{cena} centrováno, 
zatímco ostatní údaje ve sloupci jsou zarovnány
napravo. 

Makro "\table" pracuje s předdefinovanými hodnotami, které můžete změnit,
pokud chcete dosáhnout jiný vzhled tabulky:

\begtt
\def\tabiteml{\enspace} % co vkládá vlevo každé datové položky
\def\tabitemr{\enspace} % co vkládá vpravo každé datové položky
\def\tabstrut{\strut}   % podpěra vymezující výšku řádků
\def\vvkern{1pt}   % velikost mezery mezi dvojitou svislou linkou
\def\hhkern{1pt}   % velikost mezery mezi dvojitou vodorovnou linkou
\endtt

Vyzkoušejte si tabulku po "\def\tabiteml{}\def\tabitemr{}". Sloupce budete
mít na sebe nalepeny bez mezer. Příklad definice "\tabstrut": 

\begtt
\def\tabstrut{\vrule height11pt depth3pt width0pt}
\endtt
%
Tento příklad vymezuje v tabulce vzdálenost mezi účařím 14pt, z toho
11pt je rezervováno pro přetahy nad účařím a 3pt pro přetahy pod
účařím. Vyskytne-li se větší písmeno, zvětší to v~daném místě řádkování.

OPmac definuje "\strut" závislý na zvoleném řádkování (při použití příkazu
"\typosize") zhruba takto:

\begtt
\def\strut{\vrule height.709<baselineskip> depth.291<baselineskip> width0pt}
\endtt

Tip: vyzkoušejte si
"\def\tabiteml{$\enspace} \def\tabitemr{\enspace$}". Ty dolary způsobí, že
každá datová položka bude zpracována v matematickém módu. Makro "\table" se
nyní podobá \LaTeX{}ovému prostředí "array".

Chcete-li přesáhnout jedním údajem více sloupců tabulky, můžete použít
plain\TeX{}ové makro "\multispan{<číslo>}" nebo makro z OPmac
"\mspan<číslo>[<deklarace>]{<text>}", které přesáhne <číslo> sloupců a <text>
v tomto prostoru formátuje podle <deklarace>, která obsahuje deklaraci právě
jednoho sloupce podle stejných pravidel jako je <deklarace> "\table". 
Máte-li v~tabulce svislé čáry
a chcete, aby svislé čáry z "\mspan" na ně navazovaly, pak použijte v
"\mspan" čáry před údajem "c", "l" nebo "r" jedině v případě, že "\mspan" zasahuje do
prvního sloupce. V~ostatních případech používejte čáry výhradně na konci
<deklarace>, protože každý sloupec (s výjimkou prvního) přidává případné
svislé čáry jen na svůj konec.

Makro "\frame{<text>}" vytvoří rámeček kolem <textu> s vnitřními okraji
o~velikostech "\vvkern" a "\hhkern". Například "\frame{ahoj}" vytvoří \frame{ahoj}.
Povšimněte si, že účaří rámovaného textu zůstalo nezměněno. Pokud chcete mít
tabulku s dvojitými čarami, je výhodné ji vytvořit po stranách 
a nahoře a dole s~jednoduchými čarami a celou ji zabalit do "\frame":

\vbox{
\begtt
\frame{\table{|c||l||r||c|}{\crl
  \mspan4[|c|]{\bf Nadpis} \crl 
  \noalign{\kern\hhkern}\crli
  první & druhý & třetí  & čtvrtý \crlli
  sedmý & osmý  & devátý & desátý \crli}}
\endtt 
\par\removelastskip\vskip-8.3ex
\hfill\frame{\table{|c||l||r||c|}{\crl
  \mspan4[|c|]{\bf Nadpis} \crl 
  \noalign{\kern\hhkern}\crli
  první & druhý & třetí  & čtvrtý \crlli
  sedmý & osmý  & devátý & desátý \crli}}
}
\medskip

Kromě předdefinovaných znaků "c,l,r,p" lze v "<deklaraci>" "\table" použít
libovolný další symbol pro deklaraci sloupce, stačí připravit 
"\def\tabdeclare<symbol>{<vlevo>##<vpravo>}". Toto je podrobněji popsáno
v technické dokumentaci.

Tloušťka všech čar je v \TeX{}u implicitně 0,4pt. OPmac umožňuje tuto
implicitní tloušťku nastavit jinak pomocí "\rulewidth=<šířka>", například
"\rulewidth=1.5pt".

Další příklad použití makra "\table" najdete v sekci~\ref[cislovani].
Pokud potřebujete vytvořit komplikovanější tabulky, můžete se inspirovat
sekcí \ulink[http://petr.olsak.net/opmac-tricks.html]{OPmac triků}
věnující se tabulkám. Můžete si také definovat případně další vlastní tabulkové
deklarátory nebo použít přímo primitivní "\halign", což vyžaduje zřejmě  
prostudovat TBN, kapitolu čtvrtou.


\sec Vkládání obrázků
%%%%%%%%%%%%%%%%%%%%%

Makro "\inspic <jméno>.<přípona><mezera>" vloží obrázek. Je-li předem nastaven
registr "\picw", má obrázek požadovanou šířku.
Implicitní hodnota registru je 0\,pt, což znamená, že bude obrázek vložen 
ve své přirozené velikosti. Analogicky lze nastavit výšku obrázku registrem
"\picheight".
Přípony souboru s obrázkem mohou být "png", "jpg", "jbig2", "pdf".

Obrázek je vyhledán v adresáři "\picdir". Toto makro je implicitně prázdné,
tj. obrázek je vyhledán v aktuálním adresáři.

O umístění obrázku v sazbě se musíte postarat vlastními prostředky.
Například:

\begtt
\picw=.5\hsize \centerline{\inspic hodiny.jpg }\nobreak\medskip
\caption/f Hodiny na brněnském náměstí Svobody.
\endtt

Makro není vhodné použít při opakovaném použití stejného obrázku v dokumentu
(opakující se grafika na každé straně nebo obrázek jako 
odrážka ve výčtu položek). V takovém případě je vhodnější natáhnout obrázek
do PDF dokumentu jen jednou pdf\TeX{}ovým příkazem "\pdfximage" a dále
opakovat jeho zobrazení na různých místech dokumentu pomocí "\pdfrefximage".
Dokumentace k pdf\TeX{}u řekne víc.

Makro "\inspic" pracuje jen při výstupu do PDF. Pokud máte
nastaven výstup do DVI,
můžete použít makro "epsf.tex". Vzhledem k omezeným možnostem (obrázek jen
ve formátu EPS) není tento způsob práce s obrázky v makru OPmac podporován.

Chcete-li \uv{programovat} obrázky přímo ve zdrojovém textu \TeX{}u, lze
použít plain\TeX{}ové rozhraní vynikajícího makra Ti{\em k}Z, se kterým je OPmac
kompatibilní.

\sec PDF transformace
%%%%%%%%%%%%%%%%%%%%%

Veškerá sazba v pdf\TeX{}u může podléhat lineární transformaci, která je
daná transformační maticí
"\pdfsetmatrix{<a> <b> <c> <d>}". Tato matice se v lineární algebře
zapisuje do dvou řádků:
$$
  \pmatrix{a&c\cr b&d}, \quad \hbox{např. zvětšení: }
  \pmatrix{s_1&0\cr 0&s_2}, \quad \hbox{nebo rotace: }
  \pmatrix{\cos\varphi&\ -\sin\varphi \cr \sin\varphi&\ \hphantom-\cos\varphi}.
$$
Pdf\TeX{}ové primitivy "\pdfsetmatrix", "\pdfsave" a "\pdfrestore" bohužel
nejsou v dokumentaci pdf\TeX{}u uvedeny, tak je musím dokumentovat aspoň
zde. Příkaz "\pdfsave" uloží stávající transformační matici a aktuální bod
sazby. V době konání příkazu "\pdfrestore" se matice vrátí do původní podoby
a aktuální bod sazby v té době musí být na stejném místě, jako byl v~době
"\pdfsave", jinak se nám sazba rozjede a pdf\TeX{} nadává. Toho se dá
docílit např. pomocí "\pdfsave...\rlap{<text>}\pdfrestore".
Transformační matice se nastavují pomocí "\pdfsetmatrix". Opakované použití
"\pdfsetmatrix" způsobí pronásobení transformační matice novou maticí,
takže to funguje jako skládání zobrazení. OPmac nabízí dvě užitečná makra
"\pdfscale{<vodorovně>}{<svisle>}" a "\pdfrotate{<úhel>}". Parametr "<úhel>"
je interpretován ve stupních. Tato makra provedou odpovídající
"\pdfsetmatrix".

Aplikujeme-li více matic za sebou, je potřeba vědět, že výchozí text
prochází transformací jednotlivých matic \uv{odzadu dopředu}, takže
například:

\begtt
První: \pdfsave \pdfrotate{30}\pdfscale{-2}{2}\rlap{text1}\pdfrestore
      % text1 je zvětšen dvakrát a překlopen podél svislé osy,
      % dále je otočen o 30 stupňů doleva a konečně je vytištěn.
druhý: \pdfsave \pdfscale{-2}{2}\pdfrotate{30}\rlap{text2}\pdfrestore
      % text2 je otočen o 30 stupňů doleva, dále zvětšen a překlopen
      % podél svislé osy, nakonec vytištěn.
třetí: \pdfsave \pdfrotate{-15.3}\pdfsetmatrix{2 0 1.5 2}\rlap{text3}%
       \pdfrestore % nejprve zkosení, pak otočení o 15.3 stupňů doprava
\endtt

\bigskip
Ukázka dává následující výsledek. První: 
  \pdfsave \pdfrotate{30}\pdfscale{-2}{2}\rlap{text1}\pdfrestore
druhý:
  \pdfsave \pdfscale{-2}{2}\pdfrotate{30}\rlap{text2}\pdfrestore
třetí: 
  \pdfsave \pdfrotate{-15.3}\pdfsetmatrix{2 0 1.5 2}\rlap{text3}%
       \pdfrestore 
\bigskip\bigskip


\sec Poznámky pod čarou a na okraji
%%%%%%%%%%%%%%%%%%%%%%%%%%%%%%%%%%%

{\bf Poznámku pod čarou} 
vytvoříte pomocí "\fnote{<text>}". V místě tohoto zápisu
v textu se objeví automaticky generovaná značka a pod čarou dole na stránce
je tato značka zopakována a vedle ní je <text>. 

Značka je implicitně definovaná jako číslo v exponentu.
Číslování poznámek je na každé stránce započato
jedničkou\fnote
{Toto chování se dá změnit vložením příkazu {\tt\char`\\runningfnotes} na
začátek dokumentu. V takovém případě se poznámky číslují od jedné v celém
dokumentu. Další možnosti číslování jsou uvedeny v technické dokumentaci.}.
Čísla jsou vygenerována správně až po
opakovaném \TeX{}ování. Při prvním zpracování jsou místo čísel otazníky.

Implicitní značkování je možné změnit předefinováním makra "\thefnote".
Například po

\begtt
\def\thefnote{\ifcase\locfnum\or
   *\or**\or***\or$^{\dag}$\or$^{\ddag}$\or$^{\dag\dag}$\fi)}
\endtt
%
bude první poznámka mít hvězdičku se závorkou, druhá dvě hvězdičky se
závorkou atd. Uvedená definice předpokládá, že na jedné stránce nebudete mít
více než 6 poznámek.

Makro "\fnote" je možné zapsat jen v běžném textu odstavce, nikoli v boxu
(například v~tabulce). Chcete-li odkazovat třeba z tabulky, je nutné v
tabulce vytvořit jen značky a mimo tabulku (ovšem tak, aby to neuteklo na
jinou stránku) zapíšete texty poznámek. K vytvoření značky použijte
"\fnotemark<číslo>" a text (bez značky) vytvoří "\fnotetext{<text>}".
Příklad:

\begtt
{\typoscale[/1200]\table{||lc|r||}{\crl 
     měsíc     & zboží                & cena\hfil \crli \tskip.2em
     leden     & noťas\fnotemark1     & 14 kKč    \cr
     únor      & skejt\fnotemark2     &  2 kKč    \cr
     červenec  & jachtička\fnotemark3 & 3,4 MKč   \crl}}
\par\nobreak \fnotetext{notebook}\fnotetext{skateboard}\fnotetext{jachta}
\endtt
%
Čísla za slovy "\fnotemark" je třeba psát od jedné v každé tabulce/boxu.
Nemusejí souviset se skutečným číslem poznámky. Například, je-li na stejné stránce nad 
tabulkou z ukázky normální "\fnote", bude mít vytištěno číslo 1, 
odkazy v tabulce budou mít čísla~2, 3, 4 a případná další poznámka pod
tabulkou na stejné stránce obdrží číslo 5.

{\bf Poznámku na okraji stránky} vytvoříte pomocí řídicí sekvence
"\mnote{<text>}". Poznámka je vlevo (na pravou zarážku) na sudé stránce a je
vpravo (na levou zarážku) na liché stránce. Tuto vlastnost mají poznámky až
po opakovaném \TeX{}ování. Při prvním \TeX{}ování jsou všechny poznámky vpravo.
Chcete-li mít poznámky i při opakovaném \TeX{}ování jen vpravo nebo jen
vlevo, pište do úvodu dokumentu "\fixmnotes\right" nebo "\fixmnotes\left".

Řídicí sekvenci "\mnote{<text>}" můžete napsat do odstavce nebo před
odstavec. S odstavcem samotným to nic neudělá. Řádek odstavce, kde je
"\mnote" vložena jako neviditelná značka, je na stejné úrovni, jako první
řádek textu poznámky. 

Text poznámky je od sazby odsazen o "\mnoteindent" a maximální šířka
poznámky je "\mnotesize". Text poznámky se rozlomí do více řádků, aby
nepřesáhl "\mnotesize".

Není ošetřen případ, kdy je "\mnote" víceřádková a je umístěna na úroveň
například posledního řádku strany. Pak text poznámky přečuhuje poněkud 
dolů ze strany. Nebo se poznámky mohou překrývat. 
Je tedy nutné "\mnote" použít jen na velmi krátké poznámky a
případně si tento jev pohlídat a ošetřit při definitivní sazbě manuálně.
Pro manuální ošetření vertikální polohy poznámek slouží "\mnoteskip", což je
registr, který udává, o kolik se má následující poznámka (a jen ta) posunout
nahoru. Při záporné hodnotě se posune dolů. Například 
"\mnoteskip=2\baselineskip \mnote{<text>}" posune poznámku o~dva řádky výše.


\sec Bibliografické údaje
%%%%%%%%%%%%%%%%%%%%%%%%%

{\bf Odkazy.}
Pomocí "\cite[<lejblík>]" nebo "\cite[<lejblík1>,<lejblík2>,<lejblík3>]"
atd. vytvoříme v~textu odkazy na položky v seznamu literatury. V
seznamu literatury je třeba uvést záznamy, které mají odkazované lejblíky.
Tyto záznamy dostanou v seznamu automaticky vygenerovaná čísla a sekvence
"\cite" se pak promění na číselné odkazy, například [27] nebo [18,~24,~42]
atd. Souvislé řady čísel [1,~2,~3,~5,~6] se promění v~intervaly
[1--3,~5--6] jen tehdy, když je v úvodní deklaraci dokumentu napsáno
"\shortcitations". A seřadí se podle velikosti při "\sortcitations".

Při "\nonumcitations" se odkazy nepřevádějí na čísla. K tomu je potřeba
použít navazující Bib\TeX{}ový styl (např. "alpha", "apalike") nebo
rozšířenou formu příkazu "\bib", viz níže. Odkazy vypadají při stylu 
"alpha" takto [Nov08] a při stylu "apalike" 
takto [Novák, 2008].

Příkaz "\rcite[<lejblíky>]" funguje jako "\cite[<lejblíky>]", ale kolem odkazů 
nejsou přidány závorky. Možnost využití: "[\rcite[novak08],~s.~213]" vytvoří
například odkaz [17,~s.~213]. Závorky kolem musíte napsat sami. 

Příkaz "\ecite[<lejblík>]{<text>}" vytiskne pouhý "<text>", který se chová
jako odkaz na literaturu.
Příklad použití: "z~výsledů Nováka [\ecite[novak08]{2008},~s.~213] plyne..."
z~výsledků Nováka [2008, s. 213] plyne\dots\ Přitom "novak08" je registrován do
seznamu citovaných položek a třeba při "\hyperlinks" bude číslo 2008 prolinkováno 
s odpovídající položkou v~seznamu literatury. 

Příklady redefinice "\cite" pro alternativní formátování odkazů:

\begtt
\def\cite[#1]{(\rcite[#1])}     % \cite[lejblík] vytvoří (27)
\def\cite[#1]{$^{\rcite[#1]}$}  % \cite[lejblík] vytvoří^{27}
\endtt

\noindent {\bf Seznam literatury} je možné vložit do dokumentu čtyřmi různými
způsoby:

\begitems
* Manuálně: pomocí jednotlivých položek "\bib[<lejblík>]" přímo v dokumentu.
* S využitím Bib\TeX{}u makrem "\usebibtex{<bib-báze>}{<bst-styl>}".
* Využitím jednou vygenerované databáze makrem "\usebbl/<typ> <bbl-báze>".
* Přímým čtením ".bib" databáze makry \TeX{}u bez využití Bib\TeX{}u.
\enditems

Jednotlivé způsoby jsou níže probrány podrobněji.

\medskip\noindent
{\bf Manuálně vložený seznam literatury} v dokumentu vypadá například takto:
\par\nobreak

\begtt
\bib [tbn] Petr Olšák. {\it\TeX{}book naruby.} 468~s. Brno: Konvoj, 2001.
\bib [tst] Petr Olšák. {\it Typografický systém \TeX.} 
                       300~s. Brno: Konvoj, 2000.
\endtt
%
Výše uvedená ukázka dá následující výstup:
\medskip
\bib[tbn] Petr Olšák. {\it\TeX{}book naruby.} 468~s. Brno: Konvoj, 2001.
\bib[tst] Petr Olšák. {\it Typografický systém \TeX.} 
                    300~s. Brno: Konvoj, 2000.
\medskip
Je možné použít rozšířený způsob zápisu 
"\bib [<lejblík>] = {<značka>} <text záznamu>". Údaj "<značka>" se 
použije do odkazů při zapnutém "\nonumcitations". Kolem rovnítka musejí být mezery.
Například:

\begtt
\bib [tbn] = {Olšák, 2001} 
     OLŠÁK, P. {\it\TeX{}book naruby.} 468~s. Brno: Konvoj, 2001.
\endtt

\medskip\noindent
{\bf Využití Bib\TeX{}u.} Předpokládá se, že uživatel disponuje souborem
"<bib-báze>.bib", ve kterém jsou nashromážděny bibliografické údaje ve
formátu, v jakém je čte program Bib\TeX{}. V~\TeX{}ové distribuci jistě
najdete nějaký "*.bib" soubor, tak se do něj podívejte. Lejblíkem je první
údaj u~každého bibliografického záznamu. Soubor "<bib-báze>.bib" by měl
obsahovat bibliografické údaje, které jsou nadmnožinou toho, co potřebuje
uživatel vypsat ve svém dokumentu.
Na místo, kde budete chtít vypsat seznam
literatury, vložte následující pokyn:

\begtt
\usebibtex{<bib-báze>}{<bst-styl>}
\endtt

Parametr "<bib-báze>" je jméno souboru bez přípony ".bib", ve kterém jsou
připraveny bibliografické záznamy. Parametr "<bst-styl>" je jméno
stylového souboru bez přípony ".bst", který použije Bib\TeX{} pro konverzi ze zdroje
"<bib-báze>.bib" do výstupu "<dokument>.bbl". Tento výstup pak bude makrem
"\usebibtex" přečten a vložen do dokumentu. Typicky používané <bst-styly>
jsou "plain", "alpha", "apalike", "ieeetr", "unsrt". 
Existují desítky, možná stovky dalších ".bst" stylů, viz internet.

Při prvním zpracování dokumentu \TeX{}em makro "\usebibtex" připraví vstupní
pokyny pro Bib\TeX{} do souboru "<dokument>.aux" a zjistí, že soubor
"<dokument>.bbl" zatím neexistuje. To dá najevo na terminálu:

\begtt
WARNING: .bbl file doesn't exist. Use the ``bibtex <dokument>'' command.
\endtt

Přejděte tedy na příkazový řádek a napište "bibtex <dokument>". Tím se
spustí program Bib\TeX{}, který přečte ze souboru "<dokument>.aux" vstupní
pokyny (kterou otevřít ".bib" databázi, který ".bst" styl a jaké lejblíky
jsou požadovány) a na základě toho vygeneruje soubor "<dokument>.bbl", který
obsahuje výběr jen těch záznamů, které byly uživatelem citovány pomocí
"\cite". Soubor "<dokument>.bbl" je navíc zkonvertovaný z 
".bib" formátu do formátu čitelného \TeX{}em.
Tato konverze je řízena stylem ".bst".

Když znovu \TeX{}ujete dokument, makro "\usebibtex" v tomto případě shledá,
že soubor "<dokument>.bbl" existuje, načte jej a vytvoří seznam literatury. 
Seznam obsahuje jen citované položky. Druhé spuštění \TeX{}u obvykle
nestačí, protože příkazy "\cite" jsou typicky dopřednými referencemi, takže
zatím nemají ponětí o přiřazení čísel k "<lejblíkům>" v seznamu literatury. 
To se dozvědí až v~místě použití "\usebibtex", což je typicky na konci 
dokumentu. Takže teprve třetí \TeX{}ování dá vše do pořádku.

Seznam literatury obsahuje po použití Bib\TeX{}u jen citovaná dílka. Pokud
chcete do seznamu zařadit další položky, které nejsou v textu explicitně
odkazovány příkazem "\cite", použijte "\nocite[<lejblík>]". Toto makro dá
Bib\TeX{}u pokyn, aby do seznamu zahrnul i položku s~<lejblíkem>, ale v
místě použití tohoto makra se nevytiskne nic. Konečně pomocí "\nocite[*]" dáváme
Bib\TeX{}u vzkaz, že chceme mít v seznamu literatury celou ".bib" databázi.

Zdroj bibliografických záznamů může být ve více ".bib" souborech. V
takovém případě stačí jejich názvy oddělit čárkou:
"\usebibtex{<bib-báze1>,<bib-báze2>}{<bst-styl>}".

Někdy se stane, že autoři ".bib" databází nebo ".bst" stylů neopustili při
tvorbě těchto souborů \LaTeX{}ový způsob myšlení a občas jim uklouzne nějaká
\LaTeX{}ová konstrukce z prstů až do počítače. Odtud se dostane do
čteného ".bbl" souboru a náš plain\TeX{} si s~tím nebude vědět rady. K tomu
slouží seznam "\bibtexhook", kde můžete uvést definice těchto \LaTeX{}ových
konstrukcí. Tyto definice budou mít lokální platnost jen při čtení ".bbl"
souboru. Například

\begtt
\def\bibtexhook{\def\emph##1{{\em##1}}\def\frac##1##2{{##1\over##2}}}
\endtt

\noindent 
{\bf Využití jednou vygenerované databáze.} Tvorba seznamů
literatury Bib\TeX{}em má jistou nevýhodu. Pokud později do dokumentu
vložíte další "\cite[<lejblík>]", musíte veškerou anabázi s bib\TeX{}em
provést znovu. A protože v současné době probíhá inflace odborných publikací
způsobená tím, že se podle kvanta publikací a citací daňový poplatník
rozhodl odměňovat vědce, je každé zjednodušení práce s bibliografickými
záznamy přínosné.
Makro OPmac navrhuje řešení, při kterém stačí použít Bib\TeX{} pro mnoho nových
článků jen jednou.

\begitems \style n
* Vytvořte si zvláštní dokument "<mojebáze>.tex", do kterého napíšete:
\begtt
\input opmac  \genbbl{<bib-báze>}{<bst-styl>}  \end
\endtt
* Po \TeX{}ování dokumentu "<mojebáze>.tex" spusťte "bibtex <mojebáze>".
  Tím se vytvoří soubor "<mojebáze>.bbl".
* Zpracujte \TeX{}em soubor "<mojebáze>.tex" ještě jednou. Vytvoří se
  seznam veškeré literatury, který byl v souboru "<bib-báze>.bib", přitom každá
  položka je označena svým <lejblíkem>. Vytiskněte si tento výstup a 
  dejte si jej na nástěnku.
* Uložte soubor "<mojebáze>.bbl" někam, kde jej umí přečíst \TeX{} bez ohledu
  na to, v kterém pracujete adresáři. 
* Přejděte k editaci svého dokumentu, pište "\cite" nebo "\nocite" podle potřeby a
  v místě seznamu literatury dejte sekvenci "\usebbl/<typ> <mojebáze> ".
  Údaj <typ> má tyto možnosti: 
\begtt
\usebbl/a <mojebáze> % vypsat kompletně celou <mojebaze> (a=all),
\usebbl/b <mojebáze> % jen \(no)cite údaje řadit dle <mojebaze> (b=base), 
\usebbl/c <mojebáze> % jen \(no)cite řadit podle pořadí citace (c=cite).
\endtt
\removelastskip
\enditems

Kroky 2 až 4 budete muset opakovat pouze tehdy, když budete chtít přidat do
"<mojebáze>.bbl" další údaj, tj. po upgradu souboru "<bib-báze>.bib". 
Prudí-li různí odběratelé vaší vědecké činnosti požadavky na různé 
"<bst-styly>", stačí si vygenerovat
podle různých stylů různé soubory typu "mybbl-plain.bbl",
"mybbl-ieeetr.bbl".

\medskip
\noindent
{\bf Přímé čtení ".bib" databáze} je možné pomocí makra "\usebib".
Je-li toto makro použito poprvé, natáhne se dodatečný modul "opmac-bib.tex", 
který navíc potřebuje externí balíček na čtení ".bib" souborů 
"librarian.tex" od Paula Isamberta. Užití je podobné jako při "\usebbl":

\begtt
\usebib/c (<style>) <bib-báze> % řadit podle pořadí citace (c=cite),
\usebib/s (<style>) <bib-báze> % řadit podle klíče ve stylu (s=style).
\endtt

Zde "<bib-báze>" je jeden nebo více ".bib" souborů oddělených čárkou bez
mezery a bez přípony. Prametr "<style>" udává část jména souboru
"opmac-bib-<style>.tex", ve kterém je specifikace formátování položek.
Součástí balíčku je styl "simple", tedy soubor "opmac-bib-simple.tex",
který si může uživatel zkopírovat na jiný název a modifikovat dle svých
představ. Více informací je na konci souboru "opmac-bib.tex".

\medskip
\noindent 
{\bf Formátování seznamu literatury} je řízeno makrem "\printbib",
které je vloženo na začátek každé položky v seznamu. 
Implicitně makro tiskne čísla položek do
hranatých závorek a při použití "\nonumcitations" předsadí první řádek
položky a nepřidává nic. Makro může využít "\the\bibnum" pro tisk 
čísla nebo "\the\bibmark" pro tisk značky (při "\nonumcitations").
Příklady:

\begtt
% Číslování položek bez hranatých závorek:
\def\printbib{\hangindent=\parindent \indent \llap{\the\bibnum. }}

% Tisk zkratek při použití bibTeXového stylu alpha a \nonumcitations:
\def\printbib{\hangindent=\parindent \noindent [\the\bibmark]\quad}
\endtt

Další příklady (třeba jak \TeX{} změří šířku největšího čísla a podle toho
vypočítá odsazení celého seznamu) jsou uvedeny na
\url{http://petr.olsak.net/opmac-tricks.html}.

\sec Matematická sazba
%%%%%%%%%%%%%%%%%%%%%%

Následující text popisuje vlastnosti souboru maker "ams-math.tex" (resp.
"tx-math.tex" při použití některých PostScriptových fontů). Toto makro je
makrem OPmac načítáno, takže uvedené vlastnosti jsou k dispozici též
uživatelům OPmac.

V matematické sazbě (mezi dolary) nefungují přepínače textových fontů
deklarovaných příkazem "\font". Místo toho se tam přepíná mezi tzv. {\it
matematickými abecedami}. Jakmile začnete psát písmena
mezi dolary bez použití přepínače, je použita implicitní matematická abeceda
"\mit". Celkově jsou k dispozici následující abecedy:

{\def\tthook{\catcode`\$=3 \catcode`/=0 \medmuskip=0mu \adef{ }{ }}%
\begtt
\mit     % matematická kurzíva      $abc-xyz,ABC-XYZ$
\it      % textová kurzíva          $/it abc-xyz,ABC-XYZ$
\rm      % textová antikva          $/rm abc-xyz,ABC-XYZ$
\cal     % jednoduché cal. znaky    $/cal ABC-XYZ$
\script  % kroucenější cal. znaky   $/script ABC-XYZ$
\frak    % fraktura                 $/frak abc-xyz,ABC-XYZ$
\bbchar  % zdvojené tahy            $/bbchar ABC-XYZ$
\bf      % sans serif bold          $/bf abc-xyz,ABC-XYZ$
\bi      % sans serif bold slanted  $/bi abc-xyz,ABC-XYZ$
\endtt
}

Dále jsou v matematické sazbě k dispozici stovky symbolů dostupných pomocí
"\<něco>", například "\alpha" $\alpha$, "\geq" $\geq$, "\sum" $\sum$, 
"\sphericalangle" $\sphericalangle$, "\bumpeq", $\bumpeq$. Seznam všech těchto symbolů
najdete v dokumentaci k AMS\TeX{}u, která se typicky jmenuje "amsguide.ps".

Matematická sazba funguje v nastavené velikosti písma podle "\typosize" nebo
"\typoscale". V~nadpisech se doporučuje použít boldifikovanou sadu všech
fontů, která se zapíná pomocí makra "\boldmath". Například:

\begtt
\def\nadpis#1\par{\centerline{\typosize[17/]\bf\boldmath #1}}
\nadpis Tady je nadpis včetně vzorce $\int_a^b f(x) {\rm d}x$
\endtt

Makro "ams-math.tex" nastavuje jako implicitní abecedu matematickou kurzívu,
která je mírně jinak kreslena než kurzíva textová. Naopak makro
"tx-math.tex" nastavuje jako implicitní abecedu textovou kurzívu převzatou z
\uv{okolního} textového fontu. Ta se ve vzorečcích spolu s~textovým fontem
bude esteticky snášet daleko lépe. Chcete-li toto implicitní chování změnit,
je možné použít následující globální přepínače:

\begtt
\itvariables    % implicitní abeceda bude textová kurzíva,
\mitvariables   % implicitní abeceda bude matematická kurzíva.
\endtt
%
{\bf Poznámka:} Vlastnosti popsané v této kapitole (AMS symboly, "\bbchar",
"\script", "\frak", sans serifový font po přepínačích "\bf", "\bi" v math
módu) vyžadují zavést speciální fonty (AMS fonty, EC fonty). Chcete-li
se vyhnout této závislosti na fontech, můžete využít
\ulink[http://petr.olsak.net/opmac-tricks.html\#onlycm]{OPmac trik~0111}.


\sec Okraje
%%%%%%%%%%%

Plain\TeX{} nastavuje levý okraj 1~in a šířku sazby ("\hsize") nastavuje
tak, aby i pravý okraj při formátu papíru \uv{letter} byl 1\,in. Také horní a
dolní okraj (do kterého přesahuje záhlaví a čísla stran) jsou
nastaveny na 1~in při formátu papíru \uv{letter} a tím je určena výška sazby
("\vsize"). \csplain{} dělá totéž, tj. okraje jsou 1~in, ale formát papíru
je~A4.\fnote
  {Přesněji: \csplain{} nastavuje výšku sazby {\tt\char`\\vsize=239.2mm}, což vede 
   k~dolnímu okraji o 7\,mm většímu než 1\,in.} 
OPmac umožňuje toto nastavení změnit příkazem:\par\nobreak

\begtt 
\margins/<pg> <formát> (<levý>,<pravý>,<horní>,<dolní>)<jednotka>
například: 
\margins/1 b5 (2,2,2,2)cm % nastaví všechny okraje na 2 cm pro papír b5.
<pg>... 1 = shodné okraje pro všechny stránky,
<pg>... 2 = okraje pro liché stránky, sudé mají prohozeny <levý>/<pravý>,
<formát>... a3, a4, a5, b5, letter, a3l, a4l, a5l či uživatelem definovaný,
<levý>,<pravý>,<horní>,<dolní>... velikosti okrajů,
<jednotka>... mm, cm, in, pt, pc, bp, dd, cc.
\endtt

Každý z parametrů <levý>, <pravý>, <horní>, <dolní> může být prázdný.
Jsou-li prázdné oba <levý> i <pravý>, je zachováno nastavení "\hsize" a
levý i pravý okraj je stejný. Je-li jen jeden z parametrů
<levý>, <pravý> prázdný, zůstává zachováno "\hsize" a neurčený okraj se
dopočítá. Jsou-li <levý> i <pravý> neprázdné, jsou oba okraje určeny 
a je podle nich upraveno "\hsize". 
Analogické pravidlo platí pro <horní>, <dolní> v souvislosti 
s~výškou sazby "\vsize". Například

\begtt
\margins/2 a4 (,18,,)mm  % vnější okraj na dvojstraně 2*a4 je 18mm
                         % \hsize, \vsize beze změny.
\endtt

Údaj "<formát>" může být též ve tvaru "(<šířka>,<výška>)<jednotka>", kde
"<jednotka>" je nepovinná a pokud chybí, použije se jednotka za údaji s
okraji. Tedy třeba

\begtt
\margins/1 (100,200) (7,7,7,7)mm
\endtt
%
deklaruje papír o rozměru 100$\times$200\,mm a s okraji 7\,mm po každé
straně. Mezery před a za údajem "<formát>" nelze vynechat.

Uživatel může také před použitím "\margins" definovat vlastní <formát> papíru pomocí
příkazu "\sdef{pgs:<formát>}{(<šířka>,<výška>)<jednotka>}". 
%Opmac například implicitně definuje:
%
%\begtt
%\sdef{pgs:a4}{(210,297)mm}   \sdef{pgs:letter}{(8.5,11)in}
%\sdef{pgs:b5}{(176,250)mm}   \sdef{pgs:a4l}{(297,210)mm}
%\endtt

Celou sazbu na úkor okrajů je možno zvětšit/zmenšit příkazem
"\magscale[<factor>]". Například "\magscale[500]" zmenší sazbu na polovinu.
Při této změně zůstává na místě \uv{Knuthův bod}, tj. bod o souřadnicích
1\,in, 1\,in od horního a levého okraje. Sazba samotná je zalomena zcela
stejně. Jednotky
použité v dokumentu jsou od této chvíle relativní. Například po
"\magscale[2000]" je použitá jednotka v dokumentu "1mm" ve skutečnosti "2mm".
Makro "\magscale" ponechává nezměněny jen rozměry stránek dané formátem
stránek (a4, a3, atd.). Možnost použití
makra: "\magscale[1414] \margins/1 a4 (,,,)mm" umístí sazbu, která je určena
pro tisk na a5, doprostřed stránky a4 a odpovídajícím způsobem ji zvětší,
aby se to korektorům lépe četlo. 

\label[lastpage]
\sec Poslední strana
%%%%%%%%%%%%%%%%%%%%

Číslo poslední strany dokumentu (to nemusí být počet stran) je uloženo při
opakovaném zpracování \TeX{}em v~registru "\lastpage". K tomu musí být
otevřen soubor REF s daty pro křížové odkazy, rejstřík a obsah. Pokud
pracujete s těmito daty, je soubor REF automaticky otevřen. Pokud ne, 
můžete si vynutit jeho otevření příkazem "\openref".
Není-li soubor REF otevřen, je hodnota registru "\lastpage" rovna nule.
Číslování stránek ve tvaru <číslo>/<počet stran> zajistíte například takto:

\begtt
\footline={\hss \rm \thefontsize[10]\the\pageno/\the\lastpage \hss}
\endtt


\sec Další jazyk
%%%%%%%%%%%%%%%%

OPmac přímo podporuje češtinu, slovenštinu a angličtinu. Ostatní jazyky lze
použít také, ale je potřeba vynaložit nepatrné úsilí. Především je potřeba
použít \TeX{}ový stroj a formát takový, aby v něm byly podporovány vzory
dělení příslušného jazyka. \CS{}plain pro pdf\TeX{} podporuje v~implicitním
nastavení vzory dělení 16 jazyků (od března 2019) a \CS{}plain pro Lua\TeX{}
podporuje vzory dělení všech jazyků dostupných v \TeX{}ové distribuci (kolem
57 jazyků). Doporučuji tedy pro více jazyků použít Lua\TeX{}.

Následuje ukázka pro španělštinu. Ukázku můžete zpracovat \CS{}plainem v
Lua\TeX{}u, tedy příkazem
"luatex -fmt pdfcsplain dokument".

\mubyte \ntie ^^c3^^b1\endmubyte \def\ntie{\~n}
\mubyte \ccedilla ^^c3^^a7\endmubyte \def\ccedilla{\c c}
\mubyte \Ccedilla ^^c3^^87\endmubyte \def\Ccedilla{\c C}

\begtt
\input opmac
\input lmfonts  % Unicode fonty

\sdef{mt:chap:es}{Capítulo}   % "Kapitola" v jazyku es
\sdef{mt:t:es}{Cuadro}        % "Tabulka" v jazyku es
\sdef{mt:f:es}{Figura}        % "Obrázek" v jazyku es

\eslang % Španělské vzory dělení slov + deklarovaná slova v jazyku "es".

\sec Mañana

Mañana.

\caption/f Test % vytvoří text "Figura 1.1 Test"

\bye
\endtt

V ukázce je patrné, že nová tři slova ve španělštině je třeba přidat pod
zkratku jazyka "es" (podle ISO~639-1). Tato slova začnou být používána automaticky
po zapnutí vzorů dělení slov, tedy po použití přepínače "\eslang".

Pokud používáte OPmac s formátem generovaným z "etex.src" (tj. příkaz 
"xetex dokument" nebo "luatex dokument"), pak nefunguje přepínač "\eslang" a
místo něj zapnete vzory dělení pomocí "\uselanguage{espanol}".
Navíc v takovém případě musíte přiřadit dlouhému názvu jazyka jeho zkratku podle
ISO~639-1 příkazem "\isolangset{espanol}{es}".

Při abecedním řazení rejstříku pomocí "\makeindex" může OPmac ohlásit
(není-li použit \csplain) varování: ``falling back to ASCII sorting''.
Pokud chcete dodržet pravidla řazení daného jazyka, je proto nutné definovat makro 
"\sortingdata<iso-kód>". Dále můžete přidat makro
"\specsortingdata<iso-kód>". Příklad:

\begtt
\def\sortingdataes {aAäÄáÁ,bB,cCçÇ,^^P^^Q^^R,dD,...,zZ,.}
\def\specsortingdataes {ch:^^P Ch:^^Q CH:^^R}
\endtt

Prvé makro obsahuje skupiny znaků oddělené od sebe čárkami a ukončené
čárkou-tečkou. Znaky v každé skupině jsou nerozlišitelné při prvém průchodu
(primární řazení). Jsou-li z~tohoto pohledu některé údaje shodné, je na tyto
údaje použito sekundární řazení, ve kterém už jsou všechny znaky rozlišeny a
seřazeny v pořadí, jako v makru. Poznámka: tři tečky v příkladu si musí španělsky
mluvící uživatel doplnit za reálná data sám.

Řadicí algoritmus umožňuje skupiny znaků (spřežky Ch, Dz atd.) interpretovat
jako jediný znak. K tomu slouží makro "\specsortingdata<iso-kód>". V něm
jsou mezerami od sebe odděleny údaje ve tvaru "<skupina>:<token>". OPmac
provede výměnu "<skupina>" za "<token>" před zahájením řadicího algoritmu,
takže "<token>" je možné vložit do "\sortingdata<iso-kód>" jako jeden znak.
Je nutné používat znaky, které nejsou součástí abecedy jazyka, například
"^^A", "^^B" atd. Přitom je potřeba se vyhnout znakům "^^I" a "^^M" ,
protože tyto mají speciální kategorii.

OPmac definuje interní makro "\sortingdata", které
je možné použít pro češtinu, slovenštinu a angličtinu.
Makro "\sortingdata<iso-code>" má přednost před interním makrem
"\sortingdata", je-li definováno a jazyk odpovídající ISO kódu má
aktivovány vzory dělení slov.

Konečně můžete pozměnit seznam ignorovaných znaků (z hlediska řazení), který
je uveden v makru "\setignoredchars" takto:

{\catcode`\<=12
\begtt
\def\setignoredchars{\setlccodes ,.;.?.!.:.'.".|.(.).[.].<.>.=.+.{}{}}
\endtt
}%
Toto makro můžete předefinovat, ale ponechte v předu "\setlccodes", vzadu
"{}{}" a za každým znakem pište tečku, která je implicitním ignorovaným
znakem.

\maybebreak 4cm

\sec Předdefinované styly dokumentů
%%%%%%%%%%%%%%%%%%%%%%%%%%%%%%%%%%%

Implicitní styl dokumentu při použití OPmac zahrnuje výchozí parametry a
vzhled sazby
jako v plain\TeX u. Pochopitelně
uživatel si styl dokumentu obvykle upraví pomocí svých maker a třeba s
využitím maker OPmac.

Od března 2019 nabízí OPmac kromě toho zkratky s výchozím nastavením jiných
stylů, což je možné použít, pokud chce uživatel vytvořit
běžný typ dokumentu a nechce se zdržovat nastavováním parametrů a
definicí vlastních maker. Jsou připraveny deklarační makra "\report" a 
"\letter", která je možné použít v záhlaví dokumentu.

Deklarace "\report" je určena pro psaní zpráv. Písmo je nastaveno
na 11 bodů, odsazení odstavců je 1,2\,em. Makro "\tit" použije menší font
než obvykle, protože se nepředpokládá, že bude použita úroveň \uv{kapitola}.
První stránka je nečíslovaná, ostatní jsou číslovány (od strany~2). 
Poznámky pod čarou jsou číslovány vzestupně v celém dokumentu.
Při použití tohoto makra v záhlaví dokumentu je připraveno též makro
"\author<text><konec řádku>", které umístí <text> v kurzívě doprostřed
řádku. Při použití "\nl" je text rozdělen do více řádků, každý je centrován.

Deklarace "\letter" je určena pro psaní dopisů. Písmo je nastaveno na 11
bodů, odsazení není žádné a mezi odstavci je půlřádková mezera. Stránky
nejsou číslovány. Chcete-li je číslovat, můžete použít například kód uvedený
v sekci \ref[lastpage].
Při použití tohoto makra je dále připraveno makro
"\address" pro psaní víceřádkových adres a makro "\subject" které vypíše
tučně slovo \uv{Věc:} nebo \uv{Vec:} nebo \uv{Subject:} v závislosti na
použitém jazyce. Užití makra "\address" vypadá takto:

\begtt
\address
  <první řádek adresy>
  <druhý řádek adresy>
  <atd.>
  <prázdný řádek>
\endtt
%
tedy řádky se neukončují žádným speciálním znakem, ale konec řádku na 
vstupu je koncem řádku i v sazbě. 
Makro vytvoří "\vtop" se sazbou řádků tak, že jeho šířka
odpovídá nejširšímu řádku. Chcete-li mít víceřádkovou adresu u pravého
okraje stránky stačí tedy psát "\hfill\address..." 
A před adresu lze také do prvního
řádku předsadit text jednoduše pomocí "<text>\address..."

Analogický deklarátor "\book" není zahrnut, protože kniha vyžaduje individuální
péči typografa a každá kniha může být jiná. Na takový projekt je účelné si
připravit vlastní makra. Deklarátor "\slides" také není zahrnut, ale je
možné použít OPmac triky 
\ulink[http://petr.olsak.net/opmac-tricks.html\#slidy]{0017 a 0022}.


\sec Shrnutí
%%%%%%%%%%%%

\vskip-.5\baselineskip
\def\tthook{\adef!{\string\endtt}\adef&{\kern.25em}}
\begtt
\tit  <Název titulu (ukončený koncem řádku)>
\chap <Název kapitoly (ukončený koncem řádku)>
\sec  <Název sekce (ukončený koncem řádku)>
\secc <Název podsekce (ukončený koncem řádku)>

\maketoc         % generování obsahu
\ii <heslo>,<heslo> % vložení hesel do rejstříku
\makeindex       % generování rejstříku

\label[<lejblík>] % deklarace před cílem odkazu
\ref[<lejblík>]   % odkaz na kapitolu, sekci nebo podsekci
\pgref[<lejblík>] % odkaz na stránku

\caption/t      % číslovaný popisek tabulky
\caption/f      % číslovaný popisek obrázku
\eqmark         % číslovaná rovnice

\begitems           % začátek výčtu položek
\enditems           % konec výčtu položek
\begtt              % začátek verbatim výpisu
!              % konec verbatim výpisu
\activettchar <znak> % inicializace znaku pro in-text verbatim
\verbinput          % verbatim výpis ze souboru
\begmulti           % začátek vícesloupcové sazby
\endmulti           % konec vícesloupcové sazby

\cite[<lejblíky>]  % místo, odkud se odkazuje do seznamu literatury
\rcite[<lejblíky>] % jako \cite, ale bez závorek kolem
\sortcitations \shortcitations \nonumcitations % deklarace typu odkazů
\bib[<lejblík>]    % položka v seznamu literatury
\usebibtex{<bib-báze>}{<bst-styl>}  % použití bibTeXu pro seznam literatury
\genbbl{<bib-báze>}{<bst-styl>}     % předgenerování seznamu literatury
\usebbl/? <bbl-báze>  % použití předgenerované databáze, ? je z {a,b,c}
\usebib/? (<style>) <bib-báze> % přímé čtení .bib bez BibTeXu, ? je z {c,s}

\fontfam[<název fontové rodiny>]                 % výběr fontové rodiny
\typosize[<font-velikost>/<řádkování>]           & % nastavení velikosti sazby
\typoscale[<faktor-font>/<faktor-řádkování>]       % škálování velikosti sazby
\thefontsize[<velikost>] \thefontscale[<faktor>]& % velikost aktuálního fontu

\inspic <soubor>.<přípona> % vložení obrázku, přípony: jpg, png, pdf
\table{<pravidlo>}{<data>} &% makro pro tabulku
\frame{<text>}            % orámovaný text
\fnote{<text>}  % poznámka pod čarou (lokální číslování na každé stránce)
\mnote{<text>}  % poznámka na okraji (pravém nebo levém podle stránky)

\hyperlinks{<barva-in>}{<barva-out>}  % v PDF budou odkazy klikací
\outlines{<úroveň>}                 % PDF bude mít záložku s obsahem

\magscale[<faktor>]  % zvětšení/zmenšení sazby beze změny zlomu
\margins/<pg> <formát> (<levý>,<pravý>,<horní>,<dolní>)<jednotka>  % okraje

\report \letter % inicializace předdefinovaného stylu dokumentu
\endtt

\bye


