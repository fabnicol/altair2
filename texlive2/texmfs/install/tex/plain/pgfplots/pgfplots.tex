%--------------------------------------------
%
% Package pgfplots
%
% Provides a user-friendly interface to create function plots (normal
% plots, semi-logplots and double-logplots).
% 
% It is based on Till Tantau's PGF package.
%
% Copyright 2007/2008 by Christian Feuersänger.
%
% This program is free software: you can redistribute it and/or modify
% it under the terms of the GNU General Public License as published by
% the Free Software Foundation, either version 3 of the License, or
% (at your option) any later version.
% 
% This program is distributed in the hope that it will be useful,
% but WITHOUT ANY WARRANTY; without even the implied warranty of
% MERCHANTABILITY or FITNESS FOR A PARTICULAR PURPOSE.  See the
% GNU General Public License for more details.
% 
% You should have received a copy of the GNU General Public License
% along with this program.  If not, see <http://www.gnu.org/licenses/>.
%
%--------------------------------------------
% ATTENTION:
% you MAY need one of
%   \def\pgfsysdriver{pgfsys-dvipdfm.def}
%   \def\pgfsysdriver{pgfsys-pdftex.def}
%   \def\pgfsysdriver{pgfsys-dvips.def}
%
% BEFORE the first \input pgf.tex, \input tikz.tex or
% \input pgfplots.tex
% Default is
%   'dvips' for 'tex'
%   'pdftex' for 'pdftex'
% -> dvipdfm needs special attention.
%
\input tikz.tex%
%
\edef\pgfplotscatcode{\the\catcode`\@}%
\catcode`\@=11
%
\input pgfplots.revision.tex
\input pgfplots.code.tex%
\usetikzlibrary{plotmarks}%
%
\catcode`\@=\pgfplotscatcode
%
\endinput
