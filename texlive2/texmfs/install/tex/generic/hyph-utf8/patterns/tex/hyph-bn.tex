% title: Hyphenation patterns for Bengali
% copyright: Copyright (C) 2016 Santhosh Thottingal
% notice: This file is part of the hyph-utf8 package.
%     See http://www.hyphenation.org/tex for more information.
% source: https://github.com/santhoshtr/hyphenation/
% language:
%     name: Bengali
%     tag: bn
% version: 0.9.0 2016-01-16
% authors:
%   -
%     name: Santhosh Thottingal
%     contact: santhosh.thottingal (at) gmail.com
% licence:
%     - This file is available under any of the following licences:
%     -
%         name: MIT
%         url: https://opensource.org/licenses/MIT
%         text: >
%             Permission is hereby granted, free of charge, to any person
%             obtaining a copy of this software and associated documentation
%             files (the "Software"), to deal in the Software without
%             restriction, including without limitation the rights to use,
%             copy, modify, merge, publish, distribute, sublicense, and/or sell
%             copies of the Software, and to permit persons to whom the
%             Software is furnished to do so, subject to the following
%             conditions:
%
%             The above copyright notice and this permission notice shall be
%             included in all copies or substantial portions of the Software.
%
%             THE SOFTWARE IS PROVIDED "AS IS", WITHOUT WARRANTY OF ANY KIND,
%             EXPRESS OR IMPLIED, INCLUDING BUT NOT LIMITED TO THE WARRANTIES
%             OF MERCHANTABILITY, FITNESS FOR A PARTICULAR PURPOSE AND
%             NONINFRINGEMENT. IN NO EVENT SHALL THE AUTHORS OR COPYRIGHT
%             HOLDERS BE LIABLE FOR ANY CLAIM, DAMAGES OR OTHER LIABILITY,
%             WHETHER IN AN ACTION OF CONTRACT, TORT OR OTHERWISE, ARISING
%             FROM, OUT OF OR IN CONNECTION WITH THE SOFTWARE OR THE USE OR
%             OTHER DEALINGS IN THE SOFTWARE.
%     -
%         name: LGPL
%         version: 3
%         or_later: true
%         url: http://www.gnu.org/licenses/lgpl.html
%     -
%         name: GPL
%         version: 3
%         or_later: true
%         url: http://www.gnu.org/licenses/gpl.html
% hyphenmins:
%     generation:
%         left: 1
%         right: 1
% texlive:
%     babelname: bengali
%     message: Bengali hyphenation patterns
%     package: indic
%
\patterns{
% GENERAL RULE
% Do not break either side of ZERO-WIDTH JOINER  (U+200D)
2‍2
% Break on both sides of ZERO-WIDTH NON JOINER  (U+200C)
1‌1
% Break before or after any independent vowel.
অ1
আ1
ই1
ঈ1
উ1
ঊ1
ঋ1
ৠ1
ঌ1
ৡ1
এ1
ঐ1
ও1
ঔ1
% Break after any dependent vowel, but not before.
া1
ি1
ী1
ু1
ূ1
ৃ1
ৄ1
ৢ1
ৣ1
ে1
ৈ1
ো1
ৌ1
2়2
ৗ1
% Break before or after any consonant.
1ক
1খ
1গ
1ঘ
1ঙ
1চ
1ছ
1জ
1ঝ
1ঞ
1ট
1ঠ
1ড
1ড়
1ঢ
1ঢ়
1ণ
1ত
1থ
1দ
1ধ
1ন
1প
1ফ
1ব
1ভ
1ম
1য
1য়
1র
1ল
1শ
1ষ
1স
1হ
% Do not break after khanda ta.
ৎ1
% Do not break before chandrabindu, anusvara, visarga, avagraha,
% nukta and au length mark.
2ঃ1
2ং1
2ঁ1
2ঽ1
% Do not break either side of virama (may be within conjunct).
2্2
}
