% Copyright 2006 by Till Tantau
%
% This file may be distributed and/or modified
%
% 1. under the LaTeX Project Public License and/or
% 2. under the GNU Public License.
%
% See the file doc/generic/pgf/licenses/LICENSE for more details.

\ProvidesFileRCS{pgfcorepathconstruct.code.tex}


\newdimen\pgf@path@lastx
\newdimen\pgf@path@lasty

\let\pgfgetpath=\pgfsyssoftpath@getcurrentpath
\let\pgfsetpath=\pgfsyssoftpath@setcurrentpath



% Replace corners by arcs.
%
% #1 = in-size of arc
% #2 = out-size of arc
%
% Description:
%
% This command influences path construction command like
% \pgfpathlineto or \pgfpatharc. It will cause the corners at the end
% of these commands to be replaced by little arcs. If the
% corner is a 90 degrees corner and if #1=#2, a quarter-circle of
% radius #1 is put in place of the corner. If #1 and #2 are different,
% the quarter circle will instead by a quarter ellipse. If the angle
% is different from 90 degrees, a deformed quarter circle will
% result, which may or may not be desirable. For a ``perfect'' arc you
% must use the \pgfpatharc command.
%
%
% Example: One rounded corner.
%
% \pgfpathmoveto{\pgfpointxy{0}{0}}
% \pgfsetcornersarced{4pt}{4pt}
% \pgfpathlineto{\pgfpointxy{0}{1}}
% \pgfpathlineto{\pgfpointxy{1}{1}}
% \pgfstroke
%
% Example: A rounded rectangle
%
% \pgfsetcornersarced{4pt}{4pt}
% \pgfpathrectangle{\pgfpointorigin}{\pgfpoint{1cm}{1cm}}
% \pgfstroke
%
% Example: A rounded triangles
%
% \pgfsetcornersarced{4pt}{4pt}
% \pgfpathmoveto{\pgfpointorigin}
% \pgfpathlineto{\pgfpoint{1cm}{0cm}}
% \pgfpathlineto{\pgfpoint{1cm}{1cm}}
% \pgfpathclose
% \pgfstroke

\newif\ifpgf@arccorners

\def\pgfsetcornersarced#1{%
  \pgf@process{#1}%
  \edef\pgf@corner@arc{{\the\pgf@x}{\the\pgf@y}}%
  \pgf@arccornerstrue%
  \ifdim\pgf@x=0pt%
    \ifdim\pgf@y=0pt\relax%
      \pgf@arccornersfalse%
    \fi%
  \fi%
}

\def\pgf@roundcornerifneeded{%
  \ifpgf@arccorners\expandafter\pgfsyssoftpath@specialround\pgf@corner@arc\fi%
}



% The following protocol the passed sizes and all the corresponding
% softpath commands. The nonlinear transformation (nlt) module
% overwrites these commands.

\def\pgf@lt@moveto#1#2{%
  \pgf@protocolsizes{#1}{#2}%
  \pgfsyssoftpath@moveto{\the#1}{\the#2}%
}
\def\pgf@lt@lineto#1#2{%
  \pgf@protocolsizes{#1}{#2}%
  \pgfsyssoftpath@lineto{\the#1}{\the#2}%
}
\def\pgf@lt@curveto#1#2#3#4#5#6{%
  \pgf@protocolsizes{#1}{#2}%
  \pgf@protocolsizes{#3}{#4}%
  \pgf@protocolsizes{#5}{#6}%
  \pgfsyssoftpath@curveto{\the#1}{\the#2}{\the#3}{\the#4}{\the#5}{\the#6}%
}

\let\pgf@lt@closepath\pgfsyssoftpath@closepath

\let\pgf@nlt@moveto\pgf@lt@moveto
\let\pgf@nlt@lineto\pgf@lt@lineto
\let\pgf@nlt@curveto\pgf@lt@curveto
\let\pgf@nlt@closepath\pgf@lt@closepath

\let\pgf@nlt@list\pgfutil@empty % If non-empty, the nlt module is active


% Move current point to #1.
%
% #1 = new current point
%
% Example:
%
% \pgfpathmoveto{\pgfxy(0,0)}
% \pgfpathlineto{\pgfxy(0,1)}
% \pgfstroke

\def\pgfpathmoveto#1{%
  \pgfpointtransformed{#1}%
  \pgf@nlt@moveto{\pgf@x}{\pgf@y}%
  \global\pgf@path@lastx=\pgf@x%
  \global\pgf@path@lasty=\pgf@y%
}

\def\pgf@protocolsizes#1#2{%
  \ifpgf@relevantforpicturesize%
    \ifdim#1<\pgf@picminx\global\pgf@picminx#1\fi%
    \ifdim#1>\pgf@picmaxx\global\pgf@picmaxx#1\fi%
    \ifdim#2<\pgf@picminy\global\pgf@picminy#2\fi%
    \ifdim#2>\pgf@picmaxy\global\pgf@picmaxy#2\fi%
  \fi%
  \ifpgf@size@hooked%
    \let\pgf@size@hook@x#1\let\pgf@size@hook@y#2\pgf@path@size@hook%
  \fi%
  \ifdim#1<\pgf@pathminx\global\pgf@pathminx#1\fi%
  \ifdim#1>\pgf@pathmaxx\global\pgf@pathmaxx#1\fi%
  \ifdim#2<\pgf@pathminy\global\pgf@pathminy#2\fi%
  \ifdim#2>\pgf@pathmaxy\global\pgf@pathmaxy#2\fi%
}
\newif\ifpgf@size@hooked
\let\pgf@path@size@hook=\pgfutil@empty%

\def\pgf@resetpathsizes{%
  \global\pgf@pathmaxx=-16000pt\relax%
  \global\pgf@pathminx=16000pt\relax%
  \global\pgf@pathmaxy=-16000pt\relax%
  \global\pgf@pathminy=16000pt\relax%
}

\def\pgf@getpathsizes#1{%
  \edef#1{{\the\pgf@pathmaxx}{\the\pgf@pathminx}{\the\pgf@pathmaxy}{\the\pgf@pathminy}}%
}
\def\pgf@setpathsizes#1{%
  \expandafter\pgf@@setpathsizes#1%
}
\def\pgf@@setpathsizes#1#2#3#4{%
  \global\pgf@pathmaxx=#1\relax%
  \global\pgf@pathminx=#2\relax%
  \global\pgf@pathmaxy=#3\relax%
  \global\pgf@pathminy=#4\relax%
}





% Append a line from the current point to #1 to the current path.
%
% #1 = end of line
%
% Example:
%
% \pgfpathmoveto{\pgfxy(0,0)}
% \pgfpathlineto{\pgfxy(0,1)}
% \pgfstroke

\def\pgfpathlineto#1{%
  \pgfpointtransformed{#1}%
  \pgf@roundcornerifneeded%
  \pgf@nlt@lineto{\pgf@x}{\pgf@y}%
  \global\pgf@path@lastx=\pgf@x%
  \global\pgf@path@lasty=\pgf@y%
}



% Close the current path.
%
% Example:
%
% % Draws two triangles
% \pgfpathmoveto{\pgfxy(0,0)}
% \pgfpathlineto{\pgfxy(0,1)}
% \pgfpathlineto{\pgfxy(1,0)}
% \pgfclosepath
% \pgfpathmoveto{\pgfxy(2,0)}
% \pgfpathlineto{\pgfxy(2,1)}
% \pgfpathlineto{\pgfxy(3,0)}
% \pgfpathclose
% \pgfstroke

\def\pgfpathclose{%
  \pgf@roundcornerifneeded%
  \pgf@nlt@closepath%
}


% Append a cubic bezier spline from the current point to #3 with control
% points #1 and #2 to the current path.
%
% #1 = first control point
% #2 = second control point
% #3 = end point
%
% Example:
%
% \pgfpathmoveto{\pgfpointxy{0}{0}}
% \pgfpathcurveto{\pgfpointxy{0}{1}}{\pgfpointxy{1}{1}}{\pgfpointxy{1}{2}}
% \pgfstroke

\def\pgfpathcurveto#1#2#3{%
  \pgfpointtransformed{#3}%
  \pgf@xb=\pgf@x%
  \pgf@yb=\pgf@y%
  \pgfpointtransformed{#2}%
  \pgf@xa=\pgf@x%
  \pgf@ya=\pgf@y%
  \pgfpointtransformed{#1}%
  \pgf@roundcornerifneeded%
  \pgf@nlt@curveto{\pgf@x}{\pgf@y}{\pgf@xa}{\pgf@ya}{\pgf@xb}{\pgf@yb}%
  \global\pgf@path@lastx=\pgf@xb%
  \global\pgf@path@lasty=\pgf@yb%
}



% Append a quadratic bezier spline from the current point to #2 with
% control point #1 to the current path.
%
% #1 = control point
% #2 = end point
%
% Example:
%
% \pgfpathmoveto{\pgfpointxy{0}{0}}
% \pgfpathquadraticcurveto{\pgfpointxy{1}{1}}{\pgfpointxy{2}{0}}
% \pgfstroke

\def\pgfpathquadraticcurveto#1#2{%
  \pgfpointtransformed{#2}%
  \pgf@xb=\pgf@x%
  \pgf@yb=\pgf@y%
  \pgfpointtransformed{#1}%
  \pgf@xc=.6666666\pgf@x%
  \pgf@yc=.6666666\pgf@y%
  % compute second control point:
  \pgf@xa=.33333333\pgf@xb%
  \pgf@ya=.33333333\pgf@yb%
  \advance\pgf@xa by\pgf@xc%
  \advance\pgf@ya by\pgf@yc%
  % compute first control point
  \advance\pgf@xc by.3333333\pgf@path@lastx%
  \advance\pgf@yc by.3333333\pgf@path@lasty%
  \pgf@roundcornerifneeded%
  \pgf@nlt@curveto{\pgf@xc}{\pgf@yc}{\pgf@xa}{\pgf@ya}{\pgf@xb}{\pgf@yb}%
  \global\pgf@path@lastx=\pgf@xb%
  \global\pgf@path@lasty=\pgf@yb%
}




% Append an arc to the current point, where the current point is at
% angle #1 and the end is at angle #2. If #2 > #1, the arc is drawn
% counter-clockwise, otherwise it is clockwise.
%
% #1 = angle of first point
% #2 = angle of second point
% #3 = radius or x-radius/y-radius
%
% Example:
%
% \pgfpathmoveto{\pgfxy(0,0)}
% \pgfpatharc{0}{90}{2cm}
% \pgfstroke

\def\pgfpatharc#1#2#3{%
  {%
    \pgfmathparse{#1}\let\pgf@temp@a=\pgfmathresult%
    \pgfmathparse{#2}\let\pgf@temp@b=\pgfmathresult%
    \pgfutil@in@{and }{#3}%
    \ifpgfutil@in@%
      \pgf@arc@get@radii#3\pgf@arc@stop%
    \else
      \pgf@arc@get@radii#3and #3\pgf@arc@stop%
    \fi%
    \pgf@arc@local@angle@a=\pgf@temp@a pt%
    \pgf@arc@local@angle@b=\pgf@temp@b pt%
    \loop%
      \pgfutil@tempdima=\pgf@arc@local@angle@a%
      \advance\pgfutil@tempdima by-\pgf@arc@local@angle@b\relax%
      \ifdim\pgfutil@tempdima<0pt\relax%
        \pgfutil@tempdima=-\pgfutil@tempdima\relax%
      \fi%
    \ifdim\pgfutil@tempdima>90pt\relax%
      \ifdim\pgfutil@tempdima>115pt\relax%
        \pgf@arc@temp=90pt% big skip
      \else%
        \pgf@arc@temp=60pt% smaller skip to ensure wide segments
                          % (important shortened end segments because
                          % of arrow tips)
      \fi%
      \ifdim\pgf@arc@local@angle@b>\pgf@arc@local@angle@a\relax%
        {%
          \pgf@arc@local@angle@b=\pgf@arc@local@angle@a\relax%
          \advance\pgf@arc@local@angle@b by\pgf@arc@temp\relax%
          \pgf@arc%
        }
        \advance\pgf@arc@local@angle@a by\pgf@arc@temp\relax%
      \else
        {%
          \pgf@arc@local@angle@b=\pgf@arc@local@angle@a\relax%
          \advance\pgf@arc@local@angle@b by-\pgf@arc@temp\relax%
          \pgf@arc%
        }%
        \advance\pgf@arc@local@angle@a by-\pgf@arc@temp\relax%
      \fi%
    \repeat%
    \pgf@roundcornerifneeded%
    \pgf@arc%
  }%
}
\dimendef\pgf@arc@local@angle@a=0
\dimendef\pgf@arc@local@angle@b=1
\dimendef\pgf@arc@temp=2

\def\pgf@arc@get@radii#1and #2\pgf@arc@stop{%
  \pgfmathparse{#1}\let\pgf@arc@radius@a=\pgfmathresult%
  \pgfmathparse{#2}\let\pgf@arc@radius@b=\pgfmathresult%
}


\def\pgf@arc{%
  {%
  \pgfutil@tempdima=\pgf@arc@radius@a pt%
  \pgfutil@tempdimb=\pgf@arc@radius@b pt%
  %
  \pgf@xa=\pgf@arc@local@angle@a\relax%
  \pgf@xb=\pgf@arc@local@angle@b\relax%
  \advance\pgf@xb by-\pgf@xa\relax%
  \ifdim\pgf@xb<0pt\relax%
    \pgf@xb=-\pgf@xb\relax%
  \fi%
  \ifdim\pgf@xb=90.0pt%
    \def\pgfmathresult{0.55228475}%
  \else%
    \pgfmathparse{1.333333333*tan(.25*\pgf@sys@tonumber{\pgf@xb})}% many thanks to Ken Starks
  \fi%
  \pgfutil@tempdima=\pgfmathresult\pgfutil@tempdima%
  \pgfutil@tempdimb=\pgfmathresult\pgfutil@tempdimb%
  %.. controls +(\pgf@xa+90:\pgfutil@tempdima) and +(\pgf@xb-90:\pgfutil@tempdima) .. +(-(#1:#3)+(#2:#3))%
  % store first support vector in xa/ya:
  \pgf@xa=\pgf@arc@local@angle@a\relax%
  \ifdim\pgf@arc@local@angle@b>\pgf@arc@local@angle@a\relax%
    \advance\pgf@xa by 90pt\relax%
  \else%
    \advance\pgf@xa by -90pt\relax%
  \fi%
  \edef\pgf@arc@angle{\pgf@sys@tonumber{\pgf@xa}}%
  \pgfpointtransformed{\pgfpointpolar{\pgf@arc@angle}{\pgfutil@tempdima and \pgfutil@tempdimb}}%
  \advance\pgf@x by-\pgf@pt@x%
  \advance\pgf@y by-\pgf@pt@y%
  \pgf@xa=\pgf@path@lastx%
  \pgf@ya=\pgf@path@lasty%
  \advance\pgf@xa by \pgf@x%
  \advance\pgf@ya by \pgf@y%
  % store target in xb/yb:
  \pgfpointtransformed{\pgfpointpolar{\pgf@sys@tonumber{\pgf@arc@local@angle@a}}{\pgf@arc@radius@a pt and \pgf@arc@radius@b pt}}%
  \pgf@xb=\pgf@path@lastx%
  \pgf@yb=\pgf@path@lasty%
  \advance\pgf@xb by -\pgf@x%
  \advance\pgf@yb by -\pgf@y%
  \pgfpointtransformed{\pgfpointpolar{\pgf@sys@tonumber{\pgf@arc@local@angle@b}}{\pgf@arc@radius@a pt and \pgf@arc@radius@b pt}}%
  \advance\pgf@xb by \pgf@x%
  \advance\pgf@yb by \pgf@y%
  % store second support xc/yc:
  \ifdim\pgf@arc@local@angle@b>\pgf@arc@local@angle@a\relax%
    \advance\pgf@arc@local@angle@b by -90pt\relax%
  \else%
    \advance\pgf@arc@local@angle@b by 90pt\relax%
  \fi%
  \pgfpointtransformed{\pgfpointpolar{\pgf@sys@tonumber{\pgf@arc@local@angle@b}}{\pgfutil@tempdima and \pgfutil@tempdimb}}%
  \advance\pgf@x by-\pgf@pt@x%
  \advance\pgf@y by-\pgf@pt@y%
  \pgf@xc=\pgf@xb\relax%
  \pgf@yc=\pgf@yb\relax%
  \advance \pgf@xc by \pgf@x\relax%
  \advance \pgf@yc by \pgf@y\relax%
  \pgf@nlt@curveto{\pgf@xa}{\pgf@ya}{\pgf@xc}{\pgf@yc}{\pgf@xb}{\pgf@yb}%
  \global\pgf@path@lastx=\pgf@xb%
  \global\pgf@path@lasty=\pgf@yb%
  }%
}


% Append an arc to the current point, where the arc is on an ellipse
% given by two axis vectors.
%
% #1 = angle of first point
% #2 = angle of second point
% #3 = first axis
% #4 = second axis
%
% Example:
%
% \pgfpathmoveto{\pgfxy(0,0)}
% \pgfpatharcaxes{0}{90}{\pgfpointxy{2}{0}}{\pgfpointxy{0}{2}}
% \pgfstroke

\def\pgfpatharcaxes#1#2#3#4{%
  {%
    \pgftransformtriangle{\pgfpointorigin}{#3}{#4}%
    \pgfpatharc{#1}{#2}{1pt}%
  }%
}



% Append an arc to the current point that ends at a given position.
%
% #1 = x-radius
% #2 = y-radius
% #3 = x-axis-rotation (in degrees)
% #4 = large-arc-sweep-flag (0 or 1)
% #5 = sweep-flag (0 or 1)
% #6 = target point
%
% Description:
%
% This command implements an arc drawing where a given target
% coordinate (#6) is given and the task is to draw an arc of an
% ellipse with the given radii. The center point of the ellipse is not
% give, but computed automatically.
%
% This kind of "endpoint parameterization" of an arc is exactly the
% same as the one specified by the SVG-specification for the "A" and
% "a" path commands. Please see the SVG-specification for details.
%
% Note that the problem is internally converted to drawing an arc
% using \pgfpatharc. This means that there may be a heavy loss of
% accuracy.
%
% Example:
%
% \pgfpathmoveto{\pgfpoint{1cm}{1cm}}
% \pgfpatharcto{1cm}{1cm}{0}{0}{0}{\pgfpoint{0cm}{2cm}}

\def\pgfpatharcto#1#2#3#4#5#6{%
  {%
    % The following code is based on the transformation described in svg
    % 1.1 specification Section F.6.5
    %
    % Step 1: store the simple parameters (xa=x1 since TeX does not
    % allow numbers in names)
    %
    \pgfmathsetmacro\pgf@arcto@rx{abs(#1)}%
    \pgfmathsetmacro\pgf@arcto@ry{abs(#2)}%
    \ifdim\pgf@arcto@rx pt=0pt% special rule: zero radius=straight line
      \gdef\pgf@marshal{\pgfpathlineto{#6}}%
    \else
    \ifdim\pgf@arcto@ry pt=0pt% special rule: zero radius=straight line
      \gdef\pgf@marshal{\pgfpathlineto{#6}}%
    \else
    \pgfmathsetmacro\pgf@arcto@phi{#3}%
    \pgfmathsetmacro\pgf@arcto@fA{#4}%
    \ifdim\pgf@arcto@fA pt=0pt
    \else
      \pgfmathsetmacro\pgf@arcto@fA{1.0} % Special rule: every non-zero value is 1.
    \fi
    \pgfmathsetmacro\pgf@arcto@fS{#5}%
    \ifdim\pgf@arcto@fS pt=0pt
    \else
      \pgfmathsetmacro\pgf@arcto@fS{1.0} % Special rule: every non-zero value is 1.
    \fi
    \pgf@process{#6}
    \edef\pgf@arcto@xb{\the\pgf@x}%
    \edef\pgf@arcto@yb{\the\pgf@y}%
    %
    % Step 2: x1,y1 is more complicated to compute: It is given by lastx
    % and lasty, but these are transformed coordinates, we need the
    % untransformed ones. So, we inverse the transformation (arghh...)
    %
    \pgftransforminvert%
    \pgf@process{\pgfpointtransformed{\pgfqpoint{\pgf@path@lastx}{\pgf@path@lasty}}}
    \edef\pgf@arcto@xa{\the\pgf@x}
    \edef\pgf@arcto@ya{\the\pgf@y}
    \edef\pgf@temp@a{\pgf@arcto@xa,\pgf@arcto@ya}
    \edef\pgf@temp@b{\pgf@arcto@xb,\pgf@arcto@yb}
    \ifx\pgf@temp@a\pgf@temp@b% special rule: skip!
      \global\let\pgf@marshal\pgfutil@empty
    \else
    %
    % Ok, now we got all the parameters setup. Now comes the
    % computation...
    %
    %
    % Step 3: Start with a new coordinate system and rotate everything
    % by the negated phi.
    %
    \pgftransformreset
    \pgftransformrotate{-\pgf@arcto@phi}
    % Ok, using \pgfpointtransformed we now get transformed points...
    %
    % Step 4: Compute x1' and y1' (xaprime and yaprime)
    %
    \pgf@process{
      \pgfpointtransformed{\pgfpointscale{.5}{\pgfpointdiff
          {\pgfqpoint{\pgf@arcto@xb}{\pgf@arcto@yb}}
          {\pgfqpoint{\pgf@arcto@xa}{\pgf@arcto@ya}}
        }
      }
    }
    \edef\pgf@arcto@xaprime{\pgf@sys@tonumber\pgf@x}
    \edef\pgf@arcto@yaprime{\pgf@sys@tonumber\pgf@y}
    %
    % Compute Lambda
    %
    \pgfmathsetmacro\pgf@arcto@frac@x{\pgf@arcto@xaprime/\pgf@arcto@rx}
    \pgfmathsetmacro\pgf@arcto@frac@y{\pgf@arcto@yaprime/\pgf@arcto@ry}
    \pgfmathsetmacro\pgf@arcto@lambda{
      \pgf@arcto@frac@x*\pgf@arcto@frac@x+\pgf@arcto@frac@y*\pgf@arcto@frac@y
    }
    \ifdim\pgf@arcto@lambda pt>1pt%
      \pgfmathsetmacro\pgf@arcto@sqrt@lambda{sqrt(\pgf@arcto@lambda)}
      \pgfmathsetmacro\pgf@arcto@rx{\pgf@arcto@sqrt@lambda*\pgf@arcto@rx}
      \pgfmathsetmacro\pgf@arcto@ry{\pgf@arcto@sqrt@lambda*\pgf@arcto@ry}
    \fi
    %
    % Do some scaling
    %
    \pgfmathsetmacro\pgf@arcto@xaprime@abs{abs(\pgf@arcto@xaprime)}
    \pgfmathsetmacro\pgf@arcto@yaprime@abs{abs(\pgf@arcto@yaprime)}
    \pgfmathmax@{\pgf@arcto@rx,\pgf@arcto@ry,\pgf@arcto@xaprime@abs,\pgf@arcto@yaprime@abs}
    \pgfmathsetmacro\pgf@arcto@scaling{20/\pgfmathresult}
    \pgfmathsetmacro\pgf@arcto@rx@scaled{\pgf@arcto@scaling*\pgf@arcto@rx}
    \pgfmathsetmacro\pgf@arcto@ry@scaled{\pgf@arcto@scaling*\pgf@arcto@ry}
    \pgfmathsetmacro\pgf@arcto@xaprime@scaled{\pgf@arcto@scaling*\pgf@arcto@xaprime}
    \pgfmathsetmacro\pgf@arcto@yaprime@scaled{\pgf@arcto@scaling*\pgf@arcto@yaprime}
    %
    % Step 5: Now comes the messy computation of c1' and c2'.
    %
    \ifdim\pgf@arcto@rx pt>\pgf@arcto@ry pt%
      \pgfmathsetmacro\pgf@arcto@rx@over@ry{\pgf@arcto@rx/\pgf@arcto@ry}
      \pgfmathsetmacro\pgf@arcto@ry@over@rx{\pgf@arcto@ry/\pgf@arcto@rx}
      \pgfmathsetmacro\pgf@arcto@temp{\pgf@arcto@ry@over@rx*\pgf@arcto@xaprime@scaled}
      \pgfmathsetmacro\pgf@arcto@numerator{
        \pgf@arcto@ry@scaled*\pgf@arcto@ry@scaled-
        \pgf@arcto@yaprime@scaled*\pgf@arcto@yaprime@scaled-
        \pgf@arcto@temp*\pgf@arcto@temp
      }
      \pgfmathsetmacro\pgf@arcto@denominator{
        \pgf@arcto@yaprime@scaled*\pgf@arcto@yaprime@scaled+
        \pgf@arcto@temp*\pgf@arcto@temp
      }
    \else
      \pgfmathsetmacro\pgf@arcto@rx@over@ry{\pgf@arcto@rx/\pgf@arcto@ry}
      \pgfmathsetmacro\pgf@arcto@ry@over@rx{\pgf@arcto@ry/\pgf@arcto@rx}
      \pgfmathsetmacro\pgf@arcto@temp{\pgf@arcto@rx@over@ry*\pgf@arcto@yaprime@scaled}
      \pgfmathsetmacro\pgf@arcto@numerator{
        \pgf@arcto@rx@scaled*\pgf@arcto@rx@scaled-
        \pgf@arcto@xaprime@scaled*\pgf@arcto@xaprime@scaled-
        \pgf@arcto@temp*\pgf@arcto@temp
      }
      \pgfmathsetmacro\pgf@arcto@denominator{
        \pgf@arcto@xaprime@scaled*\pgf@arcto@xaprime@scaled+
        \pgf@arcto@temp*\pgf@arcto@temp
      }
    \fi
    \pgfmathsetmacro\pgf@arcto@frac{
      \pgf@arcto@numerator/\pgf@arcto@denominator
    }
    \ifdim\pgf@arcto@frac pt<0pt
      \pgfmathsetmacro\pgf@arcto@factor{0}
    \else
      \pgfmathsetmacro\pgf@arcto@factor{sqrt(\pgf@arcto@frac)}
    \fi
    \ifx\pgf@arcto@fA\pgf@arcto@fS
      \pgfmathsetmacro\pgf@arcto@factor{-\pgf@arcto@factor}
    \fi
    \pgfmathsetmacro\pgf@arcto@cxprime{
      \pgf@arcto@factor*\pgf@arcto@rx@over@ry*\pgf@arcto@yaprime
    }
    \pgfmathsetmacro\pgf@arcto@cyprime{
      -\pgf@arcto@factor*\pgf@arcto@ry@over@rx*\pgf@arcto@xaprime
    }
    %
    % Step 6: Ok, now compute cx,cy
    %
    \pgftransformreset
    \pgftransformrotate{\pgf@arcto@phi}
    \pgf@process{
      \pgfpointtransformed{\pgfqpoint{\pgf@arcto@cxprime pt}{\pgf@arcto@cyprime pt}}
    }
    \edef\pgf@arcto@temp{\noexpand\pgfqpoint{\the\pgf@x}{\the\pgf@y}}
    \pgf@process{\pgfpointadd{\pgf@arcto@temp}{
        \pgfpointscale{.5}{
          \pgfpointadd
          {\pgfqpoint{\pgf@arcto@xa}{\pgf@arcto@ya}}
          {\pgfqpoint{\pgf@arcto@xb}{\pgf@arcto@yb}}        }
      }
    }
    \edef\pgf@arcto@cx{\the\pgf@x}
    \edef\pgf@arcto@cy{\the\pgf@y}
    %
    % Step 7: Compute start angle:
    %
    \pgfmathsetmacro\pgf@arcto@vec@x{(\pgf@arcto@xaprime-\pgf@arcto@cxprime)/\pgf@arcto@rx}
    \pgfmathsetmacro\pgf@arcto@vec@y{(\pgf@arcto@yaprime-\pgf@arcto@cyprime)/\pgf@arcto@ry}
    \pgfmathsetmacro\pgf@arcto@denominator{veclen(\pgf@arcto@vec@x,\pgf@arcto@vec@y)}
    \pgfmathsetmacro\pgf@arcto@frac{\pgf@arcto@vec@x/\pgf@arcto@denominator}
    \pgfmathsetmacro\pgf@arcto@theta@start{acos(\pgf@arcto@frac)}
    \ifdim\pgf@arcto@vec@y pt<0pt
      \pgfmathsetmacro\pgf@arcto@theta@start{-\pgf@arcto@theta@start}
    \fi
    %
    % Step 8: Compute end angle:
    %
    \pgfmathsetmacro\pgf@arcto@vec@x{(-\pgf@arcto@xaprime-\pgf@arcto@cxprime)/\pgf@arcto@rx}
    \pgfmathsetmacro\pgf@arcto@vec@y{(-\pgf@arcto@yaprime-\pgf@arcto@cyprime)/\pgf@arcto@ry}
    \pgfmathsetmacro\pgf@arcto@denominator{veclen(\pgf@arcto@vec@x,\pgf@arcto@vec@y)}
    \pgfmathsetmacro\pgf@arcto@frac{\pgf@arcto@vec@x/\pgf@arcto@denominator}
    \pgfmathsetmacro\pgf@arcto@theta@end{acos(\pgf@arcto@frac)}
    \ifdim\pgf@arcto@vec@y pt<0pt
      \pgfmathsetmacro\pgf@arcto@theta@end{-\pgf@arcto@theta@end}
    \fi
    \pgfmathsetmacro\pgf@arcto@delta@theta{abs(\pgf@arcto@theta@start-\pgf@arcto@theta@end)}
    \ifdim\pgf@arcto@fA pt=0pt%
      \ifdim\pgf@arcto@delta@theta pt>180pt%
        % Ok, we need to adjust the angle!
        \ifdim\pgf@arcto@theta@end pt>\pgf@arcto@theta@start pt
          \pgfmathsetmacro\pgf@arcto@theta@end{\pgf@arcto@theta@end-360}
        \else
          \pgfmathsetmacro\pgf@arcto@theta@end{\pgf@arcto@theta@end+360}
        \fi
      \fi
    \else
      \ifdim\pgf@arcto@delta@theta pt<180pt%
        % Ok, we need to adjust the angle!
        \ifdim\pgf@arcto@theta@end pt>\pgf@arcto@theta@start pt
          \pgfmathsetmacro\pgf@arcto@theta@end{\pgf@arcto@theta@end-360}
        \else
          \pgfmathsetmacro\pgf@arcto@theta@end{\pgf@arcto@theta@end+360}
        \fi
      \fi
    \fi
    \xdef\pgf@marshal{\noexpand
      \pgfpatharcaxes{\pgf@arcto@theta@start}{\pgf@arcto@theta@end}
      {\noexpand\pgfpointpolar{\pgf@arcto@phi}{\pgf@arcto@rx}}
      {\noexpand\pgfpointpolar{\pgf@arcto@phi+90}{\pgf@arcto@ry}}
    }
    \fi\fi\fi
  }
  \pgf@marshal
}



% the quality of arc approximation by means of Bezier splines is
% controlled by a mesh width.
%
% The mesh width is provided in (full!) degrees. The smaller the mesh
% width, the more precise the arc approximation.
%
% Use an empty value to disable spline approximation (uses a single
% cubic polynomial for the complete arc).
%
% The value must be an integer!
\def\pgfpatharctomaxstepsize{45}

% A specialized arc operation for an arc on an (axis--parallel) ellipse.
%
% In contrast to \pgfpatharc, it explicitly interpolates start- and end points.
%
% In contrast to \pgfpatharcto, this routine is numerically stable and
% quite fast since it relies on a lot of precomputed information.
%
% #1 center of ellipse
% #2 angle of last path position inside of the ellipse
% #3 end angle
% #4 end point (a \pgfpoint)
% #5 xradius
% #6 yradius
% #7 the ratio xradius/yradius of the ellipse
% #8 the ratio yradius/xradius of the ellipse
% Example:
%    \def\cx{1cm}% center x
%    \def\cy{1cm}% center y
%    \def\startangle{0}%
%    \def\endangle{45}%
%    \def\a{5cm}% xradius
%    \def\b{10cm}% yradius
%    \pgfmathparse{\a/\b}\let\abratio=\pgfmathresult
%    \pgfmathparse{\b/\a}\let\baratio=\pgfmathresult
%
% \pgfpathmoveto{\pgfpoint{\cx+\a*cos(\startangle)}{\cy+\b*sin(\startangle)}}%
% \pgfpatharctoprecomputed
%    {\pgfpoint{\cx}{\cy}}
%    {\startangle}
%    {\endangle}
%    {\pgfpoint{\cx+\a*cos(\endangle)}{\cy+\b*sin(\endangle)}}%
%    {\a}
%    {\b}
%    {\abratio}
%    {\baratio}
%
\def\pgfpatharctoprecomputed#1#2#3#4#5#6#7#8{%
    \begingroup
    % Implementation idea:
    %
    % let
    %   m = center (#1)
    %   \gamma_0 = start angle
    %   \gamma_1 = end angle
    %   a = x radius
    %   b = y radius
    %
    % an axis parallel ellipse is parameterized by
    % C(\gamma) = m + ( a cos(\gamma), b sin(\gamma) ), \gamma in [0,360].
    %
    % Now, consider the segment \gamma(t),
    %   \gamma:[0,1] -> [\gamma_0,\gamma_1],
    %            t   ->  \gamma_0 + t(\gamma_1 - \gamma_0)
    % and
    %   C(\gamma(t)) which is defined on [0,1].
    %
    % I'd like to approximate the arc by one or more cubic bezier
    % splines which interpolate through the last and first provided
    % points.
    %
    % In general, a Bezier spline C:[0,1] -> \R of order n fulfills
    % C'(0) = n ( P_1 - P_0 ),
    % C'(1) = n ( P_n - P_{n-1} ).
    % For n=3 and given P_0 and P_3, I can directly compute P_1 and P_2 once I know
    % the derivatives at t=0 and t=1.
    %
    % The derivatives in our case are
    %   ( C \circ \gamma )'(t) = C'[\gamma(t)] * \gamma'(t)
    %       = ( -a pi/180 sin(\gamma(t)),  b pi/180 cos(\gamma(t)) ) * (\gamma_1 - \gamma_0).
    % The pi/180 comes into play since we are working with degrees.
    %
    % Expression (C\circ\gamma)'(0) using P_0 and (C \circ \gamma)'(1)
    % using P_3 yields the expressions
    %   (C \circ \gamma)'(0) =
    %       pi/180 * (\gamma_1 - \gamma_0)* [ - a/b(P_0^y - my), b/a (P_0^x - mx) ]
    %   (C \circ \gamma)'(1) =
    %       pi/180 * (\gamma_1 - \gamma_0)* [ - a/b(P_3^y - my), b/a (P_3^x - mx) ]
    %
    % defining
    %  scaleA = a/b * pi / (3*180) * (\gamma_1 - \gamma_0)
    % and
    %  scaleB = b/a * pi / (3*180) * (\gamma_1 - \gamma_0)
    % yields the direct expressions for the intermediate bezier
    % control points
    %
    % P_1 = [
    %   P_0^x - scaleA* ( P_0^y -my),
    %   P_0^y + scaleB* ( P_0^x -mx) ]
    % and
    % P_2 = [
    %   P_3^x + scaleA* ( P_3^y -my),
    %   P_3^y - scaleB* ( P_3^x -mx) ].
    %
    % This works fast, with few operations, if
    % - a/b and b/a are known in advance
    % - P_0 and P_3 are known in advance
    % - \gamma_0 and \gamma_1 are known.
    %
    % It is also reliable if (\gamma_1 - \gamma_0) is small
    %
    \pgf@process{#1}%
    \edef\pgfpath@center@x{\the\pgf@x}%
    \edef\pgfpath@center@y{\the\pgf@y}%
    \def\pgfpath@completearcend{#4}%
    % compute scale (#3-#2) * pi/(3*180) = (#3 - #2) * pi/27 * 1/20
    % splitting pi/(3*180) into two scales has higher TeX accuracy
    \pgf@xa=#2pt
    \pgf@xb=#3pt
    \edef\pgfpath@startangle{#2pt}%
    \edef\pgfpath@endangle{\pgf@sys@tonumber\pgf@xb}%
    %
    \pgf@ya=\pgf@xb
    \advance\pgf@ya by-\pgf@xa
    %
    \ifx\pgfpatharctomaxstepsize\pgfutil@empty
        \def\pgfpath@N{1}%
        \pgf@xc=\pgf@ya
    \else
        \pgf@xc=\pgf@ya% compute N = floor((gamma_1 - gamma_0) / max) +1
        \ifdim\pgf@xc<0pt
            \multiply\pgf@xc by-1
        \fi
        \divide\pgf@xc by\pgfpatharctomaxstepsize\relax
        \afterassignment\pgfutil@gobble@until@relax
        \c@pgf@counta=\the\pgf@xc\relax
        \advance\c@pgf@counta by1
        \edef\pgfpath@N{\the\c@pgf@counta}%
        %
        \pgf@xc=\pgf@ya
        \divide\pgf@xc by\c@pgf@counta
    \fi
    %
    \edef\pgfpath@h{\pgf@sys@tonumber\pgf@xc}%
    %
%\message{pgfpathellipse: using N =\pgfpath@N\space spline points y0 = \pgfpath@startangle, y0+i*h, yN=\pgfpath@endangle, i=1,...,(\pgfpath@N-1), with h=\pgfpath@h\space mesh width (total arc angle \pgf@sys@tonumber\pgf@ya).}%
    %
    %
    \pgf@xc=0.116355283466289\pgf@xc % pi/27
    \divide\pgf@xc by20
    \pgf@xa=#7\pgf@xc
    \edef\pgfpath@scale@A{\pgf@sys@tonumber\pgf@xa}%
    \pgf@xa=#8\pgf@xc
    \edef\pgfpath@scale@B{\pgf@sys@tonumber\pgf@xa}%
    %
    % compute intermediate spline segments for
    %  i = 1,...,N-1
    % this is a no-op for N=1.
    \c@pgf@countd=1
    \pgfutil@loop
    \ifnum\c@pgf@countd<\pgfpath@N\relax
        %
        \pgf@xa=\pgfpath@startangle % compute \pgf@xa = y_0 + i*h
        \pgf@xb=\pgfpath@h pt
        \multiply\pgf@xb by\c@pgf@countd
        \advance\pgf@xa by\pgf@xb
        \edef\pgfpath@angle@i{\pgf@sys@tonumber\pgf@xa}%
%\message{angle \the\c@pgf@countd: \pgfpath@angle@i...}%
        %
        \pgfpatharcofellipse@{%
            \pgfpoint
                {\pgfpath@center@x + #5*cos(\pgfpath@angle@i)}
                {\pgfpath@center@y + #6*sin(\pgfpath@angle@i)}%
        }%
        %
        \advance\c@pgf@countd by1
    \pgfutil@repeat
    %
    % compute final spline segment. It only differs insofar as the
    % final point is already known explicitly and should be
    % interpolated without additional math error.
%\message{angle \pgfpath@N: \pgfpath@endangle...}%
    \pgfpatharcofellipse@{\pgfpath@completearcend}%
    \endgroup
}%
\def\pgfpatharcofellipse@#1{%
    \begingroup
    \pgf@process{#1}%
    \edef\pgfpath@endpt{\global\pgf@x=\the\pgf@x\space\global\pgf@y=\the\pgf@y\space}%
    %
    \pgfpathcurveto{
        \begingroup
        \global\pgf@x=\pgf@path@lastx
        \global\pgf@y=\pgf@path@lasty
        \pgf@xa=\pgf@x \advance\pgf@xa by-\pgfpath@center@x
        \pgf@ya=\pgf@y \advance\pgf@ya by-\pgfpath@center@y
        \global\advance\pgf@x by-\pgfpath@scale@A\pgf@ya
        \global\advance\pgf@y by \pgfpath@scale@B\pgf@xa
        \endgroup
    }{%
        \begingroup
        \pgfpath@endpt
        \pgf@xa=\pgf@x \advance\pgf@xa by-\pgfpath@center@x
        \pgf@ya=\pgf@y \advance\pgf@ya by-\pgfpath@center@y
        \global\advance\pgf@x by \pgfpath@scale@A\pgf@ya
        \global\advance\pgf@y by-\pgfpath@scale@B\pgf@xa
        \endgroup
    }{%
        \pgfpath@endpt
    }%
    \endgroup
}









% Append  an ellipse to the current path.
%
% #1 = center
% #2 = first axis
% #3 = second axis
%
% Example:
%
% % Add a circle of radius 3cm around the origin
% \pgfpathellipse{\pgforigin}{\pgfxy(2,0)}{\pgfxy(0,1)}
%
% % Draw a non-filled circle of radius 1cm around the point (1,1)
% \pgfpathellipse{\pgfxy(1,1)}{\pgfxy(1,1)}{\pgfxy(-2,2)}
% \pgfstroke

\def\pgfpathellipse#1#2#3{%
  \pgfpointtransformed{#1}% store center in xc/yc
  \pgf@xc=\pgf@x%
  \pgf@yc=\pgf@y%
  \pgfpointtransformed{#2}%
  \pgf@xa=\pgf@x% store first axis in xa/ya
  \pgf@ya=\pgf@y%
  \advance\pgf@xa by-\pgf@pt@x%
  \advance\pgf@ya by-\pgf@pt@y%
  \pgfpointtransformed{#3}%
  \pgf@xb=\pgf@x% store second axis in xb/yb
  \pgf@yb=\pgf@y%
  \advance\pgf@xb by-\pgf@pt@x%
  \advance\pgf@yb by-\pgf@pt@y%
  {%
    \advance\pgf@xa by\pgf@xc%
    \advance\pgf@ya by\pgf@yc%
    \pgf@nlt@moveto{\pgf@xa}{\pgf@ya}%
  }%
  \pgf@x=0.55228475\pgf@xb% first arc
  \pgf@y=0.55228475\pgf@yb%
  \advance\pgf@x by\pgf@xa%
  \advance\pgf@y by\pgf@ya%
  \advance\pgf@x by\pgf@xc%
  \advance\pgf@y by\pgf@yc%
  \edef\pgf@temp{\pgf@xc\the\pgf@x\pgf@yc\the\pgf@y}%
  \pgf@x=0.55228475\pgf@xa%
  \pgf@y=0.55228475\pgf@ya%
  \advance\pgf@x by\pgf@xb%
  \advance\pgf@y by\pgf@yb%
  {%
    \advance\pgf@x by\pgf@xc%
    \advance\pgf@y by\pgf@yc%
    \advance\pgf@xb by\pgf@xc%
    \advance\pgf@yb by\pgf@yc%
    \pgf@temp%
    \pgf@nlt@curveto{\pgf@xc}{\pgf@yc}{\pgf@x}{\pgf@y}{\pgf@xb}{\pgf@yb}%
  }%
  \pgf@xa=-\pgf@xa% flip first axis
  \pgf@ya=-\pgf@ya%
  \pgf@x=0.55228475\pgf@xa% second arc
  \pgf@y=0.55228475\pgf@ya%
  \advance\pgf@x by\pgf@xb%
  \advance\pgf@y by\pgf@yb%
  \advance\pgf@x by\pgf@xc%
  \advance\pgf@y by\pgf@yc%
  \edef\pgf@temp{\pgf@xc\the\pgf@x\pgf@yc\the\pgf@y}%
  \pgf@x=0.55228475\pgf@xb%
  \pgf@y=0.55228475\pgf@yb%
  \advance\pgf@x by\pgf@xa%
  \advance\pgf@y by\pgf@ya%
  {%
    \advance\pgf@x by\pgf@xc%
    \advance\pgf@y by\pgf@yc%
    \advance\pgf@xa by\pgf@xc%
    \advance\pgf@ya by\pgf@yc%
    \pgf@temp%
    \pgf@nlt@curveto{\pgf@xc}{\pgf@yc}{\pgf@x}{\pgf@y}{\pgf@xa}{\pgf@ya}%
  }%
  \pgf@xb=-\pgf@xb% flip second axis
  \pgf@yb=-\pgf@yb%
  \pgf@x=0.55228475\pgf@xb% third arc
  \pgf@y=0.55228475\pgf@yb%
  \advance\pgf@x by\pgf@xa%
  \advance\pgf@y by\pgf@ya%
  \advance\pgf@x by\pgf@xc%
  \advance\pgf@y by\pgf@yc%
  \edef\pgf@temp{\pgf@xc\the\pgf@x\pgf@yc\the\pgf@y}%
  \pgf@x=0.55228475\pgf@xa%
  \pgf@y=0.55228475\pgf@ya%
  \advance\pgf@x by\pgf@xb%
  \advance\pgf@y by\pgf@yb%
  {%
    \advance\pgf@x by\pgf@xc%
    \advance\pgf@y by\pgf@yc%
    \advance\pgf@xb by\pgf@xc%
    \advance\pgf@yb by\pgf@yc%
    \pgf@temp%
    \pgf@nlt@curveto{\pgf@xc}{\pgf@yc}{\pgf@x}{\pgf@y}{\pgf@xb}{\pgf@yb}%
  }%
  \pgf@xa=-\pgf@xa% flip first axis once more
  \pgf@ya=-\pgf@ya%
  \pgf@x=0.55228475\pgf@xa% fourth arc
  \pgf@y=0.55228475\pgf@ya%
  \advance\pgf@x by\pgf@xb%
  \advance\pgf@y by\pgf@yb%
  \advance\pgf@x by\pgf@xc%
  \advance\pgf@y by\pgf@yc%
  \edef\pgf@temp{\pgf@xc\the\pgf@x\pgf@yc\the\pgf@y}%
  \pgf@x=0.55228475\pgf@xb%
  \pgf@y=0.55228475\pgf@yb%
  \advance\pgf@x by\pgf@xa%
  \advance\pgf@y by\pgf@ya%
  {%
    \advance\pgf@x by\pgf@xc%
    \advance\pgf@y by\pgf@yc%
    \advance\pgf@xa by\pgf@xc%
    \advance\pgf@ya by\pgf@yc%
    \pgf@temp%
    \pgf@nlt@curveto{\pgf@xc}{\pgf@yc}{\pgf@x}{\pgf@y}{\pgf@xa}{\pgf@ya}%
  }%
  \pgf@nlt@closepath%
  \pgf@nlt@moveto{\pgf@xc}{\pgf@yc}%
}



% Append a circle to the current path
%
% #1 = center
% #2 = radius
%
% Example:
%
% % Append a circle of radius 3cm around the point (1,1)
% \pgfpathcircle{\pgxy(1,1)}{3cm}

\def\pgfpathcircle#1#2{\pgfpathellipse{#1}{\pgfpoint{#2}{0pt}}{\pgfpoint{0pt}{#2}}}




% Append a rectangle to the current path
%
% #1 = lower left corner point of rectangle
% #2 = width and height vector
%
% Example:
%
% % A rectangle with corners (2,2) and (3,3)
% \pgfpathrectangle{\pgfpointxy{2}{2}}{\pgfpointxy{1}{1}}

\def\pgfpathrectangle{%
  \let\pgfrect@next=\pgf@specialrect%
  \ifpgf@pt@identity%
    \ifpgf@arccorners%
    \else%
      \ifx\pgf@nlt@list\pgfutil@empty%
        \let\pgfrect@next=\pgf@normalrect%
      \fi%
    \fi%
  \fi%
  \pgfrect@next%
}

\def\pgf@normalrect#1#2{%
  \pgf@process{#2}%
  \pgf@xa=\pgf@x%
  \pgf@ya=\pgf@y%
  \pgfpointtransformed{#1}%
  \pgfsyssoftpath@rect{\the\pgf@x}{\the\pgf@y}{\the\pgf@xa}{\the\pgf@ya}%
  \pgf@protocolsizes{\pgf@x}{\pgf@y}%
  \advance\pgf@x by\pgf@xa\relax%
  \advance\pgf@y by\pgf@ya\relax%
  \pgf@protocolsizes{\pgf@x}{\pgf@y}%
}

\def\pgf@specialrect#1#2{%
  \pgf@process{#2}%
  \pgf@xa=\pgf@x%
  \pgf@ya=\pgf@y%
  \pgf@process{#1}%
  \pgf@xb=\pgf@x%
  \pgf@yb=\pgf@y%
  \advance\pgf@xa by\pgf@xb%
  \advance\pgf@ya by\pgf@yb%
  \pgfpathmoveto{\pgfqpoint{\pgf@xa}{\pgf@ya}}%
  \pgfpathlineto{\pgfqpoint{\pgf@xb}{\pgf@ya}}%
  \pgfpathlineto{\pgfqpoint{\pgf@xb}{\pgf@yb}}%
  \pgfpathlineto{\pgfqpoint{\pgf@xa}{\pgf@yb}}%
  \pgfpathclose%
  \pgfpathmoveto{\pgfqpoint{\pgf@xb}{\pgf@yb}}%
}

% Append a rectangle to the current path
%
% #1 = one corner of the rectangle
% #2 = opposite corner of the rectangle
%
% Example:
%
% % A rectangle with corners (2,2) and (3,3)
% \pgfpathrectanglecorners{\pgfpointxy{2}{2}}{\pgfpointxy{3}{3}}

\def\pgfpathrectanglecorners#1#2{%
  \pgf@process{#2}%
  \pgf@xc=\pgf@x%
  \pgf@yc=\pgf@y%
  \pgf@process{#1}%
  \advance\pgf@xc by-\pgf@x%
  \advance\pgf@yc by-\pgf@y%
  \pgfpathrectangle{#1}{\pgfqpoint{\pgf@xc}{\pgf@yc}}%
}


% Append a grid to the current path.
%
% #1 = first corner point of grid
% #2 = second corner point of grid
%
% Options:
%
% stepx = x-step dimension (default 1cm)
% stepy = y-step dimension (default 1cm)
% step  = dimension vector
%
% Example:
%
% \pgfsetlinewidth{0.8pt}
% \pgfgrid{\pgfxy(0,0)}{\pgfxy(3,2)}
% \pgfsetlinewidth{0.4pt}
% \pgfgrid[stepx=1cm,stepy=1cm]{\pgfxy(0,0)}{\pgfxy(3,2)}

\pgfkeys{
  /pgf/stepx/.initial=1cm,
  /pgf/stepy/.initial=1cm,
  /pgf/step/.code={\pgf@process{#1}\pgfkeysalso{/pgf/stepx/.expanded=\the\pgf@x,/pgf/stepy/.expanded=\the\pgf@y}},
  /pgf/step/.value required
}

\def\pgfpathgrid{\pgfutil@ifnextchar[{\pgf@pathgrid}{\pgf@pathgrid[]}}
\def\pgf@pathgrid[#1]#2#3{%
  \pgfset{#1}%
  \pgfmathsetlength\pgf@xc{\pgfkeysvalueof{/pgf/stepx}}%
  \pgfmathsetlength\pgf@yc{\pgfkeysvalueof{/pgf/stepy}}%
  \pgf@process{#3}%
  \pgf@xb=\pgf@x%
  \pgf@yb=\pgf@y%
  \pgf@process{#2}%
  \pgf@xa=\pgf@x%
  \pgf@ya=\pgf@y%
  % Swap coordinates if one of them is smaller than the other:
  \ifdim\pgf@xa>\pgf@xb%
    \pgf@x=\pgf@xb%
    \pgf@xb=\pgf@xa%
    \pgf@xa=\pgf@x%
  \fi%
  \ifdim\pgf@ya>\pgf@yb%
    \pgf@y=\pgf@yb%
    \pgf@yb=\pgf@ya%
    \pgf@ya=\pgf@y%
  \fi%
  \ifdim \pgf@yc > .01pt\relax% if to draw horizontal lines
    \c@pgf@counta=\pgf@ya\relax%
    \c@pgf@countb=\pgf@yc\relax%
    \divide\c@pgf@counta by\c@pgf@countb\relax%
    \pgfutil@tempdima=\c@pgf@counta\pgf@yc\relax%
    \ifdim\pgfutil@tempdima<\pgf@ya%
      \advance\pgfutil@tempdima by\pgf@yc%
    \fi%
    \pgfutil@tempdimb\pgf@x
    \pgfutil@loop% horizontal lines
      {%
        \pgf@xa=\pgfutil@tempdimb%
        \pgf@ya=\pgfutil@tempdima%
        \pgf@pos@transform{\pgf@xa}{\pgf@ya}
        \pgf@nlt@moveto{\pgf@xa}{\pgf@ya}%
        \pgf@xa=\pgf@xb%
        \pgf@ya=\pgfutil@tempdima%
        \pgf@pos@transform{\pgf@xa}{\pgf@ya}
        \pgf@nlt@lineto{\pgf@xa}{\pgf@ya}%
      }%
    \advance\pgfutil@tempdima by\pgf@yc%
    \ifdim\pgfutil@tempdima<\pgf@yb%
    \pgfutil@repeat%
    \advance\pgfutil@tempdima by-0.01pt\relax%
    \ifdim\pgfutil@tempdima<\pgf@yb%
      {%
        \pgf@xa=\pgfutil@tempdimb%
        \pgf@ya=\pgfutil@tempdima%
        \pgf@pos@transform{\pgf@xa}{\pgf@ya}
        \pgf@nlt@moveto{\pgf@xa}{\pgf@ya}%
        \pgf@xa=\pgf@xb%
        \pgf@ya=\pgfutil@tempdima%
        \pgf@pos@transform{\pgf@xa}{\pgf@ya}
        \pgf@nlt@lineto{\pgf@xa}{\pgf@ya}%
      }%
    \fi%
  \fi%
  \ifdim \pgf@xc > .01pt\relax% if to draw vertical lines
    \c@pgf@counta=\pgf@xa\relax%
    \c@pgf@countb=\pgf@xc\relax%
    \divide\c@pgf@counta by\c@pgf@countb\relax%
    \pgfutil@tempdimb=\c@pgf@counta\pgf@xc\relax%
    \ifdim\pgfutil@tempdimb<\pgf@xa%
      \advance\pgfutil@tempdimb by\pgf@xc%
    \fi%
    \pgfutil@loop% vertical lines
      {%
        \pgf@xc=\pgfutil@tempdimb%
        \pgf@yc=\pgf@ya%
        \pgf@pos@transform{\pgf@xc}{\pgf@yc}
        \pgf@nlt@moveto{\pgf@xc}{\pgf@yc}%
        \pgf@xc=\pgfutil@tempdimb%
        \pgf@yc=\pgf@yb%
        \pgf@pos@transform{\pgf@xc}{\pgf@yc}
        \pgf@nlt@lineto{\pgf@xc}{\pgf@yc}%
      }%
      \advance\pgfutil@tempdimb by\pgf@xc%
    \ifdim\pgfutil@tempdimb<\pgf@xb%
    \pgfutil@repeat%
    \advance\pgfutil@tempdimb by-0.01pt\relax%
    \ifdim\pgfutil@tempdimb<\pgf@xb%
      {%
        \pgf@xc=\pgfutil@tempdimb%
        \pgf@yc=\pgf@ya%
        \pgf@pos@transform{\pgf@xc}{\pgf@yc}
        \pgf@nlt@moveto{\pgf@xc}{\pgf@yc}%
        \pgf@xc=\pgfutil@tempdimb%
        \pgf@yc=\pgf@yb%
        \pgf@pos@transform{\pgf@xc}{\pgf@yc}
        \pgf@nlt@lineto{\pgf@xc}{\pgf@yc}%
      }%
    \fi%
  \fi%
  \pgf@process{#3}%
  \pgf@pos@transform{\pgf@x}{\pgf@y}%
  \pgf@nlt@moveto{\pgf@x}{\pgf@y}%
}



% Append two half-parabolas to the path
%
% #1 = bend (relative to current point)
% #2 = end point (relative to bend point)
%
% Description:
%
% This command appends a half-parabola that starts at the current point
% and has its bend at #1+current point. Then, a second parabola is
% appended that starts at #1+current point, where it also has its
% minimum/maximum, and ends at #1+current point+#2, which becomes the
% new current point.
%
% By setting #2 = (0,0) you draw only a half parabola that goes from the
% current point to the bend; by setting #1 = (0,0)
% you draw a half parabola that going to current point + #2 and has its
% bend at the current point.
%
% Examples:
%
% % Half-parabola going ``up and right''
% \pgfpathmoveto{\pgfpointorigin}
% \pgfpathparabola{\pgfpointorigin}{\pgfpoint{2cm}{4cm}}
%
% % Half-parabola going ``down and right''
% \pgfpathmoveto{\pgfpointorigin}
% \pgfpathparabola{\pgfpoint{-2cm}{4cm}}}{\pgfpointorigin}
%
% % Full parabola
% \pgfpathmoveto{\pgfpointorigin}
% \pgfpathparabola{\pgfpoint{-2cm}{4cm}}{\pgfpoint{2cm}{4cm}}

\def\pgfpathparabola#1#2{%
  {%
    \pgf@process{#2}% untransformed
    \pgf@xb=\pgf@x%
    \pgf@yb=\pgf@y%
    \pgf@process{#1}% untransformed
    \pgf@xc=\pgf@x%
    \pgf@yc=\pgf@y%
    \pgfutil@tempswatrue%
    \ifdim\pgf@xb=0pt\relax%
      \ifdim\pgf@yb=0pt\relax%
        \pgfutil@tempswafalse%
      \fi%
    \fi%
    {%
      \ifpgfutil@tempswa%
        \pgf@arccornersfalse
      \else%
      \fi%
      \pgfutil@tempswatrue%
      \ifdim\pgf@xc=0pt\relax%
        \ifdim\pgf@yc=0pt\relax%
          \pgfutil@tempswafalse%
        \fi%
      \fi%
      \ifpgfutil@tempswa
      {%
        \pgf@pt@x=\pgf@path@lastx%
        \pgf@pt@y=\pgf@path@lasty%
        \pgfpathcurveto%
          {\pgfqpoint{.1125\pgf@xc}{.225\pgf@yc}}% found by trial and error
          {\pgfqpoint{.5\pgf@xc}{\pgf@yc}}% found by trial and error
          {\pgfqpoint{\pgf@xc}{\pgf@yc}}%
      }%
      \fi%
    }%
    \ifpgfutil@tempswa%
      \pgf@xc=\pgf@xb%
      \pgf@yc=\pgf@yb%
      {%
        \pgf@pt@x=\pgf@path@lastx%
        \pgf@pt@y=\pgf@path@lasty%
        \pgfpathcurveto%
          {\pgfqpoint{.5\pgf@xc}{0\pgf@yc}}% found by trial and error
          {\pgfqpoint{.8875\pgf@xc}{.775\pgf@yc}}% found by trial and error
          {\pgfqpoint{\pgf@xc}{\pgf@yc}}%
      }%
    \fi%
  }%
}




% Append a sine curve between 0 and \pi/2 to the path.
%
% #1 = vector, describing the width and height of the curve
%
% Description:
%
% This command appends a sine curve in the interval 0 and \pi/2 to the
% current path. The sine curve ends at currentpoint+#1.
%
% Examples:
%
% % One complete sine in the interval [0,\pi]
% \pgfpathmoveto{\pgfpointorigin}
% \pgfpathsine{\pgfpoint{1.57cm}{1cm}}
% \pgfpathcosine{\pgfpoint{3.141cm}{0cm}}

\def\pgfpathsine#1{%
  {%
    \pgf@process{#1}% untransformed
    \pgf@xc=\pgf@x%
    \pgf@yc=\pgf@y%
    \pgf@pt@x=\pgf@path@lastx% evil trickery to transform to the last point
    \pgf@pt@y=\pgf@path@lasty%
    \pgfpathcurveto%
      {\pgfqpoint{.3260\pgf@xc}{.5120\pgf@yc}}%
      {\pgfqpoint{.6380\pgf@xc}{\pgf@yc}}%
      {\pgfqpoint{\pgf@xc}{\pgf@yc}}%
  }%
}

% Append a cosine curve between 0 and \pi/2 to the path.
%
% #1 = vector, describing the width and height of the curve
%
% Examples:
%
% % One complete sine in the interval [0,\pi]
% \pgfpathmoveto{\pgfpointorigin}
% \pgfpathsine{\pgfpoint{1.57cm}{1cm}}
% \pgfpathcosine{\pgfpoint{3.141cm}{0cm}}

\def\pgfpathcosine#1{%
  {%
    \pgf@process{#1}% untransformed
    \pgf@xc=\pgf@x%
    \pgf@yc=\pgf@y%
    \pgf@pt@x=\pgf@path@lastx% evil trickery to transform to the last point
    \pgf@pt@y=\pgf@path@lasty%
    \pgfpathcurveto%
      {\pgfqpoint{.3620\pgf@xc}{0pt}}%
      {\pgfqpoint{.6740\pgf@xc}{.4880\pgf@yc}}%
      {\pgfqpoint{\pgf@xc}{\pgf@yc}}%
  }%
}



% Draw part of a curve between two specified times s and t.
%
% #1 - a start time s.
% #2 - an end time t.
% #3 - start point of the curve
% #4 - first control
% #5 - second control
% #6 - end point of the curve
%
% There are two versions, \pgfpathcurvebetweentime and
% \pgfpathcurvebetweentimecontinue. The latter does not insert a
% moveto to the first point.
%
\def\pgfpathcurvebetweentime{\pgf@ignoremovetofalse\pgf@@pathcurvebetweentime}
\def\pgfpathcurvebetweentimecontinue{\pgf@ignoremovetotrue\pgf@@pathcurvebetweentime}
\newif\ifpgf@ignoremoveto

\def\pgf@@pathcurvebetweentime#1#2#3#4#5#6{%
  \pgfmathparse{#1}%
  \let\pgf@time@s=\pgfmathresult%
  \pgfmathparse{#2}%
  \let\pgf@time@t=\pgfmathresult%
  \ifdim\pgf@time@s pt>\pgf@time@t pt\relax%
    \pgfmathsetmacro\pgf@time@s{1-#1}%
    \pgfmathsetmacro\pgf@time@t{1-#2}%
    \pgf@@@pathcurvebetweentime{\pgf@time@t}{#6}{#5}{#4}{#3}%
  \else%
    \pgf@@@pathcurvebetweentime{\pgf@time@t}{#3}{#4}{#5}{#6}%
  \fi%
}

\def\pgf@@@pathcurvebetweentime#1#2#3#4#5{%
  % Q1 = P1.
  \pgf@process{#2}%
  \pgf@xc=\pgf@x%
  \pgf@yc=\pgf@y%
  % Q2 = P1 + t*(P2-P1).
  \pgf@process{%
    \pgf@process{#3}%
    \pgf@xa=#1\pgf@x%
    \pgf@ya=#1\pgf@y%
    \pgf@process{#2}%
    \pgf@xb=\pgf@x%
    \pgf@yb=\pgf@y%
    \advance\pgf@x by-#1\pgf@xb%
    \advance\pgf@y by-#1\pgf@yb%
    \advance\pgf@x by\pgf@xa%
    \advance\pgf@y by\pgf@ya%
  }%
  \pgf@xb=\pgf@x%
  \pgf@yb=\pgf@y%
  % Q3 = Q2 + t*((P2 + t*(P3-P2)) - Q2).
  \pgf@process{%
    \pgf@process{#4}%
    \pgf@xa=#1\pgf@x%
    \pgf@ya=#1\pgf@y%
    %
    \pgf@process{#3}%
    \pgf@xc=\pgf@x%
    \pgf@yc=\pgf@y%
    \advance\pgf@xc by-#1\pgf@x%
    \advance\pgf@yc by-#1\pgf@y%
    %
    \pgf@x=\pgf@xb%
    \pgf@y=\pgf@yb%
    \advance\pgf@x by#1\pgf@xa%
    \advance\pgf@y by#1\pgf@ya%
    \advance\pgf@x by-#1\pgf@xb%
    \advance\pgf@y by-#1\pgf@yb%
    \advance\pgf@x by#1\pgf@xc%
    \advance\pgf@y by#1\pgf@yc%
  }%
  \pgf@xa=\pgf@x%
  \pgf@ya=\pgf@y%
  % Q4 = (1-t)^3*P1 + 3*t(1-t)^2*P2 + 3*t^2(1-t)*P3 + t^3*P4.
  \pgf@process{\pgfpointcurveattime{#1}{#2}{#3}{#4}{#5}}%
  \ifx#1\pgf@time@t%
    % First time round...
    \pgfmathdivide@{\pgf@time@s}{\pgf@time@t}%
    \pgfmathadd@{-\pgfmathresult}{1}%
    \let\pgf@time@s=\pgfmathresult%
    \edef\pgf@marshal{%
      \noexpand\pgf@@@pathcurvebetweentime{\noexpand\pgf@time@s}%
        {\noexpand\pgfqpoint{\the\pgf@x}{\the\pgf@y}}{\noexpand\pgfqpoint{\the\pgf@xa}{\the\pgf@ya}}%
        {\noexpand\pgfqpoint{\the\pgf@xb}{\the\pgf@yb}}{\noexpand\pgfqpoint{\the\pgf@xc}{\the\pgf@yc}}%
    }%
  \else%
    % ...second time round.
    \ifpgf@ignoremoveto%
      \edef\pgf@marshal{%
        \noexpand\pgfpathcurveto{\noexpand\pgfqpoint{\the\pgf@xa}{\the\pgf@ya}}%
          {\noexpand\pgfqpoint{\the\pgf@xb}{\the\pgf@yb}}{\noexpand\pgfqpoint{\the\pgf@xc}{\the\pgf@yc}}%
      }%
    \else%
      \edef\pgf@marshal{%
        \noexpand\pgfpathmoveto{\noexpand\pgfqpoint{\the\pgf@x}{\the\pgf@y}}%
        \noexpand\pgfpathcurveto{\noexpand\pgfqpoint{\the\pgf@xa}{\the\pgf@ya}}%
          {\noexpand\pgfqpoint{\the\pgf@xb}{\the\pgf@yb}}{\noexpand\pgfqpoint{\the\pgf@xc}{\the\pgf@yc}}%
      }%
    \fi%
  \fi%
  \pgf@marshal%
}


\endinput
