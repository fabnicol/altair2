%--------------------------------------------
%
% Package pgfplots
%
% Provides a user-friendly interface to create function plots (normal
% plots, semi-logplots and double-logplots).
% 
% It is based on Till Tantau's PGF package.
%
% Copyright 2009 by Nick Papior Andersen.
%
% This program is free software: you can redistribute it and/or modify
% it under the terms of the GNU General Public License as published by
% the Free Software Foundation, either version 3 of the License, or
% (at your option) any later version.
% 
% This program is distributed in the hope that it will be useful,
% but WITHOUT ANY WARRANTY; without even the implied warranty of
% MERCHANTABILITY or FITNESS FOR A PARTICULAR PURPOSE.  See the
% GNU General Public License for more details.
% 
% You should have received a copy of the GNU General Public License
% along with this program.  If not, see <http://www.gnu.org/licenses/>.
%
%--------------------------------------------

\newif\ifpgfplots@group@sharedlabels

\def\pgfplots@group@xticklabels{}
\def\pgfplots@group@yticklabels{}
\def\pgfplots@group@xlabels{}
\def\pgfplots@group@ylabels{}
\def\pgfplots@group@totalplots{1}
\def\pgfplots@group@columns{1}
\def\pgfplots@group@rows{1}
\def\pgfplots@group@name{group}
\def\pgfplots@group@style{}
\gdef\pgfplots@group@rmopts{}
\newcount\pgfplots@group@current@plot
\newcount\pgfplots@group@current@row
\newcount\pgfplots@group@current@column

\usetikzlibrary{calc}

\pgfkeys{
    /pgfplots/group/.is family,
    /pgfplots/group style/.store in=\pgfplots@group@style,
    /pgfplots/group/horizontal sep/.initial=1cm,
    /pgfplots/group/vertical sep/.initial=1cm,
    /pgfplots/group/group size/.style args={#1 by #2}{/pgfplots/group/columns=#1,/pgfplots/group/rows=#2},
    /pgfplots/group/group size/.default={1 by 1},
    /pgfplots/group/columns/.store in=\pgfplots@group@columns,
    /pgfplots/group/rows/.store in=\pgfplots@group@rows,
    /pgfplots/group/every plot/.style=,
    /pgfplots/group/shared labels/.is if=pgfplots@group@sharedlabels,
    /pgfplots/group/shared labels=false,
    /pgfplots/group/xticklabels at/.is choice,
    /pgfplots/group/xticklabels at/all/.code=\def\pgfplots@group@xticklabels{all},
    /pgfplots/group/xticklabels at/edge top/.code=\def\pgfplots@group@xticklabels{top},
    /pgfplots/group/xticklabels at/edge bottom/.code=\def\pgfplots@group@xticklabels{lower},
    /pgfplots/group/xticklabels at/edge left/.code=\def\pgfplots@group@xticklabels{left}, % TODO
    /pgfplots/group/xticklabels at/edge right/.code=\def\pgfplots@group@xticklabels{right}, % TODO
    /pgfplots/group/xticklabels at=all,
    /pgfplots/group/yticklabels at/.is choice,
    /pgfplots/group/yticklabels at/all/.code=\def\pgfplots@group@yticklabels{all},
    /pgfplots/group/yticklabels at/edge left/.code=\def\pgfplots@group@yticklabels{left},
    /pgfplots/group/yticklabels at/edge right/.code=\def\pgfplots@group@yticklabels{right},
    /pgfplots/group/yticklabels at/edge top/.code=\def\pgfplots@group@yticklabels{top}, % TODO
    /pgfplots/group/yticklabels at/edge bottom/.code=\def\pgfplots@group@yticklabels{bottom}, % TODO
    /pgfplots/group/yticklabels at=all,
    /pgfplots/group/xlabels at/.is choice,
    /pgfplots/group/xlabels at/all/.code=\def\pgfplots@group@xlabels{all},
    /pgfplots/group/xlabels at/edge top/.code=\def\pgfplots@group@xlabels{top},
    /pgfplots/group/xlabels at/edge bottom/.code=\def\pgfplots@group@xlabels{lower},
    /pgfplots/group/xlabels at/edge right/.code=\def\pgfplots@group@xlabels{right}, % TODO
    /pgfplots/group/xlabels at/edge left/.code=\def\pgfplots@group@xlabels{left},% TODO
    /pgfplots/group/xlabels at=all,
    /pgfplots/group/ylabels at/.is choice,
    /pgfplots/group/ylabels at/all/.code=\def\pgfplots@group@ylabels{all},
    /pgfplots/group/ylabels at/edge left/.code=\def\pgfplots@group@ylabels{left},
    /pgfplots/group/ylabels at/edge right/.code=\def\pgfplots@group@ylabels{right},
    /pgfplots/group/ylabels at/edge top/.code=\def\pgfplots@group@ylabels{top}, % TODO
    /pgfplots/group/ylabels at/edge bottom/.code=\def\pgfplots@group@ylabels{bottom}, % TODO
    /pgfplots/group/ylabels at=all,
    /pgfplots/group/x descriptions at/.is choice,
    /pgfplots/group/x descriptions at/all/.style={
        /pgfplots/group/xlabels at=all,
        /pgfplots/group/xticklabels at=all
    },
    /pgfplots/group/x descriptions at/edge bottom/.style={
        /pgfplots/group/xlabels at=edge bottom,
        /pgfplots/group/xticklabels at=edge bottom
    },
    /pgfplots/group/x descriptions at/edge top/.style={
        /pgfplots/group/xlabels at=edge top,
        /pgfplots/group/xticklabels at=edge top
    },
    /pgfplots/group/y descriptions at/.is choice,
    /pgfplots/group/y descriptions at/all/.style={
        /pgfplots/group/ylabels at=all,
        /pgfplots/group/yticklabels at=all
    },
    /pgfplots/group/y descriptions at/edge left/.style={
        /pgfplots/group/ylabels at=edge left,
        /pgfplots/group/yticklabels at=edge left
    },
    /pgfplots/group/y descriptions at/edge right/.style={
        /pgfplots/group/ylabels at=edge right,
	/pgfplots/group/yticklabels at=edge right
    },
    /pgfplots/group/group name/.store in=\pgfplots@group@name,
    /pgfplots/group/empty plot/.style={/pgfplots/hide axis=true},
	/pgfplots/trim axis group left/.code={%
		\pgfutil@ifundefined{pgfsettrimleft}{%
			\pgfplots@error{Sorry, the 'trim axis group left' requires a more recent PGF version.}%
		}{%
			\pgfkeysalso{/tikz/trim left={(group c1r1.south west)}}%
		}%
	},%
	/tikz/trim axis group left/.style={/pgfplots/trim axis group left},
	/pgfplots/trim axis group right/.code={%
		\pgfutil@ifundefined{pgfsettrimleft}{%
			\pgfplots@error{Sorry, the 'trim axis group right' requires a more recent PGF version.}%
		}{%
			\pgfkeysalso{/pgfplots/trim axis right}%
		}%
	},%
	/tikz/trim axis group right/.style={/pgfplots/trim axis group right},
}
\pgfkeys{
	/pgfplots/warning/groupplots/too many plots/.code 2 args={%
		\pgfplotsthrow@warning{You are adding more plots than you specified, current plots=#1\space vs. total allowed
          plots=#2.}%
	},
}

\def\nextgroupplot
}

%
% The actual addition of the new plots
%
\def\pgfplots@group@nextplot[#1]{%
    % Check if it legal to produce another plot
    \ifnum\pgfplots@group@current@plot=\pgfplots@group@totalplots\relax
      \pgfplotswarning{groupplots/too many plots}{\the\pgfplots@group@current@plot}{\pgfplots@group@totalplots}\pgfeov
    \else
      % 
      % If there has been other plots before then end the plot
      % 
      \ifnum0<\pgfplots@group@current@plot\relax
        % DEBUG: \message{DEBUG: Ending plot}
        \endpgfplots@environment@opt
      \fi
      % Prepare for next plot, increment plot, row and column
      \pgfplots@group@increment@numbers
      % DEBUG: \message{DEBUG: Cur. Plot: \the\pgfplots@group@current@plot. Current Row: \the\pgfplots@group@current@row. Current Column: \the\pgfplots@group@current@column.}    
      % Creation of the new axis environment
      \pgfplots@group@environment@create{#1}
    \fi
}
%
% This creates an environment according to which plots that have been typeset.
% The code utilizes previous plots anchor points to position themselves, unless it's the 
% first plot.
% Using the figure below to describe the code implementation:
% ____    ____   ____
%|        | |        | |        |
%|   1   | |   2   | |   3   |
%|___| |___| |___|
% ____    ____   ____
%|        | |        | |        |
%|   4   | |   5   | |   6   |
%|___| |___| |___|
% ____    ____   ____
%|        | |        | |        |
%|   7   | |   8   | |   9   |
%|___| |___| |___|
% 
% The plots anchors are as follows: 
%    1: plot is typeset normally
%    2: the "1" plot anchor "east" is anchored to "west"
%    3: the "2" plot anchor "east" is anchored to "west"
%    4: the "1" plot anchor "south" is anchored to "north"
%    5: the "2" plot anchor "south" is anchored to "north"
%    6: the "3" plot anchor "south" is anchored to "north"
%    7: the "4" plot anchor "south" is anchored to "north"
%    8: the "5" plot anchor "south" is anchored to "north"
%    9: the "6" plot anchor "south" is anchored to "north"
%
\def\pgfplots@group@environment@create#1{%
    \ifnum1=\pgfplots@group@current@row\relax
      \ifnum1=\pgfplots@group@current@column\relax
        \pgfplots@environment@opt[name={\pgfplots@group@name\space c\the\pgfplots@group@current@column r\the\pgfplots@group@current@row},%
          /pgfplots/group/every plot,%
          /pgfplots/group/plot c\the\pgfplots@group@current@column r\the\pgfplots@group@current@row,%
          #1]
      \else
        \pgfplotsgroupplotsmath@decrease\pgfplots@group@current@column
        \xdef\pgfplots@glob@TMPb{\pgfplots@group@name\space c\pgfmathresult r\the\pgfplots@group@current@row}
        \pgfplots@environment@opt[name={\pgfplots@group@name\space c\the\pgfplots@group@current@column r\the\pgfplots@group@current@row},%
          /pgfplots/group/every plot,%
          /pgfplots/group/plot c\the\pgfplots@group@current@column r\the\pgfplots@group@current@row,%
          at={($(\pgfplots@glob@TMPb.east)+(\pgfkeysvalueof{/pgfplots/group/horizontal sep},0)$)},anchor=west,%
          #1]
      \fi
    \else
      \pgfplotsgroupplotsmath@decrease\pgfplots@group@current@row
      \xdef\pgfplots@glob@TMPb{\pgfplots@group@name\space c\the\pgfplots@group@current@column r\pgfmathresult}
      \pgfplots@environment@opt[name={\pgfplots@group@name\space c\the\pgfplots@group@current@column r\the\pgfplots@group@current@row},%
        /pgfplots/group/every plot,%
        /pgfplots/group/plot c\the\pgfplots@group@current@column r\the\pgfplots@group@current@row,%
        at={($(\pgfplots@glob@TMPb.south)-(0,\pgfkeysvalueof{/pgfplots/group/vertical sep})$)},anchor=north,%
        #1]%
    \fi%
}

%
% Determine the labels of both x and y.
%
\def\pgfplots@group@determine@labels{%
    \def\pgfplots@loc@TMPa{all}%
    \def\pgfplots@loc@TMPb{lower}%
    \ifx\pgfplots@loc@TMPa\pgfplots@group@xlabels%
      % xlabels at=all
      % Do nothing as all should have xlabels
    \fi
    \gdef\pgfplots@glob@TMPa{}%
    \ifx\pgfplots@loc@TMPb\pgfplots@group@xlabels%
	  % xlabels at=lower
      \ifnum1<\pgfplots@group@rows\relax
        \foreach \pgfplots@column in {1,...,\pgfplots@group@columns} {%
            \foreach \pgfplots@row in {1,...,\pgfplots@group@rows} {%
                \ifnum\pgfplots@row=\pgfplots@group@rows\relax%
                \else
                \expandafter\xdef\expandafter\pgfplots@glob@TMPa\expandafter{%
                    \pgfplots@glob@TMPa,/pgfplots/group/plot c\pgfplots@column r\pgfplots@row/.append style={%
						/pgfplots/xlabel={},%
						\ifpgfplots@units@use/pgfplots/x unit={}\fi
					}%
				}%
                \fi
            }%
        }%
      \fi%
	\else
		\def\pgfplots@loc@TMPb{top}%
		\ifx\pgfplots@loc@TMPb\pgfplots@group@xlabels%
			% xlabels at=top
			\ifnum1<\pgfplots@group@rows\relax
				\foreach \pgfplots@column in {1,...,\pgfplots@group@columns} {%
					\foreach \pgfplots@row in {1,...,\pgfplots@group@rows} {%
						\ifnum\pgfplots@row=1 %
						\expandafter\xdef\expandafter\pgfplots@glob@TMPa\expandafter{%
							\pgfplots@glob@TMPa,/pgfplots/group/plot c\pgfplots@column r\pgfplots@row/.append style={/pgfplots/xticklabel pos=upper}}%
						\else
						\expandafter\xdef\expandafter\pgfplots@glob@TMPa\expandafter{%
							\pgfplots@glob@TMPa,/pgfplots/group/plot c\pgfplots@column r\pgfplots@row/.append style={%
								/pgfplots/xlabel={},%
								\ifpgfplots@units@use/pgfplots/x unit={}\fi%
							}%
						}%
						\fi
					}%
				}%
			\fi%
		\fi
    \fi
    % 
    % Defining styles for the y's
    % 
    \ifx\pgfplots@loc@TMPa\pgfplots@group@ylabels%
	  % ylabels at=all
      % Do nothing as all should have ylabels
	\else
		\def\pgfplots@loc@TMPb{left}%
		\ifx\pgfplots@loc@TMPb\pgfplots@group@ylabels%
			% ylabels at=left
			\ifnum1<\pgfplots@group@columns\relax
				\foreach \pgfplots@row in {1,...,\pgfplots@group@rows} {%
					\foreach \pgfplots@column in {2,...,\pgfplots@group@columns} {%
						\expandafter\xdef\expandafter\pgfplots@glob@TMPa\expandafter{%
							\pgfplots@glob@TMPa,/pgfplots/group/plot c\pgfplots@column r\pgfplots@row/.append style={%
								/pgfplots/ylabel={},%
								\ifpgfplots@units@use/pgfplots/y unit={}\fi%
							}%
						}%
					}%
				}%
			\fi%
		\else
			\def\pgfplots@loc@TMPb{right}%
			\ifx\pgfplots@loc@TMPb\pgfplots@group@ylabels%
				% ylabels at=right
				\ifnum1<\pgfplots@group@columns\relax
					\foreach \pgfplots@row in {1,...,\pgfplots@group@rows} {%
						\foreach \pgfplots@column in {1,...,\pgfplots@group@columns} {%
							\ifnum\pgfplots@column=\pgfplots@group@columns\relax
							\expandafter\xdef\expandafter\pgfplots@glob@TMPa\expandafter{%
								\pgfplots@glob@TMPa,/pgfplots/group/plot c\pgfplots@column r\pgfplots@row/.append style={/pgfplots/yticklabel pos=upper}}%
							\else
							\expandafter\xdef\expandafter\pgfplots@glob@TMPa\expandafter{%
								\pgfplots@glob@TMPa,/pgfplots/group/plot c\pgfplots@column r\pgfplots@row/.append style={%
									/pgfplots/ylabel={},%
									\ifpgfplots@units@use/pgfplots/y unit={}\fi%
								}%
							}%
							\fi
						}%
					}%
				\fi
			\fi%
		\fi
    \fi
    \expandafter\pgfkeys\expandafter{\pgfplots@glob@TMPa}%
}

%
% Determine the ticklabels of both x and y.
%
\def\pgfplots@group@determine@ticklabels{%
    \def\pgfplots@loc@TMPa{all}%
    \def\pgfplots@loc@TMPb{top}%
    \def\pgfplots@loc@TMPc{lower}%
    \ifx\pgfplots@loc@TMPa\pgfplots@group@xticklabels%
      % Do nothing as all should have xticklabels
    \fi
    \gdef\pgfplots@glob@TMPa{}
    \ifx\pgfplots@loc@TMPb\pgfplots@group@xticklabels%
      \foreach \pgfplots@column in {1,...,\pgfplots@group@columns} {%
          \expandafter\xdef\expandafter\pgfplots@glob@TMPa\expandafter{%
              \pgfplots@glob@TMPa,/pgfplots/group/plot c\pgfplots@column r1/.append style={%
                  xticklabel pos=upper}}%
          \ifnum1<\pgfplots@group@rows\relax
            \foreach \pgfplots@row in {2,...,\pgfplots@group@rows} {%
                \expandafter\xdef\expandafter\pgfplots@glob@TMPa\expandafter{%
                    \pgfplots@glob@TMPa,/pgfplots/group/plot c\pgfplots@column r\pgfplots@row/.append style={%
                        /pgfplots/xticklabels={},/pgfplots/xtick scale label code/.code={}}}%
            }%
          \fi%
      }%
    \fi
    \ifx\pgfplots@loc@TMPc\pgfplots@group@xticklabels%
      \foreach \pgfplots@column in {1,...,\pgfplots@group@columns} {%
          \expandafter\xdef\expandafter\pgfplots@glob@TMPa\expandafter{%
              \pgfplots@glob@TMPa,/pgfplots/group/plot c\pgfplots@column r\pgfplots@group@rows/.append style={%
                  xticklabel pos=lower}}%
          \ifnum1<\pgfplots@group@rows\relax
            \pgfplotsgroupplotsmath@decrease\pgfplots@group@rows
            \foreach \pgfplots@row in {1,...,\pgfmathresult} {%
                \expandafter\xdef\expandafter\pgfplots@glob@TMPa\expandafter{%
                    \pgfplots@glob@TMPa,/pgfplots/group/plot c\pgfplots@column r\pgfplots@row/.append style={%
                        /pgfplots/xticklabels={},/pgfplots/xtick scale label code/.code={}}}%
            }%
          \fi%
      }%
    \fi%
    % 
    % Defining styles for the ytick's
    % 
    \def\pgfplots@loc@TMPa{all}%
    \def\pgfplots@loc@TMPb{left}%
    \def\pgfplots@loc@TMPc{right}%
    \ifx\pgfplots@loc@TMPa\pgfplots@group@yticklabels%
      % Do nothing as all should have yticklabels
    \fi
    \ifx\pgfplots@loc@TMPb\pgfplots@group@yticklabels%
      \foreach \pgfplots@row in {1,...,\pgfplots@group@rows} {%
          \expandafter\xdef\expandafter\pgfplots@glob@TMPa\expandafter{%
              \pgfplots@glob@TMPa,/pgfplots/group/plot c1r\pgfplots@row/.append style={yticklabel pos=left}}%
          \ifnum1<\pgfplots@group@columns\relax
            \foreach \pgfplots@column in {2,...,\pgfplots@group@columns} {%
                \expandafter\xdef\expandafter\pgfplots@glob@TMPa\expandafter{%
                    \pgfplots@glob@TMPa,/pgfplots/group/plot c\pgfplots@column r\pgfplots@row/.append style={%
                        /pgfplots/yticklabels={}, /pgfplots/ytick scale label code/.code={}}}%
            }%
          \fi%
      }%
    \fi
    \ifx\pgfplots@loc@TMPc\pgfplots@group@yticklabels%
      \foreach \pgfplots@row in {1,...,\pgfplots@group@rows} {%
          \expandafter\xdef\expandafter\pgfplots@glob@TMPa\expandafter{%
              \pgfplots@glob@TMPa,/pgfplots/group/plot c\pgfplots@group@columns r\pgfplots@row/.append style={%
                  yticklabel pos=right}}%
          \ifnum1<\pgfplots@group@columns\relax
            \pgfplotsgroupplotsmath@decrease\pgfplots@group@columns
            \foreach \pgfplots@column in {1,...,\pgfmathresult} {%
                \expandafter\xdef\expandafter\pgfplots@glob@TMPa\expandafter{%
                    \pgfplots@glob@TMPa,/pgfplots/group/plot c\pgfplots@column r\pgfplots@row/.append style={%
                        /pgfplots/yticklabels={}, /pgfplots/ytick scale label code/.code={}}}%
            }%
          \fi%
      }%
    \fi%
    \expandafter\pgfkeys\expandafter{\pgfplots@glob@TMPa}%
}

%
% The counters which are used to determine the current plot:
%    \pgfplots@group@current@plot
%    \pgfplots@group@current@column
%    \pgfplots@group@current@row
% gets incremented according to the scheme below:
% ____    ____   ____
%|        | |        | |        |
%|   1   | |   2   | |   3   |
%|___| |___| |___|
% ____    ____   ____
%|        | |        | |        |
%|   4   | |   5   | |   6   |
%|___| |___| |___|
% ____    ____   ____
%|        | |        | |        |
%|   7   | |   8   | |   9   |
%|___| |___| |___|
% where the number matches the plot number.
%
\def\pgfplots@group@increment@numbers{%
    \advance\pgfplots@group@current@plot by1
    \ifnum\pgfplots@group@current@column=\pgfplots@group@columns\relax
      \ifnum1=\pgfplots@group@current@plot\relax
      \else
        \advance\pgfplots@group@current@row by1
      \fi
      \pgfplots@group@current@column=1
    \else
      \advance\pgfplots@group@current@column by1
    \fi
}

\def\pgfplots@group@get@plot@name#1,#2\in#3{%
    \edef#3{c#1r#2}%
}


%
% Setting up all possible styles in current plot
% I.e. Setting:
% /pgfplots/group/plot c1r1/.style=
% /pgfplots/group/plot c2r1/.style=
% ...
% /pgfplots/group/plot c<columns>r<rows>/.style=
%
\def\pgfplots@group@individual@setup{%
    \gdef\pgfplots@glob@TMPa{}
    \foreach \pgfplots@column in {1,...,\pgfplots@group@columns} {%
        \foreach \pgfplots@row in {1,...,\pgfplots@group@rows} {%
            \expandafter\xdef\expandafter\pgfplots@glob@TMPa\expandafter{%
                \pgfplots@glob@TMPa,/pgfplots/group/plot c\pgfplots@column r\pgfplots@row/.style=}%
        }%
    }%
    \expandafter\pgfkeys\expandafter{\pgfplots@glob@TMPa}%
}



%
% Implementation of the environment {groupplots}
\def\pgfplots@environment@groupplot{%
	\pgfutil@ifnextchar[{\pgfplots@@environment@groupplot}{\pgfplots@@environment@groupplot[\pgfutil@empty]}%
}

\def\endpgfplots@environment@groupplot{%
	\endpgfplots@environment@opt
	%
	% transport some items beyond this \endgroup:
	\pgf@remember@layerlist@globally
	\aftergroup\pgf@restore@layerlist@from@global
	\endgroup
	%
	% ... and transport them beyond the next \endgroup as well (LaTeX
	% introduces one for \end{groupplot}):
	\aftergroup\pgf@restore@layerlist@from@global
	\aftergroup\pgfplots@cell@picture@remembered
}

\def\pgfplots@@environment@groupplot[#1]{%
    \begingroup%
    % Only find the keys column and row
    % the following things appear to be complete... ?
    \pgfkeysinstallkeyfilterhandler{/pgf/key filter handlers/append filtered to}{\pgfplots@group@rmopts}%
    \let\pgfplots@group@rmopts=\pgfutil@empty
    \pgfkeysinstallkeyfilter{/pgf/key filters/equals}{/pgfplots/group style}%
    \pgfkeysinstallkeyfilter{/pgf/key filters/is descendant of}{/pgfplots/group}%
    \pgfqkeysfiltered{/pgfplots}{#1}%
    \edef\pgfplots@loc@TMPa{%
        \noexpand\pgfkeys{/pgfplots/group/.cd,\pgfplots@group@style}%
    }\pgfplots@loc@TMPa
	%
	\let\pgfplots@set@options@sanitizemode@old=\pgfplots@set@options@sanitizemode
	\let\pgfplots@set@options@sanitize@old=\pgfplots@set@options@sanitize
	\let\pgfplots@set@options@sanitize=\relax
	\let\pgfplots@set@options@sanitizemode=\relax
    \expandafter\pgfplots@set@options\expandafter{\pgfplots@group@rmopts}%
	\let\pgfplots@set@options@sanitize=\pgfplots@set@options@sanitize@old
	\let\pgfplots@set@options@sanitizemode=\pgfplots@set@options@sanitizemode@old
	%
    \pgfplots@group@individual@setup
    \pgfplots@group@determine@ticklabels
    \pgfplots@group@determine@labels
    \pgfplots@group@current@plot=0 %
    \pgfplots@group@current@row=1 %
    \pgfplots@group@current@column=0 %
    \begingroup
      \c@pgf@counta=\pgfplots@group@rows\relax
      \multiply \c@pgf@counta by\pgfplots@group@columns\relax
      \edef\pgfmathresult{\the\c@pgf@counta}%
      \pgfmath@smuggleone\pgfmathresult
    \endgroup
    \edef\pgfplots@group@totalplots{\pgfmathresult}%
    %DEBUG: \message{TEST: \pgfplots@group@totalplots, rows=\pgfplots@group@rows, columns=\pgfplots@group@columns}%
}

% Assigns \pgfmathresult as #1-1 (in integer arithmetics)
%
% #1 an integer (a register or something which expands to an integer).
\def\pgfplotsgroupplotsmath@decrease#1{%
    \begingroup
      \c@pgf@counta=#1\relax
      \advance\c@pgf@counta by-1
      \edef\pgfmathresult{\the\c@pgf@counta}%
      \pgfmath@smuggleone\pgfmathresult
    \endgroup
}

\tikzaddtikzonlycommandshortcutlet\groupplot\pgfplots@environment@groupplot
\tikzaddtikzonlycommandshortcutlet\endgroupplot\endpgfplots@environment@groupplot

\endinput
