%--------------------------------------------
%
% Package pgfplots
%
% Provides a user-friendly interface to create function plots (normal
% plots, semi-logplots and double-logplots).
%
% It is based on Till Tantau's PGF package.
%
% Copyright 2007-2015 by Christian Feuersänger.
%
% This program is free software: you can redistribute it and/or modify
% it under the terms of the GNU General Public License as published by
% the Free Software Foundation, either version 3 of the License, or
% (at your option) any later version.
%
% This program is distributed in the hope that it will be useful,
% but WITHOUT ANY WARRANTY; without even the implied warranty of
% MERCHANTABILITY or FITNESS FOR A PARTICULAR PURPOSE.  See the
% GNU General Public License for more details.
%
% You should have received a copy of the GNU General Public License
% along with this program.  If not, see <http://www.gnu.org/licenses/>.
%
%--------------------------------------------

% The low-level tikz plot handler for 3D mesh and surface plots.
%
% It expects
% a) a coordinate stream which is a linearized matrix, iterated either
% rowwise or column wise,
% b) the current set of options in /pgfplots/mesh/.
%
%
% ATTENTION:
% 	this handler is NOT as general as the others. In particular, it
% 	can't be interrupted by TeX groups: every single statement,
% 	starting with streamstart and ending with stream end, must be in
% 	the same scope!
%
% 	As a consequence, you can't use this handler inside of (all) TikZ
% 	input streams. Try it out. It will work in pgfplots, however.
%
% Furthermore, it relies on a couple of pgfplots macros. This may
% change in future versions.
% It relies on:
% 	- \ifpgfplots@curplot@threedim
% 	- \pgfplots@current@point@meta
% 		-> it also invokes \pgfplots@perpointmeta@preparetrafo and its
% 		@APPLY counterpart which checks for
% 	- \pgfplots@metamax
% 		If you use it outside of pgfplots, use
% 		\global\let\pgfplots@metamax=\pgfutil@empty
% 		This is can be used without an initialised
% 		pgfplots axes - as long as \pgfplots@current@point@meta is in
% 		the range [0,1000].
% 	- for the case 'shader=interp', a bounding box must be established
% 	before the plot is finished. This must be done using
%	\def\pgfplotspointbbupperright{\pgfqpointxyz{2}{2}{0.9}}
%	\def\pgfplotspointbblowerleft {\pgfqpointxyz{0}{0}{0.1}}
%	or something like that.
%
% Besides these internals, it also relies on all public /pgfplots
% configuration keys related to mesh/surface plots.
%
% Please note that the 'z buffer' feature will not work without
% pgfplots (especially the z buffer=sort feature).
%
% Here is an example which compiles without a pgfplots axes:
% ----------------------------------------------------------
% \begin{tikzpicture}
% 	\scope[
% 		/pgfplots/.cd,
% 		surf,
% 		shader=faceted,
% 		mesh/rows=4,
% 		mesh/cols=4,
% 		mesh/num points=,
% 		/tikz/x={(0.44237cm,-0.07439cm)},
% 		/tikz/y={(0.30942cm,0.23932cm)},
% 		/tikz/z={(0.0cm,1.5cm)},
% 	]
% \makeatletter
%
% 	\global\let\pgfplots@metamax=\pgfutil@empty
% 	\pgfplots@curplot@threedimtrue
%
% 	\pgfplotsplothandlermesh
% 	\pgfplotstreamstart
%
% 	\def\rangea{0.21}%
% 	\def\rangeb{0.9}%
%
% 	\pgfmathparse{1000/(\rangeb-\rangea)}
% 	\let\factor=\pgfmathresult
%
% 	\def\simplecoordinate(#1,#2,#3){%
% 		\pgfmathparse{\factor*(#3 - \rangea)}%
% 		\let\pgfplots@current@point@meta=\pgfmathresult
% 		\pgfplotstreampoint{\pgfqpointxyz{#1}{#2}{#3}}%
% 	}%
% 	% for the case 'shaer=interp':
%	\def\pgfplotspointbbupperright{\pgfqpointxyz{2}{2}{0.9}}
%	\def\pgfplotspointbblowerleft {\pgfqpointxyz{0}{0}{0.1}}
%
% \simplecoordinate(0,3,0.7)
% \simplecoordinate(1,3,0.5)
% \simplecoordinate(2,3,0.58)
% \simplecoordinate(3,3,0.9)
%
% \simplecoordinate(0,2,0.68)
% \simplecoordinate(1,2,0.22)
% \simplecoordinate(2,2,0.25)
% \simplecoordinate(3,2,0.4)
%
% \simplecoordinate(0,1,0.6)
% \simplecoordinate(1,1,0.3)
% \simplecoordinate(2,1,0.21)
% \simplecoordinate(3,1,0.3)
%
% \simplecoordinate(0,0,0.8)
% \simplecoordinate(1,0,0.56)
% \simplecoordinate(2,0,0.5)
% \simplecoordinate(3,0,0.75)
%
%
% 	\pgfplotstreamend
% 	\pgfusepath{stroke}
%
% 	\endscope
% \end{tikzpicture}
% ----------------------------------------------------------
%  END OF EXAMPLE
%  -> see also the legend image for mesh plots.
%

\input pgfplotsmeshplotimage.code.tex

\def\pgfplotsplothandlermesh{%
	\def\pgfplotsplothandlername{mesh}%
	%
	% acquire values as they are known right now. Note that these values may change during the survey phase, so they will be updated later.
	\pgfkeysgetvalue{/pgfplots/mesh/rows}\pgfplotsplothandlermesh@rows
	\pgfkeysgetvalue{/pgfplots/mesh/cols}\pgfplotsplothandlermesh@cols
	\pgfkeysgetvalue{/pgfplots/mesh/num points}\pgfplotsplothandlermesh@numpoints
	\edef\pgfplotsplothandlermesh@numpoints{\pgfplotsplothandlermesh@numpoints}%
	\edef\pgfplotsplothandlermesh@cols{\pgfplotsplothandlermesh@cols}%
	\edef\pgfplotsplothandlermesh@rows{\pgfplotsplothandlermesh@rows}%
	%
	\pgfkeysgetvalue{/pgfplots/patch type}\pgfplotsplothandlermesh@patchclass
	\ifx\pgfplotsplothandlermesh@patchclass\pgfplotsplothandlermesh@patchclass@defaulttext
		\let\pgfplotsplothandlermesh@patchclass=\pgfutil@empty
	\fi
	\ifx\pgfplotsplothandlermesh@patchclass\pgfutil@empty
		% assign default patch class.
		\if0\pgfplotsplothandlermesh@matrixinput
			\def\pgfplotsplothandlermesh@patchclass{triangle}%
		\else
			\def\pgfplotsplothandlermesh@patchclass{rectangle}%
		\fi
		\pgfkeyslet{/pgfplots/patch type}\pgfplotsplothandlermesh@patchclass
	\fi
	%
	\if0\pgfplotsplothandlermesh@matrixinput
		% mesh input=patches: simply skip empty input lines:
		\pgfkeyssetvalue{/pgfplots/empty line}{none}%
	\fi
	%
	\let\pgfplotsplothandlersurveystart=\pgfplotsplothandlersurveystart@mesh
	\let\pgfplotsplothandlersurveypoint=\pgfplotsplothandlersurveypoint@mesh
	\let\pgfplotsplothandlersurveyend=\pgfplotsplothandlersurveyend@mesh
	\let\pgfplotsplothandlersurveyaftersetpointmeta=\pgfplotsplothandlersurveyaftersetpointmeta@mesh@cdatanormalize
	\let\pgfplotsplothandlernotifyscanlinecomplete=\pgfplotsplothandlersurvey@mesh@scanlinecomplete
	%
	\def\pgfplotsplothandlerLUAfactory{pgfplots.MeshPlothandler.new}%
	\if2\pgfplotsplothandlermesh@matrixinput
		% mesh input=image
		% this is currently unsupported in LUA:
		\let\pgfplotsplothandlerLUAfactory=\pgfutil@empty
	\fi
	%
	\pgfkeysgetvalue{/pgfplots/patch table/value}\pgfplotsplothandlermesh@patchtable
	\ifx\pgfplotsplothandlermesh@patchtable\pgfutil@empty
	\else
		\def\pgfplotsplothandlermesh@matrixinput{0}% implies 'patch input=patches'
		\let\pgfplotsplothandlersurveystart=\pgfplotsplothandlersurveystart@mesh@patchtable
		\let\pgfplotsplothandlersurveypoint=\pgfplotsplothandlersurveypoint@mesh@patchtable
		\let\pgfplotsplothandlersurveyend=\pgfplotsplothandlersurveyend@mesh@patchtable
		%
		% this is currently unsupported in LUA:
		\let\pgfplotsplothandlerLUAfactory=\pgfutil@empty
	\fi
	\if0\pgfplotsplothandlermesh@colorinput
		% mesh/color input=colormap
	\else
		% mesh/color input=explicit
		\if0\csname pgfpmeta@\pgfplotspointmetainputhandler @issymbolic\endcsname
			\pgfplotsthrow{invalid argument}{\pgfplots@loc@TMPa}{Sorry, color input!=colormap requires 'point meta=symbolic' or 'explicit symbolic'. Please adjust the format of your input point meta data}\pgfeov%
		\fi
	\fi
	%
	%
	\pgfplotsifisvisualizationphase{%
		\pgfplotsplothandlermesh@init@variables
	}{}%
	%
	%
	\def\pgf@plotstreamstart{%
%		\scope
		%
		% this shouldn't be necessary, but it doesn't hurt either.
		\pgfplots@perpointmeta@preparetrafo
		%
		%
		\pgfplotsplothandlermesh@init@DECODE
		\pgfplotsplothandlermesh@init@REFINE
		%
		\pgfutil@IfUndefined{pgfplots@drawmodes}{%
			\let\pgfplots@drawmodes=\tikz@mode
			\let\pgfplots@drawoptions=\tikz@options
		}{}%
		%
		%
		% Prepare color data source:
		\ifx\pgfplotspointmetainputhandler\pgfutil@empty%
			% oh. There is no color data!?
			\pgfplotsplothandlermesh@init@path@without@point@meta
			%
		\else
			\pgfplotsplothandlermesh@init@cdata
			\pgfplotsplothandlermesh@init@shader
			\pgfplotsplothandlermesh@init@flat@color
		\fi
		% PREPARE THE STREAM PROCESSING:
		% handling of jumps.
		\pgfplotsplothandlermesh@init@jumphandling
		%
		\ifpgfplotsplothandlermesh@shownormals
			\let\pgfplotsplothandlermesh@show@normals@if@configured=\pgfplotsplothandlermesh@show@normals
			% actually store 3d coordinates:
			\let\pgfplotsplothandlermesh@serialize@logical@coords=\pgfplotsplothandlermesh@serialize@logical@coords@doit
		\fi
		%
		% z buffering:
		\pgfplotsplothandlermesh@init@zbuffer
		%
		\pgfplotsplothandlermesh@init@mesh@stream
		%
		\c@pgfplots@scanlineindex=0
	}%
	%
	\let\pgfplotsplothandlerifcurrentpointcanbefirstlast=\pgfplotsplothandlerifcurrentpointcanbefirstlast@mesh
}%

\def\pgfplotsplothandlermesh@init@variables{%
	% Check if the current shader needs to be reset.
	% Do this early as some processing steps depend on it:
	\if\pgfplots@meshmode m% mesh
		% this is processed by the 'shader=flat' code below.
		\def\pgfplotsplothandlermesh@shader{0}% 'flat'
	\fi
	\ifnum\pgfplotsplothandlermesh@shader>1% 'shader=interp' and 'shader=faceted interp'
		\pgflibrarysurfshadingifactive
			{}%
			{%
				\pgfutil@ifundefined{pgfplotsplothandlermesh@shader@warning}{%
					\pgfplots@error{Sorry, surface shading (shader=interp) is NOT available for the selected driver `\pgfsysdriver'. I will use 'shader=flat' if you continue now}%
					\gdef\pgfplotsplothandlermesh@shader@warning{x}%
				}{}%
				\def\pgfplotsplothandlermesh@shader{0}%
			}%
	\fi
	%
	\def\pgfplotsplothandlermesh@ONEDIMMODE{0}%
	\pgfplotsplothandlermesh@init@meshkeys
	%
	%
	\let\pgfplotsplothandlermesh@patchclass@input=\pgfplotsplothandlermesh@patchclass
	\let\pgfplotsplothandlermesh@patchclass@output=\pgfplotsplothandlermesh@patchclass@input
	\if0\pgfplotsplothandlermesh@triangulate
	\else
		\pgfplotspatchclass{\pgfplotsplothandlermesh@patchclass@input}{triangulate class}%
		\let\pgfplotsplothandlermesh@patchclass@output=\pgfplotsretval
	\fi
	%
	\pgfplotspatchclass{\pgfplotsplothandlermesh@patchclass@input}{init}%
	\ifx\pgfplotsplothandlermesh@patchclass@input\pgfplotsplothandlermesh@patchclass@output
	\else
		\pgfplotspatchclass{\pgfplotsplothandlermesh@patchclass@output}{init}%
	\fi
	%
	% assign boolean:
	\ifcase\pgfplotsplothandlermesh@matrixinput
		% mesh input=patches
		\def\b@pgfplotsplothandlermesh@matrixoutput{0}%
	\or
		% mesh input=lattice
		\def\b@pgfplotsplothandlermesh@matrixoutput{1}%
	\or
		% mesh input=image
		% while it _could_ potentially provide matrix output, the
		% implementation is quite involved. We don't have it.
		\def\b@pgfplotsplothandlermesh@matrixoutput{0}%
	\fi
	%
	\pgfplotspatchclass{\pgfplotsplothandlermesh@patchclass@output}{allow matrix}%
	\if0\pgfplotsretval
		\def\b@pgfplotsplothandlermesh@matrixoutput{0}%
	\fi
	\ifnum\pgfkeysvalueof{/pgfplots/patch refines}>0
		\def\b@pgfplotsplothandlermesh@matrixoutput{0}%
	\fi
	\if1\pgfplotsaxisplothasjumps
		\def\b@pgfplotsplothandlermesh@matrixoutput{0}%
	\fi
	\if1\pgfplotsaxisplothasunboundedpointmeta
		\def\b@pgfplotsplothandlermesh@matrixoutput{0}%
	\fi
	\if\pgfplotsplothandlermesh@zbuffer@choice4% 'z buffer=sort':
		\def\b@pgfplotsplothandlermesh@matrixoutput{0}%
	\fi
	%
	\if2\pgfplotsplothandlermesh@shader
		% shader=interp
		%
		\pgfkeysgetvalue{/pgfplots/mesh/interior colormap name}\pgfplots@loc@TMPc
		\ifx\pgfplots@loc@TMPc\pgfutil@empty
		\else
			\pgfplotswarning{interior colormap inefficient}\pgfeov%
			\pgfplotsset{shader=faceted interp,faceted color=none}%
		\fi
	\fi
	\if2\pgfplotsplothandlermesh@shader
		% shader=interp
		\if0\pgfplotsplothandlermesh@triangulate
			% patch to triangles=false
		\else
			% no matrix output possible:
			\def\b@pgfplotsplothandlermesh@matrixoutput{0}%
		\fi
	\else
		% 'shader!=interp'
		\def\b@pgfplotsplothandlermesh@matrixoutput{0}%
	\fi
	\pgfplotspatchclass{\pgfplotsplothandlermesh@patchclass@output}{get pdf shading type}%
	\if5\pgfplotsretval
	\else
		\def\b@pgfplotsplothandlermesh@matrixoutput{0}%
	\fi
	%
	% special handling: check if Type=5 shadings need to be
	% converted to Type=4
	\if0\b@pgfplotsplothandlermesh@matrixoutput
		% no matrix output possible
		\if5\pgfplotsretval
			% pdf shading type=5 (matrix output)
			\if2\pgfplotsplothandlermesh@shader
				% shader=interp
				%
				% FIXME : I am unsure if this here is still
				% necessary as of 3fc734d7ac3020600c06258b1bc80b22da53edf3
				% -> perhaps the check can be skipped entirely.
				%
				% oh. shading type=5 is *essentially* the same as
				% matrix output, but matrix output is forbidden.
				% In that case, triangulate the matrix and shade the
				% triangles:
				\def\pgfplotsplothandlermesh@triangulate{1}%
				\def\pgfplotsplothandlermesh@patchclass@output{triangle}%
				% Note shader=faceted interp  can be written as
				% matrix -- it shades individual elements anyway;
				% and one single patch can be written as matrix.
			\fi
		\fi
	\fi
	%
	%
	\pgfplotspatchclass{\pgfplotsplothandlermesh@patchclass}{LUA class}%
	\ifx\pgfplotsretval\pgfutil@empty
		\let\pgfplotsplothandlerLUAvisualizerfactory=\pgfutil@empty
	\else
		\edef\pgfplotsplothandlerLUAvisualizerfactory{
			function(plothandler) return pgfplots.MeshVisualizer.new(
				plothandler,
				\pgfplotsretval(),
				pgfplots.pgftonumber("\pgfkeysvalueof{/pgfplots/mesh/rows}"),
				pgfplots.pgftonumber("\pgfkeysvalueof{/pgfplots/mesh/cols}"),
				\if0\pgfplots@plot@mesh@ordering true\else false\fi,
				\if1\pgfplotsplothandlermesh@matrixinput true\else false\fi,
				\if1\b@pgfplotsplothandlermesh@matrixoutput true\else false\fi,
				\if4\pgfplotsplothandlermesh@zbuffer@choice true\else false\fi)% 'z buffer=sort'
			end
		}%
	\fi
}%

\def\pgfplotsplothandlerifcurrentpointcanbefirstlast@mesh#1#2{%
	\pgfplotspatchclass{\pgfplotsplothandlermesh@patchclass@output}{if current point can be first last}{#1}{#2}%
}%

% Assigns default values to 'rows' and 'cols' in case
% their values are empty.
%
% PRECONDITION:
% 	The values of
% - '/pgfplots/mesh/rows'
% - '/pgfplots/mesh/cols'
% - '/pgfplots/mesh/num points'
% - \pgfplotsscanlinelength
% - '/pgfplots/mesh/ordering'
%   are assigned properly.
%
% POSTCONDITION:
% 	'rows' and 'cols' now incorporates \pgfplotsscanlinelength if
% 	their values had been empty. However, there may still be empty
% 	values.
%
% @see \pgfplotsautocompletemeshkeys
\def\pgfplotspreparemeshkeydefaults{%
	\pgfplots@autocomplete@meshkeys0%
}%

% Completes and sanitizes 'mesh/rows' and 'mesh/cols' such that they
% have useful integer values afterwards.
%
% This method works similarly to \pgfplotspreparemeshkeydefaults.
% However, it really completes values and issues error messages if
% that proves to be impossible.
%
% PRECONDITION:
% 	The values of
% - '/pgfplots/mesh/rows'
% - '/pgfplots/mesh/cols'
% - '/pgfplots/mesh/num points'
% - '/pgfplots/mesh/ordering'
%   are assigned properly.
%   If
% - \pgfplotsscanlinelength
%   exists and is positive, it may also be used.
%
% POSTCONDITION:
% 	'rows' and 'cols' have useful integer values.
\def\pgfplotsautocompletemeshkeys{%
	\pgfplots@autocomplete@meshkeys1%
}%
\def\pgfplots@autocomplete@meshkeys#1{%
	% For #1 = 0, the values are only partially completed (which is
	% also compatible with patch plots)
	%
	% get (potentially) updated values:
	\pgfkeysgetvalue{/pgfplots/mesh/rows}\pgfplotsplothandlermesh@rows
	\pgfkeysgetvalue{/pgfplots/mesh/cols}\pgfplotsplothandlermesh@cols
	\pgfkeysgetvalue{/pgfplots/mesh/num points}\pgfplotsplothandlermesh@numpoints
	\edef\pgfplotsplothandlermesh@numpoints{\pgfplotsplothandlermesh@numpoints}%
	\edef\pgfplotsplothandlermesh@cols{\pgfplotsplothandlermesh@cols}%
	\edef\pgfplotsplothandlermesh@rows{\pgfplotsplothandlermesh@rows}%
	%
	\def\pgfplots@loc@TMPa{0}%
	\ifpgfplots@curplot@threedim
		\def\pgfplots@loc@TMPa{1}%
	\else
		\if s\pgfplots@meshmode
			% we have a 'surf' plot. That doesn't make sense unless
			% one has a matrix, even for a two--dimensional plot. Try it.
			\def\pgfplots@loc@TMPa{1}%
		\fi
	\fi
	\if1\pgfplots@loc@TMPa
		\ifx\pgfplotsplothandlermesh@rows\pgfutil@empty
			\ifx\pgfplotsplothandlermesh@cols\pgfutil@empty
				\pgfutil@ifundefined{pgfplotsscanlinelength}{\def\pgfplotsscanlinelength{-1}}{}%
				\ifpgfplots@scanline@verbose
					\ifnum\pgfplotsscanlinelength>0
						% we have scanlinelength set! Use it:
						\message{PGFPlots: Scanline length is \pgfplotsscanlinelength.}%
					\else
						\ifnum\pgfplotsscanlinelength=-2
							\message{PGFPlots: Scanline length could not be determined since multiple, different scan line lengths have been found.}%
						\else
							\message{PGFPlots: No end-of-scanline marker found in input stream.}%
						\fi
					\fi
				\fi
				\ifnum\pgfplotsscanlinelength>0
					% we have scanlinelength set! Use it:
					\if\pgfplots@plot@mesh@ordering0%
						% ordering = x varies= rowwise -> scanline is cols!
						\edef\pgfplotsplothandlermesh@cols{\pgfplotsscanlinelength}%
					\else
						% ordering = y varies = colwise: scanline is rows!
						\edef\pgfplotsplothandlermesh@rows{\pgfplotsscanlinelength}%
					\fi
				\else
					\if1#1%
						% try to get reasonable values:
						\ifx\pgfplotsplothandlermesh@numpoints\pgfutil@empty
							\pgfplots@error{Sorry, a mesh needs either '[mesh/rows=<N>]' or '[mesh/cols=<N>]' set - or at least '[mesh/num points=<N>]'. But it did not find any of them...}%
							\def\pgfplotsplothandlermesh@cols{5}%
							\def\pgfplotsplothandlermesh@rows{5}%
						\else
							% We don't have rows/cols. Well, try using
							% sqrt(num points) for both of them!
							\ifnum\pgfplotsplothandlermesh@numpoints<16380
								\pgfmath@basic@sqrt@{\pgfplotsplothandlermesh@numpoints}%
							\else
								\pgfmathfloatparsenumber{\pgfplotsplothandlermesh@numpoints}%
								\pgfmathfloatsqrt@{\pgfmathresult}%
								\pgfmathfloattofixed{\pgfmathresult}%
							\fi
							\begingroup
							% Try whether this rows*rows yields numpoints:
							\afterassignment\pgfplots@gobble@until@relax
							\c@pgf@counta=\pgfmathresult\relax
							\c@pgf@countb=\c@pgf@counta
							\multiply\c@pgf@counta by\c@pgf@counta
							\def\pgfplotsretval{1}%
							\ifnum\c@pgf@counta=\pgfplotsplothandlermesh@numpoints\relax
								% ok, rows*rows = numpoints.
								\edef\pgfplotsplothandlermesh@rows{\the\c@pgf@countb}%
							\else
								% oh. Did not work. Check for rounding errors:
								\c@pgf@counta=\c@pgf@countb
								\advance\c@pgf@counta by1
								\c@pgf@countb=\c@pgf@counta
								\multiply\c@pgf@counta by\c@pgf@counta
								\ifnum\c@pgf@counta=\pgfplotsplothandlermesh@numpoints
									% ok, (rows+1)*(rows+1) = numpoints.
									\edef\pgfplotsplothandlermesh@rows{\the\c@pgf@countb}%
								\else
									% no... did not work. num points doesn't
									% appear to be a squared number.
									\def\pgfplotsretval{0}%
									\edef\pgfplotsplothandlermesh@rows{5}%
								\fi
							\fi
							\if\pgfplotsretval1
								\pgfplotswarning{matrix delinearize uses sqrt}{\pgfplotsplothandlermesh@rows}{\pgfplotsplothandlermesh@cols}{\pgfplotsplothandlermesh@numpoints}{\pgfmathresult}\pgfeov%
							\else
								\pgfplots@error{Sorry, 'plot mesh' needs either '[mesh/rows=<N>]' or '[mesh/cols=<N>]' set,
									but it did not find any of them.
									I also tried rows = sqrt(num points) = sqrt(\pgfplotsplothandlermesh@numpoints) = \pgfmathresult\space without success.}%
							\fi
							\pgfmath@smuggleone\pgfplotsplothandlermesh@rows
							\endgroup
							\let\pgfplotsplothandlermesh@cols=\pgfplotsplothandlermesh@rows
						\fi
					\else
						% #1=0 -> leave them empty.
					\fi
				\fi
			\fi
		\fi
		\if1#1%
			\ifx\pgfplotsplothandlermesh@rows\pgfutil@empty
				\ifx\pgfplotsplothandlermesh@cols\pgfutil@empty
				\else
					% Compute 'rows' out of 'num points' and 'cols':
					\ifx\pgfplotsplothandlermesh@numpoints\pgfutil@empty
						\pgfplots@error{Sorry, matrix input needs a valid '[mesh/num points=<N>]' variable if only one of '[mesh/rows=<N>]' or '[mesh/cols=<N>]' is given.}%
						\def\pgfplotsplothandlermesh@numpoints{100}%
					\fi
					\begingroup
					\c@pgf@counta=\pgfplotsplothandlermesh@numpoints\relax
					\divide\c@pgf@counta by\pgfplotsplothandlermesh@cols\relax
					\xdef\pgfplots@glob@TMPa{\the\c@pgf@counta}%
					\endgroup
					\let\pgfplotsplothandlermesh@rows=\pgfplots@glob@TMPa
				\fi
			\else
				\ifx\pgfplotsplothandlermesh@cols\pgfutil@empty
					% Compute 'cols' out of 'num points' and 'rows':
					\ifx\pgfplotsplothandlermesh@numpoints\pgfutil@empty
						\pgfplots@error{Sorry, matrix input needs a valid '[mesh/num points=<N>]' variable if only one of '[mesh/cols=<N>]' or '[mesh/rows=<N>]' is given.}%
						\def\pgfplotsplothandlermesh@numpoints{100}%
					\fi
					\begingroup
					\c@pgf@counta=\pgfplotsplothandlermesh@numpoints\relax
					\divide\c@pgf@counta by\pgfplotsplothandlermesh@rows\relax
					\xdef\pgfplots@glob@TMPa{\the\c@pgf@counta}%
					\endgroup
					\let\pgfplotsplothandlermesh@cols=\pgfplots@glob@TMPa
				\fi
			\fi
			\ifx\pgfplotsplothandlermesh@numpoints\pgfutil@empty
				\c@pgfplots@scanlineindex=\pgfplotsplothandlermesh@rows\relax
				\multiply\c@pgfplots@scanlineindex by \pgfplotsplothandlermesh@cols\relax
				\edef\pgfplotsplothandlermesh@numpoints{\the\c@pgfplots@scanlineindex}%
			\fi
			%
			\c@pgfplots@scanlineindex=\pgfplotsplothandlermesh@rows\relax
			\multiply\c@pgfplots@scanlineindex by\pgfplotsplothandlermesh@cols\relax
			\ifnum\c@pgfplots@scanlineindex=\pgfplotsplothandlermesh@numpoints\relax
			\else
				\ifcase\pgfplots@mesh@checkmode\relax
					% mesh/check=false
					\def\pgfplots@loc@TMPa##1{}%
				\or
					% mesh/check=warning
					\def\pgfplots@loc@TMPa##1{\pgfplotsthrow@warning{##1}}%
				\or
					% mesh/check=error
					\def\pgfplots@loc@TMPa##1{\pgfplots@error{##1}}%
				\fi
				\pgfplotswarning{matrix check}{\pgfplots@loc@TMPa}{\pgfplotsplothandlermesh@rows}{\pgfplotsplothandlermesh@cols}{\pgfplotsplothandlermesh@numpoints}{\the\c@pgfplots@scanlineindex}\pgfeov%
			\fi
		\fi
	\else
		% 2d visualization has no mesh:
		\edef\pgfplotsplothandlermesh@rows{\pgfplotsplothandlermesh@numpoints}%
		\def\pgfplotsplothandlermesh@cols{1}%
	\fi
	%
	\def\pgfplots@loc@TMPa{\pgfkeyssetvalue{/pgfplots/mesh/rows}}%
	\expandafter\pgfplots@loc@TMPa\expandafter{\pgfplotsplothandlermesh@rows}%
	\def\pgfplots@loc@TMPa{\pgfkeyssetvalue{/pgfplots/mesh/cols}}%
	\expandafter\pgfplots@loc@TMPa\expandafter{\pgfplotsplothandlermesh@cols}%
}%

%%
%% Helper structs for patch plots (surfaces)

% #1: the canvas x coordinate
% #2: the canvas y coordinate
% #3: the point meta data (cdata)
% #4: the vertex'es depth (if required)
% #5: the 3d coordinates of the vertex (will only be stored if it is deemed necessary)
\def\pgfplotspatchvertex#1#2#3#4:#5\endvertex{%
	\def\pgfplotspatchvertexstruct{{#1}{#2}{#3}#4:#5}%
	\def\pgfplotspatchvertexx{#1}%
	\def\pgfplotspatchvertexy{#2}%
	\def\pgfplotspatchvertexmeta{#3}%
	\def\pgfplotspatchvertexdepth{#4}%
	\def\pgfplotspatchvertexcoords{#5}%
}%

% Assumes that #1 expands to an argument of \pgfplotspatchvertex and
% that #2 is a macro containing another patch vertex.
%
% It then replaces #2 by a new vector which is equal to #2 except for
% the point meta which is taken from #1.
\def\pgfplotspatchvertexcopymeta#1\to#2{%
	\begingroup
	\expandafter\pgfplotspatchvertex#1\endvertex
	\let\pgfplots@loc@TMPa=\pgfplotspatchvertexmeta
	\expandafter\pgfplotspatchvertex#2\endvertex
	\xdef\pgfplots@glob@TMPc{%
		{\pgfplotspatchvertexx}%
		{\pgfplotspatchvertexy}%
		{\pgfplots@loc@TMPa}%
		\pgfplotspatchvertexdepth:\pgfplotspatchvertexcoords
	}%
	\endgroup
	\let#2=\pgfplots@glob@TMPc
}%

% Same as \pgfplotspatchvertexcopymeta but it becomes a no-op if the
% point meta in #1 is bounded (according to
% \pgfplotsifpointmetaisbounded)
\def\pgfplotspatchvertexcopymetaifbounded#1\to#2{%
	\begingroup
	\expandafter\pgfplotspatchvertex#1\endvertex
	\let\pgfplots@loc@TMPa=\pgfplotspatchvertexmeta
	\pgfplotsifpointmetaisbounded{\pgfplots@loc@TMPa}{%
		\expandafter\pgfplotspatchvertex#2\endvertex
		\xdef\pgfplots@glob@TMPc{%
			{\pgfplotspatchvertexx}%
			{\pgfplotspatchvertexy}%
			{\pgfplots@loc@TMPa}%
			\pgfplotspatchvertexdepth:\pgfplotspatchvertexcoords
		}%
	}{%
		\global\let\pgfplots@glob@TMPc=#2%
	}%
	\endgroup
	\let#2=\pgfplots@glob@TMPc
}%

\def\pgfplotspointpatchvertex#1#2#3#4:#5\endvertex{%
	\global\pgf@x=#1
	\global\pgf@y=#2
}%

% Allows to combined different vertices in the form
% \pgfplotspatchvertexaccumstart
% \expandafter\pgfplotspatchvertexadd\vertexA\times{0.25}%
% \expandafter\pgfplotspatchvertexadd\vertexB\times{0.5}%
% \expandafter\pgfplotspatchvertexadd\vertexC\times{0.75}%
% \pgfplotspatchvertexfinish\finalvertex
% --> now, \finalvertex contains the result vertex, 0.25*A + 0.5*B + 0.75*C .
%
% THERE MUST NOT BE MATH INBETWEEN.
% More precisely, this overwrites the macros
% \pgf@x,\pgf@y, \pgf@xa,\pgf@ya, \pgf@yb, \pgfmathresult and \pgfplotspatchvertexdepth
\def\pgfplotspatchvertexaccumstart{%
	\pgf@xa=0pt
	\pgf@ya=0pt
	\pgf@yb=0pt
	\def\b@pgfplotspatchvertexaccum@ignore@unbounded@metadata{0}%
	\if0\pgfplotsplothandlermesh@colorinput
		% mesh/color input=colormap
		%\if m\pgfplots@colormap@access
		\iftrue
			% colormap access=map
			\pgfplotscoordmath{meta}{zero}%
		\else
			% colormap access=direct
			% FIXME : this causes lots of problems because the result is
			% of an unexpected format at consumer side:
			\begingroup
			\pgfplotscolorzero{\pgfplotsplothandlermesh@explicit@color@num@comps}%
			\let\pgfmathresult=\pgfplotsretval
			\pgfmath@smuggleone\pgfmathresult
			\endgroup
		\fi
	\else
		% mesh/color input=explicit
		\begingroup
		\pgfplotscolorzero{\pgfplotsplothandlermesh@explicit@color@num@comps}%
		\let\pgfmathresult=\pgfplotsretval
		\pgfmath@smuggleone\pgfmathresult
		\endgroup
	\fi
}%

% if this is called, all following accumulation commands will replace unbounded point meta by the fixed zero value.
\def\pgfplotspatchvertexaccum@configure@ignore@unboundedmetadata{%
	\def\b@pgfplotspatchvertexaccum@ignore@unbounded@metadata{1}%
}%

% #1 scale
% #2 meta
\def\pgfplotspatchvertexaddscaledmeta@if@applicable@colormap#1#2{%
	\let\pgfmathresult@=\pgfmathresult
	\if1\b@pgfplotspatchvertexaccum@ignore@unbounded@metadata
		\pgfplotsifpointmetaisbounded{#2}{\pgfplots@loc@tmptrue}{\pgfplots@loc@tmpfalse}%
	\else
		\pgfplots@loc@tmptrue
	\fi
	\ifpgfplots@loc@tmp
		%\if m\pgfplots@colormap@access
		\iftrue
			% colormap access=map
			\pgfplotscoordmath{meta}{parsenumber}{#1}%
			\pgfplotscoordmath{meta}{op}{multiply}{{#2}{\pgfmathresult}}%
			\pgfplotscoordmath{meta}{op}{add}{{\pgfmathresult@}{\pgfmathresult}}%
		\else
			% colormap access=direct
			% FIXME : this causes lots of problems because the result is
			% of an unexpected format at consumer side:
			% colormap access=direct
			% interpolate colors, not their indices:
			\begingroup
			\pgfplotscoordmath{meta}{parsenumber}{#1}%
			\pgfplotscoordmath{meta}{tofixed}{\pgfmathresult}%
			\pgfplotscolormapgetindex{\pgfmathresult}{\pgfplotsplothandlermesh@colormapname}%
			\global\let\pgfplots@glob@TMPc=\pgfmathresult
			\endgroup
			\pgfplotspatchvertexaddscaledmeta@if@applicable@explicit{#1}{\pgfplots@glob@TMPc}%
		\fi
	\else
		\pgfplotscoordmath{meta}{op}{multiply}{{\pgfplots@metamin}{\pgfmathresult}}%
		\pgfplotscoordmath{meta}{op}{add}{{\pgfmathresult@}{\pgfmathresult}}%
	\fi
}%

% #1 scale
% #2 meta
\def\pgfplotspatchvertexaddscaledmeta@if@applicable@explicit#1#2{%
	\if1\b@pgfplotspatchvertexaccum@ignore@unbounded@metadata
		\edef\pgfmathresult@{#2}%
		\ifx\pgfmathresult@\pgfutil@empty
			\pgfplots@loc@tmpfalse
		\else
			\pgfplots@loc@tmptrue
		\fi
	\else
		\pgfplots@loc@tmptrue
	\fi
	\let\pgfmathresult@=\pgfmathresult
	\ifpgfplots@loc@tmp
		\begingroup
		\pgfplotscoloraddweighted{\pgfplotsplothandlermesh@explicit@color@num@comps}
			{\pgfmathresult@}{#1}{#2}%
		\let\pgfmathresult=\pgfplotsretval
		\pgfmath@smuggleone\pgfmathresult
		\endgroup
	\fi
}

% #1: the patch's x coordinate
% #2: the patch's y coordinate
% #3: the patch's point meta value
% #4: depth (if required)
% #5: 3d coordinate of the vertex
\def\pgfplotspatchvertexadd#1#2#3#4:#5\times#6{%
	\global\pgf@x=#1 %
	\global\pgf@y=#2 %
	\advance\pgf@xa by#6\pgf@x
	\advance\pgf@ya by#6\pgf@y
	\if0\pgfplotsplothandlermesh@colorinput
		% mesh/color input=colormap
		\pgfplotspatchvertexaddscaledmeta@if@applicable@colormap{#6}{#3}%
	\else
		% mesh/color input=explicit
		\pgfplotspatchvertexaddscaledmeta@if@applicable@explicit{#6}{#3}%
	\fi
	\def\pgfplotspatchvertexdepth{#4}%
	\ifx\pgfplotspatchvertexdepth\pgfutil@empty
	\else
		\global\pgf@y=#4pt %
		\advance\pgf@yb by#6\pgf@y
	\fi
}%
\def\pgfplotspatchvertexfinish#1{%
	\edef#1{%
		{\the\pgf@xa}%
		{\the\pgf@ya}%
		{\pgfmathresult}%
		\ifx\pgfplotspatchvertexdepth\pgfutil@empty\else\pgf@sys@tonumber\pgf@yb\fi%
		:%
	}%
}
% Same as \pgfplotspatchvertexadd, but it only adds the X and Y
% coordinates (ignoring the rest).
\def\pgfplotspatchvertexaddXY#1#2#3#4:#5\times#6{%
	\global\pgf@x=#1
	\global\pgf@y=#2
	\advance\pgf@xa by#6\pgf@x
	\advance\pgf@ya by#6\pgf@y
}%
% Same as \pgfplotspatchvertexfinish, but it only saves X and Y.
\def\pgfplotspatchvertexfinishXY#1{%
	\edef#1{{\the\pgf@xa}{\the\pgf@ya}{}:}%
}

% #1 class name
% #2 method
% <any method arguments follow>
\def\pgfplotspatchclass#1#2{%
	\pgfutil@ifundefined{pgfpptch@#1@#2}{%
		\pgfplotspatchclasserror{#1}{#2}%
	}{%
		\edef\pgfplotspatchclassname{#1}%
		\csname pgfpptch@#1@#2\endcsname
	}%
}%
\def\pgfplotspatchclasserror#1#2{%
	\pgfutil@ifundefined{pgfpptch@#1@}{%
		\pgfplots@error{Sorry, the 'patch type=#1' is not defined. Maybe you misspelled it? Or perhaps you need to load \string\usepgfplotslibrary{patchplots}?}%
	}{%
		\pgfplots@error{Sorry, the patch class '#1' has no method '#2' (or it is not yet implemented)}%
	}%
}%

% #1: will be expanded with \edef
\def\pgfplotsplothandlermesh@setnextvertex#1{%
	\pgfplotspatchclass{\pgfplotsplothandlermesh@patchclass}{set next vertex}{#1}%
}%

% Declares a new patch class named #1.
%
% #1: a string name.
% #2: methods defining the class, provided as key values. See the
% predefined classes to adjust your own ones. Documentation about the
% available methods and their arguments can be found below in this
% file.
%
% @see \usepgfplotslibrary{patchplots}
\def\pgfplotsdeclarepatchclass#1#2{%
	\edef\pgfplotsdeclarepatchclass@{@#1@}%
	\pgfutil@ifundefined{pgfpptch@#1@}{%
		% Ok. This is the FIRST declaration of this patch class.
		% Define some default algorithms.
		%
		\expandafter\def\csname pgfpptch@#1@\endcsname{#1}%
		\pgfqkeys{/pgfplots/@declare patch}{%
			init=,
			allow matrix=\def\pgfplotsretval{0},
			init pdf shading=,
			stream to shader={%
%\message{\pgfplotspatchclassname: stream to shader: Starting  ^^J}%
				\pgfplotspatchclass{\pgfplotspatchclassname}{foreach vertex}{%
%\message{vertex (\pgfplotspatchvertexx,\pgfplotspatchvertexy[\pgfplotspatchvertexmeta]. ^^J}%
					\let\pgfplots@current@point@meta=\pgfplotspatchvertexmeta
					\pgfplotsaxisvisphasetransformpointmeta
					\pgfplotslibrarysurfstreamcoord{%
						\global\pgf@x=\pgfplotspatchvertexx\space
						\global\pgf@y=\pgfplotspatchvertexy\space
					}{\pgfplotspointmetatransformed}%
				}%
			},%
			fill path={%
				\def\b@pgfplotsmeshplothandler@fillpath@default@isfirst{1}%
				\pgfplotspatchclass{\pgfplotspatchclassname}{foreach vertex}{%
%\message{vertex (\pgfplotspatchvertexx,\pgfplotspatchvertexy[\pgfplotspatchvertexmeta]. ^^J}%
					\if1\b@pgfplotsmeshplothandler@fillpath@default@isfirst
						\expandafter\pgfpathmoveto
					\else
						\expandafter\pgfpathlineto
					\fi
					{\global\pgf@x=\pgfplotspatchvertexx\space\global\pgf@y=\pgfplotspatchvertexy\space}%
					\def\b@pgfplotsmeshplothandler@fillpath@default@isfirst{0}%
				}%
				\pgfpathclose
			},
			get num cdata vertices=\pgfplotspatchclass{\pgfplotspatchclassname}{get num vertices},
			get dimension=\def\pgfplotsretval{2},
			supports global path=\def\pgfplotsretval{0},
			set edge visible=\expandafter\edef\csname pgfppatchedge@##1\endcsname{##2},
			uses view depth=\def\pgfplotsretval{0},
			if current point can be first last=##1,
			serialize except vertices=\let\pgfplotsretval\pgfutil@empty,
			deserialize except vertices=##1,
			serialize={%
				% the default serialization is not necessarily the fastest
				% possible. But it should work.
				\pgfplotspatchclass{\pgfplotspatchclassname}{serialize except vertices}%
				\t@pgfplots@toka=\expandafter{\pgfplotsretval
					\let\pgfplotspatchready@defaultserialize=\pgfplotspatchready
					\let\pgfplotspatchready=\relax
				}%
				\edef\pgfplotsretval{%
					\noexpand\pgfplotspatchclass{\pgfplotspatchclassname}{new}%
					\the\t@pgfplots@toka
				}%
				\pgfplotspatchclass{\pgfplotspatchclassname}{foreach vertex}{%
					\t@pgfplots@toka=\expandafter{\pgfplotsretval}%
					\edef\pgfplotsretval{\the\t@pgfplots@toka
						\noexpand\pgfplotspatchclass{\pgfplotspatchclassname}{set next vertex}{\pgfplotspatchvertexstruct}%
					}%
				}%
				\expandafter\def\expandafter\pgfplotsretval\expandafter{\pgfplotsretval
					\let\pgfplotspatchready=\pgfplotspatchready@defaultserialize
				}%
			},
			deserialize=##1,
			triangulate class=\def\pgfplotsretval{triangle},
			triangulate={%
				\pgfplots@error{triangulate: not yet implemented for \pgfplotspatchclassname}%
			},%
			refine={%
				\let\pgfplotsrefinedpatchready=\pgfplotspatchready
				\pgfplotspatchclass@recursiverefine{##1}%
			},%
			recursive refine@={%
				\pgfplots@error{refine: not yet implemented for \pgfplotspatchclassname}%
			},%
			compute normal={%
				\pgfplotsplothandlermesh@compute@normal@default
			},%
			sample in unit cube={%
				\pgfplots@error{sample in unit cube: not implemented for \pgfplotspatchclassname}%
			},
			LUA class=,
		}%
	}{}%
	%
	%
	\pgfqkeys{/pgfplots/@declare patch}{#2}%
	%
	%
	\pgfutil@ifundefined{pgfpptch\pgfplotsdeclarepatchclass@ foreach cdata vertex}{%
		\pgfutil@namelet
			{pgfpptch\pgfplotsdeclarepatchclass@ foreach cdata vertex}
			{pgfpptch\pgfplotsdeclarepatchclass@ foreach vertex}%
	}{%
	}%
	\pgfutil@ifundefined{pgfpptch\pgfplotsdeclarepatchclass@ stroke path}{%
		\pgfutil@namelet
			{pgfpptch\pgfplotsdeclarepatchclass@ stroke path}
			{pgfpptch\pgfplotsdeclarepatchclass@ fill path}%
	}{%
	}%
}%
\pgfqkeys{/pgfplots/@declare patch}{%
	% may do any more or less global preparations. It is some sort (!)
	% of constructor.
	init/.code=
		{\expandafter\def\csname pgfpptch\pgfplotsdeclarepatchclass@ init\endcsname{#1}},%
	allow matrix/.code=
		{\expandafter\def\csname pgfpptch\pgfplotsdeclarepatchclass@ allow matrix\endcsname{#1}},%
	%
	% FIXME : this is NOT properly used in higher order patches!?
	% Uses view depth defines \pgfplotsretval to '1' if and only if it
	% does ALWAYS needs the view depth. Note that 'z buffer=sort' will
	% automatically provide the view depth. Use this key if view depth
	% is necessary even for 'z buffer!=sort'.
	uses view depth/.code=
		{\expandafter\def\csname pgfpptch\pgfplotsdeclarepatchclass@ uses view depth\endcsname{#1}},%
	%
	% creates a new patch. The vertices need to be added by 'set next
	% vertex'
	new/.code=
		{\expandafter\def\csname pgfpptch\pgfplotsdeclarepatchclass@ new\endcsname{#1}},%
	%
	% sets the next vertex value of the patch.
	%
	% if the patch has all expected vertices, \pgfplotspatchready is
	% invoked and the next call to 'set next vertex' will start a new
	% patch.
	%
	% #1: encodes the current vertex such that
	%   \pgfplotspatchvertex#1\endvertex
	% is accepted.
	set next vertex/.code=
		{\expandafter\def\csname pgfpptch\pgfplotsdeclarepatchclass@ set next vertex\endcsname##1{#1}},%
	%
	%
	% Creates a new patch by means of sampling. The sampling is
	% applied in the unit cube [0,1]x[0,1] (unit interval if this is an 1d patch
	% type).
	%
	% 'sample in unit cube={<code to invoke for each sampled point>}'
	%
	% It will invoke <code to invoke for each sampled point> once for
	% every sampled vertex. The complete unit cube will be covered.
	% Depending on the patch type, either one or more patches are
	% required to fill the unit cube. As soon as one patch is ready,
	% \pgfplotspatchready will be invoked.
	%
	%
	% PRECONDITION:
	%  -\pgfplotspatchready is defined.
	%
	% Arguments:
	%  #1: code which will be invoked whenever a new coordinate is
	%  ready. Inside of #1, the macros \pgfplotspatchclassx and
	%  \pgfplotspatchclassy result in the x and y coordinate (inside
	%  of the unit cube), in fixed point representation.
	%
	% Example:
	%   \def\pgfplotspatchready{}
	%   \pgfplotspatchclass{biquadratic}{sample in unit cube}{%
	%       point ready at (\pgfplotspatchclassx, \pgfplotspatchclassy)
	%   }
	sample in unit cube/.code=
		{\expandafter\def\csname pgfpptch\pgfplotsdeclarepatchclass@ sample in unit cube\endcsname##1{#1}},%
	%
	% declares whether a single edge is visible or not.
	% #1: an integer denoting the edge, either 0,1,2,...,N-1 where N = num verts
	% #2: 0 if the edge is not to be stroked and 1 if it is stroked.
	%
	% This command doesn't need to be called; but if it is, it needs
	% to be called for every edge.
	%
	% ATTENTION: this is *only* used by 'stroke path'. You will never
	% need 'stroke path' unless some edges are invisible. If all edges
	% are visible, 'stroke path' and 'fill path' are the same.
	%
	% This is currently *only* used for triangle triangulations!
	set edge visible/.code=
		{\expandafter\def\csname pgfpptch\pgfplotsdeclarepatchclass@ set edge visible\endcsname##1##2{#1}},%
	%
	% Defines \pgfplotsretval to be a serialized representation of
	% this patch.
	serialize/.code=
		{\expandafter\def\csname pgfpptch\pgfplotsdeclarepatchclass@ serialize\endcsname{#1}},%
	% Defines \pgfplotsretval to be a serialized representation of
	% this patches extra data fields, excluding vertices.
	% This is often a NO-OP (unless 'set edge visible' has been used)
	serialize except vertices/.code=
		{\expandafter\def\csname pgfpptch\pgfplotsdeclarepatchclass@ serialize except vertices\endcsname{#1}},%
	%
	% Restores a serialized representation from #1.
	deserialize/.code=
		{\expandafter\def\csname pgfpptch\pgfplotsdeclarepatchclass@ deserialize\endcsname##1{#1}},%
	% Restores a serialized representation from #1. Only vertices are
	% not stored here.
	deserialize except vertices/.code=
		{\expandafter\def\csname pgfpptch\pgfplotsdeclarepatchclass@ deserialize except vertices\endcsname##1{#1}},%
	%
	% issues \pgfplotspatchvertex{...}\endvertex for the first vertex.
	first vertex/.code=
		{\expandafter\def\csname pgfpptch\pgfplotsdeclarepatchclass@ first vertex\endcsname{#1}},%
	% issues \pgfplotspatchvertex{...}\endvertex for the vertex which
	% was the first BEFORE any z buffer reorderings changed the
	% sequence (i.e. it undoes the effect of z buffer for the patch).
	first vertex before z buffer/.code=
		{\expandafter\def\csname pgfpptch\pgfplotsdeclarepatchclass@ first vertex before z buffer\endcsname{#1}},%
	%
	% invokes
	% 	\pgfplotspatchvertex...\endvertex
	% 	#1
	% for the every vertex.
	foreach vertex/.code=
		{\expandafter\def\csname pgfpptch\pgfplotsdeclarepatchclass@ foreach vertex\endcsname##1{#1}},%
	%
	% invokes
	% 	\pgfplotspatchvertex...\endvertex
	% 	#1
	% for every colored vertex.
	% See 'get num cdata vertices'
	foreach cdata vertex/.code=
		{\expandafter\def\csname pgfpptch\pgfplotsdeclarepatchclass@ foreach cdata vertex\endcsname##1{#1}},%
	%
	% generates a pgf path which is readily usable for filling.
	fill path/.code=
		{\expandafter\def\csname pgfpptch\pgfplotsdeclarepatchclass@ fill path\endcsname{#1}},%
	%
	% Similar to 'fill path', this here paints the edges. It is only
	% useful if 'set edge visible' is in use. In all other cases, it
	% is the same as 'fill path'
	%
	% This is currently *only* used for triangle triangulations!
	stroke path/.code=
		{\expandafter\def\csname pgfpptch\pgfplotsdeclarepatchclass@ stroke path\endcsname{#1}},%
	%
	init pdf shading/.code=
		{\expandafter\def\csname pgfpptch\pgfplotsdeclarepatchclass@ init pdf shading\endcsname{#1}},%
	stream to shader/.code=
		{\expandafter\def\csname pgfpptch\pgfplotsdeclarepatchclass@ stream to shader\endcsname{#1}},%
	%
	% defines \pgfplotsretval to contain the number of vertices.
	get num vertices/.code=
		{\expandafter\def\csname pgfpptch\pgfplotsdeclarepatchclass@ get num vertices\endcsname{#1}},%
	%
	% defines \pgfplotsretval to contain the number of vertices which have color data
	% attached. This is usually the same as 'get num vertices' (but
	% see coons patches).
	get num cdata vertices/.code=
		{\expandafter\def\csname pgfpptch\pgfplotsdeclarepatchclass@ get num cdata vertices\endcsname{#1}},%
	%
	% defines \pgfplotsretval to contain the dimension of this patch
	% (either 1 for lines or 2 for patches).
	get dimension/.code=
		{\expandafter\def\csname pgfpptch\pgfplotsdeclarepatchclass@ get dimension\endcsname{#1}},%
	%
	% defines \pgfplotsretval to be 1 if and only if the patch class
	% supports global paths.
	% A global path is one which is drawn as one huge, uninterrupted
	% PGF path (except for jumps perhaps). Its area can be filled,
	% i.e. it behaves exactly like the 'sharp' plot handler. This is
	% used implicitly if 'point meta=none' (if there is point meta
	% data, we always want individual path segments and \pgfusepath
	% them after each segment).
	supports global path/.code=
		{\expandafter\def\csname pgfpptch\pgfplotsdeclarepatchclass@ supports global path\endcsname{#1}},%
	%
	% Approximates the given patch by at least one triangle.
	% For every finished triangle, \pgfplotspatchready will be invoked
	% and \pgfplotsplothandlermesh@patchclass will be set to triangle.
	% More precisely, it is expected to be set to the return value of
	% 'triangulate class'.
	triangulate/.code=
		{\expandafter\def\csname pgfpptch\pgfplotsdeclarepatchclass@ triangulate\endcsname{#1}},%
	% Defines \pgfplotsretval to be the classname returned by
	% 'triangulate'. This is usually 'triangle'
	triangulate class/.code=
		{\expandafter\def\csname pgfpptch\pgfplotsdeclarepatchclass@ triangulate class\endcsname{#1}},%
	%
	% Refines the given patch #1 times.
	% For every finished patch, \pgfplotspatchready will be invoked.
	refine/.code=
		{\expandafter\def\csname pgfpptch\pgfplotsdeclarepatchclass@ refine\endcsname##1{#1}},%
	%
	% shared helper method which can be invoked recursively by refine.
	% the default implementation of 'refine' will invoke this one.
	%
	% #1: a serialized representation of the given patch. Use it to
	% restore the current patch in intermediate steps. Note that you
	% may not use \begingroup/\endgroup in this method; otherwise the
	% 'z buffer' algorithm will need to be done with \global (and I
	% don't want that. You probably neither).
	%
	% The refinement loop as such will be controlled by the default
	% implementation of 'refine'. This macro is ONLY to be called by the
	% default implementation of 'refine'! Overwriting it should
	% automatically yield refinement facilities.
	%
	% the macro \pgfplotspatchready should be called for each refined
	% element (happens automatically if you use 'set next vertex').
	%
	% POSTCONDITION:
	% 	the element has been refined once.
	recursive refine@/.code=
		{\expandafter\def\csname pgfpptch\pgfplotsdeclarepatchclass@ recursive refine@\endcsname##1{#1}},%
	%
	get pdf shading type/.style={get pdf shading type*={\def\pgfplotsretval{#1}}},
	get pdf shading type*/.code=
		{\expandafter\def\csname pgfpptch\pgfplotsdeclarepatchclass@ get pdf shading type\endcsname{#1}},%
	%
	% A method which defines \pgfplotsretval to be the normal of the current segment.
	% The values will be given in a format understood by
	% \pgfplotsmathvectortostring{\pgfplotsretval}{default}
	%
	% This method REQUIRES that 3d coordinates are stored (see \pgfplotsplothandlermesh@serialize@logical@coords).
	compute normal/.code=
		{\expandafter\def\csname pgfpptch\pgfplotsdeclarepatchclass@ compute normal\endcsname{#1}},%
	%
	% executes '##1' if the current point can be the first or last
	% point of the patch. Executes ##2 if that is not the case.
	if current point can be first last/.code=
		{\expandafter\def\csname pgfpptch\pgfplotsdeclarepatchclass@ if current point can be first last\endcsname##1##2{#1}},%
	%
	% Defines \pgfplotsretval to contain LUA code which returns a
	% factory method (with no arguments) which returns an instance of the
	% associated LUA pgfplots.PatchType
	%
	% If the returned \pgfplotsretval is empty: there is no LUA implementation.
	LUA class/.code=
		{\expandafter\def\csname pgfpptch\pgfplotsdeclarepatchclass@ LUA class\endcsname{\def\pgfplotsretval{#1}}},%
}

% This implements 'compute normal' :
\def\pgfplotsplothandlermesh@compute@normal@default{%
	\begingroup
	\c@pgf@countd=0
	\pgfplotspatchclass{\pgfplotspatchclassname}{foreach vertex}{%
		\ifx\pgfplotspatchvertexcoords\pgfutil@empty
			\pgfplotsplothandlermesh@compute@normal@error
		\fi
		\pgfplotsmathvectorfromstring{\pgfplotspatchvertexcoords}{default}%
		\expandafter\let\csname point\the\c@pgf@countd\endcsname=\pgfplotsretval
		\let\pointlast=\pgfplotsretval
		\advance\c@pgf@countd by1 %
	}%
	\pgfplotsmathvectorsubtract{\csname point1\endcsname}{\csname point0\endcsname}{default}%
	\let\diffA=\pgfplotsretval
	% Hm. usually, I would use \point2 here... but it turned out that
	% the result looks ugly for rectangular patches -- so I used the
	% last point instead (which is, at least, consistent with the
	% patch boundaries)
	\pgfplotsmathvectorsubtract{\csname pointlast\endcsname}{\csname point0\endcsname}{default}%
	\let\diffB=\pgfplotsretval
	\pgfplotsmathvectorcrossprod{\diffA}{\diffB}{default}%
	\let\crossAB=\pgfplotsretval
	% this here is an experiment to use the MEAN of two normal
	% vectors for rectangular patches... but it did not work as good
	% as expected:
	% \pgfplotspatchclass{\pgfplotspatchclassname}{get num vertices}%
	% \ifnum\pgfplotsretval=4
	% 	\pgfplotsmathvectorsubtract{\csname point3\endcsname}{\csname point0\endcsname}{default}%
	% 	\let\diffC=\pgfplotsretval
	% 	\pgfplotsmathvectorcrossprod{\diffB}{\diffC}{default}%
	% 	\let\crossBC=\pgfplotsretval
	% 	\pgfplotsmathvectoradd{\crossAB}{\crossBC}{default}%
	% 	\pgfplotscoordmath{default}{parsenumber}{0.5}%
	% 	\pgfplotsmathvectorscale{\pgfplotsretval}{\pgfmathresult}{default}%
	% 	\let\normal=\pgfplotsretval
	% \else
		\let\normal=\crossAB
	%\fi
	%
	% specific choices of z buffering change the sequence of points.
	% If the number of reversals is off, we need to switch the
	% normal's sign:
	\def\pgfplots@switch@normal@dir{0}%
	\if1\pgfplotsplothandlermesh@zbuffer@choice
		% z buffer=reverse x seq
		\def\pgfplots@switch@normal@dir{1}%
	\else
		\if2\pgfplotsplothandlermesh@zbuffer@choice
			% z buffer=reverse y seq
			\def\pgfplots@switch@normal@dir{1}%
		\fi
	\fi
	\if1\pgfplots@switch@normal@dir
		\pgfplotscoordmath{default}{parsenumber}{-1}%
		\pgfplotsmathvectorscale{\normal}{\pgfmathresult}{default}%
	\fi
	\pgfmath@smuggleone\pgfplotsretval
	\endgroup
}%
\def\pgfplotsplothandlermesh@compute@normal@error{%
	\pgfplots@error{Cannot compute normal vectors: due to some reason, only canvas coordinates of the involved patch are available in this context, but the original input coordinates are required. This indicates an incompatible combination of variables (sorry). Please report this as a bug}%
}%

\def\pgfplotspatchclass@recursiverefine#1{%
	\c@pgf@counta=#1
	\ifnum\c@pgf@counta=1
		\let\pgfplotspatchready=\pgfplotsrefinedpatchready
	\else
		\advance\c@pgf@counta by-1
		\edef\pgfplotspatchready{\noexpand\pgfplotspatchclass@recursiverefine{\the\c@pgf@counta}}%
	\fi
	%
	\pgfplotspatchclass{\pgfplotspatchclassname}{serialize}%
	\t@pgfplots@toka=\expandafter{\pgfplotsretval}%
	\t@pgfplots@tokc=\expandafter{\pgfplotspatchready}%
	\edef\pgfplots@loc@TMPa{%
		\noexpand\pgfplotspatchclass{\pgfplotspatchclassname}{deserialize}{\the\t@pgfplots@toka}%
		\noexpand\def\noexpand\pgfplotspatchready{\the\t@pgfplots@tokc}%
		\noexpand\def\noexpand\pgfplotspatchclassname{\pgfplotspatchclassname}% % FIXME: is that correct?
	}%
	\expandafter\pgfplotspatchclass@recursiverefine@\expandafter{\pgfplots@loc@TMPa}%
}%
\def\pgfplotspatchclass@recursiverefine@{%
	\pgfplotspatchclass{\pgfplotspatchclassname}{recursive refine@}%
}%


% sequence macros preferred over integer indices.
% using characters avoids the extensive use of \csname ...\endcsname
% and \expandafter which would be necessary for integer indices.
\def\pgfpptchindexnext@A{B}% 0->1
\def\pgfpptchindexnext@B{C}% 1->2
\def\pgfpptchindexnext@C{D}% 2->3
\def\pgfpptchindexnext@D{E}% 3->4
\def\pgfpptchindexnext@E{F}% 4->5
\def\pgfpptchindexnext@F{G}% 5->6
\def\pgfpptchindexnext@G{H}% 6->7
\def\pgfpptchindexnext@H{I}% 7->8
\def\pgfpptchindexnext@I{J}% 8->9
\def\pgfpptchindexnext@J{K}% 9->10
\def\pgfpptchindexnext@K{L}% 10->11
\def\pgfpptchindexnext@L{M}% 11->12
\def\pgfpptchindexnext@M{N}% 12->13
\def\pgfpptchindexnext@N{O}% 13->14
\def\pgfpptchindexnext@O{P}% 14->15
\def\pgfpptchindexnext@P{Q}% 15->16
\def\pgfpptchindexnext@Q{R}% 16->17

\pgfplotsdeclarepatchclass{line}{%
	allow matrix=\def\pgfplotsretval{0},
	new=\def\pgfplotspatchclass@line@no{0},
	set next vertex={%
		\if0\pgfplotspatchclass@line@no\relax
			\edef\pgfplotspatchclass@line@A{#1}%
			\def\pgfplotspatchclass@line@no{1}%
		\else
			\edef\pgfplotspatchclass@line@B{#1}%
			\def\pgfplotspatchclass@line@no{0}%
			\pgfplotspatchready%
		\fi
	},
	sample in unit cube={%
		\def\pgfplotspatchclassx{0}%
		\def\pgfplotspatchclassy{0}%
		#1%
		\def\pgfplotspatchclassx{1}%
		\def\pgfplotspatchclassy{0}%
		#1%
		\pgfplotspatchready
	},%
	get dimension=\def\pgfplotsretval{1},
	supports global path=\def\pgfplotsretval{1},
	first vertex=\expandafter\pgfplotspatchvertex\pgfplotspatchclass@line@A\endvertex,
	first vertex before z buffer=\expandafter\pgfplotspatchvertex\pgfplotspatchclass@line@A\endvertex,
	foreach vertex={%
		\expandafter\pgfplotspatchvertex\pgfplotspatchclass@line@A\endvertex #1%
		\expandafter\pgfplotspatchvertex\pgfplotspatchclass@line@B\endvertex #1%
	},
	fill path={%
		\pgfplotsplothandlermesh@pathmoveto{\expandafter\pgfplotspointpatchvertex\pgfplotspatchclass@line@A\endvertex}%
		\pgfpathlineto{\expandafter\pgfplotspointpatchvertex\pgfplotspatchclass@line@B\endvertex}%
		\pgfplotsplothandlermesh@setlastpoint{\expandafter\pgfplotspointpatchvertex\pgfplotspatchclass@line@B\endvertex}%
	},
	serialize except vertices=\let\pgfplotsretval\pgfutil@empty,%
	deserialize except vertices=,
	get num vertices=\def\pgfplotsretval{2},
	get pdf shading type=0,
}%

\def\pgfplotsplothandlermesh@last@x{-16000pt}
\def\pgfplotsplothandlermesh@last@y{-16000pt}

\def\pgfplotsplothandlermesh@pathmoveto#1{%
	\pgf@process{#1}%
	\pgfpathmoveto{}%
}
\def\pgfplotsplothandlermesh@setlastpoint#1{%
}

% Applies a moveto to #1 - but only if #1 is NOT the last used path
% segment.
\def\pgfplotsplothandlermesh@pathmoveto@globalpath#1{%
	\pgf@process{#1}%
	\pgfplots@loc@tmptrue
	\ifdim\pgf@x=\pgfplotsplothandlermesh@last@x\relax
		\ifdim\pgf@y=\pgfplotsplothandlermesh@last@y\relax
			\pgfplots@loc@tmpfalse
		\fi
	\fi
	\ifpgfplots@loc@tmp
		\pgfpathmoveto{}%
	\fi
}%

\def\pgfplotsplothandlermesh@setlastpoint@globalpath#1{%
	\pgf@process{#1}%
	\xdef\pgfplotsplothandlermesh@last@x{\the\pgf@x}%
	\xdef\pgfplotsplothandlermesh@last@y{\the\pgf@y}%
}%

\pgfplotsdeclarepatchclass{triangle}{%
	allow matrix=\def\pgfplotsretval{0},
	new=\def\pgfplotspatchclass@tri@no{0}\let\pgfplotspatchclass@trie@AB\relax,
	set next vertex={%
		\ifcase\pgfplotspatchclass@tri@no\relax
			\edef\pgfplotspatchclass@tri@A{#1}%
			\def\pgfplotspatchclass@tri@no{1}%
		\or
			\edef\pgfplotspatchclass@tri@B{#1}%
			\def\pgfplotspatchclass@tri@no{2}%
		\or
			\edef\pgfplotspatchclass@tri@C{#1}%
			\def\pgfplotspatchclass@tri@no{0}%
			\pgfplotspatchready%
		\fi
	},
	sample in unit cube={%
		\def\pgfplotspatchclassx{0}%
		\def\pgfplotspatchclassy{0}%
		#1%
		\def\pgfplotspatchclassx{1}%
		\def\pgfplotspatchclassy{0}%
		#1%
		\def\pgfplotspatchclassx{0}%
		\def\pgfplotspatchclassy{1}%
		#1%
		\pgfplotspatchready
		%
		%
		\def\pgfplotspatchclassx{0}%
		\def\pgfplotspatchclassy{1}%
		#1%
		\def\pgfplotspatchclassx{1}%
		\def\pgfplotspatchclassy{0}%
		#1%
		\def\pgfplotspatchclassx{1}%
		\def\pgfplotspatchclassy{1}%
		#1%
		\pgfplotspatchready
	},%
	first vertex=\expandafter\pgfplotspatchvertex\pgfplotspatchclass@tri@A\endvertex,
	foreach vertex={%
		\expandafter\pgfplotspatchvertex\pgfplotspatchclass@tri@A\endvertex #1%
		\expandafter\pgfplotspatchvertex\pgfplotspatchclass@tri@B\endvertex #1%
		\expandafter\pgfplotspatchvertex\pgfplotspatchclass@tri@C\endvertex #1%
	},
	fill path={%
		\pgfpathmoveto{\expandafter\pgfplotspointpatchvertex\pgfplotspatchclass@tri@A\endvertex}%
		\pgfpathlineto{\expandafter\pgfplotspointpatchvertex\pgfplotspatchclass@tri@B\endvertex}%
		\pgfpathlineto{\expandafter\pgfplotspointpatchvertex\pgfplotspatchclass@tri@C\endvertex}%
		\pgfpathclose
	},
	stroke path={%
		\ifx\pgfplotspatchclass@trie@AB\relax
			\pgfpathmoveto{\expandafter\pgfplotspointpatchvertex\pgfplotspatchclass@tri@A\endvertex}%
			\pgfpathlineto{\expandafter\pgfplotspointpatchvertex\pgfplotspatchclass@tri@B\endvertex}%
			\pgfpathlineto{\expandafter\pgfplotspointpatchvertex\pgfplotspatchclass@tri@C\endvertex}%
			\pgfpathclose
		\else
			\if1\pgfplotspatchclass@trie@AB
				\pgfpathmoveto{\expandafter\pgfplotspointpatchvertex\pgfplotspatchclass@tri@A\endvertex}%
				\pgfpathlineto{\expandafter\pgfplotspointpatchvertex\pgfplotspatchclass@tri@B\endvertex}%
			\fi
			\if1\pgfplotspatchclass@trie@BC
				\pgfpathmoveto{\expandafter\pgfplotspointpatchvertex\pgfplotspatchclass@tri@B\endvertex}%
				\pgfpathlineto{\expandafter\pgfplotspointpatchvertex\pgfplotspatchclass@tri@C\endvertex}%
			\fi
			\if1\pgfplotspatchclass@trie@CA
				\pgfpathmoveto{\expandafter\pgfplotspointpatchvertex\pgfplotspatchclass@tri@C\endvertex}%
				\pgfpathlineto{\expandafter\pgfplotspointpatchvertex\pgfplotspatchclass@tri@A\endvertex}%
			\fi
		\fi
	},
	triangulate=\pgfplotspatchready,
	set edge visible={%
		\ifcase#1
			\edef\pgfplotspatchclass@trie@AB{#2}%
		\or
			\edef\pgfplotspatchclass@trie@BC{#2}%
		\or
			\edef\pgfplotspatchclass@trie@CA{#2}%
		\fi
	},
	serialize except vertices={%
		\ifx\pgfplotspatchclass@trie@AB\relax
			\let\pgfplotsretval\pgfutil@empty%
		\else
			\edef\pgfplotsretval{%
				\noexpand\def\noexpand\pgfplotspatchclass@trie@AB{\pgfplotspatchclass@trie@AB}%
				\noexpand\def\noexpand\pgfplotspatchclass@trie@BC{\pgfplotspatchclass@trie@BC}%
				\noexpand\def\noexpand\pgfplotspatchclass@trie@CA{\pgfplotspatchclass@trie@CA}%
			}%
		\fi
	},%
	recursive refine@={%
		%
		\pgfplotspatchclass{\pgfplotspatchclassname}{new}%
		%
		% Perform longest edge bisection.
		\pgfpointdiff
			{\expandafter\pgfplotspointpatchvertex\pgfplotspatchclass@tri@A\endvertex}%
			{\expandafter\pgfplotspointpatchvertex\pgfplotspatchclass@tri@B\endvertex}%
		\edef\pgfplots@loc@TMPa{{\pgf@sys@tonumber\pgf@x}{\pgf@sys@tonumber\pgf@y}}%
		\expandafter\pgfmath@basic@veclen@\pgfplots@loc@TMPa
		\pgf@xa=\pgfmathresult pt
		\pgf@yc=\pgf@xa % "max so far"
		\def\pgfplotspatch@tri@longestedgestart{A}%
		%
		\pgfpointdiff
			{\expandafter\pgfplotspointpatchvertex\pgfplotspatchclass@tri@B\endvertex}%
			{\expandafter\pgfplotspointpatchvertex\pgfplotspatchclass@tri@C\endvertex}%
		\edef\pgfplots@loc@TMPa{{\pgf@sys@tonumber\pgf@x}{\pgf@sys@tonumber\pgf@y}}%
		\expandafter\pgfmath@basic@veclen@\pgfplots@loc@TMPa
		\pgf@xa=\pgfmathresult pt
		\ifdim\pgf@xa>\pgf@yc
			\pgf@yc=\pgf@xa
			\def\pgfplotspatch@tri@longestedgestart{B}%
		\fi
		%
		\pgfpointdiff
			{\expandafter\pgfplotspointpatchvertex\pgfplotspatchclass@tri@C\endvertex}%
			{\expandafter\pgfplotspointpatchvertex\pgfplotspatchclass@tri@A\endvertex}%
		\edef\pgfplots@loc@TMPa{{\pgf@sys@tonumber\pgf@x}{\pgf@sys@tonumber\pgf@y}}%
		\expandafter\pgfmath@basic@veclen@\pgfplots@loc@TMPa
		\pgf@xa=\pgfmathresult pt
		\ifdim\pgf@xa>\pgf@yc
			\def\pgfplotspatch@tri@longestedgestart{C}%
		\fi
		% The ordering is not arbitrary
		% (unless z buffer=sort is in effect). Let's assume that the
		% AB edge is BACKGROUND and the CD edge is FOREGROUND. Then
		% this sequence here should have approximately the same depth
		% ordering:
		%
		\pgfplotspatchvertexaccumstart
		\if A\pgfplotspatch@tri@longestedgestart
			\expandafter\pgfplotspatchvertexadd\pgfplotspatchclass@tri@A\times{0.5}%
			\expandafter\pgfplotspatchvertexadd\pgfplotspatchclass@tri@B\times{0.5}%
				\def\pgfplots@loc@TMPa{ACM,BCM}%
		\else
			\if B\pgfplotspatch@tri@longestedgestart
				\expandafter\pgfplotspatchvertexadd\pgfplotspatchclass@tri@B\times{0.5}%
				\expandafter\pgfplotspatchvertexadd\pgfplotspatchclass@tri@C\times{0.5}%
				\def\pgfplots@loc@TMPa{ABM,ACM}%
			\else
				\expandafter\pgfplotspatchvertexadd\pgfplotspatchclass@tri@C\times{0.5}%
				\expandafter\pgfplotspatchvertexadd\pgfplotspatchclass@tri@A\times{0.5}%
				\def\pgfplots@loc@TMPa{ABM,BCM}%
			\fi
		\fi
		\pgfplotspatchvertexfinish\pgfplotspatchclass@tri@M
		\edef\pgfplots@loc@TMPa{\pgfplots@loc@TMPa;\pgfplotspatchclass@tri@M}%
		%
		% invoke helper method such that I don't need to restore or
		% recompute \pgfplotspatchclass@tri@M during recursion.
		\expandafter\pgfplotspatchclass@tri@recursiverefine@@\pgfplots@loc@TMPa\relax{#1}%
	},
	deserialize except vertices=%
		\let\pgfplotspatchclass@trie@AB\relax
		#1,
	get pdf shading type=4,
	get num vertices=\def\pgfplotsretval{3},
	LUA class=pgfplots.TrianglePatchType.new,
}%
\def\pgfplotspatchclass@tri@recursiverefine@@#1#2M,#3#4M;#5\relax#6{%
	\expandafter\let\expandafter\pgfplotspatchclass@tri@A@\csname pgfplotspatchclass@tri@#1\endcsname
	\expandafter\let\expandafter\pgfplotspatchclass@tri@B@\csname pgfplotspatchclass@tri@#2\endcsname
	\expandafter\pgfplotsplothandlermesh@setnextvertex\expandafter{\pgfplotspatchclass@tri@A@}%
	\expandafter\pgfplotsplothandlermesh@setnextvertex\expandafter{\pgfplotspatchclass@tri@B@}%
	\pgfplotsplothandlermesh@setnextvertex{#5}%
	#6%
	%
	\expandafter\let\expandafter\pgfplotspatchclass@tri@A@\csname pgfplotspatchclass@tri@#3\endcsname
	\expandafter\let\expandafter\pgfplotspatchclass@tri@B@\csname pgfplotspatchclass@tri@#4\endcsname
	\expandafter\pgfplotsplothandlermesh@setnextvertex\expandafter{\pgfplotspatchclass@tri@A@}%
	\expandafter\pgfplotsplothandlermesh@setnextvertex\expandafter{\pgfplotspatchclass@tri@B@}%
	\pgfplotsplothandlermesh@setnextvertex{#5}%
	#6%
}%

% implements 'first vertex before z buffer' for rectangle
% Note that 'mesh input=lattice' generates the rectangle patch as
% follows: if the input matrix has the form
%
% (i-1,j-1)  (i,j-1)
%
% (i-1,j)    (i,j)
%
% it generates the rectangle
%
% A   B
%
% D   C
%
% (if you try to map this to a plot, ensure that you have 'y dir=reverse')
\def\pgfplotspatchclass@rect@first@before@z@buffer{%
	\if\pgfplots@plot@mesh@ordering0%
		% ordering = rowwise -> scanline is cols!
		\ifcase\pgfplotsplothandlermesh@zbuffer@choice\relax
			% none
			\let\pgfplots@loc@TMPa\pgfplotspatchclass@rect@A
		\or
			% reverse x seq
			\let\pgfplots@loc@TMPa\pgfplotspatchclass@rect@B
		\or
			% reverse y seq
			\let\pgfplots@loc@TMPa\pgfplotspatchclass@rect@D
		\or
			% reverse xy seq
			\let\pgfplots@loc@TMPa\pgfplotspatchclass@rect@C
		\or
			% sort
			\let\pgfplots@loc@TMPa\pgfplotspatchclass@rect@A
		\else
			% the choices 'default' or 'auto' should have been expanded already. Nothing to do here:
			\let\pgfplots@loc@TMPa\pgfplotspatchclass@rect@A
		\fi
	\else
		% ordering = colwise: scanline is rows!
		\ifcase\pgfplotsplothandlermesh@zbuffer@choice\relax
			% none
			\let\pgfplots@loc@TMPa\pgfplotspatchclass@rect@A
		\or
			% reverse x seq
			\let\pgfplots@loc@TMPa\pgfplotspatchclass@rect@D
		\or
			% reverse y seq
			\let\pgfplots@loc@TMPa\pgfplotspatchclass@rect@B
		\or
			% reverse xy seq
			\let\pgfplots@loc@TMPa\pgfplotspatchclass@rect@C
		\or
			% sort
			\let\pgfplots@loc@TMPa\pgfplotspatchclass@rect@A
		\else
			% the choices 'default' or 'auto' should have been expanded already. Nothing to do here:
			\let\pgfplots@loc@TMPa\pgfplotspatchclass@rect@A
		\fi
	\fi
	\expandafter\pgfplotspatchvertex\pgfplots@loc@TMPa\endvertex
}%

% A rectangle specified in the sequence
% D <-- C
% |     ^
% v     |
% A --> B
%
\pgfplotsdeclarepatchclass{rectangle}{%
	allow matrix=\def\pgfplotsretval{1},
	new=\def\pgfplotspatchclass@rect@no{0},
	set next vertex={%
		\ifcase\pgfplotspatchclass@rect@no\relax
			\edef\pgfplotspatchclass@rect@A{#1}%
			\def\pgfplotspatchclass@rect@no{1}%
		\or
			\edef\pgfplotspatchclass@rect@B{#1}%
			\def\pgfplotspatchclass@rect@no{2}%
		\or
			\edef\pgfplotspatchclass@rect@C{#1}%
			\def\pgfplotspatchclass@rect@no{3}%
		\or
			\edef\pgfplotspatchclass@rect@D{#1}%
			\def\pgfplotspatchclass@rect@no{0}%
			\pgfplotspatchready%
		\fi
	},
	sample in unit cube={%
		\def\pgfplotspatchclassx{0}%
		\def\pgfplotspatchclassy{0}%
		#1%
		\def\pgfplotspatchclassx{1}%
		\def\pgfplotspatchclassy{0}%
		#1%
		\def\pgfplotspatchclassx{1}%
		\def\pgfplotspatchclassy{1}%
		#1%
		\def\pgfplotspatchclassx{0}%
		\def\pgfplotspatchclassy{1}%
		#1%
		\pgfplotspatchready
	},%
	first vertex=\expandafter\pgfplotspatchvertex\pgfplotspatchclass@rect@A\endvertex,
	first vertex before z buffer=\pgfplotspatchclass@rect@first@before@z@buffer,
	foreach vertex={%
		\expandafter\pgfplotspatchvertex\pgfplotspatchclass@rect@A\endvertex #1%
		\expandafter\pgfplotspatchvertex\pgfplotspatchclass@rect@B\endvertex #1%
		\expandafter\pgfplotspatchvertex\pgfplotspatchclass@rect@C\endvertex #1%
		\expandafter\pgfplotspatchvertex\pgfplotspatchclass@rect@D\endvertex #1%
	},
	stream to shader={%
		% a 'foreach vertex' with slightly different sequence:
		\def\pgfplots@foreachvertex@@{%
			\let\pgfplots@current@point@meta=\pgfplotspatchvertexmeta
			\pgfplotsaxisvisphasetransformpointmeta
			\pgfplotslibrarysurfstreamcoord{%
				\global\pgf@x=\pgfplotspatchvertexx\space
				\global\pgf@y=\pgfplotspatchvertexy\space
			}{\pgfplotspointmetatransformed}%
		}%
		\if1\b@pgfplotsplothandlermesh@matrixoutput
			\expandafter\pgfplotspatchvertex\pgfplotspatchclass@rect@A\endvertex\pgfplots@foreachvertex@@%
			\expandafter\pgfplotspatchvertex\pgfplotspatchclass@rect@B\endvertex\pgfplots@foreachvertex@@%
			\expandafter\pgfplotspatchvertex\pgfplotspatchclass@rect@D\endvertex\pgfplots@foreachvertex@@% EXCHANGED SEQUENCE
			\expandafter\pgfplotspatchvertex\pgfplotspatchclass@rect@C\endvertex\pgfplots@foreachvertex@@%
		\else
			\expandafter\pgfplotspatchvertex\pgfplotspatchclass@rect@A\endvertex\pgfplots@foreachvertex@@%
			\expandafter\pgfplotspatchvertex\pgfplotspatchclass@rect@B\endvertex\pgfplots@foreachvertex@@%
			\expandafter\pgfplotspatchvertex\pgfplotspatchclass@rect@D\endvertex\pgfplots@foreachvertex@@%
			\expandafter\pgfplotspatchvertex\pgfplotspatchclass@rect@B\endvertex\pgfplots@foreachvertex@@%
			\expandafter\pgfplotspatchvertex\pgfplotspatchclass@rect@C\endvertex\pgfplots@foreachvertex@@%
			\expandafter\pgfplotspatchvertex\pgfplotspatchclass@rect@D\endvertex\pgfplots@foreachvertex@@%
		\fi
	},%
	fill path={%
		\pgfpathmoveto{\expandafter\pgfplotspointpatchvertex\pgfplotspatchclass@rect@A\endvertex}%
		\pgfpathlineto{\expandafter\pgfplotspointpatchvertex\pgfplotspatchclass@rect@B\endvertex}%
		\pgfpathlineto{\expandafter\pgfplotspointpatchvertex\pgfplotspatchclass@rect@C\endvertex}%
		\pgfpathlineto{\expandafter\pgfplotspointpatchvertex\pgfplotspatchclass@rect@D\endvertex}%
		\pgfpathclose
	},
	init pdf shading={%
		\if0\b@pgfplotsplothandlermesh@matrixoutput
			\ifnum\pgfkeysvalueof{/pgfplots/surf shading/cols}=2
			\else
				\pgfplots@error{Sorry, (patch type=rectangle) and (rectangular patch shadings with more than one rectangle) is not supported (got \pgfkeysvalueof{/pgfplots/surf shading/cols} x * input points).
					It is supported if z buffer!=sort. Consider using 'patch to triangles=true' to combine it with z buffer=sort}%
			\fi
		\fi
	},
	recursive refine@={%
		\pgfplots@error{Please use \string\usepgfplotslibrary{patchplots} in order to use 'refine,patch type=\pgfplotspatchclassname'}%
	},
	triangulate={%
		\let\pgfplotsplothandlermesh@patchclass@=\pgfplotsplothandlermesh@patchclass
		\def\pgfplotsplothandlermesh@patchclass{triangle}%
		\pgfplotspatchclass{\pgfplotsplothandlermesh@patchclass}{new}%
		\pgfplotspatchclass{\pgfplotsplothandlermesh@patchclass}{set edge visible}01%
		\pgfplotspatchclass{\pgfplotsplothandlermesh@patchclass}{set edge visible}10%
		\pgfplotspatchclass{\pgfplotsplothandlermesh@patchclass}{set edge visible}21%
		\expandafter\pgfplotsplothandlermesh@setnextvertex\expandafter{\pgfplotspatchclass@rect@A}%
		\expandafter\pgfplotsplothandlermesh@setnextvertex\expandafter{\pgfplotspatchclass@rect@B}%
		\expandafter\pgfplotsplothandlermesh@setnextvertex\expandafter{\pgfplotspatchclass@rect@D}%
		%
		\pgfplotspatchclass{\pgfplotsplothandlermesh@patchclass}{set edge visible}01%
		\pgfplotspatchclass{\pgfplotsplothandlermesh@patchclass}{set edge visible}11%
		\pgfplotspatchclass{\pgfplotsplothandlermesh@patchclass}{set edge visible}20%
		\expandafter\pgfplotsplothandlermesh@setnextvertex\expandafter{\pgfplotspatchclass@rect@B}%
		\expandafter\pgfplotsplothandlermesh@setnextvertex\expandafter{\pgfplotspatchclass@rect@C}%
		\expandafter\pgfplotsplothandlermesh@setnextvertex\expandafter{\pgfplotspatchclass@rect@D}%
		\let\pgfplotsplothandlermesh@patchclass=\pgfplotsplothandlermesh@patchclass@
	},%
	serialize except vertices=\let\pgfplotsretval\pgfutil@empty,%
	deserialize except vertices=,
	get pdf shading type=5,
	get num vertices=\def\pgfplotsretval{4},
	LUA class=pgfplots.RectanglePatchType.new,
}%

\def\pgfplots@rect@next@A{B}
\def\pgfplots@rect@next@B{C}
\def\pgfplots@rect@next@C{D}
\def\pgfplots@rect@next@D{A}

\def\pgfplots@rect@rnext@A{D}
\def\pgfplots@rect@rnext@B{A}
\def\pgfplots@rect@rnext@C{B}
\def\pgfplots@rect@rnext@D{C}

% Ideas: make sure that the foreground edge is the 'v=1'
% edge of the coons patch, i.e. that it is drawn on top of
% the rest.
%
% Furthermore, for fixed v, the 'u=1' point should belong
% to the foreground (as it is drawn on top of u=0).
%
% In other words, if the coons patch looks like
%
% BB-------CC   <--v=1
%  |        |
%  |        |
%  |        |
% AA-------DD   <--v=0
%  |        |
%  u=0      u=1
%
% where the AA,BB,CC,DD are the corners of the coons
% patch, The FOREGROUND POINT of our visualization here should be CC,
% the second most foreground point BB (assuming that
% foreground point and second most foreground point are
% adjacent).
%
% Thus, if 'CC' is behind of 'BB' in the A,B,C,D ordering:
%    start the coons patch with AA := BB-1 in forward
%    direction.
% else
%    start the coons patch with AA := BB+1 in reverse
%    direction.
%
% #1: the vertex prefix for the rectangular shape.
% Thus, \csname #1A\endcsname is the first vertex, \csname #1B\endcsname the second
% and #1C, #1D the third and fourth.
%
% [output] #2: A macro name which will be filled with the FIRST point
% to stream to the shader. It will be defined to be either A,B,C or D.
% [output] #3: A macro name which indicates whether the stream shall
% be forward or backward. It will be defined to be either empty or 'r'. If
% it is empty, the coons patch shall be in the sequence
% #2,next(#2),next(next(#2),....
% If it is 'r', the coons patch shall be in the sequence
% #2,prev(#2),prev(prev(#2),...
%
% Note that this method works also for biquadratic patches since these
% patches also have #1A, #1B, #1C and #1D as their corners.
%
\def\pgfplotspatchclass@rect@to@coonspatch@findshadingsequence@for@depth#1#2#3{%
	\def#2{A}%
	\def#3{}%
	%
	\expandafter\expandafter\expandafter\pgfplotspatchvertex\csname #1A\endcsname\endvertex
	\ifx\pgfplotspatchvertexdepth\pgfutil@empty
	\else
		% aaah -- we have z buffer information. Then we will use it!
		\edef\PA{\pgfplotspatchvertexdepth pt A}%
		\expandafter\expandafter\expandafter\pgfplotspatchvertex\csname #1B\endcsname\endvertex
		\edef\PB{\pgfplotspatchvertexdepth pt B}%
		\expandafter\expandafter\expandafter\pgfplotspatchvertex\csname #1C\endcsname\endvertex
		\edef\PC{\pgfplotspatchvertexdepth pt C}%
		\expandafter\expandafter\expandafter\pgfplotspatchvertex\csname #1D\endcsname\endvertex
		\edef\PD{\pgfplotspatchvertexdepth pt D}%
		\pgfplotsutilsortfour\PA\PB\PC\PD
		\def\toCHAR##1pt ##2{##2}% get original index
		%
		%
		\edef\PA{\expandafter\toCHAR\PA}%%
		\edef#2{\expandafter\toCHAR\PB}%%
		\edef\Pnext{\csname pgfplots@rect@next@#2\endcsname}% this here uses only A,B,C and D (also for biquadratic patches!)
		\if\Pnext\PA
			% ok, next(BB) = CC.
			% that means we have to advance forward.
			\def#3{}%
			\edef#2{\csname pgfplots@rect@rnext@#2\endcsname}%
		\else
			\edef#2{\csname pgfplots@rect@next@#2\endcsname}%
			\def#3{r}%
		\fi
%\message{^^JDEPTH SEQUENCE for rectangle: \PA, \PB, \PC, \PD^^J}%
%\message{COONS PATH ORDERING: starts with #2, direction: \ifx#3\pgfutil@empty forward\else backward\fi^^J}%
	\fi
}%

\def\pgfplotspatchclass@rect@to@coonspatch@findshadingsequence@for@depth@no@op#1#2#3{%
	\def#2{A}%
	\def#3{}%
}

\def\pgfplotsplothandlermesh@patchclass@defaulttext{default}%

% This is a VERY special case. The idea is to switch to
% "normal" path mode:
% \pgfpathmoveto \pgfpathlineto \pgfpathlineto
% \pgfpathlineto \pgfusepath.
%
% Note that the mesh handler's default is to issue
% \pgfusepath after *each* elementary patch.
%
% This special use-case might not cover all aspects of the
% mesh/surf/patch plot handler.
\def\pgfplotsplothandlermesh@init@path@without@point@meta{%
	%
	\pgfplotspatchclass{\pgfplotsplothandlermesh@patchclass@output}{supports global path}%
	\ifnum\pgfplotsretval=1 %
		\def\pgfplotsplothandlermesh@definecolor{%
			% no-op
		}%
		% FIXME : this does not handle separate edges, i.e.
		% \pgfplotsplothandlermesh@VISUALIZE@std@separate@fillstroke
		\def\pgfplotsplothandlermesh@VISUALIZE{\pgfplotsplothandlermesh@VISUALIZE@globalpath@fill@andor@stroke}%
		\let\pgfplotsplothandlermesh@pathmoveto=\pgfplotsplothandlermesh@pathmoveto@globalpath
		\let\pgfplotsplothandlermesh@setlastpoint=\pgfplotsplothandlermesh@setlastpoint@globalpath
		%
		% This activates tikz colors:
		%\tikz@options
		\pgfplots@drawoptions
		% Acquire the values for \tikz@mode@* [used for faceted]
		\pgfplots@drawmodes
	\else
		\pgfplots@error{Please provide 'point meta' for your plot. Omitting point meta is only supported for 1d patch types.}%
	\fi
}

% implements 'show normals'. Keep in mind that this is more-or-less a
% debugging helper, not productive code.
\def\pgfplotsplothandlermesh@show@normals{%
	\pgfplotspatchclass{\pgfplotsplothandlermesh@patchclass}{compute normal}%
	\let\pgfplots@normal@vector=\pgfplotsretval
%\pgfplotsmathvectortostring\pgfplots@normal@vector{default}%
%\message{got normal \pgfplotsretval^^J}%
	\pgfplotsmathvectorlength{\pgfplots@normal@vector}{default}%
	\pgfplotscoordmath{default}{parsenumber}{\pgfkeysvalueof{/pgfplots/mesh/show normals length factor}}% FIXME
	\pgfplotscoordmath{default}{op}{divide}{{\pgfmathresult}{\pgfplotsretval}}%
	\pgfplotsmathvectorscale{\pgfplots@normal@vector}{\pgfmathresult}{default}%
	%\pgfplotsmathvectoradd{\pgfplotspatchvertexcoords}{\pgfplotsretval}{default}%
	\let\pgfplots@normal@vector=\pgfplotsretval
	%
	\draw[/pgfplots/every patch normal]
		\pgfextra{%
			\pgfpathmoveto
			{%
				\pgf@process{%
					\pgfplotspatchclass{\pgfplotsplothandlermesh@patchclass}{get num vertices}%
					\pgfmath@basic@reciprocal@{\pgfplotsretval}%
					\let\scale=\pgfmathresult
					\pgfplotspatchvertexaccumstart
					\pgfplotspatchclass{\pgfplotspatchclassname}{foreach vertex}{%
						\expandafter\pgfplotspatchvertexadd\pgfplotspatchvertexstruct\times{\scale}%
					}%
					\pgfplotspatchvertexfinishXY{\pgfplots@loc@TMPa}%
					\expandafter\pgfplotspointpatchvertex\pgfplots@loc@TMPa\endvertex%
				}%
			}%
			\edef\pgfplots@loc@TMPa{\pgf@x=\the\pgf@x\space\pgf@y=\the\pgf@y\space}%
			%
			\pgfpathlineto{%
				\pgfpointadd
					{\pgfplots@loc@TMPa}
					{\pgfplotspointfromcsvvector\pgfplots@normal@vector{default}}%
			}%
		}%
	;
}%

% Defines \pgfplotsretval to be either empty or the current point's
% depth according to z buffer information.
\def\pgfplotsplothandlermesh@compute@vertex@depth{\let\pgfplotsretval\pgfutil@empty}%

% If \pgfplotsplothandlermesh@compute@vertex@depth is set to this
% particular implementation, the \pgfplotspatchvertexdepth macro will
% contain the view depth.
%
\def\pgfplotsplothandlermesh@compute@vertex@depth@doit{%
	\ifx\pgfplots@current@point@x\pgfutil@empty
		\let\pgfplotsretval\pgfutil@empty
	\else
		\pgfplotsmathvectorfromstring{\pgfplots@current@point@x,\pgfplots@current@point@y,\pgfplots@current@point@z}{default}%
		\pgfplotsmathvectorviewdepth{\pgfplotsretval}%
		\pgfplotscoordmath{default}{tofixed}{\pgfplotsretval}%
		\edef\pgfplotsretval{\pgfmathresult}%
	\fi
}%

% A command which defines \pgfplotsretval to be either empty or to
% the (x,y,z) 3d coordinates of the data point. Usually, the data
% scaling transformation will already be applied to these points.
%
% Note that this is unnecessary for most applications of mesh plots.
%
% The resulting value will be available as \pgfplotspatchvertexcoords
\def\pgfplotsplothandlermesh@serialize@logical@coords{%
	% empty by default
	\let\pgfplotsretval=\pgfutil@empty
}%
\def\pgfplotsplothandlermesh@serialize@logical@coords@doit{%
	\ifx\pgfplots@current@point@x\pgfutil@empty
		\let\pgfplotsretval=\pgfutil@empty
	\else
		% note that these points are the result of the data scaling
		% trafo. That is OK because the view normal is also in the
		% same coordinate space.
		\pgfplotsmathvectorfromstring{\pgfplots@current@point@x,\pgfplots@current@point@y,\pgfplots@current@point@z}{default}%
	\fi
}

% the inverse for \pgfplotsplothandlermesh@serialize@logical@coords: it defines \pgfplots@current@point@[xyz].
% You should only call it if you know that there IS something to deserialize.
\def\pgfplotsplothandlermesh@deserialize@logical@coords#1{%
	\expandafter\pgfplotsplothandlermesh@deserialize@logical@coords@#1\relax
}%
\def\pgfplotsplothandlermesh@deserialize@logical@coords@#1,#2,#3\relax{%
	\def\pgfplots@current@point@x{#1}%
	\def\pgfplots@current@point@y{#2}%
	\def\pgfplots@current@point@z{#3}%
}%

% The patch visualization works like a PIPELINE:
%
% DECODE -> CHECK JUMP -> REFINE -> FINISH
%
% where FINISH is usually nothing but VISUALIZE; but it can also mean
% to collect patches, sort them according to view depth and visualize
% the result.
%
% Each single PIPE element can be exchanged by a separate "subclass".
% For example, DECODE can either DECODE from a matrix (a lattice) or a
% sequence of patch coords.
%
\def\pgfplotsplothandlermesh@stream#1{%
	\pgfplotsplothandlermesh@PIPE@DECODE{#1}%
}%

% accumulates a patch vertex-by-vertex.
% #1 the next vertex of a patch (#1 defines canvas coordinates \pgf@x
% and \pgf@y).
%
% @PRECONDITION \pgfplots@current@point@[xyz] and \pgfplots@current@point@meta
% need to be defined.
\def\pgfplotsplothandlermesh@PIPE@DECODE#1{%
	\let\pgfplotsplothandlermesh@patchclass\pgfplotsplothandlermesh@patchclass@input
	\pgfplotsplothandlermesh@compute@vertex@depth% <-- this modifies \pgfplotsretval
	\let\pgfplots@view@depth=\pgfplotsretval
	\pgf@process{#1}%
	% calls \pgfplotspatchready when ready
	\pgfplotsplothandlermesh@serialize@logical@coords
	\pgfplotspatchclass{\pgfplotsplothandlermesh@patchclass}{set next vertex}{{\the\pgf@x}{\the\pgf@y}{\pgfplots@current@point@meta}\pgfplots@view@depth:\pgfplotsretval}%%
}%
\def\pgfplotspatchready{%
	\pgfplotsplothandlermesh@PIPE@CHECKJUMP
}%
\let\pgfplotspatchready@orig=\pgfplotspatchready
\def\pgfplotsplothandlermesh@PIPE@CHECKJUMP{%
	\pgfplots@loc@tmptrue
	\pgfplotspatchclass{\pgfplotsplothandlermesh@patchclass}{foreach vertex}{%
		\pgfplotsplothandlermesh@check@unbounded%
	}%
	%
	\ifpgfplots@loc@tmp
		\pgfplotsplothandlermesh@PIPE@CHECKJUMP@meta
	\fi
	%
	\ifpgfplots@loc@tmp
		\expandafter\pgfplotsplothandlermesh@PIPE@REFINE
	\fi
}%

\def\pgfplotsplothandlermesh@PIPE@CHECKJUMP@meta@noop{%
}

\let\pgfplotsplothandlermesh@PIPE@CHECKJUMP@meta=\pgfplotsplothandlermesh@PIPE@CHECKJUMP@meta@noop

\def\pgfplotsplothandlermesh@PIPE@CHECKJUMP@meta@everyvertex{%
	\pgfplotspatchclass{\pgfplotsplothandlermesh@patchclass}{foreach cdata vertex}{%
		\pgfplotsplothandlermesh@PIPE@CHECKJUMP@meta@check
	}%
}%
\def\pgfplotsplothandlermesh@PIPE@CHECKJUMP@meta@firstvertex{%
	\pgfplotspatchclass{\pgfplotsplothandlermesh@patchclass}{first vertex before z buffer}%
	\pgfplotsplothandlermesh@PIPE@CHECKJUMP@meta@check
}%
\def\pgfplotsplothandlermesh@PIPE@CHECKJUMP@meta@firstvertex@legacy{%
	\pgfplotspatchclass{\pgfplotsplothandlermesh@patchclass}{first vertex}%
	\pgfplotsplothandlermesh@PIPE@CHECKJUMP@meta@check
}%
\def\pgfplotsplothandlermesh@PIPE@CHECKJUMP@meta@check{%
	\if0\pgfplotsplothandlermesh@colorinput
		% mesh/color input=colormap
		\pgfplotsifpointmetaisbounded{\pgfplotspatchvertexmeta}{}{%
			\pgfplots@loc@tmpfalse
		}%
	\else
		% mesh/color input=explicit
		\ifx\pgfplotspatchvertexmeta\pgfutil@empty
			\pgfplots@loc@tmpfalse
		\fi
	\fi
}

\def\pgfplotsplothandlermesh@PIPE@REFINE@no@op{%
	\pgfplotsplothandlermesh@PIPE@FINISH
}%
\let\pgfplotsplothandlermesh@PIPE@REFINE=\pgfplotsplothandlermesh@PIPE@REFINE@no@op

% invoked when a patch is completely processed.
% This might be replaced by the z buffering algorithm
\def\pgfplotsplothandlermesh@PIPE@FINISH{%
	\pgfplotsplothandlermesh@PIPE@interior@exterior@styles
}%

\def\pgfplotsplothandlermesh@PIPE@interior@exterior@styles{%
	\pgfplotsplothandlermesh@interior@exterior@hook
	\pgfplotsplothandlermesh@VISUALIZE
}%

% Standard implementation for VISUALIZE
\def\pgfplotsplothandlermesh@VISUALIZE@std@fill@andor@stroke{%
	\pgfplotspatchclass{\pgfplotsplothandlermesh@patchclass}{fill path}%
	\pgfplotsplothandlermesh@definecolor
	\pgfplotsplothandlermesh@usepath
	\pgfplotsplothandlermesh@show@normals@if@configured
}%

% a special VISUALIZE implementation which allows to draw a CONTINUOS
% global path, i.e. one which is not "\pgfusepath"ed after each patch
% segment. This is only used if 'point meta=none'
\def\pgfplotsplothandlermesh@VISUALIZE@globalpath@fill@andor@stroke{%
	\pgfplotspatchclass{\pgfplotsplothandlermesh@patchclass}{fill path}%
}%

\def\pgfplotsplothandlermesh@VISUALIZE@std{\pgfplotsplothandlermesh@VISUALIZE@std@fill@andor@stroke}

\def\pgfplotsplothandlermesh@VISUALIZE{\pgfplotsplothandlermesh@VISUALIZE@std}

\let\pgfplotsplothandlermesh@show@normals@if@configured\relax


%%%%%%%%%%%%%%%%%%%%%%%%%%%%%%%%%%%%%%%%
\def\pgfplotsplothandlermesh@init@DECODE{%
	\if0\pgfplotsplothandlermesh@matrixinput
%\message{initialised DECODE for patch output (from patch).^^J}%
	\else
		% mesh input=lattice|image
		% Init memory:
		%
		% And now finally: initialise the conversion from
		%  LINEARIZED COORD STREAM -> DATA MATRIX:
		\if\pgfplots@plot@mesh@ordering0%
			% x varies (=rowwise):
			\let\pgfplotsplothandlermesh@scanlinelength=\pgfplotsplothandlermesh@cols
			\let\pgfplotsplothandlermesh@numscanlines=\pgfplotsplothandlermesh@rows
		\else
			% y varies (=colwise):
			\let\pgfplotsplothandlermesh@scanlinelength=\pgfplotsplothandlermesh@rows
			\let\pgfplotsplothandlermesh@numscanlines=\pgfplotsplothandlermesh@cols
		\fi
		%
		\if1\pgfplotsplothandlermesh@ONEDIMMODE
%\message{initialised DECODE for 1d mode.^^J}%
			% FIXME : needs LUA support...
			\if2\pgfplotsplothandlermesh@matrixinput
				% mesh input=image
				\pgfplots@error{'matrix input=image' is unsupported for line plots (or matrix plots with just 1 row or 1 column). Please provide at least 2 rows and 2 cols for 'matrix input=image'}%
			\fi
			\let\pgfplotsplothandlermesh@PIPE@DECODE=\pgfplotsplothandlermesh@PIPE@DECODE@matrix@ONEDIMMODE
		\else
			\if0\b@pgfplotsplothandlermesh@matrixoutput
				\if1\b@pgfplots@LUA@visualization@enabled
%\message{initialised DECODE for patch output (from matrix which has already been decoded in LUA).^^J}%
				\else
					% in fact, this does only remember the maximum capacity;
					% it doesn't fill the memory.
					\c@pgfplots@scanlineindex=\pgfplotsplothandlermesh@scanlinelength\relax
					\advance\c@pgfplots@scanlineindex by1
					\pgfplotsdequenewempty{lastscanline}\capacity{\the\c@pgfplots@scanlineindex}%
					\if1\pgfplotsplothandlermesh@matrixinput
						% mesh input=lattice
						\let\pgfplotsplothandlermesh@PIPE@DECODE=\pgfplotsplothandlermesh@PIPE@DECODE@matrix@fillscanline
%\message{initialised DECODE for patch output (from matrix).^^J}%
					\else
						% mesh input=image
						% lastinterpscanline contains a line of
						% interpolated points, and there is one more than points. Preallocate it:
						\advance\c@pgfplots@scanlineindex by1
						\pgfplotsdequenewempty{lastinterpscanline}\capacity{\the\c@pgfplots@scanlineindex}%
						% NOTE: implementation is in a separate file.
						\let\pgfplotsplothandlermesh@PIPE@DECODE=\pgfplotsplothandlermesh@PIPE@DECODE@image@fillscanline
%\message{initialised DECODE for IMAGE output (from matrix).^^J}%
					\fi
				\fi
			\else
				% SPECIAL HANDLING!
				\if1\pgfplotsplothandlermesh@matrixinput
					% mesh input=lattice
					\let\pgfplotsplothandlermesh@PIPE@DECODE=\pgfplotsplothandlermesh@PIPE@DECODE@to@shader
				\else
					% mesh input=image
					% this should never happen because matrix output
					% is off for mesh input=image
					\pgfplots@error{Illegal internal state encountered}%
				\fi
%\message{initialised DECODE for direct matrix output (from matrix).^^J}%
			\fi
		\fi
	\fi
}%


\def\pgfplotsplothandlermesh@PIPE@REFINE@triangulate{%
	\let\pgfplotspatchready\pgfplotsplothandlermesh@PIPE@FINISH
	\pgfplotspatchclass{\pgfplotsplothandlermesh@patchclass}{triangulate}%
	\let\pgfplotspatchready\pgfplotspatchready@orig
}%
\def\pgfplotsplothandlermesh@PIPE@REFINE@recursively{%
	\let\pgfplotspatchready\pgfplotsplothandlermesh@PIPE@FINISH
	\pgfplotspatchclass{\pgfplotsplothandlermesh@patchclass}{refine}{\pgfplotsplothandlermesh@refinecount}%
	\let\pgfplotspatchready\pgfplotspatchready@orig
}%
\def\pgfplotsplothandlermesh@PIPE@REFINE@triangulate@recursively{%
	\let\pgfplotspatchready=\pgfplotsplothandlermesh@PIPE@REFINE@triangulate@recursively@patchready
	\pgfplotspatchclass{\pgfplotsplothandlermesh@patchclass}{refine}{\pgfplotsplothandlermesh@refinecount}%
	\let\pgfplotspatchready\pgfplotspatchready@orig
}%
\def\pgfplotsplothandlermesh@PIPE@REFINE@triangulate@recursively@patchready{%
	\let\pgfplotspatchready\pgfplotsplothandlermesh@PIPE@FINISH
	\pgfplotspatchclass{\pgfplotsplothandlermesh@patchclass}{triangulate}%
	\let\pgfplotspatchready\pgfplotsplothandlermesh@PIPE@REFINE@triangulate@recursively@patchready
}%
\def\pgfplotsplothandlermesh@init@REFINE{%
	\pgfkeysgetvalue{/pgfplots/patch refines}\pgfplotsplothandlermesh@refinecount
	\if0\pgfplotsplothandlermesh@triangulate
		\ifnum\pgfplotsplothandlermesh@refinecount>0
%\message{initiased REFINE to REFINE.}%
			\let\pgfplotsplothandlermesh@PIPE@REFINE=\pgfplotsplothandlermesh@PIPE@REFINE@recursively
		\else
%\message{initiased REFINE to NOP.}%
			\let\pgfplotsplothandlermesh@PIPE@REFINE=\pgfplotsplothandlermesh@PIPE@REFINE@no@op
		\fi
	\else
%\message{initiased REFINE to triangulate.}%
		\ifnum\pgfplotsplothandlermesh@refinecount>0
			\let\pgfplotsplothandlermesh@PIPE@REFINE=\pgfplotsplothandlermesh@PIPE@REFINE@triangulate@recursively
		\else
			\let\pgfplotsplothandlermesh@PIPE@REFINE=\pgfplotsplothandlermesh@PIPE@REFINE@triangulate
		\fi
	\fi
}%

%%%%%%%%%%%%%%%%%%%%%%%%%%%%%%%%%%%%%%%%
% MESH SURVEY
\def\pgfplotsplothandlersurveystart@mesh{%
	\pgfplotsplothandlersurveystart@default
	\if0\pgfplotsplothandlermesh@colorinput
		% mesh/color input=colormap
	\else
		% mesh/color input=explicit
		\if1\pgfplotsplothandlermesh@colorinput@mathparse
			\pgfkeysdef{/pgfplots/color sequence/every input component}{%
				\expandafter\pgfmathparse\expandafter{\pgfmathresult}%
				\ifpgfplots@usefpu
					\pgfmathfloattofixed\pgfmathresult
				\fi
			}%
		\else
			\pgfkeysdef{/pgfplots/color sequence/every input component}{}%
		\fi
		\pgfplotscolornormalizesequence[
			colorspace=\pgfkeysvalueof{/pgfplots/mesh/colorspace explicit color output},
			default input colorspace=\pgfkeysvalueof{/pgfplots/mesh/colorspace explicit color input},
			context message={point meta '\pgfplots@current@point@meta' of coord no \the\c@pgfplots@coordindex\space (\pgfplots@current@point@x,\pgfplots@current@point@y\ifpgfplots@curplot@threedim,\pgfplots@current@point@z\fi)},
		]%
	\fi
	%
	\if2\pgfplotsplothandlermesh@matrixinput
		% mesh input=image
		\pgfplotsplothandlersurveystart@mesh@image
		\pgfplotsplothandlermesh@needs@scanlinecomplete@notificationstrue
	\fi
	%
	\pgfplotsplothandlersurveypoint@mesh@check@scanline@length@init
}%

\def\pgfplotsplothandlersurveypoint@mesh{%
	\pgfplotsplothandlersurveypoint@default
	%
	\pgfplotsplothandlersurveypoint@mesh@common
}

\def\pgfplotsplothandlersurveypoint@mesh@common{%
	\if2\pgfplotsplothandlermesh@matrixinput
		\pgfplotsplothandlersurveypoint@mesh@image
	\fi
	%
	\ifcase\pgfplotsplothandlermesh@matrixinput
	\else
		% mesh input=lattice|image
		\pgfplotsplothandlersurveypoint@mesh@check@scanline@length
	\fi
}%

\def\pgfplotsplothandlersurveypoint@pseudoconstant@row@index{%
	\if\pgfplots@plot@mesh@ordering0%
		% ordering = x varies= rowwise -> scanline is cols!
	\else
		% ordering = y varies = colwise: scanline is rows!
		\ifx\pgfplotsplothandlermesh@rows\pgfutil@empty
	\fi
}%

% initializes \pgfplotsplothandlersurveypoint@mesh@check@scanline@length, see its comments.
\def\pgfplotsplothandlersurveypoint@mesh@check@scanline@length@init{%
	\pgfkeysgetvalue{/pgfplots/empty line}\pgfplots@loc@TMPa%
	\def\pgfplots@loc@TMPb{scanline}
	\ifx\pgfplots@loc@TMPb\pgfplots@loc@TMPa
	\else
		% without end-of-scanline markers, the notification
		% \pgfplotsscanlinecomplete does not make sense anyway in this
		% context. Disable it:
		\let\pgfplotsplothandlersurveypoint@mesh@check@scanline@length=\relax
	\fi
	\if\pgfplots@plot@mesh@ordering0%
		% ordering = x varies= rowwise -> scanline is cols!
		\ifx\pgfplotsplothandlermesh@cols\pgfutil@empty
			% ... and without columns, we cannot generate the notification.
			\let\pgfplotsplothandlersurveypoint@mesh@check@scanline@length=\relax
		\fi
	\else
		% ordering = y varies = colwise: scanline is rows!
		\ifx\pgfplotsplothandlermesh@rows\pgfutil@empty
			% ... and without rows, we cannot generate the notification.
			\let\pgfplotsplothandlersurveypoint@mesh@check@scanline@length=\relax
		\fi
	\fi
}

% Invokes \pgfplotsscanlinecomplete if the scanline is complete according to the keys 'mesh/rows' and/or 'mesh/cols'
%
% This routine improves the precision of the scanline computation, but not much.
% At the time of this writing, its only purpose is to help 'matrix input=image'
% as that requires information about the current scanline in order to update its
% axis limits.
%
% Note that this macro is unnecessary in many cases: if the user input _contains_ end-of-scanline markers, it is redundant.
%
% Limitations of this routine:
% 1. it requires 'empty line=scanline'.
% 2. if mesh/ordering=x varies, it requires that mesh/cols is set.
%    if mesh/ordering=y varies, it requires that mesh/rows is set.
%
% In all other cases, it simply becomes a no-op without further notice (i.e. it will be \let to \relax)
\def\pgfplotsplothandlersurveypoint@mesh@check@scanline@length{%
%\message{checking scanline length for \the\c@pgfplots@scanlineindex. cols=\pgfplotsplothandlermesh@cols, rows=\pgfplotsplothandlermesh@rows. ordering = \pgfplots@plot@mesh@ordering^^J}%
	\if\pgfplots@plot@mesh@ordering0%
		% ordering = x varies= rowwise -> scanline is cols!
		\ifnum\pgfplotsplothandlermesh@cols=\c@pgfplots@scanlineindex\relax
			% it does not hurt to invoke this here multiple times -- it has to handle that case anyway:
			\pgfplotsscanlinecomplete
		\fi
	\else
		% ordering = y varies = colwise: scanline is rows!
		\ifnum\pgfplotsplothandlermesh@rows=\c@pgfplots@scanlineindex\relax
			% it does not hurt to invoke this here multiple times -- it has to handle that case anyway:
			\pgfplotsscanlinecomplete
		\fi
	\fi
}%

\def\pgfplotsplothandlersurveypoint@mesh@noreset@of@pointmeta{%
	\pgfplotsplothandlersurveypoint@default@noreset@of@pointmeta
	\pgfplotsplothandlersurveypoint@mesh@common
}%

\newif\ifpgfplotsplothandlermesh@needs@scanlinecomplete@notifications

\def\pgfplotsplothandlersurveyend@mesh{%
	\pgfplotsplothandlersurveyend@default
	\if0\pgfplotsplothandlermesh@colorinput
		% mesh/color input=colormap
	\else
		% mesh/color input=explicit
		\endpgfplotscolornormalizesequence
	\fi
	%
	\if2\pgfplotsplothandlermesh@matrixinput
		% mesh input=image
		\pgfplotsplothandlersurveyend@mesh@image
	\fi
	%
	\ifpgfplotsplothandlermesh@needs@scanlinecomplete@notifications
		%
		% OK, we EXPECT that the code generated
		% notifications of sorts \pgfplotsscanlinecomplete.
		%
		% If this turns out to be NOT the case, we generate a
		% warning/error.
		%
		% This, in turn, is restricted to the following cases:
		% 1. there have been end-of-scanline markers in the input
		% stream (that's why we can only decide it here, after the
		% input has been surveyed)
		% 2. OR \pgfplotsscanlinecomplete has been called due to scan
		% line length comparison with mesh/cols or mesh/rows. In this
		% case, the required argument must have been set, see
		%  \pgfplotsplothandlersurveypoint@mesh@check@scanline@length@init
		%
		\ifnum\c@pgfplotsplothandlermesh@image@numscanlines<2
			\pgfplotsautocompletemeshkeys%
			\ifx\pgfplotsplothandlersurveypoint@mesh@check@scanline@length\relax
				% ah -- this routine has been cleared because the keys are
				% missing.
				%
				% Please see the comments on
				% \pgfplotsplothandlersurveypoint@mesh@check@scanline@length
				% for details.
				%
				\if\pgfplots@plot@mesh@ordering0%
					% ordering = x varies= rowwise -> scanline is cols!
					\pgfplotswarning{imagesc limits incomplete}{mesh/cols=\pgfplotsplothandlermesh@cols}\pgfeov%
				\else
					% ordering = y varies = colwise: scanline is rows!
					\pgfplotswarning{imagesc limits incomplete}{mesh/rows=\pgfplotsplothandlermesh@rows}\pgfeov%
				\fi
			\fi
		\fi
	\fi
	%
}

\def\pgfplotsplothandlersurvey@mesh@scanlinecomplete{%
	%
	%
	\if2\pgfplotsplothandlermesh@matrixinput
		% mesh input=image
		\pgfplotsplothandlersurveyend@mesh@image@scanlinecomplete
	\fi
}%

\def\pgfplotsplothandlersurveyaftersetpointmeta@mesh@cdatanormalize{%
	\pgfplotsplothandlersurveyaftersetpointmeta@default
	\if0\pgfplotsplothandlermesh@colorinput
		% mesh/color input=colormap
	\else
		% mesh/color input=explicit
		\pgfplotscolornormalizesequencenext{\pgfplots@current@point@meta}%
		\let\pgfplots@current@point@meta=\pgfplotsretval
	\fi
}%

\newif\ifpgfplotsplothandlermesh@patchtable@hascdata
\newif\ifpgfplotsplothandlermesh@patchtable@cdata@individual
\def\pgfplotsplothandlersurveystart@mesh@patchtable{%
	\pgfplotsplothandlersurveystart@mesh
	\pgfplotsarraynewempty\pgfplots@verts
}%
\def\pgfplotsplothandlersurveypoint@mesh@patchtable{%
	\pgfplotscoordmath{x}{parsenumber}{\pgfplots@current@point@x}%
	\let\pgfplots@current@point@x=\pgfmathresult
	%
	\pgfplotscoordmath{y}{parsenumber}{\pgfplots@current@point@y}%
	\let\pgfplots@current@point@y=\pgfmathresult
	%
	\ifpgfplots@curplot@threedim
		\pgfplotscoordmath{z}{parsenumber}{\pgfplots@current@point@z}%
		\let\pgfplots@current@point@z=\pgfmathresult
	\fi
	%
	\ifpgfplotsplothandlermesh@patchtable@hascdata
		% will be handled later, using data from within 'patch table':
		\let\pgfplots@current@point@meta=\pgfutil@empty
	\else
		\pgfplotsaxissurveysetpointmeta
	\fi
	%
	\def\pgfplots@loc@TMPa{1}%
	\ifx\pgfplots@current@point@x\pgfutil@empty
		\def\pgfplots@loc@TMPa{0}%
	\else
		\ifx\pgfplots@current@point@y\pgfutil@empty
			\def\pgfplots@loc@TMPa{0}%
		\else
			\ifx\pgfplots@current@point@z\pgfutil@empty
				\ifpgfplots@curplot@threedim
					\def\pgfplots@loc@TMPa{0}%
				\fi
			\fi
		\fi
	\fi
	\if1\pgfplots@loc@TMPa
		\edef\pgfplots@loc@TMPa{\pgfplots@current@point@x,\pgfplots@current@point@y,\pgfplots@current@point@z;\pgfplots@current@point@meta;}%
		\expandafter\pgfplotsarraypushback\pgfplots@loc@TMPa\to\pgfplots@verts
	\fi
}%
\def\pgfplotsplothandlersurveyend@mesh@patchtable{%
	\pgfplotsplothandlersurveyend@mesh
	\let\pgfplotsplothandlersurveystart=\pgfplotsplothandlersurveystart@default
	\let\pgfplotsplothandlersurveyend=\pgfplotsplothandlersurveyend@default
	\let\pgfplotsplothandlersurveypoint=\pgfplotsplothandlersurveypoint@mesh@noreset@of@pointmeta
	\pgfplotsplothandlersurveystart
	\expandafter\pgfplotstable@isloadedtable\expandafter{\pgfplotsplothandlermesh@patchtable}{%
		\pgfplots@error{Sorry, 'patch table={<loaded table>}' has not been implemented; it would probably be too slow :-(. Please provide a file name or inline table data}%
	}{%
		\expandafter\pgfplotstableread\expandafter*\expandafter{\pgfplotsplothandlermesh@patchtable} to listener\pgfplotsplothandlermesh@patchtable@rowlistener
	}%
	\pgfplotsplothandlersurveyend
}%

\def\pgfplotsplothandlermesh@patchtable@init{%
	\pgfplotspatchclass{\pgfplotsplothandlermesh@patchclass}{get num vertices}%
	\let\pgfplotsplothandlermesh@patchtable@numverts=\pgfplotsretval
	%
	\ifpgfplotsplothandlermesh@patchtable@hascdata
		\ifpgfplotsplothandlermesh@patchtable@cdata@individual
		\else
			\ifnum\pgfplotsplothandlermesh@patchtable@numverts<\pgfplotstablecols\relax
				% prepare a pointer to data column: use the column
				% after the last vertex index.
				% this thing constitutes the "default" cdata source for
				% 'point meta=explicit'.
				\pgfplotstablereadgetptrtocolindex
					{\pgfplotsplothandlermesh@patchtable@numverts}
					{\pgfplots@table@PTR@CDATA}%
			\else
				\let\pgfplots@table@PTR@CDATA=\relax
			\fi
		\fi
	\else
		\pgfplotsplothandlermesh@patchtable@cdata@individualfalse
	\fi
	%
	\c@pgf@countb=0
	\pgfutil@loop
	\ifnum\c@pgf@countb<\pgfplotsplothandlermesh@patchtable@numverts\relax
		\edef\pgfplots@loc@TMPa{%
			\noexpand\pgfplotstablereadgetptrtocolindex{\the\c@pgf@countb}{\expandafter\noexpand\csname pgfplots@table@PTR@\the\c@pgf@countb\endcsname}%
		}%
		\pgfplots@loc@TMPa
		\advance\c@pgf@countb by1
	\pgfutil@repeat
	\let\pgfplotsplothandlermesh@patchtable@init=\relax
}%
\def\pgfplotsplothandlermesh@patchtable@rowlistener@exceptionmsg{%
	patch table \pgfplotsplothandlermesh@patchtable\space line \pgfplotstablelineno: could not read column no \the\c@pgf@countb: expected '\pgfplots@current@vertexindex' to be a valid index into the vertex arrays (the \pgfplotsarraysizeof\pgfplots@verts\space input coordinates)%
}%
\def\pgfplotsplothandlermesh@patchtable@rowlistener{%
	\pgfplotsplothandlermesh@patchtable@init%
	%
	\ifpgfplotsplothandlermesh@patchtable@hascdata
		\ifpgfplotsplothandlermesh@patchtable@cdata@individual
		\else
			\ifx\pgfplots@table@PTR@CDATA\relax
				\let\pgfplots@current@point@meta=\pgfutil@empty
			\else
				\pgfplotstablereadevalptr\pgfplots@table@PTR@CDATA\pgfplots@current@point@meta
			\fi
		\fi
	\fi
	%
	\c@pgf@countb=0
	\pgfutil@loop
	\ifnum\c@pgf@countb<\pgfplotsplothandlermesh@patchtable@numverts\relax
		\expandafter\pgfplotstablereadevalptr\csname pgfplots@table@PTR@\the\c@pgf@countb\endcsname\pgfplots@current@vertexindex%
		%
		\ifx\pgfplots@current@vertexindex\pgfutil@empty
		\else
			\pgfmathfloatparsenumber{\pgfplots@current@vertexindex}%
			\pgfmathfloattoint\pgfmathresult
			\let\pgfplots@current@vertexindex=\pgfmathresult
			%
			\let\pgfplotsexceptionmsg=\pgfplotsplothandlermesh@patchtable@rowlistener@exceptionmsg
			\pgfplotsarrayselect\pgfplots@current@vertexindex\of\pgfplots@verts\to\pgfplots@stored@pt
			\let\pgfplotsexceptionmsg=\relax
			%
			\ifpgfplotsplothandlermesh@patchtable@cdata@individual
				\c@pgf@countc=\c@pgf@countb\relax
				\advance\c@pgf@countc by\pgfplotsplothandlermesh@patchtable@numverts\relax
				\expandafter\getthisrowno\expandafter{\the\c@pgf@countc}\pgfplots@current@point@meta
			\fi
			%
			\expandafter\pgfplotsplothandlermesh@patchtable@stream\pgfplots@stored@pt
		\fi
		\advance\c@pgf@countb by1
	\pgfutil@repeat
}
\def\pgfplotsplothandlermesh@patchtable@stream#1,#2,#3;#4;{%
	\def\pgfplots@current@point@x{#1}%
	\def\pgfplots@current@point@y{#2}%
	\def\pgfplots@current@point@z{#3}%
	\ifpgfplotsplothandlermesh@patchtable@hascdata
		\def\pgfplots@set@perpointmeta@done{0}%
		\pgfplotsaxissurveysetpointmeta
	\else
		\def\pgfplots@current@point@meta{#4}%
	\fi
	\pgfplotsplothandlersurveypoint
}%
% MESH SURVEY
%%%%%%%%%%%%%%%%%%%%%%%%%%%%%%%%%%%%%%%%

\def\pgfplotsplothandlermesh@init@meshkeys{%
	\ifcase\pgfplotsplothandlermesh@matrixinput\relax
		% mesh input=patches
		\pgfkeys{/pgfplots/mesh/ordering/x varies}% disable any special handling.
	\else
		% mesh input=lattice|image
		\pgfplotsautocompletemeshkeys%
		\pgfkeysgetvalue{/pgfplots/mesh/rows}\pgfplotsplothandlermesh@rows
		\pgfkeysgetvalue{/pgfplots/mesh/cols}\pgfplotsplothandlermesh@cols
		\pgfkeysgetvalue{/pgfplots/mesh/num points}\pgfplotsplothandlermesh@numpoints
		%
		% the \pgfplotsplothandlermesh@ONEDIMMODE macro is ONLY
		% interesting for matrix input.
		\ifnum\pgfplotsplothandlermesh@cols>1
		\else
			\def\pgfplotsplothandlermesh@ONEDIMMODE{1}%
		\fi
		\ifnum\pgfplotsplothandlermesh@rows>1
		\else
			\def\pgfplotsplothandlermesh@ONEDIMMODE{1}%
		\fi
		\if1\pgfplotsplothandlermesh@ONEDIMMODE
			\def\pgfplotsplothandlermesh@patchclass{line}%
			\pgfkeyslet{/pgfplots/patch type}\pgfplotsplothandlermesh@patchclass
		\fi
	\fi
	%
	\pgfplotspatchclass{\pgfkeysvalueof{/pgfplots/patch type}}{get dimension}%
	\ifnum\pgfplotsretval=1 %
		\def\pgfplots@meshmode{m}% mesh.
	\fi
	%
}%

\let\pgfplotsplothandlermesh@interior@exterior@hook=\relax
\def\pgfplotsplothandlermesh@interior@exterior@hook@active@zerothresh{%
	\begingroup
	\pgfplotspatchclass{\pgfplotsplothandlermesh@patchclass}{compute normal}%
% \let\pgfplots@normal=\pgfplotsretval
% \pgfplotsmathvectorlength{\pgfplotsretval}{default}%
% \pgfplotscoordmath{default}{op}{reciprocal}{{\pgfplotsretval}}%
% \pgfplotsmathvectorscale{\pgfplots@normal}{\pgfmathresult}{default}%
	%
	\pgfplotsmathvectorscalarproduct{\pgfplots@view@dir@threedim}{\pgfplotsretval}{default}%
	\pgfplotscoordmath{default}{if is}{\pgfplotsretval}{-}{%
		% <0:
		\global\let\pgfplots@glob@TMPc\pgfplotsplothandlermesh@is@interior
	}{%
		% >=0:
		\global\let\pgfplots@glob@TMPc\pgfplotsplothandlermesh@is@exterior
	}%
	\endgroup
	\pgfplots@glob@TMPc
}%

\def\pgfplotsplothandlermesh@interior@exterior@hook@active{%
	\begingroup
	\pgfplotspatchclass{\pgfplotsplothandlermesh@patchclass}{compute normal}%
	\let\pgfplots@normal=\pgfplotsretval
	\pgfplotsmathvectorlength{\pgfplotsretval}{default}%
	\let\pgfplots@normal@len=\pgfplotsretval
	%
	\pgfplotsmathvectorscalarproduct{\pgfplots@view@dir@threedim}{\pgfplots@normal}{default}%
	\let\pgfplots@dot=\pgfplotsretval
	\pgfplotscoordmath{default}{op}{multiply}{{\pgfplots@view@dir@threedim@len}{\pgfplots@normal@len}}%
	\pgfplotscoordmath{default}{op}{divide}{{\pgfplots@dot}{\pgfmathresult}}%

	\pgfplotscoordmath{default}{if less than}{\pgfmathresult}{\pgfplotsplothandlermesh@interior@thresh}{%
		% <0:
		\global\let\pgfplots@glob@TMPc\pgfplotsplothandlermesh@is@interior
	}{%
		% >=0:
		\global\let\pgfplots@glob@TMPc\pgfplotsplothandlermesh@is@exterior
	}%
	\endgroup
	\pgfplots@glob@TMPc
}%


% only used if
% \pgfplotsplothandlermesh@interior@exterior@hook@active is
% active
\def\pgfplotsplothandlermesh@is@interior{%
%\def\pgfplotsplothandlermesh@usepath{\pgfusepath{discard}}%
	\pgfkeyslet{/pgfplots/colormap name}\pgfplotsplothandlermesh@colormapname@interior
	\pgfkeyslet{/pgfplots/surf shading/colormap}\pgfplotsplothandlermesh@shader@colormap@interior
}%
\def\pgfplotsplothandlermesh@is@exterior{%
%\let\pgfplotsplothandlermesh@usepath=\pgfplotsplothandlermeshusepathfillstroke
	\pgfkeyslet{/pgfplots/colormap name}\pgfplotsplothandlermesh@colormapname
	\pgfkeyslet{/pgfplots/surf shading/colormap}\pgfplotsplothandlermesh@shader@colormap
}%

\def\pgfplotsplothandlermesh@init@cdata{%
	% Ok, we have color data.
	\pgfkeysgetvalue{/pgfplots/colormap name}\pgfplotsplothandlermesh@colormapname
	%
	%
	\if0\pgfplotsplothandlermesh@colorinput
		% mesh/color input=colormap
		\pgfplotscolormapgetcolorcomps{\pgfplotsplothandlermesh@colormapname}%
		\let\pgfplotsplothandlermesh@explicit@color@num@comps=\pgfplotsretval
	\else
		% mesh/color input=explicit
		\pgfkeysgetvalue{/pgfplots/mesh/colorspace explicit color output}\pgfplots@loc@TMPa
		\pgfplotscolorspacegetcomponents{\pgfplots@loc@TMPa}%
		\let\pgfplotsplothandlermesh@explicit@color@num@comps=\pgfplotsretval
	\fi
	%
	\pgfkeysgetvalue{/pgfplots/mesh/interior colormap name}\pgfplotsplothandlermesh@colormapname@interior
	\ifx\pgfplotsplothandlermesh@colormapname@interior\pgfutil@empty
	\else
		\pgfkeysgetvalue{/pgfplots/mesh/interior colormap thresh}\pgfplotsplothandlermesh@interior@thresh
		\pgfplotscoordmath{default}{parsenumber}{\pgfplotsplothandlermesh@interior@thresh}%
		\let\pgfplotsplothandlermesh@interior@thresh=\pgfmathresult
		\pgfplotscoordmath{default}{if is}{\pgfplotsplothandlermesh@interior@thresh}{0}{%
			\let\pgfplotsplothandlermesh@interior@exterior@hook=\pgfplotsplothandlermesh@interior@exterior@hook@active@zerothresh
		}{%
			\pgfplotsmathvectorlength{\pgfplots@view@dir@threedim}{default}% should be 1, but never mind
			\let\pgfplots@view@dir@threedim@len=\pgfplotsretval
			\let\pgfplotsplothandlermesh@interior@exterior@hook=\pgfplotsplothandlermesh@interior@exterior@hook@active
		}%
		% actually store 3d coordinates:
		\let\pgfplotsplothandlermesh@serialize@logical@coords=\pgfplotsplothandlermesh@serialize@logical@coords@doit
	\fi
	%
	% Prepare:
	\def\pgfplotsplothandlermesh@definecolor{%
		% FIXME : use lua backend to improve this here! it is
		% _extremely_ slow!
		\pgfplotsplothandlermesh@get@flat@color
		\pgfplotsplothandlermesh@definemappedcolor{\pgfplotsretval}%
		\pgfplots@drawoptions
	}%
	%
	% Acquire the values for \tikz@mode@* [used for faceted]
	\pgfplots@drawmodes
}

\def\pgfplotsplothandlermesh@definemappedcolor#1{%
	\if0\pgfplotsplothandlermesh@colorinput
		% mesh/color input=colormap
		\if d\pgfplots@colormap@access
			% colormap access=direct
			% we assume that the color *is* already normalized,
			% i.e. it is {<comma-separate-components>} and to be
			% interpreted in the correct color space.
			\edef\pgfplots@loc@TMPb{%
				\noexpand\pgfutil@definecolor{mapped color}
					{\pgfplotscolormapcolorspaceof{\pgfplotsplothandlermesh@colormapname}}%
					{#1}%
			}%
			\pgfplots@loc@TMPb
		\else
			% colormap access=map|piecewise constant
			\pgfplotscolormapdefinemappedcolor{#1}%
		\fi
	\else
		% mesh/color input=explicit
		%
		% we assume that the color *is* already normalized,
		% i.e. it is {<comma-separate-components>} and to be
		% interpreted in the correct color space.
		\edef\pgfplots@loc@TMPb{%
			\noexpand\pgfutil@definecolor{mapped color}
				{\pgfkeysvalueof{/pgfplots/mesh/colorspace explicit color output}}
				{#1}%
		}%
		\pgfplots@loc@TMPb
	\fi
}%

\let\pgfsetstrokecolor@orig=\pgfsetstrokecolor
\def\pgfplotsplothandlermesh@VISUALIZE@std@separate@fillstroke@setstrokecolor#1{%
	\pgfsetstrokecolor@orig{.}%
}%
\def\pgfplotsplothandlermesh@VISUALIZE@std@separate@fillstroke{%
	\pgfplotsplothandlermesh@get@flat@color
	\pgfplotsplothandlermesh@definemappedcolor{\pgfplotsretval}%
	%
	\ifx\pgfplotsplothandlermesh@usepathfill\relax
	\else
		\let\pgfsetstrokecolor=\pgfplotsplothandlermesh@VISUALIZE@std@separate@fillstroke@setstrokecolor
		\pgfplots@drawoptions
		\pgfplotspatchclass{\pgfplotsplothandlermesh@patchclass}{fill path}%
		\pgfplotsplothandlermesh@usepathfill
	\fi
	\ifx\pgfplotsplothandlermesh@usepathstroke\relax
	\else
		\let\pgfsetstrokecolor=\pgfsetstrokecolor@orig
		\pgfplots@drawoptions
		\pgfplotspatchclass{\pgfplotsplothandlermesh@patchclass}{stroke path}%
		\pgfplotsplothandlermesh@usepathstroke
	\fi
}%

% This is one of two places where shadings are served with
% coordinates; the other one uses matrix output (which does not work
% with 'stream to shader')
\def\pgfplotsplothandlermesh@VISUALIZE@interp{%
	\pgfplotspatchclass{\pgfplotsplothandlermesh@patchclass}{stream to shader}%
	\ifpgf@relevantforpicturesize
		\pgfplotspatchclass{\pgfplotsplothandlermesh@patchclass}{foreach vertex}{%
			\global\pgf@x=\pgfplotspatchvertexx\space
			\global\pgf@y=\pgfplotspatchvertexy\space
			% compute the bounding box using the vertices:
			\pgfpointtransformed{}%
			\pgf@protocolsizes{\pgf@x}{\pgf@y}%
		}%
	\fi
}%

\def\pgfplotsplothandlermesh@VISUALIZE@faceted@interp{%
	\pgfplotslibrarysurfstreamstart
	\pgfplotsplothandlermesh@VISUALIZE@interp
	\pgfplotslibrarysurfstreamend
	\pgfplotslibrarysurfdrawinpicture
	%
	\pgfplotsplothandlermesh@VISUALIZE@std
}%

\def\pgfplotsplothandlermesh@PIPE@CHECKJUMP@disable{%
	\def\pgfplotsplothandlermesh@PIPE@CHECKJUMP{\pgfplotsplothandlermesh@PIPE@REFINE}%
}%

\def\pgfplotsplothandlermesh@init@shader{%
	%
	\ifnum\pgfplotsplothandlermesh@shader>1
		% 'shader=2': 'interp':
		% very simple: defer work to surf library. Only
		% provide the mapped colordata.
		%
		% 'shader=3': 'faceted interp':
		% a mix of 'faceted' and 'interp'. (might be inefficient in
		% terms of pdf memory and rendering times)
		%
		% This won't be used for mesh plots, neither in one
		% nor in two dimensions.
		\pgfkeyssetvalue{/pgfplots/surf shading/anchor}{\pgfpointorigin}%
		%
 		% do NOT update the picture's BB using the streamed low-level coordinates
		% (which are bezier control points in most cases).
		% We do that on our own using the input vertices.
		\pgfkeys{/pgfplots/surf shading/update BB=false}%
		\if1\b@pgfplotsplothandlermesh@matrixoutput
			% this *implies* a compatible patch class, see the
			% definition of \b@pgfplotsplothandlermesh@matrixoutput
			\def\pgfplotsretval{5}%
		\else
			\pgfplotspatchclass{\pgfplotsplothandlermesh@patchclass@output}{get pdf shading type}%
			\if5\pgfplotsretval
				% we have no matrix output->
				% downgrade lattice triangular grid to individual
				% triangles:
				\def\pgfplotsretval{4}%
			\fi
		\fi
		\pgfkeyslet{/pgfplots/surf shading/shading type}\pgfplotsretval%
		%
		\pgfplotsplothandlermesh@set@shader@colormap
		\pgfkeysgetvalue{/pgfplots/mesh/interior colormap name}\pgfplots@loc@TMPa
		\ifx\pgfplots@loc@TMPa\pgfutil@empty
		\else
			\let\pgfplotsplothandlermesh@shader@colormap=\pgfplotsretval
			\begingroup
			\pgfkeyslet{/pgfplots/colormap name}\pgfplots@loc@TMPa
			\pgfplotsplothandlermesh@set@shader@colormap
			\pgfmath@smuggleone\pgfplotsretval
			\endgroup
			\let\pgfplotsplothandlermesh@shader@colormap@interior=\pgfplotsretval
		\fi
		%
		\if0\pgfplotsplothandlermesh@colorinput
			% mesh/color input=colormap
			\pgfkeysgetvalue{/pgfplots/colormap name}\pgfplots@loc@TMPa
			\pgfplotscolormapgetcolorspace\pgfplots@loc@TMPa
			\pgfkeyslet{/pgfplots/surf shading/colorspace}\pgfplotsretval
		\else
			% mesh/color input=explicit
			\pgfkeysgetvalue{/pgfplots/mesh/colorspace explicit color output}\pgfplots@loc@TMPa
			\edef\pgfplots@loc@TMPa{\pgfplots@loc@TMPa}%
			\pgfkeyslet{/pgfplots/surf shading/colorspace}\pgfplots@loc@TMPa
		\fi
		\if2\pgfplotsplothandlermesh@shader
			% shader=interp
			\let\pgfplotsplothandlermesh@definecolor=\relax
			\let\pgfplotsplothandlermesh@VISUALIZE=\pgfplotsplothandlermesh@VISUALIZE@interp%
			\pgfkeyslet{/pgfplots/surf shading/cols}\pgfplotsplothandlermesh@scanlinelength
			%\if1\b@pgfplotsplothandlermesh@matrixoutput
			\iffalse% FIXME : is this special handling necessary!?
				\if\pgfplots@plot@mesh@ordering0%
					% ordering=x varies:
					% nothing special to do
				\else
					% FIXME
					% [this happens only for mesh input=lattice]
					% colwise:
					%
					% The low level shader NEEDS rowwise ordering. So, we need to transpose the data matrix.
					% As usual in TeX, we should save macros as much as possible. I guess we NEED at least one macro for each matrix row.
					% Idea:
					% while we iterate colwise through the input data, accumulate row vectors.
					% Then, process the row vectors.
					% Since vectors are expensive (in terms of number of macros), we use fast "append" lists instead of vectors.
					\let\pgfplotsplothandlermesh@shader@interp=\pgfplotsplothandlermesh@shader@interp@incremental@transpose
					%
					% init all required row vectors:
					\c@pgfplots@scanlineindex=0
					\pgfutil@loop
					\ifnum\c@pgfplots@scanlineindex<\pgfplotsplothandlermesh@rows\relax
						\expandafter\pgfplotsapplistXnewempty\csname pgfp@transprow@\the\c@pgfplots@scanlineindex\endcsname%
						\advance\c@pgfplots@scanlineindex by1
					\pgfutil@repeat
					%\pgfkeyslet{/pgfplots/surf shading/cols}\pgfplotsplothandlermesh@rows
					%\pgfplots@error{Sorry, the 'shader=interp' does not yet support 'mesh/ordering=colwise'... You can continue now, but the plot will be transposed.}%
				\fi
			\fi
			\pgfplotspatchclass{\pgfplotsplothandlermesh@patchclass@output}{init pdf shading}%
			\pgfplotslibrarysurfstreamstart
		\else
			% shader='faceted interp'
			\def\pgfplotsplothandlermesh@VISUALIZE{\pgfplotsplothandlermesh@VISUALIZE@faceted@interp}%
			\pgfkeyssetvalue{/pgfplots/surf shading/cols}{2}%
			\pgfplotspatchclass{\pgfplotsplothandlermesh@patchclass@output}{init pdf shading}%
		\fi
	\else
		\def\pgfplotsplothandlermesh@VISUALIZE{\pgfplotsplothandlermesh@VISUALIZE@std}%
	\fi
	%
	%%
	\ifpgfplotsplothandlermesh@internaldepth
	\else
		% disable internal sorting.
		\let\pgfplotspatchclass@rect@to@coonspatch@findshadingsequence@for@depth=\pgfplotspatchclass@rect@to@coonspatch@findshadingsequence@for@depth@no@op%
		\let\pgfplotspatchclass@tri@to@coonspatch@findshadingsequence@for@depth=\pgfplotspatchclass@tri@to@coonspatch@findshadingsequence@for@depth@no@op
	\fi
}%

\def\pgfplotsplothandlermesh@init@flat@color{%
	\ifcase\pgfplotsplothandlermesh@shader
		% 0 == 'flat'
		\def\pgfplotsplothandlermesh@needs@flat@color{1}%
	\or
		% 1 == 'faceted'
		\def\pgfplotsplothandlermesh@needs@flat@color{1}%
	\or
		% 2 == 'interp'
		\def\pgfplotsplothandlermesh@needs@flat@color{0}%
	\or
		% 3 == 'faceted interp'
		\def\pgfplotsplothandlermesh@needs@flat@color{1}%
	\fi
	\if1\pgfplotsplothandlermesh@needs@flat@color
		\pgfplotsplothandlermesh@init@flat@color@getter
		%
		\pgfplotsplothandlermesh@init@flat@color@usepath
		%
		%
		\pgfplotsplothandlermesh@init@flat@color@visualize@pipe@step
	\fi
}%

\def\pgfplotsplothandlermeshusepathstroke{\pgfusepathqstroke}%
\def\pgfplotsplothandlermeshusepathfill{\pgfusepathqfill}%
\def\pgfplotsplothandlermeshusepathfillstroke{\pgfusepathqfillstroke}%

\def\pgfplotsplothandlermesh@init@flat@color@visualize@pipe@step{%
	\ifx\pgfplotsplothandlermesh@PIPE@REFINE\pgfplotsplothandlermesh@PIPE@REFINE@no@op
		\let\pgfplotsplothandlermesh@VISUALIZE@std=\pgfplotsplothandlermesh@VISUALIZE@std@fill@andor@stroke
	\else
		\ifpgfplotsplothandlermesh@hide@refined@edges
			% this is (at most) partially implemented.
			\let\pgfplotsplothandlermesh@VISUALIZE@std=\pgfplotsplothandlermesh@VISUALIZE@std@separate@fillstroke
		\else
			\let\pgfplotsplothandlermesh@VISUALIZE@std=\pgfplotsplothandlermesh@VISUALIZE@std@fill@andor@stroke
		\fi
		%
		% FIXME : that doesn't work.
		% filling *always* seems to need a stroke operation,
		% otherwise there will be translucent edges.
		% I need to adjust the stroke color :-(
		%
		% FIXME : even that doesn't work correctly. The problem
		% is: drawing adjacent patches on top of each other
		% OVERDRAWS edges: once with the correct stroke color;
		% once with the fill color.
		%
		\ifx\pgfplotsplothandlermesh@usepath\pgfplotsplothandlermeshusepathstroke
			\let\pgfplotsplothandlermesh@usepathfill=\relax
			\let\pgfplotsplothandlermesh@usepathstroke=\plotsplothandlermeshusepathstroke
		\else
			\ifx\pgfplotsplothandlermesh@usepath\pgfplotsplothandlermeshusepathfillstroke
				\let\pgfplotsplothandlermesh@usepathfill=\pgfplotsplothandlermeshusepathfillstroke
				\let\pgfplotsplothandlermesh@usepathstroke=\pgfplotsplothandlermeshusepathstroke
			\else
				\let\pgfplotsplothandlermesh@usepathfill=\relax
				\let\pgfplotsplothandlermesh@usepathstroke=\relax
			\fi
		\fi
	\fi
}%

\def\pgfplotsplothandlermesh@init@flat@color@usepath{%
	\if\pgfplots@meshmode m%
		% 'mesh' plot: stroke only and use the 'flat' code
		% here just to define the color for each mesh
		% segment.
		\let\pgfplotsplothandlermesh@usepath=\pgfplotsplothandlermeshusepathstroke
	\else
		% 'surf' plot:
		\let\pgfplotsplothandlermesh@usepath=\pgfplotsplothandlermeshusepathfillstroke
		\if\pgfplotsplothandlermesh@shader3% faceted interp
			% it has an extra fill command (shading)
			\let\pgfplotsplothandlermesh@usepath=\pgfplotsplothandlermeshusepathstroke
		\fi
		%
		% the distinction between 'faceted' and 'flat' is
		% technical: it is just the default draw color.
		%
		% 'flat,draw=black' is the same as 'faceted'.
		\pgfkeysgetvalue{/pgfplots/faceted color}\pgfplots@loc@TMPb
		\def\pgfplots@loc@TMPc{none}%
		\ifx\pgfplots@loc@TMPb\pgfplots@loc@TMPc
			\pgfkeyslet{/pgfplots/faceted color}\pgfutil@empty
			\def\pgfplots@loc@TMPb{}%
		\fi
		%
		\def\pgfplots@loc@TMPa{0}%
		\if\pgfplotsplothandlermesh@shader1% faceted
			\def\pgfplots@loc@TMPa{1}%
			\ifx\pgfplots@loc@TMPb\pgfutil@empty
				% oh. no faceted color. Then: fill only, use
				% whatever color is available.
				\def\pgfplots@loc@TMPa{0}%
				%\let\pgfplotsplothandlermesh@usepath=\pgfplotsplothandlermeshusepathfill
			\fi
		\fi
		\if\pgfplotsplothandlermesh@shader3% faceted interp
			\def\pgfplots@loc@TMPa{1}%
			\ifx\pgfplots@loc@TMPb\pgfutil@empty
				% oh. no faceted color. Then: no-op!
				% This is the same as interp with more overhead
				% (but allows z buffering).
				\def\pgfplots@loc@TMPa{0}%
				\def\pgfplotsplothandlermesh@usepath{\pgfusepath{discard}}%
			\fi
		\fi
		\if1\pgfplots@loc@TMPa
			% make sure there is a the correct color.
			\expandafter\def\expandafter\pgfplots@drawoptions\expandafter{%
				\pgfplots@drawoptions
				\pgfsetstrokecolor{\pgfkeysvalueof{/pgfplots/faceted color}}%
			}%
		\fi
		%
	\fi
}

\def\pgfplotsplothandlermesh@init@flat@color@getter{%
	\if\pgfplotsplothandlermesh@flatmode c% 'flat corner':
		%
		% use colordata of ONE corner point. That's easy, not
		% much to do.
		\if1\b@pgfplots@correct@shader@flat
			\def\pgfplotsplothandlermesh@get@flat@color{%
				\pgfplotspatchclass{\pgfplotsplothandlermesh@patchclass@output}{first vertex before z buffer}%
				\let\pgfplots@current@point@meta=\pgfplotspatchvertexmeta
				\pgfplotsaxisvisphasetransformpointmeta
				\let\pgfplotsretval=\pgfplotspointmetatransformed%
			}%
		\else
			\def\pgfplotsplothandlermesh@get@flat@color{%
				\pgfplotspatchclass{\pgfplotsplothandlermesh@patchclass@output}{first vertex}%
				\let\pgfplots@current@point@meta=\pgfplotspatchvertexmeta
				\pgfplotsaxisvisphasetransformpointmeta
				\let\pgfplotsretval=\pgfplotspointmetatransformed%
			}%
		\fi
	\else
		% 'flat mean':
		%
		% use colordata as mean of all corners for shading.
		\if0\pgfplotsplothandlermesh@colorinput
			% mesh/color input=colormap
			\if d\pgfplots@colormap@access
				% colormap access=direct
				% interpolate adjacent colors, not their indices:
				\pgfplotsplothandlermesh@get@flat@color@explicit@prepare
				\def\pgfplotsplothandlermesh@get@flat@color{%
					\pgfplotsplothandlermesh@get@flat@color@explicit@accum{%
						\begingroup
						\let\pgfplots@current@point@meta=\pgfplotspatchvertexmeta
						\pgfplotsaxisvisphasetransformpointmeta
						\let\pgfplotspatchvertexmeta=\pgfplotspointmetatransformed
						\pgfmath@smuggleone\pgfplotspatchvertexmeta
						\endgroup
					}%
				}%
			\else
				% colormap access=map|piecewise constant
				\def\pgfplotsplothandlermesh@get@flat@color{%
					\global\pgfplots@tmpa=0pt
					\pgfplotspatchclass{\pgfplotsplothandlermesh@patchclass@output}{foreach cdata vertex}{%
						\let\pgfplots@current@point@meta=\pgfplotspatchvertexmeta
						\pgfplotsaxisvisphasetransformpointmeta
						\global\advance\pgfplots@tmpa by\pgfplotspointmetatransformed pt
					}%
					\pgfplotspatchclass{\pgfplotsplothandlermesh@patchclass@output}{get num cdata vertices}%
					\global\divide\pgfplots@tmpa by\pgfplotsretval\relax
					\edef\pgfplotsretval{\pgf@sys@tonumber\pgfplots@tmpa}%
					\global\pgfplots@tmpa=0pt
				}%
			\fi
		\else
			% mesh/color input=explicit
			\pgfplotsplothandlermesh@get@flat@color@explicit@prepare
			%
			\def\pgfplotsplothandlermesh@get@flat@color{%
				\pgfplotsplothandlermesh@get@flat@color@explicit@accum{}%
				\pgfplotsaxisvisphasetransformpointmeta
			}%
		\fi
	\fi
}%

\def\pgfplotsplothandlermesh@get@flat@color@explicit@prepare{%
	\pgfplotspatchclass{\pgfplotsplothandlermesh@patchclass@output}{get num cdata vertices}%
	\pgf@xa=1pt %
	\divide\pgf@xa by\pgfplotsretval\relax
	\edef\pgfplotsplothandlermesh@get@flat@color@@{\pgf@sys@tonumber\pgf@xa}%
}%
\def\pgfplotsplothandlermesh@get@flat@color@explicit@accum#1{%
	\pgfplotscolorzero{\pgfplotsplothandlermesh@explicit@color@num@comps}%
	\pgfplotspatchclass{\pgfplotsplothandlermesh@patchclass@output}{foreach cdata vertex}{%
		#1%
		\pgfplotscoloraddweighted
			{\pgfplotsplothandlermesh@explicit@color@num@comps}%
			{\pgfplotsretval}
			{\pgfplotsplothandlermesh@get@flat@color@@}%
			{\pgfplotspatchvertexmeta}%
	}%
	% FIXME : this here sounds wrong: !? what is this good for:
	\let\pgfplots@current@point@meta=\pgfplotspatchvertexmeta
}%

\def\pgfplotsplothandlermesh@init@jumphandling{%
	% For matrix input, a "jump" should result in something
	% visible although it must not destroy the matrix structure.
	%
	% try this here:
	\def\pgfplotsplothandlervisualizejump{%
		\pgfplotstreampoint{\global\pgf@x=-16000.0pt \global\pgf@y=-16000.0pt }%
	}%
	%
	% Prepare jumps according to point meta:
	\ifx\pgfplotspointmetainputhandler\pgfutil@empty%
		% oh. No point meta!?
		\let\pgfplotsplothandlermesh@PIPE@CHECKJUMP@meta=\pgfplotsplothandlermesh@PIPE@CHECKJUMP@meta@noop
	\else
		% we have point meta. This is the standard case.
		%
		% prepare the routine for "jump on unbounded point meta":
		\ifcase\pgfplotsplothandlermesh@shader
			% 0 == 'flat'
			\def\pgfplots@loc@TMPa{1}%
		\or
			% 1 == 'faceted'
			\def\pgfplots@loc@TMPa{1}%
		\else
			\def\pgfplots@loc@TMPa{0}%
		\fi
		%
		\if 1\pgfplots@loc@TMPa
			% ok, this uses (only) 'shader=flat'
			\if c\pgfplotsplothandlermesh@flatmode
				% 'flat corner':
				\if1\b@pgfplots@correct@shader@flat
					\let\pgfplotsplothandlermesh@PIPE@CHECKJUMP@meta=\pgfplotsplothandlermesh@PIPE@CHECKJUMP@meta@firstvertex
				\else
					\let\pgfplotsplothandlermesh@PIPE@CHECKJUMP@meta=\pgfplotsplothandlermesh@PIPE@CHECKJUMP@meta@firstvertex@legacy
				\fi
			\else
				% 'flat mean':
				\let\pgfplotsplothandlermesh@PIPE@CHECKJUMP@meta=\pgfplotsplothandlermesh@PIPE@CHECKJUMP@meta@everyvertex
			\fi
		\else
			\let\pgfplotsplothandlermesh@PIPE@CHECKJUMP@meta=\pgfplotsplothandlermesh@PIPE@CHECKJUMP@meta@everyvertex
		\fi
	\fi
	%
	\if0\pgfplotsaxisplothasunboundedpointmeta
		\let\pgfplotsplothandlermesh@PIPE@CHECKJUMP@meta=\pgfplotsplothandlermesh@PIPE@CHECKJUMP@meta@noop
	\fi
	\if0\pgfplotsaxisplothasjumps
		\ifx\pgfplotsplothandlermesh@PIPE@CHECKJUMP@meta\pgfplotsplothandlermesh@PIPE@CHECKJUMP@meta@noop
			\pgfplotsplothandlermesh@PIPE@CHECKJUMP@disable
		\fi
	\fi
}%


\def\pgfplotsplothandlermesh@init@zbuffer{%
	\ifpgfplots@curplot@threedim
		\pgfplotspatchclass{\pgfplotsplothandlermesh@patchclass@output}{uses view depth}%
		\if1\pgfplotsretval
			\let\pgfplotsplothandlermesh@compute@vertex@depth=\pgfplotsplothandlermesh@compute@vertex@depth@doit%
		\fi
		%
		\if\pgfplotsplothandlermesh@zbuffer@choice4%
			\def\pgfplotsplothandlermesh@zbuffer@sort@in@TeX{1}%
		\else
			\def\pgfplotsplothandlermesh@zbuffer@sort@in@TeX{0}%
		\fi
		\if1\b@pgfplots@LUA@visualization@enabled
			% set it to 'z buffer=none' -- LUA did the job already.
			\def\pgfplotsplothandlermesh@zbuffer@sort@in@TeX{0}%
		\fi
		%
		\if\pgfplotsplothandlermesh@zbuffer@sort@in@TeX1%
			% z buffer=sort:
			% create a z buffer:
			% the z buffer contains elements {<view depth>}{{<pt1>}{<pt2>}{<pt3>}{<pt4>}}
			\pgfplotsarraynewempty\pgfplots@zbuffer@local
			%
			\let\pgfplotsplothandlermesh@compute@vertex@depth=\pgfplotsplothandlermesh@compute@vertex@depth@doit%
			%
			% the "element depth" is defined to be the MEAN of all
			% vertex depths.
			% And since the mean is 1/n * sum_{i=1}^n V_i, we can
			% directly omit the 1/n --- it is the same for every
			% vertex anyway, and we only want to compare the depth
			% values.
			\let\pgfplotsplothandlermesh@accum@element@depth=\pgfmath@basic@add@
			%
			% and collect each point into the z buffer
			% (instead of drawing it):
			\def\pgfplotsplothandlermesh@PIPE@FINISH{%
				%
				% compute DEPTH(lattice element):= min{DEPTH(A),DEPTH(B),DEPTH(C),DEPTH(D)}
				\def\pgfmathresult{0}%
				\def\pgfplotsplothandlermesh@zbuffer@tmp{}%
				\pgfplotspatchclass{\pgfplotsplothandlermesh@patchclass@output}{foreach vertex}{%
					\pgfplotsplothandlermesh@accum@element@depth\pgfplotspatchvertexdepth\pgfmathresult
					\t@pgfplots@toka=\expandafter{\pgfplotsplothandlermesh@zbuffer@tmp}%
					\t@pgfplots@tokb=\expandafter{\pgfplotspatchvertexstruct}%
					\edef\pgfplotsplothandlermesh@zbuffer@tmp{%
						\the\t@pgfplots@toka
						{\the\t@pgfplots@tokb}%
					}%
				}%
				\pgfplotspatchclass{\pgfplotsplothandlermesh@patchclass@output}{serialize except vertices}%
				\t@pgfplots@toka=\expandafter{\pgfplotsretval}%
				\t@pgfplots@tokb=\expandafter{\pgfplotsplothandlermesh@zbuffer@tmp}%
				\edef\pgfplotsplothandlermesh@zbuffer@tmp{%
					{\the\t@pgfplots@toka}% first argument is meta data, then follow the vertices
					\the\t@pgfplots@tokb%
				}%
				%
				% assemble array element:
				\t@pgfplots@toka=\expandafter{\pgfplotsplothandlermesh@zbuffer@tmp}%
				\edef\pgfplots@loc@TMPa{%
					{\pgfmathresult}%
					{%
						\the\t@pgfplots@toka
					}%
				}%
				\expandafter\pgfplotsarraypushback\pgfplots@loc@TMPa\to\pgfplots@zbuffer@local
			}%
		\fi
	\else
		\def\pgfplotsplothandlermesh@zbuffer@sort@in@TeX{0}%
	\fi
}%
\def\pgfplotsplothandlermesh@init@mesh@stream{%
	\global\let\pgf@plotstreamspecial=\pgfutil@gobble%
	\global\let\pgf@plotstreamend=\pgfplotsplothandlermesh@finish%
	\global\let\pgf@plotstreampoint=\pgfplotsplothandlermesh@stream
	\pgfplotspatchclass{\pgfplotsplothandlermesh@patchclass}{new}%
}

\def\pgfplotsplothandlermesh@set@shader@colormap{%
	\if0\pgfplotsplothandlermesh@colorinput
		% mesh/color input=colormap
		\if d\pgfplots@colormap@access
			% colormap access=direct
			% tell it to expect explicit colors:
			\pgfkeyslet{/pgfplots/surf shading/colormap}\pgfutil@empty
		\else
			% colormap access=map|piecewise constant
			\pgfkeysgetvalue{/pgfplots/colormap name}\pgfplots@loc@TMPa
			\expandafter\pgfplotscolormaptopdffunction\expandafter{\pgfplots@loc@TMPa}%
			\pgfkeyslet{/pgfplots/surf shading/colormap}\pgfplotsretval
		\fi
	\else
		% mesh/color input=explicit
		%
		% tell it to expect explicit colors:
		\pgfkeyslet{/pgfplots/surf shading/colormap}\pgfutil@empty
	\fi
}%

\def\pgfplotsplothandlermesh@check@unbounded@value{-16000.0pt}%
{
\dimen0=\pgfplotsplothandlermesh@check@unbounded@value
\xdef\pgfplotsplothandlermesh@check@unbounded@text{\the\dimen0}%
}

\def\pgfplotsplothandlermesh@check@unbounded{%
	\ifx\pgfplotspatchvertexx\pgfplotsplothandlermesh@check@unbounded@text
		\pgfplots@loc@tmpfalse
	\fi
}%

\long\def\pgfplotsplothandlermesh@defaultlegend@img#1{%
	\scope[%
		#1,
		/pgfplots/mesh/rows=3,
		/pgfplots/mesh/cols=3,
		/pgfplots/mesh/num points=,
		/pgfplots/mesh input=lattice,
		/pgfplots/patch type=rectangle,
		/pgfplots/z buffer=default,
		/tikz/x={(0.44237cm,-0.07439cm)},
		/tikz/y={(0.30942cm,0.23932cm)},
		/tikz/z={(0.0cm,1.5cm)},
		scale=0.4,
		yshift=-0.7cm,
		/pgfplots/point meta=explicit,
		/pgfplots/mesh/color input=colormap,
	]
	\let\pgfplots@metamax=\pgfutil@empty
	\pgfplots@curplot@threedimtrue

	\pgfplotsplothandlermesh
	\pgfplotstreamstart

	\def\rangea{0.21}%
	\def\rangeb{0.7}%

	\pgfmathparse{1000/(\rangeb-\rangea)}
	\let\factor=\pgfmathresult

	\def\simplecoordinate(##1,##2,##3){%
		\pgfmathparse{\factor*(##3 - \rangea)}%
		\pgfmathfloatparsenumber\pgfmathresult
		\let\pgfplots@current@point@meta=\pgfmathresult
		\pgfplotstreampoint{\pgfqpointxyz{##1}{##2}{##3}}%
	}%
	% for the case 'shader=interp':
	%\def\pgfplotspointbbupperright{\pgfqpointxyz{2}{2}{0.9}}
	%\def\pgfplotspointbblowerleft {\pgfqpointxyz{0}{0}{0.1}}

	\simplecoordinate(0,2,0.7)
	\simplecoordinate(1,2,0.5)
	\simplecoordinate(2,2,0.43)

	\simplecoordinate(0,1,0.68)
	\simplecoordinate(1,1,0.4)
	\simplecoordinate(2,1,0.25)

	\simplecoordinate(0,0,0.6)
	\simplecoordinate(1,0,0.35)
	\simplecoordinate(2,0,0.21)

	\pgfplotstreamend
	\pgfusepath{stroke}
	\endscope
}%

\def\pgfplotsplothandlermesh@finish{%
	\if\pgfplotsplothandlermesh@zbuffer@sort@in@TeX1%
		% z buffer=sort:
		% process the z buffer:
		\pgfkeyslet{/pgfplots/iflessthan/.@cmd}\pgfplotsplothandlermesh@zbuffer@sort@iflessthen
		\pgfkeysdef{/pgfplots/array/unscope pre}{%
			\let\pgfplotsplothandlermesh@patchclass=\pgfplotsplothandlermesh@patchclass@output
			\def\pgfplotspatchready{\pgfplotsplothandlermesh@PIPE@interior@exterior@styles}%
			\pgfplotsarrayforeachungrouped\pgfplots@zbuffer@local\as\curelem{%
				\expandafter\pgfplotsplothandlermesh@draw@zbufferelement\curelem%
			}%
		}%
		\pgfkeysdef{/pgfplots/array/unscope post}{}%
		\pgfplotsarraysort\pgfplots@zbuffer@local
	\fi
	%
	\if2\pgfplotsplothandlermesh@shader
		% shader=interp:
		% FIXME : broken!
		%\ifx\pgfplotsplothandlermesh@shader@interp\pgfplotsplothandlermesh@shader@interp@incremental@transpose%
			% This is the case for 'mesh/ordering=colwise'. We need to finish the transpose procedure:
		%	\pgfplotsplothandlermesh@shader@interp@transpose@finish
		%\fi
		\pgfplotslibrarysurfstreamend
		\pgfplotslibrarysurfdrawinpicture
	\fi
}%


% This is one of two places where the shader is served with
% coordinates: it handles the special matrix-form shading which is
% incompatible with the patch-wise sequence of coordinates
\def\pgfplotsplothandlermesh@PIPE@DECODE@to@shader#1{%
	\pgfplotsaxisvisphasetransformpointmeta
	\ifpgf@relevantforpicturesize
		\pgfpointtransformed{#1}%
		\pgf@protocolsizes{\pgf@x}{\pgf@y}%
	\fi
	\pgfplotslibrarysurfstreamcoord
		{#1}%
		{\pgfplotspointmetatransformed}%
}%

\def\pgfplotsplothandlermesh@PIPE@DECODE@matrix@ONEDIMMODE#1{%
	\let\pgfplotsplothandlermesh@patchclass\pgfplotsplothandlermesh@patchclass@input
	\pgfplotsplothandlermesh@compute@vertex@depth% <-- this modifies \pgfplotsretval
	\let\pgfplots@view@depth=\pgfplotsretval
	\pgf@process{#1}%
	\pgfplotsplothandlermesh@serialize@logical@coords
	\edef\pgfplotsplothandlermesh@i{{\the\pgf@x}{\the\pgf@y}{\pgfplots@current@point@meta}\pgfplots@view@depth:\pgfplotsretval}%
	\ifnum\c@pgfplots@scanlineindex>0
		\expandafter\pgfplotsplothandlermesh@setnextvertex\expandafter{\pgfplotsplothandlermesh@im}%
		\expandafter\pgfplotsplothandlermesh@setnextvertex\expandafter{\pgfplotsplothandlermesh@i}%
	\else
		\advance\c@pgfplots@scanlineindex by1
	\fi
	\let\pgfplotsplothandlermesh@im=\pgfplotsplothandlermesh@i
}%

\def\pgfplotsplothandlermesh@PIPE@DECODE@matrix@fillscanline@common#1{%
	\pgfplotsplothandlermesh@compute@vertex@depth% <-- this modifies \pgfplotsretval
	\let\pgfplots@view@depth=\pgfplotsretval
	\pgf@process{#1}%
	\pgfplotsplothandlermesh@serialize@logical@coords
	\edef\pgfplots@loc@TMPa{{\the\pgf@x}{\the\pgf@y}{\pgfplots@current@point@meta}\pgfplots@view@depth:\pgfplotsretval}%
	\expandafter\pgfplotsdequepushback\pgfplots@loc@TMPa\to{lastscanline}%
	\advance\c@pgfplots@scanlineindex by1
}%

% the pgf@plotstreampoint routine for the first scan line:
\def\pgfplotsplothandlermesh@PIPE@DECODE@matrix@fillscanline#1{%
	\pgfplotsplothandlermesh@PIPE@DECODE@matrix@fillscanline@common{#1}%
	\ifnum\c@pgfplots@scanlineindex=\pgfplotsplothandlermesh@scanlinelength\relax
		% second line!
		\let\pgfplotsplothandlermesh@PIPE@DECODE=\pgfplotsplothandlermesh@PIPE@DECODE@matrix%%
		\c@pgfplots@scanlineindex=0
	\fi
}%

% This is the pgf@plotstreampoint for all scan lines starting with the
% second.
%
% We perform
% // i == y
% // j == x
% for i = 1  to rows (note the 1! 0 would be the first row)
% 	for j = 0 to cols
% 		get (i-1, j) from front of deque // remember: i: rows, j : cols. We iterate rowwise.
% 		push (i,j) to back of deque
% 		if j > 0
% 			draw
% 				(i-1,j-1) [from last iteration]  -- (i-1,j  ) [from deque]
% 			--  (  i,j  ) [from last iteration]  -- (i  ,j-1) [from #1]
% 			-- cycle
% 		endif
% 		let (i-1,j-1) := (i-1,j) // for next iteration
% 		let (  i,j-1) := (  i,j) // for next iteration
% 	end
% end
%
% This is a SWEEP through the data matrix.
% #1: the current point of the linearized data matrix
% #2: a callback (TeX code) which will be executed for each point with j>0
% #3: a callback (TeX code) which will be executed if the scanline is completed, i.e. j=cols
\def\pgfplotsplothandlermesh@PIPE@DECODE@matrix@collect@and@callback#1#2#3{%
	% NOTATION:
	% 	'im == (i-1)'
	% 	'jm == (j-1)'
	\pgfplotsplothandlermesh@compute@vertex@depth% <-- this modifies \pgfplotsretval
	\let\pgfplots@view@depth=\pgfplotsretval
	\pgf@process{#1}%
	\pgfplotsplothandlermesh@serialize@logical@coords
	\edef\pgfplotsplothandlermesh@i@j{{\the\pgf@x}{\the\pgf@y}{\pgfplots@current@point@meta}\pgfplots@view@depth:\pgfplotsretval}% FIXME relies on pgfplots macros
%\message{##\the\c@pgfplots@scanlineindex: prepare rectangle with corner \meaning\pgfplotsplothandlermesh@i@j.}%
	\pgfplotsdequepopfront{lastscanline}\to\pgfplotsplothandlermesh@im@j
	\expandafter\pgfplotsdequepushback\pgfplotsplothandlermesh@i@j\to{lastscanline}%
	\ifnum\c@pgfplots@scanlineindex>0
		#2%
	\fi
	\let\pgfplotsplothandlermesh@im@jm=\pgfplotsplothandlermesh@im@j
	\let\pgfplotsplothandlermesh@i@jm=\pgfplotsplothandlermesh@i@j
	\advance\c@pgfplots@scanlineindex by1
	\ifnum\c@pgfplots@scanlineindex=\pgfplotsplothandlermesh@scanlinelength\relax
		#3%
	\fi
}%
\def\pgfplotsplothandlermesh@PIPE@DECODE@matrix#1{%
	\pgfplotsplothandlermesh@PIPE@DECODE@matrix@collect@and@callback{#1}{%
		\let\pgfplotsplothandlermesh@patchclass\pgfplotsplothandlermesh@patchclass@input
		\expandafter\pgfplotsplothandlermesh@setnextvertex\expandafter{\pgfplotsplothandlermesh@im@jm}%
		\expandafter\pgfplotsplothandlermesh@setnextvertex\expandafter{\pgfplotsplothandlermesh@im@j}%
		\expandafter\pgfplotsplothandlermesh@setnextvertex\expandafter{\pgfplotsplothandlermesh@i@j}%
		\expandafter\pgfplotsplothandlermesh@setnextvertex\expandafter{\pgfplotsplothandlermesh@i@jm}%
	}{%
		% if scan line is completed, start the next one:
		\c@pgfplots@scanlineindex=0
	}%
}%


\def\pgfplotsplothandlermesh@shader@interp@incremental@transpose#1{%
	\pgfplotsaxisvisphasetransformpointmeta
	\pgf@process{#1}%
	\edef\pgfplotsplothandlermesh@i{\noexpand\pgfplotsplothandlermesh@shader@interp@next{\the\pgf@x}{\the\pgf@y}{\pgfplotspointmetatransformed}}%
	%
	% Store the postprocessed point into the appropriate row vector:
	\def\pgfplots@loc@TMPa{\expandafter\pgfplotsapplistXpushback\expandafter{\pgfplotsplothandlermesh@i}\to}
	\expandafter\pgfplots@loc@TMPa\csname pgfp@transprow@\the\c@pgfplots@scanlineindex\endcsname%
	%
	\advance\c@pgfplots@scanlineindex by1
	\ifnum\c@pgfplots@scanlineindex=\pgfplotsplothandlermesh@scanlinelength\relax
		\c@pgfplots@scanlineindex=0
	\fi
}%
\def\pgfplotsplothandlermesh@shader@interp@transpose@finish{%
	% Now, "execute" each row vector:
	\def\pgfplotsplothandlermesh@shader@interp@next##1##2##3{%
		\pgfplotslibrarysurfstreamcoord{\global\pgf@x=##1 \global\pgf@y=##2 }{##3}%
	}%
	\c@pgfplots@scanlineindex=0
	\pgfutil@loop
	\ifnum\c@pgfplots@scanlineindex<\pgfplotsplothandlermesh@rows\relax
		\expandafter\pgfplotsapplistXlet\expandafter\pgfplots@loc@TMPd\expandafter=\csname pgfp@transprow@\the\c@pgfplots@scanlineindex\endcsname%
		\expandafter\pgfplotsapplistXnewempty\csname pgfp@transprow@\the\c@pgfplots@scanlineindex\endcsname% clear
		% execute the row:
		\pgfplots@loc@TMPd
		\advance\c@pgfplots@scanlineindex by1
	\pgfutil@repeat
	\let\pgfplots@loc@TMPd=\relax
}%

\def\pgfplotsplothandlermesh@draw@zbufferelement#1#2{%
	\pgfplotsplothandlermesh@draw@zbufferelement@#2\pgfplots@EOI
}
\def\pgfplotsplothandlermesh@draw@zbufferelement@#1{%
	\pgfplotspatchclass{\pgfplotsplothandlermesh@patchclass@output}{deserialize except vertices}{#1}%
	\pgfplotsplothandlermesh@draw@zbufferelement@@
}
\def\pgfplotsplothandlermesh@draw@zbufferelement@@#1{%
	\def\pgfplots@loc@TMPa{#1}%
	\ifx\pgfplots@loc@TMPa\pgfplots@EOI
	\else
		\pgfplotspatchclass{\pgfplotsplothandlermesh@patchclass@output}{set next vertex}{#1}%
		\expandafter\pgfplotsplothandlermesh@draw@zbufferelement@@
	\fi
}%


\def\pgfplotsplothandlermesh@zbuffer@sort@iflessthen#1#2#3#4\pgfeov{%
	\edef\pgfplots@loc@TMPa{#1}%
	\expandafter\pgfplotsplothandlermesh@zbuffer@sort@iflessthen@\pgfplots@loc@TMPa\relax
	\let\pgfplotsplothandlermesh@zbuffer@sort@iflessthen@a=\pgfmathresult
	\edef\pgfplots@loc@TMPb{#2}%
	\expandafter\pgfplotsplothandlermesh@zbuffer@sort@iflessthen@\pgfplots@loc@TMPb\relax
	\let\pgfplotsplothandlermesh@zbuffer@sort@iflessthen@b=\pgfmathresult
	\ifdim\pgfplotsplothandlermesh@zbuffer@sort@iflessthen@b pt<\pgfplotsplothandlermesh@zbuffer@sort@iflessthen@a pt%
		#3\relax
	\else
		#4\relax
	\fi
}%
\def\pgfplotsplothandlermesh@zbuffer@sort@iflessthen@#1#2{%
	\def\pgfmathresult{#1}%
}%
\def\pgfplotsplothandlermesh@unpack@zbuffer@pointdepth#1{%
	\expandafter\pgfplotsplothandlermesh@unpack@zbuffer@pointdepth@#1%
}%
\def\pgfplotsplothandlermesh@unpack@zbuffer@pointdepth@#1#2#3#4{%
	\def\pgfplotsretval{#4}%
}%
