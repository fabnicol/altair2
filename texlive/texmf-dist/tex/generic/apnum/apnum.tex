% Arbitrary precision numbers
%%%%%%%%%%%%%%%%%%%%%%%%%%%%%
% 2014, 2015       Petr Olsak

% See the documentation apnum.pdf or apnum.d for more information

\def\apnumversion{1.1 <Jan. 2015>}
\message{The Arbitrary Precision Numbers, \apnumversion}

%%%%%%%%%%%% Internal registers:

\newcount\apnumA \newcount\apnumB \newcount\apnumC \newcount\apnumD 
\newcount\apnumE \newcount\apnumF \newcount\apnumG \newcount\apnumH
\newcount\apnumO \newcount\apnumL
\newcount\apnumX \newcount\apnumY \newcount\apnumZ
\newcount\apSIGNa \newcount\apSIGNb  \newcount\apEa \newcount\apEb
\newif\ifapX

\apnumZ=\catcode`\@ \catcode`\@=12

%%%%%%%%%%%% Public macros, sec. 2.1 in apnum.pdf

\def\evaldef{\relax \apEVALa}
\def\PLUS{\relax \apPPab\apPLUSa}
\def\MINUS#1#2{\relax \apPPab\apPLUSa{#1}{-#2}}
\def\MUL{\relax \apPPab\apMULa}
\def\DIV{\relax \apPPab\apDIVa}
\def\POW{\relax \apPPab\apPOWa}
\def\ABS{\relax \apEVALone\apABSa}
\def\iDIV{\relax \apEVALtwo\apiDIVa}
\def\iMOD{\relax \apEVALtwo\apiMODa}
\def\iROUND#1{\relax \evaldef\OUT{#1}\apiROUNDa}
\def\iFRAC{\relax \apEVALone\apiFRACa}
\def\FAC{\relax \apEVALone\apFACa}
\def\ROUND{\apPPs\apROUNDa}
\def\ROLL{\apPPs\apROLLa}
\def\NORM{\apPPs\apNORMa}
\def\addE#1{\edef#1{#1\ifnum\apE=0 \else E\ifnum\apE>0+\fi\the\apE\fi}}

\newcount\apSIGN  \let\SIGN=\apSIGN
\newcount\apE
\newcount\apTOT   \apTOT=-30
\newcount\apFRAC  \apFRAC=20

%%%%%%%%%%%% Evaluation of the expression, sec. 2.2 in apnum.pdf

\def\apEVALa#1#2{{\apnumA=0 \apnumE=1 \apEVALb#2\end\expandafter}\tmpb \let#1=\OUT}
\def\apEVALb{\def\tmpa{}\apEVALc}
\def\apEVALc#1{%
   \ifx+#1\apEVALd \apEVALc \fi
   \ifx-#1\edef\tmpa{\tmpa-}\apEVALd\apEVALc \fi
   \ifx(#1\apEVALd \apEVALe \fi
   \ifx\the#1\apEVALd \apEVALf\the\fi
   \ifx\number#1\apEVALd \apEVALf\number\fi
   \apTESTdigit#1\iftrue 
      \ifx E#1\let\tmpb=\tmpa \expandafter\apEVALd\expandafter\apEVALk
      \else \edef\tmpb{\tmpa#1}\expandafter\apEVALd\expandafter\apEVALn\fi\fi
   \edef\tmpb{\tmpa\noexpand#1}\futurelet\apNext\apEVALg
}
\def\apEVALd#1\fi#2\apNext\apEVALg{\fi#1}
\def\apEVALe{%
   \ifx\tmpa\empty \else \ifnum\tmpa1<0 \def\tmpb{-1}\apEVALp \MUL 4\fi\fi
   \advance\apnumA by4
   \apEVALb
}
\def\apEVALf#1#2{\expandafter\def\expandafter\tmpb\expandafter{\tmpa#1#2}\apEVALo}
\def\apEVALg{\ifx\apNext \bgroup \expandafter\apEVALh \else \expandafter\apEVALo \fi}
\def\apEVALh#1{\expandafter\def\expandafter\tmpb\expandafter{\tmpb{#1}}\futurelet\apNext\apEVALg}
\def\apEVALk{\afterassignment\apEVALm\apE=}
\def\apEVALm{\edef\tmpb{\tmpb E\the\apE}\apEVALo}
\def\apEVALn#1{\apTESTdigit#1%
   \iftrue \ifx E#1\afterassignment\apEVALm\expandafter\expandafter\expandafter\apE
           \else\edef\tmpb{\tmpb#1}\expandafter\expandafter\expandafter\apEVALn\fi
   \else \expandafter\apEVALo\expandafter#1\fi
}
\def\apEVALo#1{\let\apNext=\apEVALb
   \ifx+#1\apEVALp \apEPLUS 1\fi
   \ifx-#1\apEVALp \apEMINUS 1\fi
   \ifx*#1\apEVALp \apEMUL 2\fi
   \ifx/#1\apEVALp \apEDIV 2\fi
   \ifx^#1\apEVALp \apEPOW 3\fi
   \ifx)#1\advance\apnumA by-4 \let\apNext=\apEVALo \let\tmpa=\relax
      \ifnum\apnumA<0 \apEVALerror{many brackets ")"}\fi
   \fi
   \ifx\end#1%
      \ifnum\apnumA>0 \apEVALerror{missing bracket ")"}% 
      \else \apEVALp\END 0\fi
      \let\apNext=\relax
   \fi
   \ifx\tmpa\relax \else \apEVALerror{unknown operator "\string#1"}\fi
   \apnumE=0 \apNext
}
\def\apEVALp#1#2{%
   \apnumB=#2 \advance\apnumB by\apnumA
   \toks0=\expandafter{\expandafter{\tmpb}{#1}}%
   \expandafter\apEVALpush\the\toks0\expandafter{\the\apnumB}% {value}{op}{priority}
   \let\tmpa=\relax
}
\let\apEPLUS=\PLUS \let\apEMINUS=\MINUS \let\apEMUL=\MUL \let\apEDIV=\DIV \let\apEPOW=\POW
\def\apEVALstack{{}{}{0}.}
\def\apEVALpush#1#2#3{% value, operator, priority
   \toks0={{#1}{#2}{#3}}%
   \expandafter\def\expandafter\apEVALstack\expandafter{\the\toks0\apEVALstack}%   
   \expandafter\apEVALdo\apEVALstack@%
}
\def\apEVALdo#1#2#3#4#5#6#7@{%
   \apnumB=#3 \ifx#2\POW \advance\apnumB by1 \fi
   \ifnum\apnumB>#6\else
      \ifnum#6=0 \def\tmpb{#1}%\toks0={#1}\message{RESULT: \the\toks0}
                 \ifnum\apnumE=1 \def\tmpb{\apPPn{#1}}\fi
      \else \def\apEVALstack{#7}\apEVALpush{#5{#4}{#1}}{#2}{#3}%
   \fi\fi
}
\def\apEVALerror#1{\message{\noexpand\evaldef ERROR: #1.}%
   \def\tmpb{\def\OUT{??}}\def\apNext##1\end{}%
}
\def\apTESTdigit#1#2{%
   \ifx E#1\apXtrue \else
      \ifcat.\noexpand#1%
         \ifx.#1\apXtrue \else
            \ifnum`#1<`0 \apXfalse\else 
               \ifnum`#1>`9 \apXfalse\else \apXtrue\fi
         \fi\fi
      \else \apXfalse
   \fi\fi
   \ifapX
}

%%%%%%%%%%%% Preparation of the parameter, sec. 2.3 in apnum.pdf

\def\apPPa#1#2{\expandafter\apPPb#2@#1}
\def\apPPb{\def\tmpc{}\apSIGN=1 \apE=0 \apXfalse \expandafter\expandafter\expandafter\apPPc}
\def\apPPc#1{%
   \ifx+#1\apPPd \fi
   \ifx-#1\apSIGN=-\apSIGN \apPPd \fi
   \ifx\relax#1\apPPe \fi
   \apPPg#1%
}
\def\apPPd#1\apPPg#2{\fi\expandafter\expandafter\expandafter\apPPc}
\def\apPPe#1\apPPg#2#3@{\fi\apXtrue{#3% execution of the parameter in the group
   \edef\tmpc{\apE=\the\apE\relax\noexpand\apPPf\OUT@}\expandafter}\tmpc
}
\def\apPPf#1{\ifx-#1\apSIGN=-\apSIGN \expandafter\apPPg\else\expandafter\apPPg\expandafter#1\fi}
\def\apPPg#1{%
   \ifx.#1\def\tmpc{.}\apPPh\fi
   \ifx\tmpc\empty\else\edef\tmpc{\tmpc#1}\fi
   \ifx0#1\apPPh\fi
   \ifx\tmpc\empty\edef\tmpc{#1}\fi
   \ifx@#1\def\tmpc{@}\fi
   \expandafter\apPPi\tmpc
}
\def\apPPh#1\apPPi\tmpc{\fi\apPPg}
\def\apPPi{\ifapX \expandafter\apPPj \else \expandafter\apPPk \fi}
\def\apPPj#1@#2{\def#2{#1}}
\def\apPPk#1@#2{\ifx@#1@\apSIGN=0 \def#2{0}\else \apPPl#1E@#2\fi}
\def\apPPl#1E#2@#3{%
   \ifx@#1@\def#3{1}\else\def#3{#1}\fi
   \ifx@#2@\else \afterassignment\apPPm \apE=#2\fi
}
\def\apPPm E{}
\def\apPPn#1{\expandafter\apPPb#1@\OUT \edef\OUT{\ifnum\apSIGN<0-\fi\OUT}}
\def\apPPab#1#2#3{%
   \expandafter\apPPb#2@\tmpa \apSIGNa=\apSIGN \apEa=\apE
   \expandafter\apPPb#3@\tmpb \apSIGNb=\apSIGN \apEb=\apE
   #1%
}
\def\apPPs#1#2#3{\def\tmpc{#3}\expandafter\apPPt\expandafter#1#2.@#2}
\def\apPPt#1#2{%
   \ifx-#2\apnumG=-1 \def\apNext{#1}%
   \else \ifx0#2\apnumG=0 \def\apNext{\apPPu#1}\else \apnumG=1 \def\apNext{#1#2}\fi\fi 
   \apNext
}
\def\apPPu#1#2.@#3{\ifx@#2@\apnumG=0 \ifx#1\apROUNDa\def\XOUT{}\fi
   \else\def\apNext{\apPPt#1#2.@#3}\expandafter\apNext\fi
}
\def\apEVALone#1#2{\evaldef\OUT{#2}\ifnum\apSIGN<0 \expandafter\apNOminus\OUT@\OUT\fi #1}
\def\apEVALtwo#1#2#3{%
   {\evaldef\OUT{#2}\apOUTtmpb}\tmpb \let\tmpa=\OUT \apSIGNa=\apSIGN \apEa=\apE
   \ifnum\apSIGNa<0 \expandafter\apNOminus\tmpa@\tmpa\fi 
   {\evaldef\OUT{#3}\apOUTtmpb}\tmpb \let\tmpb=\OUT \apSIGNb=\apSIGN \apEb=\apE
   \ifnum\apSIGNb<0 \expandafter\apNOminus\tmpb@\tmpb\fi   
   #1%
}
\def\apNOminus-#1@#2{\def#2{#1}}

%%%%%%%%%%%% Addition and Subtraction, sec. 2.4 in apnum.pdf

\def\apPLUSa{%
   \ifnum\apEa=\apEb \apE=\apEa \else \apPLUSxE \fi
   \apDIG\tmpa\relax \apnumA=\apnumD % digits before decimal point
   \apDIG\tmpb\relax \apnumB=\apnumD
   \apIVmod \apnumA \apnumE \advance\apnumA by-\apnumE % digits in the first Digit
   \apIVmod \apnumB \apnumF \advance\apnumB by-\apnumF
   \apnumC=\apnumB \advance\apnumC by-\apnumA % difference between Digits
   \ifnum\apSIGNa<0 \def\apPLUSxA{-}\else \def\apPLUSxA{}\fi
   \ifnum\apSIGNb<0 \def\apPLUSxB{-}\else \def\apPLUSxB{}\fi
   \apSIGN=0 % \apSIGN=0 means that we are doing subtraction
   \ifx\apPLUSxA\empty \ifx\apPLUSxB\empty \apSIGN=1 \fi\fi
   \if\apPLUSxA-\relax \if\apPLUSxB-\relax \apSIGN=-1 \def\apPLUSxA{}\def\apPLUSxB{}\fi\fi
   \ifnum\apnumC>0 \apPLUSg  \apPLUSb \tmpb\apnumF \tmpa\apnumE \apnumB % first pass
   \else \apnumC=-\apnumC    \apPLUSb \tmpa\apnumE \tmpb\apnumF \apnumA
   \fi
   \ifnum\apnumG=0 \def\OUT{0}\apSIGN=0 \apE=0 \else
      \ifnum\apSIGN=0 \apSIGN=\apnumG \let\apNext=\apPLUSm \else \let\apNext=\apPLUSp \fi
      \apnumX=0 \edef\OUT{\expandafter}\expandafter \apNext \OUT@% second pass
      \ifnum\apnumD<1 % result in the form .000123
         \apnumZ=-\apnumD
         \def\tmpa{.}%
         \ifnum\apnumZ>0 \apADDzeros\tmpa \fi % adding dot and left zeros
         \edef\OUT{\ifnum\apSIGN<0-\fi\tmpa\OUT}%
      \else
         \edef\OUT{\expandafter}\expandafter\apPLUSy \OUT@% removing left zeros
   \fi\fi
}
\def\apPLUSb#1#2#3#4#5{%
   \edef\tmpd{\ifcase#4\or{}{}{}\or{}{}\or{}\fi#3}%
   \edef\tmpc{\ifcase#2\or{}{}{}\or{}{}\or{}\fi}%
   \let\apNext=\apPLUSc \apnumD=#5\advance\apnumD by4 \apnumG=0 \apnumZ=0 \def\OUT{}%
   \expandafter\expandafter\expandafter\apPLUSc\expandafter\tmpc#1\apNL\apNL\apNL\apNL@%
}
\def\apPLUSc#1#2#3#4{\apnumY=\apPLUSxA#1#2#3#4\relax 
   \ifx\apNL#4\let\apNext=\apPLUSd\fi 
   \ifx\apNL#1\relax \ifx\tmpd\empty \expandafter\expandafter\expandafter\apPLUSf \fi\fi
   \apPLUSe
}
\def\apPLUSd{\apnumY=0 \ifx\tmpd\empty \expandafter\apPLUSf \else\expandafter \apPLUSe\fi}
\def\apPLUSe{%
   \ifnum\apnumC>0 \advance\apnumC by-4 
   \else \apIVread\tmpd \advance\apnumY by\apPLUSxB\apnumX \fi
   \ifnum\apnumZ=0 \apPLUSh \fi
   \edef\OUT{{\the\apnumY}\OUT}% 
   \advance\apnumD by-4
   \apNext
}
\def\apPLUSf#1@{}
\def\apPLUSg{\let\tmpc=\apPLUSxA \let\apPLUSxA=\apPLUSxB \let\apPLUSxB=\tmpc}
\def\apPLUSh{\apnumZ=\apnumY
   \ifnum\apnumY=0 \else \ifnum\apnumY<0 \apnumG=-1 \apnumY=-\apnumY \apPLUSg \else\apnumG=1 \fi\fi
}
\def\apPLUSm#1{%
   \ifx@#1\else
      \apnumA=#1 \advance\apnumA by-\apnumX
      \ifnum\apnumA<0 \advance\apnumA by\apIVbase \apnumX=1 \else \apnumX=0 \fi
      \apPLUSw
      \expandafter\apPLUSm
   \fi
}
\def\apPLUSp#1{%
   \ifx@#1\ifnum\apnumX>0 \edef\OUT{1\OUT}\fi
   \else
      \apnumA=\apnumX \advance\apnumA by#1
      \ifnum\apnumA<\apIVbase \apnumX=0 \else \apnumX=1 \advance\apnumA by-\apIVbase \fi
      \apPLUSw
      \expandafter\apPLUSp
   \fi
}
\def\apPLUSw{%
   \ifnum\apnumD=0 \ifx\OUT\empty \def\OUT{\empty}\else \edef\OUT{.\OUT}\fi \fi
   \advance\apnumD by4
   \ifx\OUT\empty  \edef\tmpa{\apIVwrite\apnumA}\edef\OUT{\apREMzerosR\tmpa}%
   \else \edef\OUT{\apIVwrite\apnumA\OUT}\fi
}
\def\apPLUSy#1{\ifx0#1\expandafter\apPLUSy\else \expandafter\apPLUSz\expandafter#1\fi}
\def\apPLUSz#1@{\edef\OUT{\ifnum\apSIGN<0-\fi#1}}
\def\apPLUSxE{%
   \apnumE=\apEa \advance\apnumE by-\apEb
   \ifnum\apEa>\apEb \apPPs\apROLLa\tmpb{-\apnumE}\apE=\apEa
   \else \apPPs\apROLLa\tmpa{\apnumE}\apE=\apEb \fi
}

%%%%%%%%%%%% Multiplication, sec. 2.5 in apnum.pdf

\def\apMULa{%
   \apE=\apEa \advance\apE by\apEb 
   \apSIGN=\apSIGNa \multiply\apSIGN by\apSIGNb 
   \ifnum\apSIGN=0 \def\OUT{0}\apE=0 \else
      \apDIG\tmpa\apnumA \apnumX=\apnumA \advance\apnumA by\apnumD
      \apDIG\tmpb\apnumB \advance\apnumX by\apnumB \advance\apnumB by\apnumD
      \apnumD=\apnumX % \apnumD = the number of digits after decimal point in the result
      \apIVmod \apnumA \apnumF % \apnumF = digits in the first Digit of \tmpa
      \edef\tmpc{\ifcase\apnumF\or{}{}{}\or{}{}\or{}\fi}\def\OUT{}%
      \expandafter\expandafter\expandafter \apMULb \expandafter \tmpc \tmpa @@@@%
      \edef\OUT{*.\OUT}%
      \apIVmod \apnumB \apnumF % \apnumF = digits in the first Digit of \tmpb
      \edef\tmpc{\ifcase\apnumF\or{}{}{}\or{}{}\or{}\fi}\def\tmpa{}%
      \expandafter\expandafter\expandafter \apMULc \expandafter \tmpc \tmpb @@@@%      
      \expandafter\apMULd \tmpa@%
      \expandafter\apMULg \OUT
      \edef\tmpa{\ifnum\apSIGN<0-\fi}%
      \ifnum\apnumD>0 \apnumZ=\apnumD \edef\tmpa{\tmpa.}\apADDzeros\tmpa \fi
      \ifx\tmpa\empty \else \edef\OUT{\tmpa\OUT}\fi      
   \fi
}
\def\apMULb#1#2#3#4{\ifx@#4\else 
   \ifx\OUT\empty \edef\OUT{{#1#2#3#4}*}\else\edef\OUT{{#1#2#3#4}0\OUT}\fi
   \expandafter\apMULb\fi
}
\def\apMULc#1#2#3#4{\ifx@#4\else \edef\tmpa{{#1#2#3#4}\tmpa}\expandafter\apMULc\fi}
\def\apMULd#1{\ifx@#1\else
     \apnumA=#1 \expandafter\apMULe \OUT
     \expandafter\apMULd
  \fi
}
\def\apMULe#1*#2{\apnumX=0 \def\OUT{#1{#2}*}\def\apOUTl{}\apnumO=1 \apnumL=0 \apMULf}
\def\apMULf#1#2{%
  \advance\apnumO by-1 \ifnum\apnumO=0 \apOUTx \fi
  \apnumB=#1 \multiply\apnumB by\apnumA \advance\apnumB by\apnumX
  \ifx*#2%
     \ifnum\apnumB<\apIVbase
       \edef\OUT{\OUT\expandafter\apOUTs\apOUTl.,\ifnum\the\apnumB#1=0 \else{\the\apnumB}{#1}\fi*}%
     \else \apIVtrans
        \expandafter \edef\csname apOUT:\apOUTn\endcsname
                 {\csname apOUT:\apOUTn\endcsname{\the\apnumB}{#1}}%
        \apMULf0*\fi
  \else \advance\apnumB by#2
     \ifnum\apnumB<\apIVbase \apnumX=0 \else \apIVtrans \fi
     \expandafter
        \edef\csname apOUT:\apOUTn\endcsname{\csname apOUT:\apOUTn\endcsname{\the\apnumB}{#1}}%
     \expandafter\apMULf \fi
}
\def\apMULg#1{\def\OUT{}\apMULh}
\def\apMULh#1{\ifx*#1\expandafter\apMULi
   \else \apnumA=#1 \apMULo4{\apIVwrite\apnumA}%
         \expandafter\apMULh
   \fi
}
\def\apMULi#1#2#3{\apnumA=#1
   \ifx*#3\apMULo{\apNUMdigits\tmpa}{\the\apnumA}\expandafter\apMULj
   \else  \apMULo4{\apIVwrite\apnumA}\expandafter\apMULi
   \fi{#3}%
}
\def\apMULj#1{}
\def\apMULo#1#2{\edef\tmpa{#2}%
   \advance\apnumD by-#1
   \ifnum\apnumD<1 \ifnum\apnumD>-4 \apMULt\fi\fi
   \edef\OUT{\tmpa\OUT}%
}
\def\apMULt{\edef\tmpa{\apIVdot{-\apnumD}\tmpa}\edef\tmpa{\tmpa}}

%%%%%%%%%%%% Division, sec. 2.6 in apnum.pdf

\def\apDIVa{%
   \ifnum\apSIGNb=0 \errmessage{Dividing by zero}\else
      \apSIGN=\apSIGNa \multiply\apSIGN by\apSIGNb
      \ifnum\apSIGNa=0 \def\OUT{0}\def\XOUT{0}\apE=0 \apSIGN=0 \else
         \apE=\apEa \advance\apE by-\apEb 
         \apDIG\tmpb\relax \apnumB=\apnumD 
         \apDIG\tmpa\relax \apnumH=\apnumD
         \advance\apnumD by-\apnumB % \apnumD = num. of digits before decimal point in the result
         \apDIVcomp\tmpa\tmpb  % apXtrue <=>  A>=B, i.e 1 digit from A/B
         \ifapX \advance\apnumD by1 \advance\apnumH by1 \fi
         \apnumC=\apTOT
         \ifnum\apTOT<0 \apnumC=-\apnumC 
            \ifnum\apnumD>\apnumC \apnumC=\apnumD \fi
         \fi
         \ifnum\apTOT=0 \apnumC=\apFRAC \advance\apnumC by\apnumD
         \else \apnumX=\apFRAC \advance\apnumX by\apnumD 
               \ifnum\apnumC>\apnumX \apnumC=\apnumX \fi
         \fi
         \ifnum\apnumC>0  % \apnumC = the number of digits in the result
            \advance\apnumH by-\apnumC % \apnumH = the position of decimal point in the remainder
            \apIVmod \apnumC \apnumF   % \apnumF = the number of digits in the first Digit 
            \apIVread\tmpb \apnumB=\apnumX % \apnumB = partial divisor
            \apnumX=\apnumF \ifapX \advance\apnumX by-1 \fi
            \apIVreadX\apnumX\tmpa
            \apnumA=\apnumX % \apnumA = first Digit of the partial dividend
            \apIVread\tmpa  % \apnumX = second Digit of the partial dividend
            \edef\apDIVxA{\the\apnumA\apIVwrite\apnumX}% first partial dividend
            \edef\apDIVxB{\the\apnumB}% partial divisor
            \edef\XOUT{{\apDIVxB}{\the\apnumX}@{\the\apnumA}}% the \XOUT is initialized
            \edef\OUT{\ifnum\apSIGN<0-\fi}%
            \ifnum\apnumD<0 \edef\OUT{\OUT.}\apnumZ=-\apnumD \apADDzeros\OUT \fi
            \apnumE=1 \apnumZ=0
            \let\apNext=\apDIVg \apNext % <--- the main calculation loop is here
            \ifnum\apnumD>0 \apnumZ=\apnumD \apADDzeros\OUT \fi 
            \ifnum\apnumE=0 \def\XOUT{0}\else % extracting remainder from \XOUT
               \edef\XOUT{\expandafter}\expandafter\apDIVv\XOUT
               \def\tmpc{\apnumH}\apnumG=\apSIGNa \expandafter\apROLLa\XOUT.@\XOUT
           \fi
         \else \def\OUT{0}\def\XOUT{0}\apE=0 \apSIGN=0
   \fi\fi\fi
}
\def\apDIVcomp#1#2{%
   \expandafter\def\expandafter\tmpc\expandafter{#1\apNL\apNL\apNL\apNL\apNL\apNL\apNL\apNL@}%
   \expandafter\def\expandafter\tmpd\expandafter{#2\apNL\apNL\apNL\apNL\apNL\apNL\apNL\apNL@}%
   \def\apNext{\expandafter\expandafter\expandafter\apDIVcompA\expandafter\tmpc\tmpd}%
   \apXtrue \apNext
}
\def\apDIVcompA#1#2#3#4#5#6#7#8#9@{%
   \ifx#8\apNL \def\tmpc{0000000\apNL@}\else\def\tmpc{#9@}\fi
   \apnumX=#1#2#3#4#5#6#7#8\relax
   \apDIVcompB
}
\def\apDIVcompB#1#2#3#4#5#6#7#8#9@{%
   \ifnum\apnumX<#1#2#3#4#5#6#7#8 \let\apNext=\relax \apXfalse \else
   \ifnum\apnumX>#1#2#3#4#5#6#7#8 \let\apNext=\relax \apXtrue
   \fi\fi
   \ifx\apNext\relax\else 
      \ifx#8\apNL \def\tmpd{0000000\apNL@}\ifx\tmpc\tmpd\let\apNext=\relax\fi 
      \else\def\tmpd{#9@}\fi
   \fi
   \apNext
}
\def\apDIVg{%
   \ifx\tmpb\empty
      \ifx\tmpa\empty  \def\apNext{\apDIVi!}\let\apNexti=\apDIVi
      \else \def\apNext{\expandafter\apDIVh\tmpa\apNL\apNL\apNL\apNL!}\let\apNexti=\apDIVh
   \fi\fi
   \ifx\apNext\apDIVg
      \apIVread\tmpa \apnumA=\apnumX
      \apIVread\tmpb
      \edef\XOUT{{\the\apnumX}{\the\apnumA}\XOUT}%
   \fi
   \apNext
}
\def\apDIVh#1#2#3#4{\apnumZ=#1#2#3#4
   \ifx\apNL#4\let\apNexti=\apDIVi\fi
   \apDIVi
}
\def\apDIVi{%
   \ifnum\apnumE=0 \apnumC=0 \fi
   \ifnum\apnumC>0
      \expandafter\apDIVp\XOUT
      \advance\apnumC by-4  
      \apnumZ=0
      \expandafter\apNexti
   \else 
      \expandafter\apDIVj
   \fi
}
\def\apDIVj#1!{}
\def\apDIVp{%
   \apnumA=\apDIVxA \divide\apnumA by\apDIVxB
   \def\apOUTl{}\apnumO=1 \apnumL=0
   \apnumX=0 \apnumB=0 \apnumE=0 
   \let\apNext=\apDIVq \apNext 0\apnumZ
}
\def\apDIVq#1#2#3{% B A B
   \advance\apnumO by-1 \ifnum\apnumO=0 \apOUTx \fi
   \apnumY=\apnumB 
   \apnumB=#1\multiply\apnumB by-\apnumA
   \advance\apnumB by#2\advance\apnumB by-\apnumX
   \ifnum\apnumB<0 \apnumX=\apnumB \advance\apnumX by1
      \divide\apnumX by-\apIVbase \advance\apnumX by1
      \advance\apnumB by\the\apnumX 0000
   \else \apnumX=0 \fi
   \expandafter
      \edef\csname apOUT:\apOUTn\endcsname{\csname apOUT:\apOUTn\endcsname{#3}{\the\apnumB}}%
   \ifnum\apnumE<\apnumB \apnumE=\apnumB \fi
   \ifx@#3\let\apNext=\apDIVr \fi
   \apNext{#3}%
}
\def\apDIVr#1#2{%
   \ifnum\apnumX=#2  % the calculated Digit is OK, we save it
      \edef\XOUT{\expandafter\apOUTs\apOUTl.,}%
      \edef\tmpa{\ifnum\apnumF=4 \expandafter\apIVwrite\else \expandafter\the\fi\apnumA}%
      \ifnum\apnumD<\apnumF \ifnum\apnumD>-1 \apDIVt \fi\fi %adding dot
      \ifx\apNexti\apDIVh \apnumE=1 \fi
      \ifnum\apnumE=0 \apDIVu % removing zeros
            \advance\apnumD by-\apNUMdigits\tmpa \relax
      \else \advance\apnumD by-\apnumF \apnumF=4 \fi
      \edef\OUT{\OUT\tmpa}% save the Digit
      \edef\apDIVxA{\the\apnumB\apIVwrite\apnumY}% next partial dvividend
   \else  % we need do correction and run the remainder calculation again
      \advance\apnumA by-1 \apnumX=0 \apnumB=0 \apnumE=0
      \def\apOUTl{}\apnumO=1 \apnumL=0
      \def\apNext{\let\apNext=\apDIVq
      \expandafter\apNext\expandafter0\expandafter\apnumZ\XOUT}%
      \expandafter\apNext
   \fi
}
\def\apDIVt{\edef\tmpa{\apIVdot\apnumD\tmpa}\edef\tmpa{\tmpa}}
\def\apDIVu{\edef\tmpa{\apREMzerosR\tmpa}\edef\tmpa{\apREMdotR\tmpa}}
\def\apDIVv#1#2{\apnumX=#2 
   \ifx@#1\apDIVw{.\apIVwrite\apnumX}\else\apDIVw{\apIVwrite\apnumX}\expandafter\apDIVv\fi
}
\def\apDIVw#1{%
   \ifx\XOUT\empty \ifnum\apnumX=0 
                   \else \edef\tmpa{#1}\edef\XOUT{\apREMzerosR\tmpa\XOUT}%
                   \fi
   \else \edef\XOUT{#1\XOUT}\fi
}

%%%%%%%%%%%% Power to the integer, sec. 2.7 in apnum.pdf

\def\apPOWa{%
   \ifnum\apSIGNa=0 \def\OUT{0}\apSIGN=0 \apE=0 \else
      \ifnum\apSIGNb=0 \def\OUT{1}\apSIGN=1 \apE=0 \else
         \apDIG\tmpb\apnumB 
         \ifnum\apnumB>0 \errmessage{POW: non-integer exponent is not implemented yet}\apPOWe\fi
         \ifnum\apEb=0 \else \errmessage{POW: the E notation of exponent isn't allowed}\apPOWe\fi
         \ifnum\apnumD>8 \errmessage{POW: too big exponent. 
                       Do you really need about 10^\the\apnumD\space digits in output?}\apPOWe\fi
         \apE=\apEa \multiply\apE by\tmpb\relax
         \apSIGN=\apSIGNa
         \ifodd\tmpb \else \apSIGN=1 \fi
         \apDIG\tmpa\apnumA \apnumC=\apnumA \advance\apnumC by\apnumD  
         \apnumD=\apnumA \multiply\apnumD by\tmpb
         \apIVmod \apnumC \apnumA
         \edef\tmpc{\ifcase\apnumA\or{}{}{}\or{}{}\or{}\fi}\def\OUT{}%
         \expandafter\expandafter\expandafter \apMULb \expandafter \tmpc \tmpa @@@@%
         \edef\OUT{*.\OUT}%    \OUT := \tmpa in interleaved format
         \def\tmpc{*.1*}%
         \apnumE=\tmpb\relax \apPOWb
         \expandafter\apPOWg \tmpc % \OUT := \tmpc in human raedable form
         \ifnum\apnumD=0 \ifnum \apSIGN<0 \edef\OUT{-\OUT}\fi
         \else \def\tmpc{-\apnumD}\apnumG=\apSIGN \expandafter\apROLLa\OUT.@\OUT\fi
         \ifnum\apSIGNb<0 \apPPab\apDIVa 1\OUT \fi
         \relax
   \fi\fi
}
\def\apPOWb{%
   \ifodd\apnumE   \def\tmpb{}\expandafter\apPOWd\OUT
                   \let\tmpd=\OUT \let\OUT=\tmpc
                   \expandafter\apMULd \tmpb@\expandafter\apPOWn\OUT@% 
                   \let\tmpc=\OUT \let\OUT=\tmpd
   \fi
   \divide\apnumE by2
   \ifnum\apnumE>0 \expandafter\apPOWt\OUT \expandafter\apPOWn\OUT@%
                   \expandafter\apPOWb
   \fi
}                 
\def\apPOWd#1#2{% \apPOWd <spec format> => \tmpb (in simple reverse format)
   \ifx*#1\expandafter\apPOWd \else
      \edef\tmpb{\tmpb{#1}}%
      \ifx*#2\else \expandafter\expandafter\expandafter\apPOWd\fi
   \fi
}
\def\apPOWe#1\relax{\fi}
\def\apPOWg#1#2{\def\OUT{}\apPOWh} % conversion to the human readable form
\def\apPOWh#1#2{\apnumA=#1 
   \ifx*#2\edef\OUT{\the\apnumA\OUT}\else \edef\OUT{\apIVwrite\apnumA\OUT}\expandafter\apPOWh\fi
}
\def\apPOWn#1{\def\OUT{*}\apPOWna}
\def\apPOWna#1{\ifx*#1\expandafter\apPOWnn\else \edef\OUT{\OUT0{#1}}\expandafter\apPOWna\fi}
\def\apPOWnn#1#2{\ifx*#1\edef\OUT{\OUT*}\else\edef\OUT{\OUT0{#1}}\expandafter\apPOWnn\fi}
\def\apPOWt#1#2{\apPOWu} % power to two
\def\apPOWu#1#2{\apnumA=#1 
   \expandafter\apPOWv\OUT
   \ifx*#2\else \expandafter\apPOWu\fi
}
\def\apPOWv#1*#2#3#4{\def\apOUTl{}\apnumO=1 \apnumL=0
   \apnumB=\apnumA \multiply\apnumB by\apnumB \multiply\apnumA by2
   \ifx*#4\else\advance\apnumB by#4 \fi
   \ifx\apnumB<\apIVbase \apnumX=0 \else \apIVtrans \fi
   \edef\OUT{#1{#2}{\the\apnumB}*}%
   \ifx*#4\apMULf0*\else\expandafter\apMULf\fi
}

%%%%%%%%%%%% ROLL, ROUND and NORM macros, sec. 2.8 in apnum.pdf

\def\apROLLa{\apnumA=\tmpc\relax \ifnum\apnumA<0 \expandafter\apROLLc\else \expandafter\apROLLg\fi}
\def\apROLLc{\edef\tmpc{}\edef\tmpd{\ifnum\apnumG<0-\fi}\apnumB=0 \apROLLd}
\def\apROLLd#1{%
   \ifx.#1\expandafter\apROLLe
   \else \edef\tmpc{\tmpc#1}%
      \advance\apnumB by1
      \expandafter\apROLLd
   \fi
}
\def\apROLLe#1{\ifx@#1\edef\tmpc{\tmpc.@}\else\edef\tmpc{\tmpc#1}\fi
   \advance\apnumB by\apnumA
   \ifnum\apnumB<0
       \apnumZ=-\apnumB \edef\tmpd{\tmpd.}\apADDzeros\tmpd
       \expandafter\expandafter\expandafter\apROLLf\expandafter\tmpc
   \else
       \apnumA=\apnumB
       \expandafter\expandafter\expandafter\apROLLi\expandafter\tmpc
   \fi
}
\def\apROLLf#1.@#2{\edef#2{\tmpd#1}}
\def\apROLLg#1{\edef\tmpd{\ifnum\apnumG<0-\fi}\ifx.#1\apnumB=0 \else\apnumB=1 \fi \apROLLh#1}
\def\apROLLh#1{\ifx.#1\expandafter\apROLLi\else \edef\tmpd{\tmpd#1}\expandafter\apROLLh\fi}
\def\apROLLi#1{\ifx.#1\expandafter\apROLLi\else
   \ifnum\apnumA>0 \else \apROLLj \apROLLk#1\fi 
   \ifx@#1\apROLLj \apROLLi0@\fi
   \advance\apnumA by-1
   \ifx0#1\else \apnumB=1 \fi
   \ifnum\apnumB>0 \edef\tmpd{\tmpd#1}\fi
   \expandafter\apROLLi\fi
}
\def\apROLLj#1\fi#2\apROLLi\fi{\fi\fi#1}
\def\apROLLk#1{\ifx@#1\expandafter\apROLLo\expandafter@\else 
   \def\tmpc{}\apnumB=0 \expandafter\apROLLn\expandafter#1\fi
}
\def\apROLLn#1{%
   \ifx.#1\ifnum\apnumB>0 \edef\tmpd{\tmpd.\tmpc}\fi \expandafter\apROLLo
   \else \edef\tmpc{\tmpc#1}\advance\apnumB by#1 \expandafter\apROLLn
   \fi
}
\def\apROLLo@#1{\let#1=\tmpd}

\def\apROUNDa{\apnumD=\tmpc\relax
   \ifnum\apnumD<0 \expandafter\apROUNDe
   \else \expandafter\apROUNDb
   \fi
}
\def\apROUNDb#1.{\edef\tmpc{#1}\apnumX=0 \def\tmpd{}\let\apNext=\apROUNDc \apNext}
\def\apROUNDc#1{\ifx@#1\def\apNext{\apROUNDd.@}%
   \else \advance\apnumD by-1 
         \ifnum\apnumD<0 \def\apNext{\apROUNDd#1}%
         \else \ifx.#1\else \advance\apnumX by#1 \edef\tmpd{\tmpd#1}\fi
         \fi
   \fi \apNext
}
\def\apROUNDd#1.@#2{\def\XOUT{#1}%
   \ifnum\apnumX=0 \def\tmpd{}\fi
   \ifx\tmpd\empty 
      \ifx\tmpc\empty \def#2{0}% 
      \else \edef#2{\ifnum\apnumG<0-\fi\tmpc}\fi
   \else\edef#2{\ifnum\apnumG<0-\fi\tmpc.\tmpd}\fi
}
\def\apROUNDe#1.@#2{\apnumC=\apnumD 
   \apPPs\apROLLa#2{\apnumC}\apPPs\apROUNDa#2{0}\apPPs\apROLLa#2{-\apnumC}%
}
\def\apNORMa#1.@#2{\ifnum\apnumG<0 \def#2{#1}\fi \expandafter\apNORMb\expandafter#2\tmpc@}
\def\apNORMb#1#2#3@{%
   \ifx.#2\apnumC=#3\relax  \apDIG#1\apnumA \apNORMc#1%
   \else \apnumC=#2#3\relax \apDIG#1\relax  \apNORMd#1%
   \fi
}
\def\apNORMc#1{\advance\apE by-\apnumA \advance\apE by\apnumC
   \def\tmpc{-\apnumC}\expandafter\apROLLa#1.@#1%
}
\def\apNORMd#1{\advance\apE by\apnumD \advance\apE by-\apnumC
   \def\tmpc{\apnumC}\expandafter\apROLLa\expandafter.#1.@#1%
}

%%%%%%%%%%%% Function-like macros, sec. 2.9 in apnum.pdf

\def\apABSa{\ifnum\apSIGN<0 \apSIGN=1 \fi}
\def\apiDIVa{{\apFRAC=0 \apTOT=0 \apDIVa \apOUTtmpb}\tmpb}
\def\apiMODa{{\apFRAC=0 \apTOT=0 \apDIVa \let\OUT=\XOUT \apOUTtmpb}\tmpb}
\def\apiROUNDa{\apROUNDa\OUT0}
\def\apiFRACa{\apROUNDa\OUT0\ifx\XOUT\empty\def\OUT{0}\else\edef\OUT{.\XOUT}\fi}
\def\apFACa{{\apnumC=\OUT\relax
   \loop \ifnum \apnumC>2 \advance\apnumC by-1 
         \MUL{\OUT}{\the\apnumC}\repeat
   \global\let\OUT=\OUT}%
}

%%%%%%%%%%%% Auxiliary macros, sec. 2.10 in apnum.pdf

\def\apREV#1{\expandafter\apREVa#1@!}
\def\apREVa#1#2!{\ifx@#1\else\apREVa#2!#1\fi}

\def\apDIG#1#2{\ifx\relax#2\def\tmpc{}\else #2=0 \def\tmpc{\advance#2 by1 }\fi 
   \apnumD=0 \expandafter\apDIGa#1..@#1%
}
\def\apDIGa#1{\ifx.#1\csname apDIG\ifnum\apnumD>0 c\else b\fi\expandafter\endcsname 
   \else \advance\apnumD by1 \expandafter\apDIGa\fi}
\def\apDIGb#1{%
   \ifx0#1\advance\apnumD by-1 \tmpc \expandafter\apDIGb
   \else \expandafter\apDIGc \expandafter#1\fi
}
\def\apDIGc#1.{\def\tmpd{#1}%
   \ifx\tmpc\empty \let\apNext=\apDIGe
   \else \def\apNext{\expandafter\apDIGd\tmpd@}%
   \fi \apNext
}
\def\apDIGd#1{\ifx@#1\expandafter\apDIGe \else \tmpc \expandafter\apDIGd \fi}
\def\apDIGe#1@#2{%
   \ifx@#1@\else  % #1=empty <=> the param has no dot, we need to do nothing 
      \ifnum\apnumD>0 \edef#2{\expandafter\apDIGf#2@}% the dot plus digits before dot
      \else \let#2=\tmpd  % there are only digits after dot, use \tmpd
   \fi\fi
}
\def\apDIGf#1.#2@{#1#2}

\def\apNL{0}
\def\apIVread#1{\expandafter\apIVreadA#1\apNL\apNL\apNL\apNL\apNL@#1}
\def\apIVreadA#1#2#3#4#5\apNL#6@#7{\apnumX=#1#2#3#4\relax \def#7{#5}}
\def\apIVreadX#1#2{\edef\tmpc{\ifcase#1{}{}{}0\or{}{}{}\or{}{}\or{}\fi}%
   \expandafter\expandafter\expandafter\apIVreadA\expandafter\tmpc#2\apNL\apNL\apNL\apNL\apNL@#2%
}
\def\apIVwrite#1{\ifnum#1<1000 0\ifnum#1<100 0\ifnum#1<10 0\fi\fi\fi\the#1}

\mathchardef\apIVbase=10000
\def\apIVtrans{\apnumX=\apnumB \divide\apnumB by\apIVbase \multiply\apnumB by-\apIVbase 
   \advance\apnumB by\apnumX \divide\apnumX by\apIVbase
}
\def\apIVmod#1#2{#2=#1\divide#2by4 \multiply#2by-4 \advance#2by#1\relax
   \ifnum#2>0 \else \advance#2by4 \fi
}

\def\apIVdot#1#2{\noexpand\apIVdotA\ifcase#1....\or...\or..\or.\fi #2....@}
\def\apIVdotA#1#2#3#4#5.#6@{\ifx.#1\else#1\fi
   \ifx.#2\else#2\fi \ifx.#3\else#3\fi \ifx.#4\else#4\fi\ifx.#5.\else.#5\fi
}
\def\apNUMdigits#1{\expandafter\apNUMdigitsA#1@@@@!}
\def\apNUMdigitsA#1#2#3#4#5!{\ifx@#4\ifx@#3\ifx@#2\ifx@#10\else1\fi \else2\fi \else3\fi \else4\fi}

\def\apADDzeros#1{\edef#1{#10}\advance\apnumZ by-1 
   \ifnum\apnumZ>0 \expandafter\apADDzeros\expandafter#1\fi
}
\def\apREMzerosR#1{\expandafter\apREMzerosRa#1@0@!}
\def\apREMzerosRa#10@#2!{\ifx!#2!\apREMzerosRb#1\else\apREMzerosRa#1@0@!\fi}
\def\apREMzerosRb#1@{#1}
\def\apREMdotR#1{\expandafter\apREMdotRa#1@.@!}
\def\apREMdotRa#1.@#2!{\ifx!#2!\apREMzerosRb#1\else#1\fi}

\def\apOUTx{\apnumO=7
   \edef\apOUTn{\the\apnumL}\edef\apOUTl{\apOUTl\apOUTn,}%
   \expandafter\def\csname apOUT:\apOUTn\endcsname{}%
   \advance\apnumL by1
}
\def\apOUTs#1,{\ifx.#1\else\csname apOUT:#1\expandafter\endcsname\expandafter\apOUTs\fi}

\def\apOUTtmpb{\edef\tmpb{\apSIGN=\the\apSIGN \apE=\the\apE \edef\noexpand\OUT{\OUT}}\expandafter}

%%%%%%%%%%%% Conclusion, sec. 2.11 in apnum.pdf

\ifx\documentclass\undefined \else % please, don't remove this message
\message{SORRY, you are using LaTeX. I don't recommend this. Petr Olsak}\fi
\catcode`\@=\apnumZ
\endinput

1.0 <Nov. 2014> - First version released
1.1 <Jan. 2015> 
    - POW implemented more simple (by base 2 of exponent)
    - \next renamed in order to avoid name conflict
