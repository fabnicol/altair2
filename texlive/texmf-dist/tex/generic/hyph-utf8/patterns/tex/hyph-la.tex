%
%                  ********** hyph-la.tex *************
%
% Copyright 1999-2014 Claudio Beccari
%                [latin hyphenation patterns]
%
% -----------------------------------------------------------------
% IMPORTANT NOTICE:
%
% This program can be redistributed and/or modified under the terms
% of the LaTeX Project Public License Distributed from CTAN
% archives in directory macros/latex/base/lppl.txt; either
% version 1 of the License, or any later version.
% -----------------------------------------------------------------
%
% Patterns for the latin language mainly in modern spelling
% (u when u is needed and v when v is needed); medieval spelling
% with the ligatures \ae and \oe  and the (uncial) lowercase `v'
% written as a `u' is also supported; apparently there is no conflict
% between the patterns of modern  Latin and those of medieval Latin.
%
%
% Prepared by  Claudio Beccari
%              Politecnico di Torino
%              Torino, Italy
%              e-mail claudio dot beccari at gmail.com
%
% \versionnumber{3.2a}   \versiondate{2014/06/04}
%
% For more information please read the babel-latin documentation.
%
%%%%%%%%%%%%%%%%%%%%%%%%%%%%%%%%%%%%%%%%%%%%%%%%%%%%%%%%%%%%%%%%%%%%%%%%%%%%%%
%
%  For documentation see:
%  C. Beccari, "Computer aided hyphenation for Italian and Modern
%        Latin", TUG vol. 13, n. 1, pp. 23-33 (1992)
%
%  see also
%
%  C. Beccari, "Typesetting of ancient languages",
%              TUG vol.15, n.1, pp. 9-16 (1994)
%
%  In the former paper  the  code  was  described  as  being contained in file
%  ITALAT.TEX; this is substantially the same code,  but  the  file  has  been
%  renamed and included in hyph-utf8.
%
%  A corresponding file (ITHYPH.TEX) has been extracted in order to  eliminate
%  the  (few)  patterns specific to Latin and leave those specific to Italian;
%  ITHYPH.TEX has been further  extended  with  many  new patterns in order to
%  cope with the many neologisms and technical terms with foreign roots.
%
%  Should you find any word that gets hyphenated in a wrong way, please, AFTER
%  CHECKING  ON A RELIABLE MODERN DICTIONARY, report to the author, preferably
%  by e-mail.  Please  do  not  report  about  wrong  break  points concerning
%  prefixes and/or suffixes; see at the bottom of this file.
%
%  Compared with the previous versions, this file has been extended so  as  to
%  cope also with the medieval Latin spelling, where the letter `V' played the
%  roles of both `U' and `V', as in the Roman times, save that the Romans used
%  only capitals. In the middle ages the availability of soft writing supports
%  and the necessity of copying books with a reasonable speed, several scripts
%  evolved  in  (practically)  all  of  which  there was a lower case alphabet
%  different from the upper case  one,  and  where  the lower case `v' had the
%  rounded shape of our modern lower case `u', and where the Latin  diphthongs
%  `AE'  and  `OE',  both in upper and lower case, where written as ligatures,
%  not to mention the habit of  substituting  them with their sound, that is a
%  simple `E'.
%
%  According  to  Leon  Battista  Alberti,  who  in  1466  wrote  a  book   on
%  cryptography  where  he  thoroughly  analyzed  the hyphenation of the Latin
%  language of his (still  medieval)  times,  the  differences from the Tuscan
%  language (the Italian language, as it was named  at  his  time)  were  very
%  limited,  in particular for what concerns the handling of the ascending and
%  descending diphthongs; in  Central  and  Northern  Europe,  and later on in
%  North America, the Scholars perceived the above diphthongs as made  of  two
%  distinct  vowels;  the  hyphenation of medieval Latin, therefore, was quite
%  different in the northern countries compared to the southern ones, at least
%  for what concerns these  diphthongs.  If  you need hyphenation patterns for
%  medieval Latin that suite you better according to the  habits  of  Northern
%  Europe  you  should  resort  to the hyphenation patterns prepared by Yannis
%  Haralambous (TUGboat, vol.13 n.4 (1992)).
%
%
%
%                            PREFIXES AND SUFFIXES
%
% For what concerns prefixes and suffixes, the latter are generally  separated
% according  to  "natural"  syllabification,  while  the  former are generally
% divided etimologically. In order to  avoid  an excessive number of patterns,
% care has been paid to some prefixes,  especially  "ex",  "trans",  "circum",
% "prae",  but  this set of patterns is NOT capable of separating the prefixes
% in all circumstances.
%
%                         BABEL SHORTCUTS AND FACILITIES
%
% Read  the  documentation  coming  with the discription of the Latin language
% interface of  Babel  in  order  to  see  the  shortcuts  and  the facilities
% introduced in order to facilitate the insertion  of  "compound  word  marks"
% which are very useful for inserting etimological break points.
%
% Happy Latin and multilingual typesetting!
%
%%%%%%%%%%%%%%%%%%%%%%%%%%%%%%%%%%%%%%%%%%%%%%%%%%%%%%%%%%%%%%%%%%%%%%%%%%%%%%
%
% \message{Latin Hyphenation Patterns Version 3.2a <2014/06/04>}
%
\patterns{%
.a2b3l
.anti1  .anti3m2n
.circu2m1
.co2n1iun
.di2s3cine
.e2x1
.o2b3                                % .o2b3l  .o2b3r .o2b3s
.para1i  .para1u
.su2b3lu .su2b3r
2s3que.  2s3dem.
3p2sic
3p2neu
æ1 œ1
a1ia a1ie  a1io  a1iu ae1a ae1o ae1u
e1iu
io1i
o1ia o1ie  o1io  o1iu
uo3u                                % quousque
1b   2bb   2bd   b2l   2bm  2bn  b2r  2bt  2bs  2b.
1c   2cc   c2h2  c2l   2cm  2cn  2cq  c2r  2cs  2ct  2cz  2c.
1d   2dd   2dg   2dm   d2r  2ds  2dv  2d.
1f   2ff   f2l   2fn   f2r  2ft  2f.
1g   2gg   2gd   2gf   g2l  2gm  g2n  g2r  2gs  2gv  2g.
1h   2hp   2ht   2h.
1j
1k   2kk   k2h2
1l   2lb   2lc   2ld   2lf  l3f2t 2lg 2lk  2ll  2lm  2ln  2lp  2lq  2lr
     2ls   2lt   2lv   2l.
1m   2mm   2mb   2mp   2ml  2mn  2mq  2mr  2mv  2m.
1n   2nb   2nc   2nd   2nf  2ng  2nl  2nm  2nn  2np  2nq  2nr  2ns
     n2s3m n2s3f 2nt   2nv  2nx  2n.
1p   p2h   p2l   2pn   2pp  p2r  2ps  2pt  2pz  2php 2pht 2p.
1qu2
1r   2rb   2rc   2rd   2rf  2rg  r2h  2rl  2rm  2rn  2rp  2rq  2rr  2rs  2rt
     2rv   2rz   2r.
1s2  2s3ph 2s3s  2stb  2stc 2std 2stf 2stg 2st3l     2stm 2stn 2stp 2stq
     2sts  2stt  2stv  2s.  2st.
1t   2tb   2tc   2td   2tf  2tg  t2h  t2l  t2r  2tm  2tn  2tp  2tq  2tt
     2tv   2t.
1v   v2l   v2r   2vv
1x   2xt   2xx   2x.
1z   2z.
% For medieval Latin
a1ua a1ue a1ui a1uo a1uu
e1ua e1ue e1ui e1uo e1uu
i1ua i1ue i1ui i1uo i1uu
o1ua o1ue o1ui o1uo o1uu
u1ua u1ue u1ui u1uo u1uu
%
a2l1ua a2l1ue a2l1ui a2l1uo a2l1uu
e2l1ua e2l1ue e2l1ui e2l1uo e2l1uu
i2l1ua i2l1ue i2l1ui i2l1uo i2l1uu
o2l1ua o2l1ue o2l1ui o2l1uo o2l1uu
u2l1ua u2l1ue u2l1ui u2l1uo u2l1uu
%
a2m1ua a2m1ue a2m1ui a2m1uo a2m1uu
e2m1ua e2m1ue e2m1ui e2m1uo e2m1uu
i2m1ua i2m1ue i2m1ui i2m1uo i2m1uu
o2m1ua o2m1ue o2m1ui o2m1uo o2m1uu
u2m1ua u2m1ue u2m1ui u2m1uo u2m1uu
%
a2n1ua a2n1ue a2n1ui a2n1uo a2n1uu
e2n1ua e2n1ue e2n1ui e2n1uo e2n1uu
i2n1ua i2n1ue i2n1ui i2n1uo i2n1uu
o2n1ua o2n1ue o2n1ui o2n1uo o2n1uu
u2n1ua u2n1ue u2n1ui u2n1uo u2n1uu
%
a2r1ua a2r1ue a2r1ui a2r1uo a2r1uu
e2r1ua e2r1ue e2r1ui e2r1uo e2r1uu
i2r1ua i2r1ue i2r1ui i2r1uo i2r1uu
o2r1ua o2r1ue o2r1ui o2r1uo o2r1uu
u2r1ua u2r1ue u2r1ui u2r1uo u2r1uu
%
}
