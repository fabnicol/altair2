% This is file treetex.tex of TreeTeX, Version 2.1 (May 23, 1989).
% For changes search for "vs. 2.1".
% 
% TreeTeX is a public domain macro package for drawing
% trees with TeX. It may be freely distributed, provided
% that the following files are kept together:
%  
%   classes.tex, l_pic.tex, readme, tree_doc.aux, tree_doc.bbl
%   tree_doc.dvi, tree_doc.tex, treetex.tex
% 
% Copyright is with Anne Brueggemann-Klein and Derick Wood. 
% Print tree_doc.dvi to get more information about TreeTeX.
% 
% All remarks, bug reports etc. should be directed to
% 
%   Dr. Anne Brueggemann-Klein
%   Institut fuer Informatik
%   Rheinstr. 10--12
%   7800 Freiburg, West Germany
% 
%   email: abk@sun1.ruf.uni-freiburg.dbp.de
% 

\catcode`\@=11                                                                  
                                                                                
%%%%%%%%%%%%%%%%%%%%%%%%%%%%%%%%%%%%%%%%%%%%%%%%%%%%%%%%%%%%%%%%%%%%%%%%%%%%    
%%% Only for testing, delete later                                       %%%    
%%%%%%%%%%%%%%%%%%%%%%%%%%%%%%%%%%%%%%%%%%%%%%%%%%%%%%%%%%%%%%%%%%%%%%%%%%%%    
                                                                                
\tracingonline=2                          %                                      
% \showboxbreadth=100                     % Only for testing                     
% \showboxdepth=100                       %                                      

\newcount\cnta\newcount\cntb\newcount\cntc

\def\showlasttree{%
     \g\cnta\count\l@stdiminfo
     \g\cntb\cnta
     \g\advance\cntb 5 
     \g\advance\cntb \count\l@sttreeheight
     \g\advance\cntb \count\l@sttreeheight
     \ifnum\count\l@sttreeheight=-1\relax
           \g\advance\cntb by 2
           \immediate\write16{Tree contour for dummy node:}
      \else\immediate\write16{Tree contour:}% 
        \fi
     \for\cntc:=\cnta\to\cntb\do\immediate\write16{\the\dimen\cntc}\od}
                                                                               
                                                                                
%%%%%%%%%%%%%%%%%%%%%%%%%%%%%%%%%%%%%%%%%%%%%%%%%%%%%%%%%%%%%%%%%%%%%%%%%%%%    
%%% TeX vs. LaTeX                                                        %%%    
%%%%%%%%%%%%%%%%%%%%%%%%%%%%%%%%%%%%%%%%%%%%%%%%%%%%%%%%%%%%%%%%%%%%%%%%%%%%    
                                                                                
\def\lplain{lplain}                     % Set \LaTeXtrue if TreeTeX is          
\newif\ifLaTeX                          % used together with LaTeX,             
\ifx\fmtname\lplain\LaTeXtrue           % otherwise set \LaTeXfalse             
    \else\LaTeXfalse\fi                 % (LaTeX defines \fmtname=={lplain}).   

\immediate\write16{This is TreeTeX, Version 2.1, for use with \ifLaTeX LaTeX%     
                   \else plain TeX\fi.}                                         
                                                                                
\ifLaTeX \let\lineseg\line              % latex_picture is part of latex.tex,   
   \else \let\@line\line                % so you don't need it if you use       
         \input l_pic                   % TreeTeX together with LaTeX. LaTeX    
         \let\lineseg\line              % has the command \line for geometric   
         \let\line\@line                % lines, and plain TeX has the same     
   \fi                                  % command for lines of text. Because    
                                        % both versions of \line play an        
                                        % important role in the respective      
                                        % macro packages, we introduce a new    
                                        % command \lineseg in TreeTeX for the   
                                        % geometric lines, and \line will have  
                                        % the LaTeX-meaning if TreeTeX is used  
                                        % with LaTeX, and the plain \TeX        
                                        % meaning otherwise.                    
                                                                                
%%%%%%%%%%%%%%%%%%%%%%%%%%%%%%%%%%%%%%%%%%%%%%%%%%%%%%%%%%%%%%%%%%%%%%%%%%%%    
%%% General programming environment                                      %%%    
%%%%%%%%%%%%%%%%%%%%%%%%%%%%%%%%%%%%%%%%%%%%%%%%%%%%%%%%%%%%%%%%%%%%%%%%%%%%    
                                                                                
\catcode`\@=11                                                                  
\let\g\global                                                                   
\def\gxdef{\global\xdef}                                                        
                                                                                
% The command \newcount is redefined such that it can be used           
% inside a definition (i.e. it is no longer an \outer-command).          
                                                                                
\def\newcount{\alloc@0\count\countdef\insc@unt}                                 
                                                                                
% Implementing a for-loop (first argument must be a counter).                   
% Usage: \for<counter>:=<start value>\to<stop value>\do<operations>\od          
% Semantics: the same as a PASCAL for-loop                                      
% Precautions: Don't change the counter-value inside the loop!                  
%              for-loops cannot be nested (nor can the \loop-commands!).        
                                                                                
\def\for#1:=#2\to#3\do#4\od{%
   \def\f@rcount{#1}\def\upp@rlimit{#3}\def\b@dy{#4}\f@rcount=#2\relax\dof@r}

\def\dof@r{\ifnum\f@rcount>\upp@rlimit\relax\let\n@xt\relax
                 \else\b@dy\advance\f@rcount\@ne\let\n@xt\dof@r\fi
           \n@xt}
                                                                                
% \ex repeats a sequence of commands a predetermined number of times.           
% Usage: \ex<number>\times<operations>\xe                                       
% Semantics: <operations> is executed as often as <number> says                 
% Precautions: \ex commands cannot be nested.                                   
                                                                                
\newcount\@xcount
\newcount\t@mes                                                                    
                                                                                
\def\ex#1\times#2\xe{%                                                          
   \@xcount1 \t@mes#1\def\b@dy{#2}\do@x}

\def\do@x{\ifnum\@xcount>\t@mes\let\n@xt\relax
                \else\b@dy\advance\@xcount\@ne\let\n@xt\do@x\fi
          \n@xt}
                                                                              
% \rect@ngle produces a rectangle with horizontal edge length #1, vertical
% edge length #2 and line thickness #3. The reference point is in the center of
% the rectangle. The width is 0pt.

\newskip\thickn@ss
\newskip\@nner
\newskip\@uter

\def\rect@ngle#1#2#3{\hbox to 0pt{%
     \thickn@ss#3%
     \g\@nner#2\g\advance\@nner-\thickn@ss
     \g\divide\@nner\tw@
     \g\@uter#2\g\advance\@uter\thickn@ss
     \g\divide\@uter\tw@
     \hskip 0pt minus .5fil%
     \vrule height\@uter depth\@nner width\thickn@ss
     \vrule height\@uter depth-\@nner width#1%
     \hskip 0pt minus 1fil%
     \vrule height-\@nner depth\@uter width#1%
     \vrule height\@nner depth\@uter width\thickn@ss
     \hskip 0pt minus .5fil%
     }% \hbox
     }% \def

% \s@ries takes two arguments. The first one is a name, say XXX, and            
% the second is a series of arguments, devided by two slashs (//).              
% \s@ries assigns this last series of arguments one after another to the        
% control sequences \XXXi, \XXXii, and so on. Furthermore, a control           
% sequence \XXX is defined, which takes a number k as its argument and          
% expands to \XXXk', where k' is the roman numeral equivalent to k.             

\def\s@ries#1#2{%                                                               
     \g\t@mpcnta1                                                               
     \gdef\t@mp{#1}%                                                            
     \@ssign#2/\l@st  % \l@st is a sentinal element                                                        
     \expandafter\gdef\csname#1\endcsname##1{%                                  
                      \csname#1\romannumeral##1\endcsname}%                     
     }                                                                          
                                                                                
\def\@ssign#1/#2{%
      \expandafter\gdef\csname\t@mp\romannumeral\t@mpcnta\endcsname{#1}%    
      \g\advance\t@mpcnta\@ne                                            
      \ifx#2\l@st                                                               
          \g\let\n@xt\relax                                                     
     \else\g\let\n@xt\@ssign                                                    
       \fi                                                                      
      \n@xt}
                                                                                 
%%%%%%%%%%%%%%%%%%%%%%%%%%%%%%%%%%%%%%%%%%%%%%%%%%%%%%%%%%%%%%%%%%%%%%%%%%%%    
%%% Allocation of internal registers                                     %%%    
%%%%%%%%%%%%%%%%%%%%%%%%%%%%%%%%%%%%%%%%%%%%%%%%%%%%%%%%%%%%%%%%%%%%%%%%%%%%    
                                                                                
\newdimen\leftdist                                                              
\newdimen\rightdist                                                             
\newbox\TeXTree                                                                 
                                                                                
\newcount\sl@pe                                                                 
\newcount\l@vels                                                                
\newcount\s@ze                                                                  
                                                                                
\newbox\circleb@x                                                               
\newbox\squareb@x                                                               
\newbox\dotb@x                                                                  
\newbox\triangleb@x
\newbox\textb@x                                                             
\newbox\frameb@x
                                                                                
\newdimen\circlew@dth                                                           
\newdimen\squarew@dth                                                           
\newdimen\dotw@dth                                                              
\newdimen\trianglew@dth                                                         
\newdimen\textw@dth                                                             
\newdimen\framew@dth
                                                         
\newdimen\vd@st                                                                 
\newdimen\hd@st                                                                 
\newdimen\based@st                                                              
\newdimen\dummyhalfcenterdim@n                                                  
                                                                                
\newcount\t@mpcnta                                                              
\newcount\t@mpcntb                                                              
\newcount\t@mpcntc                                                              
\newcount\t@mpcntd                                                              
\newdimen\t@mpdima                                                              
\newdimen\t@mpdimb                                                              
\newdimen\t@mpdimc                                                              
\newbox\t@mpboxa                                                                
\newbox\t@mpboxb                                                                
                                                                                
\newbox\leftb@x                                                                 
\newbox\rightb@x                                                                
\newbox\centerb@x                                                               
\newbox\beneathb@x                                                              
\newtoks\typ@                                                                   
\newbox\centerb@@x                                                              
\newdimen\centerdim@n                                                           
\newdimen\halfcenterdim@n                                                       
                                                                                
\newdimen\mins@p                                                                
\newdimen\halfmins@p                                                            
\newdimen\tots@p                                                                
\newdimen\halftots@p                                                            
\newdimen\currs@p                                                               
\newdimen\adds@p                                                                
\newcount\l@ftht                                                                
\newcount\r@ghtht                                                               
\newcount\l@ftinfo                                                              
\newcount\r@ghtinfo                                                             
\newbox\l@ftbox                                                                 
\newbox\r@ghtbox                                                                
                                                                                
\newif\ifr@ghthigher  % true iff the right subtree is higher than the left one  
\newif\ifadds@p                                                                 
                                                                                
\newcount\@larg                                                                 
\newcount\@rarg                                                                 
                                                                                
\newif\ifl@fttop                                                                
\newif\ifl@ftonly                                                               
\newif\ifr@ghtonly                                                              
\newif\if@xt                                                                    
\newif\ifl@ftedge                                                               
\newif\ifr@ghtedge                                                              
\newif\ifext@nded                                                               
                                                                                
\newdimen\lm@ff                                                                 
\newdimen\rm@ff                                                                 
\newdimen\lb@ff                                                                 
\newdimen\rb@ff                                                                 
\newdimen\lt@p                                                                  
\newdimen\rt@p                                                                  
                                                                                
\newcount\l@sttreebox     % These four counter allocations have been copied
\newcount\l@sttreeheight  % to this position from the \Tree command
\newcount\l@stdiminfo     % (vs. 2.1). Previously each tree allocated its own
\newcount\l@sttreetype    % counters, using up counters for nothing.

%%%%%%%%%%%%%%%%%%%%%%%%%%%%%%%%%%%%%%%%%%%%%%%%%%%%%%%%%%%%%%%%%%%%%%%%%%%%    
%%% Slope handling for the edges                                         %%%    
%%%%%%%%%%%%%%%%%%%%%%%%%%%%%%%%%%%%%%%%%%%%%%%%%%%%%%%%%%%%%%%%%%%%%%%%%%%%    
                                                                                
% The picture environment of LaTeX gives us a choice of 24 positive             
% slopes for lines (i.e. edges of trees in this context),                       
% including vertical ones. The slope of a line is                               
% given by an x- and an y-value, see L. Lamport, LaTeX, pp. 105f for            
% further details. x/y can have the following values (decreasing amount         
% of slope): 0/1 1/6 1/5 1/4 1/3 2/5 1/2 3/5 2/3 3/4 4/5 5/6 1/1 6/5 5/4        
% 4/3 3/2 5/3 2/1 5/2 3/1 4/1 5/1 6/1.                                          
% The x-values are allocated to \xv@li, \xv@lii,..., \xv@lxxiv, and             
% they can conveniently be accessed by the command \xv@l{<number>}.             
% The same holds for the y-values.                                              
                                                                                
\s@ries{xv@l}{0//1//1//1//1//2//1//3//2//3//4//5//1//6//%                       
              5//4//3//5//2//5//3//4//5//6}                                     
\s@ries{yv@l}{1//6//5//4//3//5//2//5//3//4//5//6//1//5//%                       
              4//3//2//3//1//2//1//1//1//1}                                     
                                                                                
% \hv@ldef calculates \hv@li, \hv@lii,..., \hv@lxxiv for a given dimen          
% \vd@st according to the following picture:                                    
%                                                                               
%          /-|                                                                  
%         /  |                                                                  
%        /   |                                                                  
%       /    |                                                                  
%      /     |\vd@st                                                            
%     /      |                                                                  
%    /       |                                                                  
%   / -|     |                                                                  
%  /   |\yv@l|                                                                  
% /   _|    _|                                                                  
%                                                                               
% |___|                                                                         
% \xv@l                                                                         
% |_________|                                                                   
%  .5\hv@l                                                                      
%                                                                               
% \hv@li,..., \hv@lxxiv are initialized in \beginTree, when the                 
% actual value for \vd@st is known (\vd@st will depend on the point size of     
% the picture). As before, these values can conveniently be accessed by the     
% command \hv@l{<number>}.                                                      
                                                                                
\def\hv@ldef{%                                                                  
     \for\t@mpcnta:=1\to24%                                                     
      \do\g\t@mpdima\vd@st\g\multiply\t@mpdima by\xv@l{\t@mpcnta}%              
         \g\divide\t@mpdima by\yv@l{\t@mpcnta}\g\multiply\t@mpdima by 2         
         \expandafter\gxdef\csname hv@l\romannumeral\t@mpcnta\endcsname{%       
                          \the\t@mpdima}%                                       
      \od}                                                                      
                                                                                
\def\hv@l#1{\csname hv@l\romannumeral#1\endcsname}                              
                                                                                
%%%%%%%%%%%%%%%%%%%%%%%%%%%%%%%%%%%%%%%%%%%%%%%%%%%%%%%%%%%%%%%%%%%%%%%%%%%%    
%%% Naming trees                                                         %%%    
%%%%%%%%%%%%%%%%%%%%%%%%%%%%%%%%%%%%%%%%%%%%%%%%%%%%%%%%%%%%%%%%%%%%%%%%%%%%    
                                                                                
% A TeXtree is stored in TeX's internal registers in the following way:         
% A TeXtree of height h has associated to itself the following internal         
% TeX registers: a box, holding the graphical appearance of the tree,           
% a consecutive number of 6+2h internal dimen registers, holding the        
% additional information about the contour of the tree, a counter               
% holding the height h of the tree, a counter holding the first                 
% position of the additional information registers, and a toks register         
% holding the type of the node (circle, square, dot, triangle, text, or frame).              
% The height and position                                                       
% of additional information are stored in consecutive order for                 
% consecutive trees. The same is true for the boxes and toks.                   
                                                                                
% Four key numbers, the register numbers (addresses)
% for the height, diminfo, box, and type 
% of a tree, enable you to access all information which is stored about the     
% tree. For the last tree on the stack, the four key numbers are stored in the  
% counters \l@sttreeheight, \l@stdiminfo, \l@sttreebox, and \l@sttreetype,      
% the key numbers for the next tree are \l@sttreeheight-2, \l@stdiminfo-2,      
% \l@sttreebox-1, and \l@sttreetype-1, and so on.                                 
                                                                                
% The macro \n@metree gives names to some registers associated with the tree.   
% The tree to be named is specified by its four key numbers. \n@metree takes    
% five arguments, namely the keynumbers for height, info, box and type of the   
% tree, and the name to be given to the tree. If the tree gets the name XXX and 
% the key numbers are h, i, b, and t, the commands on the left side of the      
% following list get the meaning on the right side.                             
                                                                                
% \XXXht	<---	\count h                                                          
% \XXXinfo	<---	\count i                                                        
% \XXXbox	<---	b                                                                
% \XXXtype	<---	\toks t                                                         
% \XXXlmoff	<---	\dimen f          where f is the address stored                
%                                         in \count i                           
% \XXXrmoff	<---	\dimen (f+1)                                                   
% \XXXlboff	<---	\dimen (f+2)                                                   
% \XXXrboff	<---	\dimen (f+3)                                                   
% \XXXltop	<---	\dimen (f+4)                                                    
% \XXXrtop	<---	\dimen (f+5)                                                    
% \XXXloff	<---	\dimen (f+4+2g)   where g is the height stored in               
% 					  \count h, i.e. \dimen (f+4+2g)                                         
%  					  holds loff(1) of the tree, if g>0                                           
% \XXXroff	<---	\dimen (f+5+2g)                                                 
                                                                                
% The macro \pr@vioustree sets \l@sttreeheight, \l@stdiminfo,                   
% \l@sttreebox and \l@sttreetype to the key numbers of the previous tree        
% and gives the name `l@st' to this tree.                                       
                                                                                
% The macro \@ddname which has two names n1 and n2 as arguments, gives          
% the tree with name n1 the additional name n2.                                 
                                                                                
% The macro \n@mel@st gives the name `l@st' to the tree with the                
% key numbers \l@sttreeheight, \l@stdiminfo, \l@sttreebox and \l@sttreetype.    
                                                                                
% The macro \n@xttree sets \l@sttreebox, \l@sttreeheight, \l@stdiminfo,         
% \count\l@stdiminfo, and \l@sttreetype to the next free position.              
                                                                                
% The macro \@ppenddummy pushs a dummy onto the stack and names it `l@st'.      
% The dummy has height -1, its box is the empty box, its type is circle,         
% and all dimensions are 0pt.                                                   
                                                                                
% \p@s#1#2#3 sets counter #1 to position #2 of tree #3. #1 must be a counter,   
% #3 must be a name for the tree. If the tree has the name XXX, \XXXinfo        
% must be a number holding the first position of the dimen-parameters of        
% the tree and \XXXht must hold the height of the tree.                         
% #2 must be one of the following control sequences indicating the              
% desired position: \lmoff, \rmoff, \lboff, \rboff, \ltop, \rtop, \loff, or     
% \roff. \loff and \roff give the left resp. right offset of the *top*          
% level of the tree.                                                            
                                                                                
\def\p@s#1#2#3{%                                                                
     \g#1\csname#3info\endcsname                                                
     \gxdef\t@mp{\csname#3ht\endcsname}%                                        
     \ifnum\t@mp<0 \gxdef\t@mp{0}\fi                                            
     #2{#1}%                                                                    
     }                                                                          
                                                                                
\chardef\@lmoff0 \chardef\@rmoff1 \chardef\@ltop4 \chardef\@rtop5               
\chardef\@lboff2 \chardef\@rboff3 \chardef\@loff4 \chardef\@roff5               
                                                                                
\def\lmoff#1{\g\advance#1 by\@lmoff}                                            
\def\rmoff#1{\g\advance#1 by\@rmoff}                                            
\def\lboff#1{\g\advance#1 by\@lboff}                                            
\def\rboff#1{\g\advance#1 by\@rboff}                                            
\def\ltop#1{\g\advance#1 by\@ltop}                                              
\def\rtop#1{\g\advance#1 by\@rtop}                                              
\def\loff#1{\g\advance#1 by\@loff\g\advance#1 by\t@mp                           
     \g\advance#1 by\t@mp\relax}                                                
\def\roff#1{\g\advance#1 by\@roff\g\advance#1 by\t@mp                           
     \g\advance#1 by\t@mp\relax}                                                
                                                                                
% \n@meinfo#1 defines for an argument XXX (name of a tree) \XXXlmoff,           
% \XXXrmoff, ... as lmoff(XXX), rmoff(XXX),... .                                
% The following arguments will be used: l@ft, r@ght, l@st,                      
% m@n, and m@x.                                                                 
                                                                                
\def\n@meinfo#1{%                                                               
     \n@me@nfo{#1}{lmoff}\n@me@nfo{#1}{rmoff}%                                  
     \n@me@nfo{#1}{lboff}\n@me@nfo{#1}{rboff}%                                  
     \n@me@nfo{#1}{ltop}\n@me@nfo{#1}{rtop}%                                    
     \n@me@nfo{#1}{loff}\n@me@nfo{#1}{roff}%                                    
     }                                                                          
                                                                                
\def\n@me@nfo#1#2{%                                                             
     \p@s\t@mpcnta{\csname#2\endcsname}{#1}%                                    
     \expandafter\gxdef\csname#1#2\endcsname{\dimen\the\t@mpcnta}}              
                                                                                
\def\n@metree#1#2#3#4#5{%                                                       
     \expandafter\gxdef\csname#5ht\endcsname{\count\the#1}%                     
     \expandafter\gxdef\csname#5info\endcsname{\count\the#2}%                   
     \expandafter\gxdef\csname#5box\endcsname{\the#3}%                          
     \expandafter\gxdef\csname#5type\endcsname{\toks\the#4}%                    
     \n@meinfo{#5}%                                                             
     }                                                                          
                                                                                
\chardef\@cntoff3 \chardef\@boxoff1 \chardef\@dimoff2 \chardef\@typeoff1        
                                                                                
\def\pr@vioustree{%                                                             
     \g\advance\l@sttreeheight by-\@cntoff                                      
     \g\advance\l@stdiminfo by-\@cntoff                                         
     \g\advance\l@sttreetype by-\@cntoff                                        
     \g\advance\l@sttreebox by-\@boxoff                                         
     \n@mel@st                                                                  
     }                                                                          
                                                                                
\def\@ddname#1#2{%                                                              
     \expandafter\gxdef\csname#2ht\endcsname{\csname#1ht\endcsname}%            
     \expandafter\gxdef\csname#2info\endcsname{\csname#1info\endcsname}%        
     \expandafter\gxdef\csname#2type\endcsname{\csname#1type\endcsname}%        
     \expandafter\gxdef\csname#2box\endcsname{\csname#1box\endcsname}%          
     \n@meinfo{#2}%                                                             
     }                                                                          
                                                                                
\def\n@xttree{%                                                                 
     \p@s\t@mpcnta\loff{l@st}\g\advance\t@mpcnta by\@dimoff                     
     \g\advance\l@sttreeheight by\@cntoff                                       
     \g\advance\l@stdiminfo by\@cntoff                                          
     \g\advance\l@sttreetype by\@cntoff                                         
     \g\advance\l@sttreebox by\@boxoff                                          
     \g\count\l@stdiminfo\t@mpcnta                                              
     }                                                                          
                                                                                
\def\@ppenddummy{% pushs a dummy onto the stack and names it `l@st'             
                 % The dummy has height -1, its box is the empty box, the type   
                 % is circle, and all dimensions are 0pt.                       
     \n@xttree \g\count\l@sttreeheight-\@ne\n@mel@st                            
     \l@sttype{circle}%                                                         
     \g\setbox\l@stbox\copy\voidb@x                                             
     \g\l@stlmoff=0pt\g\l@strmoff=0pt\g\l@stlboff=0pt\g\l@strboff=0pt%          
     \g\l@stltop=0pt\g\l@strtop=0pt\g\l@stloff=0pt\g\l@stroff=0pt%              
     }                                                                          
                                                                                
\def\g@tchildren{% enables us to talk about the left and the right child
                 % (names l@ft resp. r@ght) and the smaller and the larger
                 % child (names m@n resp. m@x)                                                             
     \ifl@fttop\@ddname{l@st}{l@ft}%                                            
               \pr@vioustree                                                    
               \@ddname{l@st}{r@ght}%                                           
          \else\@ddname{l@st}{r@ght}%                                           
               \pr@vioustree                                                    
               \@ddname{l@st}{l@ft}%                                            
            \fi                                                                 
     \ifnum\r@ghtht>\l@ftht\relax                                               
               \r@ghthighertrue                                                 
               \@ddname{r@ght}{m@x}%                                            
               \@ddname{l@ft}{m@n}%                                             
          \else\r@ghthigherfalse                                                
               \@ddname{l@ft}{m@x}%                                             
               \@ddname{r@ght}{m@n}%                                            
            \fi                                                                 
               }                                                                
                                                                                
\def\n@mel@st{%                                                                 
     \n@metree\l@sttreeheight\l@stdiminfo\l@sttreebox\l@sttreetype{l@st}}       
                                                                                
%%%%%%%%%%%%%%%%%%%%%%%%%%%%%%%%%%%%%%%%%%%%%%%%%%%%%%%%%%%%%%%%%%%%%%%%%%%%    
%%% Initialization of the tree environment                               %%%    
%%%%%%%%%%%%%%%%%%%%%%%%%%%%%%%%%%%%%%%%%%%%%%%%%%%%%%%%%%%%%%%%%%%%%%%%%%%%    
                                                                                
\def\beginTree{%                                                                
     \begingroup                                                                
     \unitlength 1pt%                                                           
     \divide\unitlength by 65536
     \l@sttreebox\count14                                                       
     \l@sttreeheight\count10                                                    
     \advance\l@sttreeheight by \@ne                                            
     \count\l@sttreeheight=-1                                                   
     \l@stdiminfo\l@sttreeheight                                                
     \advance\l@stdiminfo by \@ne                                               
     \count\l@stdiminfo\count11                                                 
     \advance\count\l@stdiminfo by -5                                           
     \l@sttreetype\l@stdiminfo                                                  
     \advance\l@sttreetype by\@ne                                               
     \count\l@sttreetype\count15                                                
     \n@mel@st\ignorespaces                                                     
     }                                                                          
                                                                                
\let\Tree\beginTree                                                             
                                                                                
\def\endTree{%                                                                  
     \g\leftdist-\l@stlmoff\g\advance\leftdist by \l@stltop                     
     \g\rightdist\l@strmoff\g\advance\rightdist by\l@strtop                     
     \g\setbox\TeXTree\box\l@stbox\endgroup\ignorespaces}                       
                                                                                
%%%%%%%%%%%%%%%%%%%%%%%%%%%%%%%%%%%%%%%%%%%%%%%%%%%%%%%%%%%%%%%%%%%%%%%%%%%%    
%%% Specification of nodes                                               %%%    
%%%%%%%%%%%%%%%%%%%%%%%%%%%%%%%%%%%%%%%%%%%%%%%%%%%%%%%%%%%%%%%%%%%%%%%%%%%%    
                                                                                
% A node is defined by the command \node{<specifications>}.                     
% <specifications> defines the labels, graphical appearence and the order       
% of the node and the thickness of the edges.                                   
% Labels are defined by the commands \lft, \rght, \cntr,                        
% and \bnth. The specification of empty labels may be omitted.                  
% The graphical appearence is defined by the command \type{<type>}.             
% <type> can have the values `circle', `square', `dot',                         
% `triangle', 'text', or 'frame'. The thickness of the edges is defined by                      
% \leftthick and \rightthick that give thick edges instead of the               
% normal thin ones. Furthermore,                                                
% the order of the node is given by the following commands:                     
% \external (if the node is an external node), \leftonly (if the node has a     
% left successor only), \rightonly (analogous), \unary (if the node is an       
% unary one), and \lefttop (the                                                 
% node which has been defined before this one, is supposed to be the left       
% successor instead of the right one).                                          
% Default: two children, no labels, type circle, thin edges,                    
% left child has been entered first.                                            
                                                                                
\def\th@ck{\let\@linefnt\tenlnw                                                 
     \@wholewidth\fontdimen8\tenlnw\@halfwidth.5\@wholewidth}                   
                                                                                
\def\leftthick{\g\let\l@ftthick\th@ck}                                          
\def\rightthick{\g\let\r@ghtthick\th@ck}                                        
\def\lft#1{\g\setbox\leftb@x\hbox{#1\ }}                                        
\def\rght#1{\g\setbox\rightb@x\hbox{\ #1}}                                      
\def\cntr#1{\g\setbox\centerb@x\hbox{#1\strut}}                   
\def\bnth#1{\g\setbox\beneathb@x\hbox to0pt{\hss\strut#1\hss}}                        
\def\type#1{%                                                                   
     \g\setbox\centerb@@x\copy\csname#1b@x\endcsname                            
     \g\centerdim@n\csname#1w@dth\endcsname                                     
     \typ@{#1}%                                                                 
     \g\halfcenterdim@n=.5\centerdim@n}                                         
                                                                                
\def\ext@nded{\g\ext@ndedfalse} % This definition must precede                  
                                % \input TreeTeX.sty  (outdated now!)                           
                                                                                
\def\node#1{%                                                                   
     %%% Initialization (node type and labels), defaults and actual values      
     \g\setbox\leftb@x\copy\voidb@x                                             
     \g\setbox\rightb@x\copy\voidb@x                                            
     \g\setbox\centerb@x\copy\voidb@x                                           
     \g\setbox\beneathb@x\copy\voidb@x                                          
     \type{circle}%                                                             
     \g\l@fttopfalse\g\l@ftonlyfalse\g\l@ftedgetrue                             
     \g\r@ghtonlyfalse\g\r@ghtedgetrue\g\@xtfalse\ext@nded\n@dummy              
     \g\let\l@ftthick\relax\g\let\r@ghtthick\relax                              
     #1% 
     \@pdcenter                                                                       
     \d@mmy
     \n@de
     \ignorespaces                                                              
     }

\def\@pdcenter{\csname\the\typ@ @cntr\endcsname}

\let\circle@cntr\relax
\let\square@cntr\relax
\let\triangle@cntr\relax
\let\dot@cntr\relax

\def\text@cntr{%
     \g\centerdim@n\wd\centerb@x
     \g\halfcenterdim@n.5\centerdim@n}                                                                          
     
\def\frame@cntr{% 
     \g\setbox\centerb@x\hbox{\ \unhcopy\centerb@x\ }
     \g\centerdim@n\wd\centerb@x
     % \g\advance\centerdim@n\fontdimen2\font       
     \g\halfcenterdim@n.5\centerdim@n
     \g\setbox\centerb@@x\rect@ngle{\centerdim@n}{\squarew@dth}{.4pt}}
                                                                           
\def\leftonly{\g\l@ftonlytrue\g\r@ghtedgefalse\g\let\d@mmy\l@ftdummy}           
\def\rightonly{\g\r@ghtonlytrue\g\l@ftedgefalse\g\let\d@mmy\r@ghtdummy}         
\def\unary{\g\r@ghtedgefalse\g\let\d@mmy\@ndummy}                               
\def\external{\g\@xttrue\g\l@ftedgefalse\g\r@ghtedgefalse\g\let\d@mmy\@xtdummy} 
                                                                                
\def\lefttop{\g\l@fttoptrue}                                                    
                                                                                
\def\@xtdummy{%                                                                 
     \@ppenddummy                                                     
     \g\l@strtop-\halfmins@p                                         
     \@ppenddummy
     \g\l@stltop-\halfmins@p                                         
     }                                                                          
                                                                                
\def\n@dummy{\g\let\d@mmy\relax}                                                
                                                                                
\def\l@ftdummy{% cf. \g@tposition                                                               
     \@ppenddummy                                                      
     \g\l@stltop=\dummyhalfcenterdim@n                                          
     \g\l@strtop=\dummyhalfcenterdim@n                                          
     }                                                                          
                                                                                
\def\r@ghtdummy{% cf. \g@tposition                                                              
     \lefttop                                                                   
     \@ppenddummy                                                      
     \g\l@stltop=\dummyhalfcenterdim@n                                          
     \g\l@strtop=\dummyhalfcenterdim@n                                          
     }                                                                          
                                                                                
\def\@ndummy{%                                                                  
     \g\t@mpdima\l@strtop\relax                                                 
     \@ppenddummy                                                     
     \g\l@stltop-\mins@p\g\advance\l@stltop by-\t@mpdima                        
     \g\l@strtop=\t@mpdima                                                      
     }                                                                          
                                                                                
\def\n@de{%                                                                     
     \g@tposition       % naming children and calculating \sl@pe and \tots@p    
     \g@tlt@p\g@trt@p   % calculating \lt@p and \rt@p                           
     \g@tlm@ff\g@trm@ff % calculating \lm@ff and \rm@ff                         
     \g@tlb@ff\g@trb@ff % calculating \lb@ff and \rb@ff                         
     \@pdlroff          % updating loff and roff for all levels but the top one 
     \@pdloffl\@pdroffl % updating loff(1) and roff(1) of the parent tree             
     \@pddim            % updating ltop, rtop, lmoff, rmoff, lboff, and rboff   
     \@pdinfo\@pdht     % updating diminfo and treeheight                       
     \@pdbox            % updating treebox                                      
     \@pdtype           % updating type                                         
     \n@mel@st          % giving the name `l@st' to the new tree 
     \ignorespaces                                                              
     }                                                                          
                                                                                
\def\g@tposition{% naming children and calculating \sl@pe, \tots@p, and node offsets
     \g@tchildren\c@lcsep\c@lcslope\c@lcoffsets
     \ifext@nded\relax                                                          
           \else\ifl@ftonly\g\r@ghtrtop=-\tots@p                                
                           \g\advance\r@ghtrtop by\l@ftrtop                     
                        \fi                                                     
                \ifr@ghtonly\g\l@ftltop=-\tots@p                                
                            \g\advance\l@ftltop by\r@ghtltop                    
                         \fi                                                    
             \fi % cf. \l@ftdummy and \r@ghtdummy                                                               
     }                                                                          
                                                                                
\def\@pdinfo{% updating diminfo                                                 
     \g\l@stinfo=\m@xinfo\relax                                                 
     }                                                                          
                                                                                
\def\@pdht{% updating treeheight                                                
     \g\l@stht=\m@xht                                                           
     \g\advance\l@stht by\@ne                                                   
     }                                                                          
                                                                                
\def\@pdtype{% updating type                                                    
     \g\l@sttype\typ@                                                           
     }                                                                          
                                                                                
\def\g@tlt@p{% calculating \lt@p                                                
     \g\lt@p\wd\leftb@x\g\advance\lt@p by\halfcenterdim@n                       
     }                                                                          
                                                                                
\def\g@trt@p{% calculating \rt@p                                                
     \g\rt@p\wd\rightb@x\g\advance\rt@p by\halfcenterdim@n                      
     }                                                                          
                                                                                
\def\g@tlm@ff{% calculating \lm@ff                                              
     % \lm@ff:=lmoff(left tree)-ltop(left tree)                                 
     %          -.5\tots@p+\lt@p                                                
     \g\lm@ff\l@ftlmoff                                                         
     \g\advance\lm@ff by-\l@ftltop                                              
     \g\advance\lm@ff by-\halftots@p                                            
     \g\advance\lm@ff by\lt@p\relax                                             
     % if ht(left tree) < ht(right tree)                                        
     %    \t@mpdima:=lmoff(right tree)-ltop(right tree)+.5\tots@p+\lt@p         
     %    \lm@ff:=min(\lm@ff,\t@mpdima) fi                                      
     \ifnum\l@ftht<\r@ghtht\relax                                               
           \g\t@mpdima\r@ghtlmoff                                               
           \g\advance\t@mpdima by-\r@ghtltop                                    
           \g\advance\t@mpdima by\halftots@p                                    
           \g\advance\t@mpdima by\lt@p\relax                                    
           \ifdim\t@mpdima<\lm@ff\relax                                         
                 \g\lm@ff\t@mpdima                                              
              \fi                                                               
        \fi                                                                     
     % \lm@ff:=min(\lm@ff,0pt)                                                  
     \ifdim0pt<\lm@ff\relax                                                     
           \g\lm@ff=0pt%                                                        
        \fi                                                                     
     }                                                                          
                                                                                
\def\g@trm@ff{% calculating \rm@ff                                              
     % analog to lm@ff                                                          
     % \rm@ff:=rmoff(right tree)+rtop(right tree)                               
     %         +.5\tots@p-\rt@p                                                 
     \g\rm@ff\r@ghtrmoff                                                        
     \g\advance\rm@ff by\r@ghtrtop                                              
     \g\advance\rm@ff by\halftots@p                                             
     \g\advance\rm@ff by-\rt@p\relax                                            
     % \t@mpdima:=rmoff(left tree)+rtop(left tree)-.5\tots@p-\rt@p              
     \ifnum\r@ghtht<\l@ftht\relax                                               
           \g\t@mpdima\l@ftrmoff                                                
           \g\advance\t@mpdima by\l@ftrtop                                      
           \g\advance\t@mpdima by-\halftots@p                                   
           \g\advance\t@mpdima by-\rt@p\relax                                   
           \ifdim\t@mpdima>\rm@ff\relax                                         
                 \g\rm@ff\t@mpdima                                              
              \fi                                                               
        \fi                                                                     
     % \rm@ff:=max(\rm@ff,0pt)                                                  
     \ifdim0pt>\rm@ff\relax                                                     
           \g\rm@ff=0pt                                                         
        \fi                                                                     
     }                                                                          
                                                                                
\def\g@tlb@ff{% calculating \lb@ff                                              
     % \lb@ff:=lboff(right tree)-ltop(right tree)+.5\tots@p+\lt@p               
     %  resp.:=lboff(left tree)-ltop(left tree)                                 
     %         -.5\tots@p+\lt@p 
     \if@xt\g\lb@ff0pt%
      \else\ifnum\l@ftht<\r@ghtht\relax                                               
                 \g\lb@ff\r@ghtlboff                                                  
                 \g\advance\lb@ff by-\r@ghtltop                                       
                 \g\advance\lb@ff by\halftots@p                                       
                 \g\advance\lb@ff by\lt@p\relax                                       
            \else\g\lb@ff\l@ftlboff                                                   
                 \g\advance\lb@ff by-\l@ftltop                                        
                 \g\advance\lb@ff by-\halftots@p                                      
                 \g\advance\lb@ff by\lt@p\relax                                       
              \fi
        \fi                                                                     
     }                                                                          
                                                                                
\def\g@trb@ff{% calculating \rb@ff                                              
     % \rb@ff:=rboff(left tree)+rtop(left tree)-.5\tots@p-\rt@p                 
     %  resp.:=rboff(right tree)+rtop(right tree)                               
     %         +.5\tots@p-\rt@p                                                 
     \if@xt\g\rb@ff0pt%
      \else\ifnum\r@ghtht<\l@ftht\relax                                               
                 \g\rb@ff\l@ftrboff                                                   
                 \g\advance\rb@ff by\l@ftrtop                                         
                 \advance\rb@ff by-\halftots@p                                        
                 \g\advance\rb@ff by-\rt@p\relax                                      
            \else\g\rb@ff\r@ghtrboff                                                  
                 \g\advance\rb@ff by\r@ghtrtop                                        
                 \g\advance\rb@ff by\halftots@p                                       
                 \g\advance\rb@ff by-\rt@p\relax                                      
              \fi
        \fi                                                                     
     }                                                                          
                                                                                
\def\@pdlroff{% updating loff and roff for all levels but the top one           
     % if right tree higher \t@mpdima:=-ltop(right tree)                        
     %                      \t@mpdimb:=lboff(left tree)-ltop(left tree)         
     %                      % \t@mpdimb holds the offset between the node       
     %                      % and the left edge of the bottom of the left tree  
     %                 else \t@mpdima:=rtop(left tree)                          
     %                      \t@mpdimb:=lboff(right tree)+rtop(right tree)       
     %                      % substitute left by right                          
     %                   fi                                                     
     \ifr@ghthigher\g\t@mpdima-\r@ghtltop\relax                                 
                   \g\t@mpdimb\l@ftlboff                                        
                   \g\advance\t@mpdimb by-\l@ftltop\relax                       
              \else\g\t@mpdima\l@ftrtop\relax                                   
                   \g\t@mpdimb\r@ghtlboff                                       
                   \g\advance\t@mpdimb by\r@ghtrtop\relax                       
                \fi                                                             
     \ifr@ghthigher\p@s\t@mpcnta\loff{m@n}% pointer to loff(1) of smaller tree  
                   \p@s\t@mpcntb\loff{m@x}% pointer to loff(1) of larger tree  
              \else\p@s\t@mpcnta\roff{m@n}% pointer to roff(1) of smaller tree  
                   \p@s\t@mpcntb\roff{m@x}% pointer to roff(1) of larger tree  
                \fi  % if the right tree is the higher one you have to shift    
                     % the left profile of the smaller tree, otherwise the      
                     % right one                                                
     % For every level, \t@mpdima contains the offset between the node of the   
     % higher tree and the inner edge of the next level. Furthermore, if the    
     % right tree is the higher one, the left profile of the left tree becomes  
     % the upper part of the left profile of the parent tree, otherwise         
     % substitute `left' by `right.'                                            
     \ex\m@nht\times                                                            
        \g\advance\t@mpdima by\dimen\t@mpcntb                                   
        \g\dimen\t@mpcntb\dimen\t@mpcnta                                        
        \g\advance\t@mpcnta by-\@dimoff                                         
        \g\advance\t@mpcntb by-\@dimoff\relax                                   
     \xe                                                                        
     % The link between the last outer level of the smaller tree and the next   
     % level of the higher one:                                                 
     % \dimen\t@mpcntb:=\dimen\t@mpcntb+\t@mpdima+\tots@p-\t@mpdimb             
     %                   if left tree is smaller than right tree                
     %              resp.\dimen\t@mpcntb+\t@mpdima-\tots@p-\t@mpdimb            
     %                   if right tree is smaller than left tree                
     \ifnum\m@xht=\m@nht\relax                                                  
      \else\g\advance\dimen\t@mpcntb by\t@mpdima                                
           \ifnum\l@ftht<\r@ghtht\relax                                         
                 \g\advance\dimen\t@mpcntb by\tots@p                            
            \else\g\advance\dimen\t@mpcntb by-\tots@p                           
              \fi                                                               
           \g\advance\dimen\t@mpcntb by-\t@mpdimb                               
        \fi                                                                     
     }                                                                          
                                                                                
\def\@pdloffl{% updating loff(1) of parent tree                                 
     % loff(1) of parent tree:=+\lt@p-.5\tots@p-ltop(left tree)                 
     \p@s\t@mpcnta\loff{m@x}%                                                   
     \g\advance\t@mpcnta by \@dimoff\relax % pointer to loff(0) of parent tree  
     \g\dimen\t@mpcnta\lt@p                                                     
     \g\advance\dimen\t@mpcnta by-\halftots@p                                   
     \g\advance\dimen\t@mpcnta by-\l@ftltop\relax                               
     }                                                                          
                                                                                
\def\@pdroffl{% updating roff(1) of parent tree                                 
     % roff(l) of parent tree:=-\rt@p+.5\tots@p+rtop(right tree)                
     \p@s\t@mpcnta\roff{m@x}%                                                   
     \g\advance\t@mpcnta by \@dimoff\relax % pointer to roff(0) of parent tree  
     \g\dimen\t@mpcnta-\rt@p                                                    
     \g\advance\dimen\t@mpcnta by\halftots@p                                    
     \g\advance\dimen\t@mpcnta by\r@ghtrtop\relax                               
     }                                                                          
                                                                                
\def\@pddim{% updating ltop, rtop, lmoff, rmoff, lboff, and rboff               
     \g\m@xlmoff=\lm@ff\g\m@xrmoff=\rm@ff                                       
     \g\m@xlboff=\lb@ff\g\m@xrboff=\rb@ff                                       
     \g\m@xltop=\lt@p\g\m@xrtop=\rt@p                                           
     }                                                                          
                                                                                
\def\@pdbox{% pushing the nodebox on the stack: updating treebox                
     \g\@xarg\xv@l\sl@pe\g\@yarg\yv@l\sl@pe                                     
     \ifnum\sl@pe=1 % vertical edge                                             
           \g\t@mpdima\vd@st                                                    
           \g\advance\t@mpdima by-\y@ff\typ@                                    
           \g\advance\t@mpdima by-\y@ff\l@fttype                                
           \g\@larg\t@mpdima % \@larg is a number register!                     
           \g\t@mpdima\vd@st                                                    
           \g\advance\t@mpdima by-\y@ff\typ@                                    
           \g\advance\t@mpdima by-\y@ff\r@ghttype                               
           \g\@rarg\t@mpdima % \@rarg is a number register!                     
      \else\g\t@mpdima\halftots@p                                               
           \g\advance\t@mpdima by-\x@ff\typ@                                    
           \g\advance\t@mpdima by-\x@ff\l@fttype                                
           \g\@larg\t@mpdima % \@larg is a number register!                     
           \g\t@mpdima\halftots@p                                               
           \g\advance\t@mpdima by-\x@ff\typ@                                    
           \g\advance\t@mpdima by-\x@ff\r@ghttype                               
           \g\@rarg\t@mpdima % \@rarg is a number register!                     
        \fi                                                                     
     \g\setbox\l@sttreebox\hbox{%
           \ifvoid\leftb@x\relax
             \else\hskip-\halfcenterdim@n\hskip-\wd\leftb@x
                  \unhcopy\leftb@x\hskip\halfcenterdim@n
               \fi
           \ifvoid\centerb@x\relax
             \else\g\t@mpdima-.5\wd\centerb@x\hskip\t@mpdima
                  \unhbox\centerb@x\hskip\t@mpdima
               \fi
           \ifvoid\rightb@x\relax
             \else\g\t@mpdima-\wd\rightb@x\hskip\halfcenterdim@n
                  \unhbox\rightb@x\hskip\t@mpdima\hskip-\halfcenterdim@n
               \fi
           \raise\based@st\copy\centerb@@x
           \if@xt\relax
                 \lower\s@ze pt\hbox to0pt{\hss\unhbox\beneathb@x\hss}%
            \else\hskip-\halftots@p
                 \lower\vd@st\box\l@ftbox
                 \ifl@ftedge\drawl@ftedge\else\hskip\halftots@p\fi
                 \ifr@ghtedge\drawr@ghtedge\else\hskip\halftots@p\fi
                 \lower\vd@st\box\r@ghtbox
                 \hskip-\halftots@p
              \fi
           }% of hbox
     }                                                                          
                                                                                
\def\drawl@ftedge{%                                                             
           \hskip\x@ff\l@fttype                                                 
           \g\t@mpdimc\y@ff\l@fttype\g\advance\t@mpdimc by\based@st
           \g\advance\t@mpdimc-\vd@st
           \raise\t@mpdimc                                                      
           \hbox{\l@ftthick\lineseg(\@xarg,\@yarg){\@larg}}%
           \hskip\x@ff\typ@                                                     
     }                                                                          
                                                                                
\def\drawr@ghtedge{%                                                            
           \hskip\x@ff\typ@                                                     
           \g\t@mpdimc\vd@st                                                      
           \g\advance\t@mpdimc by \based@st
           \g\advance\t@mpdimc by -\y@ff\typ@\relax
           \g\advance\t@mpdimc by- \vd@st
           \raise\t@mpdimc                                                      
           \hbox{\r@ghtthick\lineseg(\@xarg,-\@yarg){\@rarg}}% 
           \hskip\x@ff\r@ghttype                                                
     }                                                                          
                                                                                
\def\x@ff#1{%                                                                   
     \csname\the#1x@ff\endcsname\sl@pe                                          
     }                                                                          
                                                                                
\def\y@ff#1{%                                                                   
     \csname\the#1y@ff\endcsname\sl@pe                                          
     }                                                                          
                                                                                
%%%%%%%%%%%%%%%%%%%%%%%%%%%%%%%%%%%%%%%%%%%%%%%%%%%%%%%%%%%%%%%%%%%%%%%%%%%%    
%%% Calculating the separation of subtrees                               %%%    
%%%%%%%%%%%%%%%%%%%%%%%%%%%%%%%%%%%%%%%%%%%%%%%%%%%%%%%%%%%%%%%%%%%%%%%%%%%%    
                                                                                
% \c@lcslope will calculate the required slope of the edges                     
% when the nodes are to be \vd@st apart vertically                              
% and at least \tots@p apart horizontally. This slope is returned by            
% the value of the counter \sl@pe (a number between 1 and 23).                  
% Furthermore, \tots@p is updated in order to fit to this slope.                
                                                                                
\def\c@lcslope{%                                                                
     \g\sl@pe1                                                                  
       \loop                                                                    
      \ifdim\hv@l\sl@pe < \tots@p                                               
            \g\advance\sl@pe by1                                                
     \repeat
     \g\tots@p\hv@l\sl@pe                                                       
     \g\halftots@p.5\tots@p}                                                    

                                                                                
\def\c@lcsep{%                                                                  
     %%% \tots@p:=\mins@p + rtop(left tree) + ltop(right tree)                  
     %%% \currs@p:=\mins@p                                                      
     \g\tots@p\mins@p                                                           
     \g\advance\tots@p by\l@ftrtop                                              
     \g\advance\tots@p by\r@ghtltop\relax                                       
     \g\currs@p\mins@p                                                          
     %%% \t@mpcnta:= pointer to roff(0) of left tree                            
     %%% \t@mpcntb:= pointer to loff(0) of right tree                           
     \p@s\t@mpcnta\roff{l@ft}%                                                  
     \p@s\t@mpcntb\loff{r@ght}%                                                 
     %%% Calculate \currs@p and update \tots@p for each level of the           
     %%% smaller tree                                                           
     %%% If at any level the subtrees are as close or closer than at            
     %%% the level of their roots, they will be moved apart by the additional   
     %%% amount of \adds@p                                                      
     \g\adds@pfalse                                                             
     \g\t@mpcntc\m@nht                               
     \ex\t@mpcntc\times                                                         
        \g\advance\currs@p by-\dimen\t@mpcnta                                   
        \g\advance\currs@p by \dimen\t@mpcntb                                   
        \ifdim\mins@p<\currs@p                                                  
         \else\g\adds@ptrue                                                     
           \fi                                                                  
        \ifdim\currs@p<\mins@p                                                  
              \g\advance\tots@p by\mins@p                                       
              \g\advance\tots@p by -\currs@p                                    
              \g\currs@p\mins@p                                                 
           \fi                                                                  
        \g\advance\t@mpcnta by -\@dimoff                                        
        \g\advance\t@mpcntb by -\@dimoff                                        
     \xe                                                                        
     \ifadds@p\g\advance\tots@p by\adds@p\fi}                                   
                                                                                
%%%%%%%%%%%%%%%%%%%%%%%%%%%%%%%%%%%%%%%%%%%%%%%%%%%%%%%%%%%%%%%%%%%%%%%%%%%%    
%%% Predefined trees                                                     %%%    
%%%%%%%%%%%%%%%%%%%%%%%%%%%%%%%%%%%%%%%%%%%%%%%%%%%%%%%%%%%%%%%%%%%%%%%%%%%%    
                                                                                
% \tri@ngle produces a triangle which covers \l@vels many level of a tree.      
% The slope of the edges is given by \sl@pe, the reference point of the         
% triangle is in the top, the width is 0pt.                                     
                                                                                
\def\tri@ngle{%                                                                 
     \vtop{\g\@xarg\xv@l\sl@pe \g\@yarg\yv@l\sl@pe                              
           \g\t@mpdimc\l@vels\vd@st 
           \g\advance\t@mpdimc by .5\squarew@dth
           \g\multiply\t@mpdimc\xv@l\sl@pe
           \g\divide\t@mpdimc\yv@l\sl@pe
           \g\@larg\t@mpdimc                                                    
           \offinterlineskip                                                    
           \vskip0pt% Force the reference point to the top                                                         
           \hbox to0pt{\hss\lineseg(\@xarg,\@yarg){\@larg}%                     
                       \hskip\t@mpdimc\rlap{\lineseg(-\@xarg,\@yarg){\@larg}}%  
                       \hss}%                                                   
           \setbox\t@mpboxa                                                          
           \hbox to0pt{\hss\vrule height.2pt depth.2pt width2\t@mpdimc\hss}%
           \t@mpdimc-.5\squarew@dth\advance\t@mpdimc\based@st
           \ht\t@mpboxa0pt\dp\t@mpboxa\t@mpdimc\copy\t@mpboxa    
          }%                                                                    
     }                                                                          
                                                                                
\def\lvls#1{\g\l@vels#1}                                                        
\def\slnt#1{\g\sl@pe#1}                                                         
                                                                                
\def\treesymbol#1{%                                                             
     \g\setbox\leftb@x\copy\voidb@x                                             
     \g\setbox\rightb@x\copy\voidb@x                                            
     \g\setbox\centerb@x\copy\voidb@x                                           
     \g\setbox\beneathb@x\copy\voidb@x                                          
     \lvls{1}\slnt{3}%                                                          
     #1%                                                                        
     \g\centerdim@n\trianglew@dth                                               
     \g\halfcenterdim@n.5\trianglew@dth                                         
     \n@xttree                                                                  
     \g\count\l@sttreeheight\l@vels% \g\advance\count\l@sttreeheight by\tw@  
     \g\toks\l@sttreetype{triangle}%                                            
     \n@mel@st                                                                  
     \g\hd@st\hv@l\sl@pe \g\divide\hd@st by\tw@                                 
     \g\l@stltop=\halfcenterdim@n\g\advance\l@stltop by\wd\leftb@x              
     \g\l@strtop=\halfcenterdim@n\g\advance\l@strtop by\wd\rightb@x             
     \g\l@stlboff=-\hd@st \g\multiply\l@stlboff by\l@vels   
     \g\advance\l@stlboff by\wd\leftb@x
     \g\l@strboff=\hd@st \g\multiply\l@strboff by\l@vels 
     \g\advance\l@strboff by-\wd\rightb@x
     \g\l@stlmoff=\l@stlboff\relax
     \ifdim\l@stlmoff>0pt\relax\g\l@stlmoff=0pt\fi   
     \g\l@strmoff=\l@strboff
     \ifdim\l@strmoff<0pt\relax\g\l@strmoff=0pt\fi 
     \g\t@mpcnta\l@stinfo\g\advance\t@mpcnta by6% preliminary                   
     \ex\l@vels\times                                                           
        \g\dimen\t@mpcnta-\hd@st\g\advance\t@mpcnta by\@ne                      
        \g\dimen\t@mpcnta\hd@st\g\advance\t@mpcnta by\@ne                       
     \xe                                                                        
     \g\advance\t@mpcnta by-\tw@                                                
     \g\advance\dimen\t@mpcnta by\wd\leftb@x                                    
     \g\advance\t@mpcnta by\@ne                                                 
     \g\advance\dimen\t@mpcnta by-\wd\rightb@x                                  
     \g\setbox\l@stbox\vtop % to\l@vels\vd@st
            {\offinterlineskip                   
             \g\setbox\t@mpboxa                                                 
             \hbox{\hskip-\halfcenterdim@n\hskip-\wd\leftb@x\unhbox\leftb@x
                   \hskip\halfcenterdim@n
                   \raise\based@st\tri@ngle
                   \hskip\halfcenterdim@n\t@mpdima-\wd\rightb@x
                   \unhbox\rightb@x\hskip\t@mpdima\hskip-\halfcenterdim@n}                                     
             \g\ht\t@mpboxa=0pt\box\t@mpboxa                                    
             \setbox\centerb@x\hbox to0pt{\hss\unhbox\centerb@x\hss}%
             \ht\centerb@x0pt\dp\centerb@x0pt\box\centerb@x
             \vskip\s@ze pt
             \ht\beneathb@x0pt\box\beneathb@x                                                 
             \vskip-\dp\beneathb@x\vskip-\ht\beneathb@x}%
     \ignorespaces                                                              
     }                                                                          
                                                                                
%%%%%%%%%%%%%%%%%%%%%%%%%%%%%%%%%%%%%%%%%%%%%%%%%%%%%%%%%%%%%%%%%%%%%%%%%%%%    
%%% Node sizes                                                           %%%    
%%%%%%%%%%%%%%%%%%%%%%%%%%%%%%%%%%%%%%%%%%%%%%%%%%%%%%%%%%%%%%%%%%%%%%%%%%%% 

% The algorithm (macros \@pdbox, \drawl@ftedge, \drawr@ghtedge) accesses the
% horizontal and vertical offset fo any node type XXX (circle, square,
% dot, triangle, text, and frame) via the macros \XXXx@ff resp. \XXXy@ff.
% These two macros take the slope of the outgoing edges
% of the current node as their argument. Though horizontal offsets can
% be calculated from the vertical offsets and vice versa, there is no
% room to do so in the \XXXx@ff or \YYYy@ff macros, because these macros
% have to expand to a dimension!
%
% Let k be a <number>. Actually k will be the slope of the outgoing
% edges of the current node.
% 
% There are three methods how, e.g., \XXXx@ff can be defined , depending
% on the node type and the x- or y-orientation.
% First, \XXXx@ff can be defined to expand to \XXXx@ffk', k' beeing
% the roman numeral representation of the value of k, that in turn
% expands to the appropriate dimension (example: \circley@ff).
% In this case, \XXXx@ffi,...,\XXXx@ffiv are predefined by a \s@ries
% command.
% Second, \XXXx@ff can be defined to expand to a fixed dimension,
% independent of its argument (example: \dotx@ff).
% Third, the value of \XXXx@ff can be defined by the macro
% \c@lcoffsets, when the slope k is already known. In this case,
% the computation of \XXXx@ff makes use of the actual value
% of k and possibly other offsets that are already predefined by method I
% (examples: \circlex@ff, \squarex@ff).
                                                                                
\def\norm@ff{% everything set up for 10pt node size                                                                 
\s@ries{circley@ff}{0.50000pt//0.49320pt//0.49029pt//0.48507pt//%               
                    0.47434pt//0.46424pt//0.44721pt//0.42875pt//%               
                    0.41603pt//0.40000pt//0.39043pt//0.38411pt//%               
                    0.35355pt//0.32009pt//0.31235pt//0.30000pt//%               
                    0.27735pt//0.25725pt//0.22361pt//0.18570pt//%               
                    0.15811pt//0.12127pt//0.09806pt//0.08220pt}%                
     }                                                                          
                                                                                
\def\dotx@ff#1{0pt}
\def\doty@ff#1{0pt} 

\def\trianglex@ff#1{0pt}
\def\triangley@ff#1{0pt} 

\def\c@lcoffsets{%
     % \circlex@ff uses predefined \circley@ffi, \circley@ffii etc.
     \ifnum\sl@pe=\@ne\relax
           \xdef\circlex@ff##1{0pt}%
      \else\g\t@mpcnta26 % number of slopes + 2
           \g\advance\t@mpcnta-\sl@pe
           \xdef\circlex@ff##1{\circley@ff\t@mpcnta}%
        \fi
     % \squarex@ff and \squarey@ff are computed directly from \sl@pe and \squarew@dth
     \ifnum\sl@pe<13\relax % incoming edge meets upper border of a square node
                           % (slope 13 corresponds to 45 degrees)
           \g\t@mpdima.5\squarew@dth
           \xdef\squarey@ff##1{\the\t@mpdima}%
           \g\multiply\t@mpdima\xv@l\sl@pe
           \g\divide\t@mpdima\yv@l\sl@pe
           \xdef\squarex@ff##1{\the\t@mpdima}%
      \else\g\t@mpdima.5\squarew@dth
           \xdef\squarex@ff##1{\the\t@mpdima}%
           \g\multiply\t@mpdima\yv@l\sl@pe
           \g\divide\t@mpdima\xv@l\sl@pe
           \xdef\squarey@ff##1{\the\t@mpdima}%
        \fi
     % \texty@ff is .5\squarew@dth, \textx@ff is adjusted accordingly
     \g\t@mpdima.5\squarew@dth
     \xdef\texty@ff##1{\the\t@mpdima}%
     \g\multiply\t@mpdima\xv@l\sl@pe
     \g\divide\t@mpdima\yv@l\sl@pe
     \xdef\textx@ff##1{\the\t@mpdima}% 
     \let\framex@ff\textx@ff
     \let\framey@ff\texty@ff
    }
                                                                             
\def\upds@ze#1{%                                                                
     \for\t@mpcntc:=1\to24                                                      
      \do\g\t@mpdimc=\csname#1\romannumeral\t@mpcntc\endcsname\relax            
         \g\multiply\t@mpdimc by\s@ze                                           
         \expandafter\gxdef\csname#1\romannumeral\t@mpcntc\endcsname            
                            {\the\t@mpdimc}%                                    
      \od}                                                                      
                                                                                
\def\nodes@ze{%                                                                 
     \begingroup                                                                
     \unitlength 1pt%                                                           
     \divide\unitlength by 65536                                                
     \g\based@st\s@ze pt\g\divide\based@st by 10 % \based@st is 10 % of         
                                                 % circle diameter
     \g\dummyhalfcenterdim@n=\s@ze pt\g\divide\dummyhalfcenterdim@n by\tw@      
     \g\circlew@dth=\s@ze pt%                                                   
     \g\t@mpcntc\s@ze\g\multiply\t@mpcntc by 65536                              
     \g\setbox\circleb@x\hbox to0pt{\circle{\t@mpcntc}\hss}%                     
     \upds@ze{circley@ff}%                                  
     \g\squarew@dth.9pt\g\multiply\squarew@dth by\s@ze                          
     \g\setbox\squareb@x\rect@ngle{\squarew@dth}{\squarew@dth}{.4pt}%      
     \g\dotw@dth=\s@ze pt\g\divide\dotw@dth by 5                                
     \ifdim\dotw@dth < 1pt\relax                                                
           \g\dotw@dth1pt\relax                                                 
        \fi                                                                     
     \g\t@mpcntc\dotw@dth                                               
     \g\setbox\dotb@x\hbox to 0pt{\circle*{\t@mpcntc}\hss}%
     \g\trianglew@dth=\s@ze pt\g\multiply\trianglew@dth by \tw@                 
     \g\divide\trianglew@dth by 3 
     \g\textw@dth=0pt%
     \g\setbox\textb@x\copy\voidb@x
     \g\framew@dth0pt%
     \g\setbox\frameb@x\copy\voidb@x
     \hv@ldef                                                                   
     \endgroup                                                                  
     }                                                                          
                                                                                
%%%%%%%%%%%%%%%%%%%%%%%%%%%%%%%%%%%%%%%%%%%%%%%%%%%%%%%%%%%%%%%%%%%%%%%%%%%%    
%%% Changing the style                                                   %%%    
%%%%%%%%%%%%%%%%%%%%%%%%%%%%%%%%%%%%%%%%%%%%%%%%%%%%%%%%%%%%%%%%%%%%%%%%%%%%    
                                                                                
\def\treefonts#1{#1}                                                            
\def\vdist#1{\g\vd@st=#1\relax}                                                 
\def\minsep#1{\g\mins@p=#1\relax\g\halfmins@p=.5\mins@p}                        
\def\addsep#1{\g\adds@p=#1\relax}                                               
\def\extended{\def\ext@nded{\g\ext@ndedtrue}}                                   
\def\noextended{\def\ext@nded{\g\ext@ndedfalse}}                                
\def\nodesize#1{\g\t@mpdima=#1\relax\g\s@ze=\t@mpdima                           
     \g\divide\s@ze by 65536\relax} % conversion from dimension to number       
\def\Treestyle#1{\norm@ff#1\nodes@ze\ignorespaces}                                           
                                                                                
\input classes                                                             
                                                                                
\Treestyle{%                                                                    
     \ifLaTeX\treefonts{\normalsize\rm}%                                        
        \else\treefonts{\tenrm}%                                                
          \fi                                                                   
     \vdist{60pt}%                                                              
     \minsep{20pt}%                                                             
     \addsep{0pt}%                                                              
     \nodesize{20pt}%                                                           
     }                                                                          
                                                                                
                                                                                
%


