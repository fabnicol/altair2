\newif\ifenigmaisrunningplain
\ifcsname ver@enigma.sty\endcsname\else
  \enigmaisrunningplaintrue
  %%
%% This is file `luatexbase.sty',
%% generated with the docstrip utility.
%%
%% The original source files were:
%%
%% luatexbase.dtx  (with options: `texpackage')
%% 
%% See the aforementioned source file(s) for copyright and licensing information.
%% 
\begingroup\catcode61\catcode48\catcode32=10\relax% = and space
  \catcode123 1 % {
  \catcode125 2 % }
  \catcode 35 6 % #
  \toks0\expandafter{\expandafter\endlinechar\the\endlinechar}%
  \edef\x{\endlinechar13}%
  \def\y#1 #2 {%
    \toks0\expandafter{\the\toks0 \catcode#1 \the\catcode#1}%
    \edef\x{\x \catcode#1 #2}}%
  \y  13  5 % carriage return
  \y  61 12 % =
  \y  32 10 % space
  \y 123  1 % {
  \y 125  2 % }
  \y  35  6 % #
  \y  64 11 % @ (letter)
  \y  10 12 % new line ^^J
  \y  39 12 % '
  \y  40 12 % (
  \y  41 12 % )
  \y  45 12 % -
  \y  46 12 % .
  \y  47 12 % /
  \y  58 12 % :
  \y  91 12 % [
  \y  93 12 % ]
  \y  94  7 % ^
  \y  96 12 % `
  \toks0\expandafter{\the\toks0 \relax\noexpand\endinput}%
  \edef\y#1{\noexpand\expandafter\endgroup%
    \noexpand\ifx#1\relax \edef#1{\the\toks0}\x\relax%
    \noexpand\else \noexpand\expandafter\noexpand\endinput%
    \noexpand\fi}%
\expandafter\y\csname luatexbase@sty@endinput\endcsname%
\begingroup
  \expandafter\ifx\csname ProvidesPackage\endcsname\relax
    \def\x#1[#2]{\immediate\write16{Package: #1 #2}}
  \else
    \let\x\ProvidesPackage
  \fi
\expandafter\endgroup
\x{luatexbase}[2013/05/11 v0.6 Resource management for the LuaTeX macro programmer]
\begingroup\expandafter\expandafter\expandafter\endgroup
\expandafter\ifx\csname RequirePackage\endcsname\relax
  \input ifluatex.sty
\else
  \RequirePackage{ifluatex}
\fi
\ifluatex\else
  \begingroup
    \expandafter\ifx\csname PackageError\endcsname\relax
      \def\x#1#2#3{\begingroup \newlinechar10
        \errhelp{#3}\errmessage{Package #1 error: #2}\endgroup}
    \else
      \let\x\PackageError
    \fi
  \expandafter\endgroup
  \x{luatexbase}{LuaTeX is required for this package. Aborting.}{%
    This package can only be used with the LuaTeX engine^^J%
    (command `lualatex' or `luatex').^^J%
    Package loading has been stopped to prevent additional errors.}
  \expandafter\luatexbase@sty@endinput%
\fi
\expandafter\ifx\csname RequirePackage\endcsname\relax
  \input luatex.sty
\else
  \RequirePackage{luatex}
\fi
\begingroup\expandafter\expandafter\expandafter\endgroup
\expandafter\ifx\csname RequirePackage\endcsname\relax
  \input luatexbase-compat.sty
  \input luatexbase-modutils.sty
  \input luatexbase-loader.sty
  \input luatexbase-regs.sty
  \input luatexbase-attr.sty
  \input luatexbase-cctb.sty
  \input luatexbase-mcb.sty
\else
  \RequirePackage{luatexbase-compat}
  \RequirePackage{luatexbase-modutils}
  \RequirePackage{luatexbase-loader}
  \RequirePackage{luatexbase-regs}
  \RequirePackage{luatexbase-attr}
  \RequirePackage{luatexbase-cctb}
  \RequirePackage{luatexbase-mcb}
\fi
\luatexbase@sty@endinput%
\endinput
%%
%% End of file `luatexbase.sty'.

  \catcode`\@=11
% \else latex
\fi
\catcode`\_=11 % There’s no reason why this shouldn’t be the case.
\catcode`\!=11
%D Nice tool from luat-ini.mkiv. This really helps with those annoying
%D string separators of Lua’s that clutter the source.
% this permits \typefile{self} otherwise nested b/e sep problems
\def\luastringsep{===}
\edef\!!bs{[\luastringsep[}
\edef\!!es{]\luastringsep]}
%D \startdocsection[title=Prerequisites]
%D \startparagraph
%D Package loading and the namespacing issue are commented on in
%D \identifier{enigma.lua}.
%D \stopparagraph
\directlua{
  packagedata = packagedata or { }
  dofile(kpse.find_file\!!bs enigma.lua\!!es)
}

%D \startparagraph
%D First, create somthing like \CONTEXT’s asciimode. We found
%D \texmacro{newluatexcatcodetable} in \identifier{luacode.sty} and it
%D seems to get the job done.
%D \stopparagraph
\newluatexcatcodetable \enigmasetupcatcodes
\bgroup
  \def\escapecatcode      {0}
  \def\begingroupcatcode  {1}
  \def\endgroupcatcode    {2}
  \def\spacecatcode      {10}
  \def\lettercatcode     {11}
  \setluatexcatcodetable\enigmasetupcatcodes {
      \catcode`\^^I = \spacecatcode % tab
      \catcode`\    = \spacecatcode
      \catcode`\{   = \begingroupcatcode
      \catcode`\}   = \endgroupcatcode
      \catcode`\^^L = \lettercatcode    % form feed
      \catcode`\^^M = \lettercatcode    % eol
  }
\egroup
%D \stopdocsection

%D \startdocsection[title=Setups]
%D \startparagraph
%D Once the proper catcodes are in place, the setup macro
%D \texmacro{do_setup_enigma} doesn’t to anything besides passing stuff
%D through to Lua.
%D \stopparagraph
\def\do_setup_enigma#1{%
    \directlua{
      local enigma = packagedata.enigma
      local current_args = enigma.parse_args(\!!bs\detokenize{#1}\!!es)
      enigma.save_raw_args(current_args, \!!bs\current_enigma_id\!!es)
      enigma.new_callback(
        enigma.new_machine(\!!bs\current_enigma_id\!!es),
        \!!bs\current_enigma_id\!!es)
    }%
  \egroup%
}

%D The module setup \texmacro{setupenigma} expects key=value, notation.
%D All the logic is at the Lua end, not much to see here …
\def\setupenigma#1{%
  \bgroup
    \edef\current_enigma_id{#1}
    \luatexcatcodetable \enigmasetupcatcodes
    \do_setup_enigma%
}
%D \stopdocsection

%D \startdocsection[title=Encoding Macros]
%D \startparagraph
%D The environment of \texmacro{begin<enigmaid>} and
%D \texmacro{end<enigmaid>} toggles Enigma encoding.
%D (Regarding environment delimiters we adhere to Knuth’s
%D practice of prefixing with \type{begin}/\type{end}.)
%D \stopparagraph

\def\e!start{begin} %{start}
\def \e!stop{end}   %{stop}
\edef\c!pre_linebreak_filter{pre_linebreak_filter}
\def\do_define_enigma#1{%
  \@EA\gdef\csname \e!start\current_enigma_id\endcsname{%
    \endgraf
    \bgroup%
    \directlua{%
      if packagedata.enigma                         and
         packagedata.enigma.machines[ \!!bs#1\!!es] then
        luatexbase.add_to_callback(
          \!!bs\c!pre_linebreak_filter\!!es,
          packagedata.enigma.callbacks[ \!!bs#1\!!es],
          \!!bs#1\!!es)
      else
        print\!!bs ENIGMA: No machine of that name: #1!\!!es
        os.exit()
      end
    }%
  }%
  \@EA\gdef\csname \e!stop\current_enigma_id\endcsname{%
    \endgraf
    \directlua{
      luatexbase.remove_from_callback(
        \!!bs\c!pre_linebreak_filter\!!es,
        \!!bs#1\!!es)
      packagedata.enigma.machines[ \!!bs#1\!!es]:processed_chars()
    }%
    \egroup%
  }%
}

\def\defineenigma#1{%
  \begingroup
  \let\@EA\expandafter
  \edef\current_enigma_id{#1}%
  \@EA\do_define_enigma\@EA{\current_enigma_id}%
  \endgroup%
}

%D \stopdocsection

\catcode`\_=8  % \popcatcodes
\catcode`\!=12 % reserved according to source2e
\ifenigmaisrunningplain\catcode`\@=12\fi
% vim:ft=plaintex:sw=2:ts=2:expandtab:tw=71
