% FIGFLOW:  plain TEX macro by Ian Hutchinson, 21 Oct 95.
% Copyright 1995 Ian Hutchinson.
% You may freely use, modify, and/or distribute this file, without limitation.
% Make text flow round figure.
% Usage: \figflow{<width>}{<height>}{<[Figure+][Caption]>}
% at start of new paragraph. Figure top starts at insert.
% #1 figure width dimen. If negative, fig on right, else left.
% #2 figure height (including caption) dimen. (E.g.: 4 truein)
% #3 \special for figure if desired, then \vfill caption. (Both optional).
% Example: figflow{4 truein}{5 truein}{\epsfbox{figure.ps}\vfill Figure 1.}
% User is responsible for the figure fitting within the space defined.
% If figure won't fit on page, it is moved over the page break.
% If a new figflow starts before the old one is finished, a message is given
% and the second figure is skipped. Fix manually.
% Does not work for Latex.

\newdimen\pageremains\newdimen\pdepth
\newdimen\figwidth
\newdimen\figheight
\newcount\figlines
\newcount\flevel

\def\figflow#1#2#3{
\ifnum\flevel>0
\message{******Figure collision. Ignoring second figure.******}
\else
\figwidth=#1
\figheight=#2
\def\contents{#3}
% Put figure contents in an appropriate box.
\def\figure{\let\temp=\par \let\par=\plainpar
  \line{\overfullrule=0pt%  Avoid black box.
   \ifdim \figwidth<0pt \hsize=-\figwidth \hss\else \hsize=\figwidth\fi
   \advance \hsize by -10pt% Give a little extra hspace.
   \vbox to \figheight{\vfil\noindent\contents}
   \ifdim \figwidth>0pt \hss\fi
  } \vskip-\figheight
  \let\par=\temp%
}
\advance\figheight by \baselineskip
\divide\figheight by \baselineskip% convert height to lines.
\figlines=\figheight \multiply\figheight by \baselineskip
\begingroup\overfullrule=0pt% Turn off black box outside fig
\tolerance=1000% Allow more spaced out lines.
\flevel=1
% Store \par
\let\plainpar=\par
% Define new \par to process figures each paragraph.
\def\par{
  \ifnum\flevel=1
% We are starting a new figure. Set to look for enough room.
   \plainpar
% End the previous paragraph.
   \pageremains=\pagegoal \advance\pageremains by -\pagetotal
   \ifdim\pageremains<\figheight \message{Moving figure...}%keep looking
   \else
% Found the starting place. Store prevdepth. Remove glue. Place the figure.
      \pdepth=\prevdepth
      \nointerlineskip
      \figure
      \hangindent \figwidth \hangafter -\figlines \hfuzz 5 pt
      \flevel=2
      \prevgraf=0
      \figheight=\baselineskip% Removed parskip adjust.
   \fi
  \else
   \ifnum\flevel=2%level 2, making the figure.
    \ifdim\figheight<\parskip
       \advance\figlines -1 \advance\hangafter 1
       \advance\figheight\baselineskip
    \else
       \advance\figheight -\parskip%\message{else \the\figheight}
    \fi
    \hangcarrypar\relax% I don't know why, but this is needed.
   \fi
  \fi
}
\par
\vskip-\pdepth%Restore the prevdepth from the previous paragraph.
\fi
}
% Macros.
\def\endflow{\global\let\par=\plainpar\endgroup}% terminate main group.
\def\hangcarrypar{% Carry the hangindent to next par.
\edef\next{\hangafter=\the\hangafter\hangindent=\the\hangindent}
\plainpar\next
\edef\next{\prevgraf=\the\prevgraf}
\ifnum\prevgraf>0
   \ifnum\prevgraf>\figlines \endflow \flevel=0
   \else
     \message{FIGFLOW: line \the\prevgraf, of \the\figlines.}
     \leavevmode% Sets prevgraf to 0. So reset it using next.
     \next
   \fi
\fi
}
