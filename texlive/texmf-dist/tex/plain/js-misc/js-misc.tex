% $Id: js-misc.tex,v 1.1 1995/03/13 23:18:12 schrod Exp $
%--------------------------------------------------
% Written by Joachim Schrod <schrod@iti.informatik.th-darmstadt.de>.

% This is a german description for cassette.tex & schild.tex.


% Manual fuer ``Kleine TeX-Makros, aufbauend auf PLAIN'':
%     Tonbandkassetten-Schilder
%     Buecherruecken-Schilder
%
% [Formate]


\berichtformat

\input idverb
\def\.{$\bullet$}




\titel{}{Kleine \TeX{}-Makros}{, aufbauend auf {\tt PLAIN}}
\autor{Joachim Schrod}
\datum{R�dermark, den 08.02.88}





\kapitel{Beschriftung von Tonband-Kassetten}{}

Die Beschriftung von Tonband-Kassetten wird durch
|\input cassette| geladen.
Durch diese Makros werden Schilder erzeugt, die
in eine Kassettenh�lle gelegt werden k�nnen.
%Jedes Schild wird auf eine einzelne Seite gedruckt.
Das Layout eines Schildes ist in Abb.~1.1 zu sehen.
Das Schild wird an den �u�eren R�ndern ausgeschnitten
und entlang der horizontalen Linien gefaltet.

\abbvoll{
{\input cassette
\let\eject=\relax
\begincassette{Kassettenname}
\titelcass{Frontseite mit Titeln}
\seitea{Musikst�cke der Seite A}
\seiteb{Musikst�cke der Seite B}
\endcassette
}}{Layout eines Kassettenschildes}

\noindent Folgende Befehle erzeugen das gew�nschte Layout:

\beginitemize

\item{\.} |\begincassette|\quad (1 Parameter)\nl
Mit diesem Befehl beginnt die Beschreibung eines Kassettenschilds.
Der Parameter ist der Name der Kassette, der in |\cassfont|
(Kapit�lchen, 10pt) gesetzt wird.

\item{\.} |\titelcass|\quad (1 Parameter)\nl
Dieser Befehl gibt den Titel der Kassette an, der auf die
Frontseite gesetzt wird.
Der Titel wird in |\titfont| (Serifenlose, 10pt) gesetzt.
Dieser Befehl ist optional, da Titel auch durch |\titel| (s.u.)\
angegeben werden k�nnen.

\item{\.} |\seitea|\quad (1 Parameter)\nl
Durch den Parameter von |\seitea| wird die Seite~A der
Kassette beschrieben.
Diese Beschreibung umfa�t einen optionalen Titel und eine
Folge von Musikst�cken.
Sie ist zeilenorientiert, d.h.\ jeweils ein Musikst�ck
(mit eventuellem Interpret) steht auf einer Eingabezeile.
Wenn der Text zu lang wird, kann durch die Angabe von |%|
am Zeilenende die Folgezeile mitbenutzt werden.
Innerhalb eines Musikst�cks kann durch |\nl| gezielt
umgebrochen werden, wenn man mit dem von \TeX{} gew�hltem
Umbruch nicht zufrieden ist.
Die Musikst�cke werden in |\norm| (Serifenlose, 8pt) gesetzt.
Dieser Befehl ist optional.
\itempar
Folgende Hilfsbefehle stehen zur Beschreibung zur Verf�gung:

{\nextitemlevel\itemskip=0pt
\item{--} |\cnt|\quad (1 Parameter, beendet durch |:|)\nl
Mit |\cnt| kann die Z�hlerstellung des Beginns eines
Musikst�cks angegeben werden.
Die angegebene Zahl wird in eckige Klammern gesetzt.

\item{--} |\von|\quad (1 Parameter, beendet durch |:|)\nl
Durch |\von| kann ein Interpret dem Musikst�ck vorgestellt werden.
Ein Beispiel f�r eine vollst�ndige Angabe eines Musikst�cks
kann also
$$
   \hbox{|\von Led Zeppelin: Stairway to heaven|}
$$
sein.
Der Interpret wird in |\intfont| (Serifenlose Kursiv, 8pt) gesetzt.

\item{--} |\titel|\quad (1 Parameter)\nl
Mit diesem Befehl kann in die Seite eine Angabe eines Titels
(z.B.~einer Schallplatte) �bernommen werden.
Dieser Titel wird gleichzeitig auf die Frontseite gesetzt.
Innerhalb des Titels kann |\von| zur Angabe eines Interpreten
benutzt werden.
Dieser Befehl kann mehrmals vorkommen.
Der Titel wird auf der Seite in |\titfont@page| (Kapit�lchen, 8pt)
gesetzt, wobei der Interpret in Versalien gesetzt wird.
Auf der Frontseite wird der Titel in |\titfont| (Serifenlose, 10pt)
gesetzt, wobei der Interpret in |\titintfont| (Serifenlose Kursiv, 10pt)
gesetzt wird.
\par}

\item{\.} |\seiteb|\quad (1 Parameter)\nl
Dieser Befehl beschreibt die Seite~B der Kassette.
Die Beschreibungsm�glichkeiten sind mit denen von |\seitea|
identisch.

\item{\.} |\endcassette|\nl
Durch diesen Befehl wird die Beschreibung des Kassettenschildes
beendet und das Kassettenschild gesetzt.
Gleichzeitig wird eine neue Seite begonnen.

\enditemize





\kapitel{Beschriftung von B�cherr�cken}{}

Die Beschriftung von B�cherr�cken wird durch
|\input schild| geladen.
Durch diese Makrodatei ist es m�glich Schilder f�r
B�cherr�cken zu erzeugen, die anschlie�end auf die B�cher
geklebt werden k�nnen.
Das Layout eines Schildes ist in der Abb.~2.1 gezeigt,
es wird an den �u�eren R�ndern ausgeschnitten.

\def\schild{%
   \vbox{%
      \hrule
      \hbox{%
         \vrule
         \vbox to 10mm{%
            \vfill
            \hbox{\kern 25mm \svtnrm Name des Buches\kern 25mm}%
            \vfill
            }%
         \vrule
         }%
      \hrule
      }%
   }

\def\vertlen{%
   \llap{%
      \vbox to 10mm{%
         \vfill
         \hbox{{\tt \#1\/} mm }%
         \vfill
         }
      }%
   }
\def\vertmass{%
   \vbox{%
      \hrule width 4mm
      \hbox{%
         \vertlen
         \kern 2mm
         \vrule height 10mm
         }%
      \hrule width 4mm
      }%
   }

\def\horizmass{%
   \hbox{%
      \vrule height 2mm depth 2mm
      \vtop{%
         \hrule width 25mm
         \kern 2mm
         \hbox to 25mm{\ctr{\tt \char`\\rand}}%
         }%
      \vrule height 2mm depth 2mm
      }%
   }

\abbvoll{
   \offinterlineskip
   \hbox{%
      \llap{%
         \vertmass
         \hskip 2mm
         }%
      \schild
      }%
   \vskip 2mm
   \hbox{%
      \horizmass
      \phantom{\svtnrm Name des Buches}%
      \horizmass
      }%
}{Layout eines Buchr�ckenschildes}

\noindent Um das Buchr�ckenschild zu erzeugen, existieren
folgende Makros und Parameter:

\beginitemize

\item{\.} |\schild|\quad (2 Parameter)\nl
Durch dieses Makro wird ein Schild erzeugt.
Der erste Parameter ist die H�he des Schildes, der zweite
der Text, der auf das Buchr�ckenschild soll.

\item{\.} |\rand|\nl
|\rand| ist eine Dimensionsangabe, die beschreibt, wieviel
Platz zwischen dem Buchnamen und dem Rand des Schildes
horizontal verbleibt.

\item{\.} |\namfont|\nl
In |\namfont| wird der Buchname gesetzt.

\enditemize





\bye


%%%%%%%%%%%%%%%%%%%%%%%%%%%%%%%%%%%%%%%%%%%%%%%%%%%%%%%%%%%%%%%%%%%%%%
%
% $Log: js-misc.tex,v $
% Revision 1.1  1995/03/13  23:18:12  schrod
%     Started to manage this package with CVS. Made minor code cleanup.
%

%
% Pre-CVS Log:
%
% 88-02-08 js  Initial revision
