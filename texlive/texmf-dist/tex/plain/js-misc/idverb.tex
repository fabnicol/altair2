% $Id: idverb.doc,v 1.1 1995/03/13 23:18:11 schrod Exp $
%----------------------------------------------------------------------
% Written by Joachim Schrod <schrod@iti.informatik.th-darmstadt.de>.
% Copyright conditions see below.

%
% idverb.doc  ---  typeset identifiers verbatim
%
% [plain TeX in MAKEPROG]
% (history at end)


%%%%
%%%%
%%%% These TeX macros were documented with the documentation system
%%%% MAKEPROG and automatically converted to the current form.
%%%% If you have MAKEPROG available you may transform it back to
%%%% the original input: Remove every occurence of three percents
%%%% and one optional blank from the beginning of a line and remove
%%%% every line which starts with four percents.  The following lex
%%%% program will do this:
%%%%
%%%%	%%
%%%%
%%%%	^%%%\ ?   ;
%%%%	^%%%%.*\n ;
%%%%
%%%%	If you just want to print the documentation you may fetch
%%%% the archive print-makeprog.tar.Z from ftp.th-darmstadt.de (directory
%%%% pub/tex/latex). It contains *all* used styles -- but beware, they
%%%% may not be in a documented form...
%%%%
%%%%
%%% \input progdoc

%%% \input names.sty
%%% \def\v{{\tt \vbar\/}}



%%% \title{Identifiers Verbatim \`a la {\ttitlefont WEB}}



%%% \chap Introduction.

%%% This macro files allows to write identifiers in a \WEB{} like style,
%%% i.e.\ as {\tt \origvert identifier\origvert\/}. The identifiers are
%%% typeset verbatim in the monospace type face. The macros
%%% |\makevertother| and |\makevertactive| are available to activate and
%%% deactivate the special behaviour of the vertical bar.


%%% \sect This macro file uses the namespace |idv|.

%%% \beginprog
\ifx \IdvLoaded\undefined
    \def\IdvLoaded{$Revision: 1.1 $}
\else \endinput \fi                     % <-- one line!
%%% \endprog


%%% \sect These macros are supported. Send bug reports, comments and
%%% repairs.

%%% The reference version may be retrieved via anonymous ftp from
%%% |ftp.th-darmstadt.de| [130.83.47.112], directory |pub/tex/plain|. It's
%%% placed there as a gzipped tar file. (The information on the
%%% IP~number is dated March 13, 1995. It might have changed, also this is very
%%% unlikely. Use your friendly nameserver.)


%%% \sect This is freely distributable software; you can redistribute it
%%% and/or modify it under the terms of the GNU General Public License as
%%% published by the Free Software Foundation; either version~2 of the
%%% License, or (at your option) any later version.

%%% This software is distributed in the hope that it will be useful, but
%%% {\bf without any warranty\/}; without even the implied warranty of
%%% {\bf merchantability\/} or {\bf fitness for a particular purpose}.  See
%%% the GNU General Public License for more details.

%%% You should have received a copy of the GNU General Public License in
%%% the file |License| along with this package; if not, write to the Free
%%% Software Foundation, Inc., 675~Mass Ave, Cambridge, MA~02139,~USA.


%%% \sect But before we start we declare some shorthands for category
%%% codes. By declaring the at sign~(`|@|') as well as the
%%% underscore~`(|_|)' as letters we can use them in our macros. (I agree
%%% with D.~Knuth that |\identifier_several_words_long| is more readable
%%% than |\IdentifierSeveralWordsLong| and in every case better than
%%% |\p@@@s|.) With the at sign we can use the ``private'' Plain macros
%%% and with the underscore we can make our own macros more readable. But
%%% as we have to restore these category codes at the end of this macro
%%% file we store their former values in control sequences. This method is
%%% better than to use a group because not all macros have to be defined
%%% global this way.

%%% Only the first macro file read in defines the |Cat| cseqs.

%%% \beginprog
\ifx \CatEscape\undefined
    \chardef\CatEscape=0
    \chardef\CatOpen=1
    \chardef\CatClose=2
    \chardef\CatIgnore=9
    \chardef\CatLetter=11
    \chardef\CatOther=12
    \chardef\CatActive=13               % \active of plain.tex
    \chardef\CatInvalid=15

    \chardef\CatAtCode=\catcode`\@
    \chardef\CatUsCode=\catcode`\_
\fi

\catcode`\@=\CatLetter                  % top level macro file
\catcode`\_=\CatLetter
%%% \endprog


%%% \sect Let's identify this macro file against the user and in the Log file.

%%% \beginprog
\begingroup
    \catcode`\$=\CatIgnore
    \catcode`\:=\CatIgnore
    \message{Verbatim identifiers, $Revision: 1.1 $}
\endgroup
%%% \endprog


%%% \sect The usual verbatim macros use |\dospecials| to change the
%%% catcode of all special characters. We have to add `\v' to this list.
%%% The old meaning is stored in |\idv_OrigDospecials|.

%%% \beginprog
\let\idv_OrigDospecials=\dospecials
\begingroup
    \def\do#1{\noexpand\do\noexpand#1}
    \xdef\dospecials{\idv_OrigDospecials\do\|}
\endgroup
\let\idv_dospecials=\dospecials

\def\makevertactive{\catcode`\|\CatActive \let\dospecials\idv_dospecials}
\def\makevertother{\catcode`\|\CatOther \let\dospecials\idv_OrigDospecials}

\makevertactive
%%% \endprog


%%% \sect Now we can define `\v'. This definition is a little bit tricky,
%%% as it redefines itself to |\egroup| to close the hbox. But if the hbox
%%% is ended, `\v' is restored to its original meaning. |\origvert| is
%%% given the original meaning of `\v'. |\vbar| is the character from the
%%% current font that has the ASCII code of a vertical bar, sometimes this
%%% {\it is\/} a vertical bar.

%%% \beginprog
\let\origvert=|
\chardef\vbar=`\|

\def\idv_setup_verbatim{%
    \def\do##1{\catcode`##1\CatOther}\idv_OrigDospecials
    \parskip\z@skip \parindent\z@
    \obeylines \obeyspaces \frenchspacing
    \tt
    }

\def|{%
    \leavevmode
    \hbox\bgroup
        \let\par\space \idv_setup_verbatim
        \let|\egroup
    }
%%% \endprog


%%% \sect We are finished;
%%% restore the catcodes and prevent from following garbage.

%%% \beginprog
\catcode`\@=\CatAtCode
\catcode`\_=\CatUsCode

\endinput
%%% \endprog


%%% %% \sect {\it Acknowledgements:}\quad I would like to thank XXX


%%% \bye

%%% 
%%% %%%%%%%%%%%%%%%%%%%%%%%%%%%%%%%%%%%%%%%%%%%%%%%%%%%%%%%%%%%%%%%%%%%%%%
%%% %
%%% % $Log: idverb.doc,v $
%%% % Revision 1.1  1995/03/13  23:18:11  schrod
%%% %     Started to manage this package with CVS. Made minor code cleanup.
%%% %

%%% %
%%% % Pre-CVS Log:
%%% %
%%% % 27 Jul 89 js  Initial revision.


%%% 
%%% %%%%%%%%%%%%%%%%%%%%%%%%%%%%%%%%%%%%%%%%%%%%%%%%%%%%%%%%%%%%%%%%%%%%%%
%%% Local Variables:
%%% mode: plain-TeX
%%% TeX-master: t
%%% TeX-brace-indent-level: 4
%%% End:
