%%% This is part of the hyplain package by Enrico Gregorio
%%% You are allowed to do anything you want with this as long
%%% as you cite the source and change name to the file
%%% (but don't rename it plain.tex)
%%% 
%%% Its purpose is to load plain.tex but not hyphen.tex
%%% in order to be able to define more languages than
%%% American English.
%%%
%%% By running iniTeX on this file, you end up with a
%%% plain TeX format with some new facilities for changing
%%% the language (see hyrules.tex and hydoc.tex)

\catcode`\{=1 % left brace is begin-group character
\catcode`\}=2 % right brace is end-group character
\catcode`\@=11 % Now @ is a letter

%%% Save the original meaning of \input
\let\orig@input\input

%%% D. E. Knuth has decreed that plain.tex cannot be modified
%%% except for preloaded fonts. But we can always use some
%%% TeX trick; since the file is immutable, it will contain
%%% the line `\input hyphen'; at that point we restore the
%%% original meaning of \input and input hyrules.tex instead
%%% of hyphen.tex
\def\input hyphen {\let\input\orig@input \input hyrules }

%%% Here we load plain.tex
\orig@input plain

%%% We change the contents of \fmtname
\def\fmtname{plain (multiple language support)}
%%% Just a reminder for \fmtversion which we keep identical
%%% to the plain TeX version number
% \def\fmtversion{3.14159265}
\def\hyplainversion{1.0}

\dump
