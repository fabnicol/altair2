\documentclass{article}
\usepackage{luamplib}
\usepackage{xcolor}
\everymplib{ beginfig(0); }
\everyendmplib{ endfig; }
\begin{document}
\tracingcommands1
A%
\begin{mplibcode}
%% test all printable ascii chars in comments
%%    (  2  <  F  P  Z  d   n   x
%%    )  3  =  G  Q  [  e   o   y
%%    *  4  >  H  R  \  f   p   z
%% !  +  5  ?  I  S  ]  g   q   {
%% "  ,  6  @  J  T  ^  h   r   |
%% #  -  7  A  K  U  _  i   s   }
%% $  .  8  B  L  V  `  j   t   ~
%% %  /  9  C  M  W  a  k   u  DEL
%% &  0  :  D  N  X  b  l   v
%% ´  1  ;  E  O  Y  c  m   w
	fill fullcircle scaled 20; %% actual <tab> to make sure it works
\end{mplibcode}%
B\par
\everymplib{}\everyendmplib{}% reset toks
A%
\begin{mplibcode}
verbatimtex \lower.2em etex
beginfig(0);
draw origin--(1cm,0) withcolor \mpcolor{teal};
draw btex g etex withcolor \mpcolor{orange};
draw btex\tracingcommands0
  multi%
  \begin{huge}line\end{huge}%
  texcode
  with \%
  etex shifted (10,0);
draw bbox currentpicture;
endfig;

beginfig(18);
numeric u;
u = 1cm;
draw (0,2u)--(0,0)--(4u,0);
pickup pencircle scaled 1pt;
draw (0,0){up}
  for i=1 upto 8: ..(i/2,sqrt(i/2))*u  endfor;
label.lrt(btex $\sqrt x$ etex, (3,sqrt 3)*u);
label.bot(btex $x$ etex, (2u,0));
label.lft(btex $y$ etex, (0,u));
endfig;
\end{mplibcode}%
B\par
A%
\begin{mplibcode}
beginfig(2);
numeric u; u=1cm;
z1=-z2=(-u,0);
for i = 1 upto 3:
  draw z1..(0, i*u)..z2;
  label.top(TEX("$e_{" & decimal(i) & "}$"), (0, i*u))
  withcolor \mpcolor{red!70!blue};
endfor;
endfig;
\end{mplibcode}%
B\par
\mplibsetformat{metafun}%
\begin{mplibcode}
verbatimtex \moveright 0.4\hsize etex
beginfig(0);
path p; p:= fullcircle scaled 2cm yshifted .5cm;
fill p withcolor transparent("normal", 0.5, red);
fill p rotated 120 withcolor transparent("normal", 0.5, green);
fill p rotated 240 withcolor transparent("normal", 0.5, blue);
endfig;
verbatimtex \leavevmode etex
picture p; p := btex MetaPost etex scaled 2;
beginfig(1);
linear_shade(bbox p,0,blue,.7white);
draw p withcolor white;
endfig;
verbatimtex \kern10pt etex
beginfig(2);
circular_shade(bbox p,0,blue,.7white);
draw p withcolor white;
endfig;
\end{mplibcode}%

\newbox\mympbox
\begin{mplibcode}
verbatimtex \global\setbox\mympbox etex
beginfig(0);
breadth=.667\mpdim\linewidth;
height=2pt;
x1=0; x2=x6=.333x4; x5=x3=.667x4;
x4=breadth;
y1=y4=height/2; y2=y3=height; y5=y6=0;
fill z1--z2--z3--z4--z5--z6--cycle;
endfig;
\end{mplibcode}%
\copy\mympbox
\copy\mympbox
\copy\mympbox
\copy\mympbox

\mplibnumbersystem{double}%
\begin{mplibcode}
beginfig(0);
u := 10**5*(10**-4);
draw unitsquare scaled u;
endfig;
\end{mplibcode}%
\mplibsetformat{plain}%
\begin{mplibcode}
  input graph;
  beginfig(0);
  draw begingraph(100,100);
    gdraw (10,10)--(30,35)--(50,25)--(70,80)--(90,90);
    autogrid(otick.bot,);
    for y=20,40,60,80:
      grid.lft(format("%e",1000y), y) withcolor .85white;
    endfor
    endgraph;
  endfig;
\end{mplibcode}%
\mplibtextextlabel{enable}%
\begin{mplibcode}
beginfig(0);
dotlabel.rt("$\sqrt2$",origin);
endfig;
\end{mplibcode}%
\leavevmode
\begin{mplibcode}
   D := sqrt(2)**7;
   beginfig(0);
   draw fullcircle scaled D;
   VerbatimTeX("\gdef\Dia{" & decimal D & "}");
   endfig;
\end{mplibcode}%
diameter:\Dia bp.%
\begin{mplibcode}
  vardef rotatedlabel@#(expr str, loc, angl) =
    draw thelabel@#(str, loc) rotatedaround(loc, angl)
  enddef;

  beginfig(1);
    rotatedlabel.top(textext("Rotated!"), origin, 45);
  endfig;
\end{mplibcode}%
\end{document}
