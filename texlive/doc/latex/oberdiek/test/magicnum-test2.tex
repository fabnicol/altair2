%%
%% This is file `magicnum-test2.tex',
%% generated with the docstrip utility.
%%
%% The original source files were:
%%
%% magicnum.dtx  (with options: `testplain,testdata')
%% 
%% This is a generated file.
%% 
%% Project: magicnum
%% Version: 2011/04/10 v1.4
%% 
%% Copyright (C) 2007, 2009-2011 by
%%    Heiko Oberdiek <heiko.oberdiek at googlemail.com>
%% 
%% This work may be distributed and/or modified under the
%% conditions of the LaTeX Project Public License, either
%% version 1.3c of this license or (at your option) any later
%% version. This version of this license is in
%%    http://www.latex-project.org/lppl/lppl-1-3c.txt
%% and the latest version of this license is in
%%    http://www.latex-project.org/lppl.txt
%% and version 1.3 or later is part of all distributions of
%% LaTeX version 2005/12/01 or later.
%% 
%% This work has the LPPL maintenance status "maintained".
%% 
%% This Current Maintainer of this work is Heiko Oberdiek.
%% 
%% The Base Interpreter refers to any `TeX-Format',
%% because some files are installed in TDS:tex/generic//.
%% 
%% This work consists of the main source file magicnum.dtx
%% and the derived files
%%    magicnum.sty, magicnum.pdf, magicnum.ins, magicnum.drv, magicnum.txt,
%%    magicnum-test1.tex, magicnum-test2.tex, magicnum-test3.tex,
%%    magicnum-test4.tex, magicnum.lua, oberdiek.magicnum.lua.
%% 
\input magicnum.sty\relax
\def\Test#1#2{%
  \edef\result{\magicnum{#1}}%
  \edef\expect{#2}%
  \edef\expect{\expandafter\stripprefix\meaning\expect}%
  \ifx\result\expect
  \else
    \errmessage{%
      Failed: [#1] % hash-ok
      returns [\result] instead of [\expect]%
    }%
  \fi
}
\def\stripprefix#1->{}
\Test{tex.catcode.escape}{0}
\Test{tex.catcode.invalid}{15}
\Test{tex.catcode.unknown}{}
\Test{tex.catcode.0}{escape}
\Test{tex.catcode.15}{invalid}
\Test{etex.iftype.true}{15}
\Test{etex.iftype.false}{16}
\Test{etex.iftype.15}{true}
\Test{etex.iftype.16}{false}
\Test{etex.nodetype.none}{-1}
\Test{etex.nodetype.-1}{none}
\Test{luatex.pdfliteral.mode.direct}{2}
\Test{luatex.pdfliteral.mode.1}{page}
\Test{}{}
\Test{unknown}{}
\Test{unknown.foo.bar}{}
\Test{unknown.foo.4}{}
\csname @@end\endcsname
\end
\endinput
%%
%% End of file `magicnum-test2.tex'.
