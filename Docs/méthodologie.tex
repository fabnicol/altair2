\documentclass[]{article}
\usepackage[T1]{fontenc}
\usepackage{lmodern}
\usepackage{amssymb,amsmath}
\usepackage{ifxetex,ifluatex}
\usepackage{fixltx2e} % provides \textsubscript
% Set line spacing
% use upquote if available, for straight quotes in verbatim environments
\IfFileExists{upquote.sty}{\usepackage{upquote}}{}
\ifnum 0\ifxetex 1\fi\ifluatex 1\fi=0 % if pdftex
  \usepackage[utf8]{inputenc}
\else % if luatex or xelatex
  \ifxetex
    \usepackage{mathspec}
    \usepackage{xltxtra,xunicode}
  \else
    \usepackage{fontspec}
  \fi
  \defaultfontfeatures{Mapping=tex-text,Scale=MatchLowercase}
  \newcommand{\euro}{€}
\fi
% use microtype if available
\IfFileExists{microtype.sty}{\usepackage{microtype}}{}
\usepackage[margin=1in]{geometry}
\ifxetex
  \usepackage[setpagesize=false, % page size defined by xetex
              unicode=false, % unicode breaks when used with xetex
              xetex]{hyperref}
\else
  \usepackage[unicode=true]{hyperref}
\fi
\hypersetup{breaklinks=true,
            bookmarks=true,
            pdfauthor={},
            pdftitle={Méthodologie de calcul des rémunérations individuelles},
            colorlinks=true,
            citecolor=blue,
            urlcolor=blue,
            linkcolor=magenta,
            pdfborder={0 0 0}}
\urlstyle{same}  % don't use monospace font for urls
\setlength{\parindent}{0pt}
\setlength{\parskip}{6pt plus 2pt minus 1pt}
\setlength{\emergencystretch}{3em}  % prevent overfull lines
\setcounter{secnumdepth}{5}

%%% Change title format to be more compact
\usepackage{titling}
\setlength{\droptitle}{-2em}
  \title{Méthodologie de calcul des rémunérations individuelles}
  \pretitle{\vspace{\droptitle}\centering\huge}
  \posttitle{\par}
  \author{}
  \preauthor{}\postauthor{}
  \date{}
  \predate{}\postdate{}




\begin{document}

\maketitle


\section{Les convergences avec les statistiques de
l'INSEE}\label{les-convergences-avec-les-statistiques-de-linsee}

Les périmètres retenus sont autant que possible ajustés pour être
proches de ceux du système d'information sur les agents du service
public (SIASP).

L'étude des salaires est réalisée à partir des postes, restreints aux
postes actifs et non annexes.

Un certain nombre de différences méthodologiques avec les calculs de
l'INSEE et de la DGAFP doivent toutefois être relevées.

\subsection{La caractérisation des postes actifs non
annexes}\label{la-caracterisation-des-postes-actifs-non-annexes}

Dans le SIASP, les postes actifs et non annexes se définissent par trois
variables : rémunération, nombre d'heures, durée d'emploi.

On définit tout d'abord une période non annexe.

Une période est dite non annexe sous les conditions (C) suivantes :

\begin{enumerate}
\def\labelenumi{\arabic{enumi}.}
\itemsep1pt\parskip0pt\parsep0pt
\item
  si le salaire net du poste, indépendamment de sa durée, est supérieur
  à 3 fois le Smic mensuel net; ou :\\
\item
  si les trois conditions suivantes sont simultanément remplies :

  \begin{enumerate}
  \def\labelenumii{\alph{enumii})}
  \itemsep1pt\parskip0pt\parsep0pt
  \item
    la durée du poste est supérieure à 30 jours ;\\
  \item
    le nombre d'heures du poste est supérieur à 120 heures ;\\
  \item
    le nombre d'heures travaillées par jour au cours du poste est
    supérieur à 1,5 heure.
  \end{enumerate}
\end{enumerate}

Un poste est constitué d'une ou plusieurs périodes de travail d'un
salarié dans un même établissement.

On qualifie une période d'active si elle correspond à un temps complet,
à un temps partiel ou à un forfait, à une cessation progressive
d'activité, à un congé de maladie ordinaire (CMO) ou à un congé
formation (CF).

Sont considérées comme périodes inactives, les rappels, les périodes non
rémunérées, les périodes de chômage, les congés de fin d'activité (CFA)
et les congés longue maladie (CLM).

Un poste est actif s'il est constitué d'au moins une période active.

On qualifie un poste de non annexe sous les conditions suivantes :

\begin{itemize}
\itemsep1pt\parskip0pt\parsep0pt
\item
  si le salarié a été rémunéré pour au moins une période non annexe ;\\
\item
  sinon, si les conditions (C) sont vérifiées pour l'ensemble du poste,
  à défaut de l'être pour une période donnée.
\end{itemize}

Dans le cas contraire, le poste est dit annexe.

Pour les calculs réalisés par Altair, les périodes sont systématiquement
annuelles ou infra-annuelles et les filtres (C) sont appréciés sur une
seule période globalisée. Le filtrage par la condition d'activité et les
conditions (C) est réalisé, moyennant quelques approximations qui sont
précisées \emph{infra}.

Comme dans le SIASP, les élus locaux sont exclus du calcul des
rémunérations moyennes et taux de primes.\\Ils ne sont inclus que pour
les calculs de masses de rémunérations globales et dans ce cas leur
inclusion est signalée.

\subsection{Le calcul des quotités}\label{le-calcul-des-quotites}

Les quotités sont calculées à partir des postes actifs et non annexes et
de la variable Heures de travail.

Sauf indication contraire, les données de salaire sont exprimées en «
équivalent temps plein annualisé ». En matière d'effectifs, cette notion
est équivalente à la notion budgétaire d'ETPT (équivalents temps plein
travaillés) ou à celle d'EQTP utilisée dans le secteur privé.

Pour chaque poste, le salaire est transformé en un salaire en équivalent
temps plein annualisé (EQTP annualisé), correspondant au salaire qui
aurait été perçu par un poste à temps plein (notion de quotité) toute
l'année (notion de durée).

Pour chaque poste, ce salaire est pondéré par son poids en « équivalent
temps plein annualisé », autrement dit au prorata de la durée rémunérée
et de la quotité travaillée. Par exemple, un poste occupé durant 6 mois
à temps plein et rémunéré 10 000 euros compte pour 0,5 « équivalent
temps plein annualisé », rémunéré par an 20 000 euros. Un poste occupé
toute l'année avec une quotité travaillée de 60 \% et rémunéré 12 000
euros compte pour 0,6 « équivalent temps plein annualisé » rémunéré 20
000 euros par an.

Le salaire net annuel moyen en « équivalent temps plein annualisé » est
obtenu en pondérant les salaires annualisés des postes par le nombre d'«
équivalents temps plein annualisés ».

Le calcul de la quotité consiste à déterminer une durée médiane annuelle
de travail pour les postes actifs à temps complet, par strate. La durée
estimée au sein d'une strate sert ensuite de pondération pour ramener,
au sein de ces strates, les postes à temps partiel ou à temps non
complet à des « équivalent temps plein ».

\subsection{L'unicité du SIREN}\label{lunicite-du-siren}

Dans les fichiers SIASP, l'unité d'intérêt est le poste, ce qui
correspond à une ou plusieurs périodes de travail d'un salarié dans un
même établissement, défini dans la FPT par son SIREN.

Pour les traitements réalisés pour une seule collectivité simultanément,
il convient donc de n'utiliser, en entrée du logiciel, que des fichiers
de paye comportant des codes SIRET correspondant au même SIREN (les 9
premiers chiffres du SIRET). Cette condition n'est pas vérifiée par les
filtres d'importation des fichiers .xhl du logiciel à ce jour ( version
14.10) mais l'est en pratique. En effet, l'ouverture d'un contrôle étant
conditionnée à l'unicité de la personne morale vérifiée, les structures
rattachées à une même collectivité (régies non personnalisées, services
diversement autonomes,\ldots{}) ont des numéros SIRET correspondant au
même SIREN.

\section{Les approximations retenues par le logiciel
Altair}\label{les-approximations-retenues-par-le-logiciel-altair}

\subsection{La caractérisation des postes
actifs}\label{la-caracterisation-des-postes-actifs}

Les postes actifs sont caractéristés de manière plus rudimentaire : les
agents actifs sont ceux qui ont simultanément un nombre d'heures
travaillées non nul et un traitement strictement positif.

En principe ce double critère permet d'exclure les CFA et CLM. Il peut
aussi conduire à exclure, à tort, des CMO ou des CF de plus d'un mois.
Il est estimé que l'impact de cette aproximation reste raisonnable à
l'échelle de la précision visée. Cette approximation devrait faire
l'objet d'un examen spécifique.

\subsection{L'utilisation de données
brutes}\label{lutilisation-de-donnees-brutes}

La principale différence avec les traitements réalisés par les services
statistiques ministériels (SSM) a trait a l'absence de redressements
pour un certain nombre de variables connues pour leur qualité
insatisfaisante (grade et statut, nombre d'heures, entre autres).
L'impact de l'utilisation de données brutes, par rapport aux données
redressées des SSM, n'est pas évaluable à ce stade.

\subsection{Autres différences et
approximations}\label{autres-differences-et-approximations}

Pour la version 14.10 (novembre 2014), tout mois rémunéré est réputé
correspondre à 30 jours de travail, pour les critères de caractérisation
des périodes et postes annexes du 1.1, plus précisément pour le calcul
du nombre de jours travaillés et du nombre d'heures travaillées par jour
au cours de la période.

Cette approximation est provisoire et sera ultérieurement levée.

Elle a pour conséquence deux effets contraires : un nombre plus élevé de
postes et périodes annexes est pris en compte au titre du critère a).
Inversement un nombre plus faible de ces postes et périodes est pris en
compte au titre du critère c).

Par ailleurs, la définition des strates de calcul des quotités a dû être
revue pour éviter que les effectifs des strates ne soient trop faibles,
notamment pour les collectivités à effectifs limités. Là où l'INSEE
retient comme critères définitoires d'une strate ``la catégorie
hiérarchique, le statut, le sexe et la tranche d'âge'', seuls sont
retenus par Altair le statut modifié, l'emploi modifié et le sexe.

Par statut modifié, on entend une variable comprenant la variable statut
incluse dans le SIASP (avec les mêmes modalités) augmentée des modalités
suivantes : statut de vacataire (lorsque détecté), d'élu ou d'assistante
maternelle. Ces trois catégories statutaires sont exclues de l'analyse
d'office. En effet, pour les vacataires, les faibles effectifs et les
problèmes de qualité sur les variables heures et temps de travail
conduisent à des valeurs de rémunérations EQTP aberrantes dans un nombre
excessif de cas, compte tenu de l'absence de redressement. Les
assistantes maternelles, par ailleurs, ne sont pas comprises dans les
périmètres des statistiques publiées dans le cadre du Rapport annuel sur
l'état de la fonction publique. Enfin les élus locaux sont exclus des
calculs par l'INSEE.

Par emploi modifié on entend soit l'emploi, s'il y a au moins 5 emplois
à temps complet dans la strate, soit, à défaut, la médiane du même
statut.

Il est conjecturé qu'en l'absence de catégorie hiérarchique, l'emploi
modifié peut représenter une variable suffisamment bien liée à la
catégorie hiérarchique d'une part, à la tranche d'âge d'autre part, pour
se substituer à ces deux variables dans la détermination des strates.

\section{Algorithmes}\label{algorithmes}

Ces considérations méthodologiques permettent de dégager les algorithmes
dont l'exposé suit en pseudo-code.

\subsection{Filtrage des périodes}\label{filtrage-des-periodes}

Pour chaque Matricule et chaque année :

\begin{verbatim}
   Si il existe un bulletin mensuel dans l'année tel que pour la première ligne de paie vérifiant :
        Type == "TRAITEMENT" [le traitement indiciaire]  
      on a  : Montant > minimum.positif  
   et si Heures > minimum.positif  pour ce mois
   alors   
      Filtre_actif = vrai.  
   sinon  
      Filtre_actif = faux.  
      
   Pour année:
    2013 :
      si Rémunération.annuelle.nette  > 3361
    2012 :
      si Rémunération.annuelle.nette  > 3322
    2011 :
      si Rémunération.annuelle.nette  > 3222
    2010 :
      si Rémunération.annuelle.nette  > 3169
    2009 :
      si Rémunération.annuelle.nette  > 3132
    2008 :
      si Rémunération.annuelle.nette  > 3076  
      
   ou si :
    nombre.mois > 1 et   
    somme(Heures) > 120 heures et  
    somme(Heures)/nombre(jours) > 1,5.   
   
   alors :
    Filtre_non_annexe = vrai
    
   Pour les distributions, moyennes et médianes de rémunérations :   
   Retenir le périmètre combiné : 
      Filtre_non_annexe == vrai et Filtre_actif == vrai
    
\end{verbatim}

\subsection{Calcul des quotités}\label{calcul-des-quotites}

\begin{verbatim}
Pour chacun des statuts S dans {TITULAIRE, STAGIAIRE, AUTRE_STATUT},    
     et chacun des sexes G dans {HOMME, FEMME} :  

  calculer la médiane M0[S, G] des Heures des agents tels que :  
  
     Temps.de.travail == 100   
     et Heures est renseigné   
     et Heures > minimum.positif.  
\end{verbatim}

\begin{verbatim}
Pour chaque Emploi et Sexe dans la période calculer l'effectif E des agents tels que :
   
     Temps.de.travail == 100   
     et Filtre_actif == vrai
     et Heures est renseigné   
     et Heures > minimum.positif.  
\end{verbatim}

\begin{verbatim}
  Si E > 4  
  alors 
     calculer M = médiane(Heures) sur cet effectif
  sinon
     M = M0[S, G], pour S le statut de l'agent et G son sexe.   
\end{verbatim}

\begin{verbatim}
Alors si M est non-manquante et M > minimum.positif :  

         Quotité = Heures / M

      sinon M = M0[S, G]   
\end{verbatim}

\section{Références}\label{references}

Nouveaux compléments méthodologiques suite à l'introduction du système
d'information SIASP, INSEE- DGAFP mars 2013.

Décompte des emplois et mesure des évolutions de salaires dans les trois
versants de la fonction publique, INSEE-DGAFP novembre 2011.

\end{document}
